\documentclass[10pt]{report}
\usepackage[utf8]{inputenc}

\usepackage[spanish]{babel} % Idioma
\selectlanguage{spanish}
\usepackage{graphicx} %imagenes
\usepackage{verbatim} % Para comment


\usepackage{appendix} %Para el apéndice


\usepackage{ragged2e} %para alinear texto izqda y derecha



\usepackage{geometry}
%\geometry{ % Margenes
%  a4paper,
 % left=20mm,
 % right=20mm,
  %top=20mm,
 % bottom=20mm
%}
%\setlength{\parindent}{0pt} % Quitar indentado parrafos automatico

\usepackage{amsmath} % Movidas útiles
\usepackage{amssymb} % Simbolos mates
\usepackage{amsthm} % Personalizar teoremas (mas abajo continuacion)
\usepackage{thmtools} % Mas movidas teoremas

\usepackage{tikz} % Para pictures
\usetikzlibrary{decorations.markings,arrows} %decoracion en tikz
\usetikzlibrary{shadows,arrows,positioning,shapes.geometric}
\usetikzlibrary{decorations,decorations.markings}
\usepackage{pgfplots}
\usetikzlibrary{intersections, pgfplots.fillbetween}
\usetikzlibrary {arrows.meta}

\usepackage{etoolbox}
\usetikzlibrary{optics}
\usepackage{pgf,tikz}
\usetikzlibrary{arrows}
\usetikzlibrary{babel}
\usepackage{hyperref}
\usepackage[all]{hypcap}


\usepackage{sectsty} % Personalizar titulos secciones
%\sectionfont{\underline} % Titulo seccion subrayado

% Comandos simbolos utiles 
\newcommand{\C}{\mathbb{C}}
\newcommand{\R}{\mathbb{R}}
\newcommand{\Q}{\mathbb{Q}}
\newcommand{\Z}{\mathbb{Z}}
\newcommand{\N}{\mathbb{N}}
\newcommand{\Epsilon}{\mathcal{E}}
\DeclareMathOperator{\interior}{int} %interior


\newcommand{\norm}[1]{\left\lVert#1\right\rVert} % Comando para normas
\newcommand{\Esfera}{\mathbb{S}^2}
\newcommand{\Toro}{\mathbb{T}^2}
\newcommand{\Proyectivo}{\mathbb{P}^2}
\newcommand{\enfatiza}[1]{\textbf{\textit{#1}}}

% Estilos teoremas (Mejorable)
\theoremstyle{definition}
\newtheorem{defin}{Definición}[section]
\newtheorem{tma}[defin]{Teorema}
\newtheorem*{tma*}{Teorema}
\newtheorem{corol}[defin]{Corolario}
\newtheorem{prop}[defin]{Proposición}
\newtheorem{lema}[defin]{Lema}
%\newcommand{\demo}{Demostración.\\}
%\newcommand{\ok}{\hfill$\square$}
%\theoremstyle{remark}
\newtheorem{obs}[defin]{Observación}
\newtheorem{eje}[defin]{Ejemplo}
\graphicspath{ {images/} }
\usepackage{afterpage}

\newcommand\blankpage{%
    \null
    \thispagestyle{empty}%
    \newpage}


% Cabeceras
\renewcommand{\title}{ZIP}
\newcommand{\subtitle}{Trabajo fin de grado}
\renewcommand{\maketitle}{{\Large{\textbf{\title}}}\\\\{\Large \subtitle}\\\rule{17cm}{0.4pt}\\}

\begin{document}

%%%%%%%%%%%%%%%%%%%%%%%%%%%%%%%%%%%%%%%%%%%%%%%%%%%%%%%%%%%%%%%%%%%%%%%%%%%%%%%%%%%%%%%%%%%%%%%%%%%%%%%%%%%%%%%%%%%%%%%%%%%%%%%%%%%%%%%%%%%%%%%%%%%%%%%%%%%%%%%%%%%%%%%%%%%%%%%%%%%%%%%%%%%%%%%%%%%%%%%%%%%%%%%%%%%%%%%%%%%%%%%%%%%%%%%%%%%%%%%%%%%%%%%%%%%%%%%%%%%%%%%%%%%%%%%%%%%%%%%%%%%%%%%%%%%%%%%%%%%%%%%%%%%%%%%%%%%%%%%%%%%%%%%%%%%%%%%%%%%%%%%%%%%%%%%%%%%%%%%%%%%%%%%%%%%%%%%%%%%%%%%%%%%%%%%%%%%%%%%%%%%%%%%%%%%%%%%%%%%%%%%%%%%%%%%%%%%%%%%%%%%%%%%%%%%%%%%%%%%%%%%%%%%%%%%%%%%%%%%%%%%%%%%%%%%%%%%%%%%%%%%%%%%%%%%%%%%%%%%%%%%%%%%%%%%%%%%%%%%%%%%%%%%%%%%%%%%%%%%%%%%%%%%%%%%%%%%%%%%%%%%%%%%%%%%%%%%%%%%%%%%%%%%%%%%%%%%%%%%%%%%%%%%%%%%%%%%%%%%%%%%%%%%%%%%%%%%%%%%%%%%%%%%%%%%%%%%%%%%%%%%%%%%%%%%%%%%%%%%%%%%%%%%%%%%%%%%%%%%%%%%%%%%%%%%%%%%%%%%%%%%%%%%%%%%%%%%%%%%%%%%%%%%%%%%%%%%%%%%%%%%%%%%%%%%%%%%%%%%%%%%%%%%%%%%%%%%%%%%%%%%%%%%%%%%%%%%%%%%%%%%%%%%%%%%%%%%%%%%%%%%%%%%%%%%%%%%%%%%%%%%%%%%%%%%%%%%%%%%%%%%%%%%%%%%%%%%%%%%%%%%%%%%%%%%%%%%%%%%%%%%%%%%%%%%%%%%%%%%%%%%%%%%%%%%%%%%%%%%%%%%%%%%%%%%%%%%%%%%%%%%%%%%%%%%%%%%%%%%%%%%%%%%%%%%%%%%%%%%%%%%%%%%%%%%%%%%%%%%%%%%%%%%%%%%%%%%%%%%%%%%%%%%%%%%%%%%%%%%%%%%%%%%%%%%%%%%%%%%%%%%%%%%%%%%%%%%%%%%%%%%%%%%%%%%%%%%%%%%%%%%%%%%%%%%%%%%%%%%%%%%%%%%%%%%%%%%%%%%%%%%%%%%%%%%%%%%%%%%%%%%%%%%%%%%%%%%%%%%%%%%%%%%%%%%%%%%%%%%%%%%%%%%%%%%%%%%%%%%%%%%%%%%%%%%%%%%%%%%%%%%%%%%%%%%%%%%%%%%%%%%%%%%%%%%%%%%%%%%%%%%%%%%%%%%%%%%%%%%%%%%%%%%%%%%%%%%%
%%%%%%%%PORTADA%%%%%%%%%%%

%%% PORTADA%%%%%%
\begin{titlepage} %Creo que esto es para la numeración de páginas
\begin{center} %Que todo quede centradito

% Todo esto de abajo habría que retocarlo pero así sirve de ejemplo
\huge\textsc{Universidad Complutense de Madrid}\\[0.2in]
\includegraphics[scale=0.8]{comlu}\\[0.1in] %Introduce la imagen y la reescala, inserta un pequeño hueco con lo de debajo

\Large{Facultad de Matemáticas}\\[0.5in] %inserta un hueco mayor con lo de abajo
 %la linea horizontal
\Large{Trabajo de Fin de Grado}\\[.1in]
\Huge {Un tratamiento riguroso de la prueba ZIP}\\[0.2in]



\vfill %Llenar verticalmente
\Large {Juan Valero Oliet}\\[0.5in]
\vfill 
Dirigido por:\\
Manuel Alonso Morón\\[.1in]
\Large{Junio de 2020}
\end{center}

\end{titlepage}


\afterpage{\blankpage}

%%%%%%%%%%%%%PORTADA%%%%%%%%%%%%%%%%%%%%%
%%%%%%%%%%%%%%%%%%%%%%%%%%%%%%%%%%%%%%%%%%%%%%%%%%%%%%%%%%%%%%%%%%%%%%%%%%%%%%%%%%%%%%%%%%%%%%%%%%%%%%%%%%%%%%%%%%%%%%%%%%%%%%%%%%%%%%%%%%%%%%%%%%%%%%%%%%%%%%%%%%%%%%%%%%%%%%%%%%%%%%%%%%%%%%%%%%%%%%%%%%%%%%%%%%%%%%%%%%%%%%%%%%%%%%%%%%%%%%%%%%%%%%%%%%%%%%%%%%%%%%%%%%%%%%%%%%%%%%%%%%%%%%%%%%%%%%%%%%%%%%%%%%%%%%%%%%%%%%%%%%%%%%%%%%%%%%%%%%%%%%%%%%%%%%%%%%%%%%%%%%%%%%%%%%%%%%%%%%%%%%%%%%%%%%%%%%%%%%%%%%%%%%%%%%%%%%%%%%%%%%%%%%%%%%%%%%%%%%%%%%%%%%%%%%%%%%%%%%%%%%%%%%%%%%%%%%%%%%%%%%%%%%%%%%%%%%%%%%%%%%%%%%%%%%%%%%%%%%%%%%%%%%%%%%%%%%%%%%%%%%%%%%%%%%%%%%%%%%%%%%%%%%%%%%%%%%%%%%%%%%%%%%%%%%%%%%%%%%%%%%%%%%%%%%%%%%%%%%%%%%%%%%%%%%%%%%%%%%%%%%%%%%%%%%%%%%%%%%%%%%%%%%%%%%%%%%%%%%%%%%%%%%%%%%%%%%%%%%%%%%%%%%%%%%%%%%%%%%%%%%%%%%%%%%%%%%%%%%%%%%%%%%%%%%%%%%%%%%%%%%%%%%%%%%%%%%%%%%%%%%%%%%%%%%%%%%%%%%%%%%%%%%%%%%%%%%%%%%%%%%%%%%%%%%%%%%%%%%%%%%%%%%%%%%%%%%%%%%%%%%%%%%%%%%%%%%%%%%%%%%%%%%%%%%%%%%%%%%%%%%%%%%%%%%%%%%%%%%%%%%%%%%%%%%%%%%%%%%%%%%%%%%%%%%%%%%%%%%%%%%%%%%%%%%%%%%%%%%%%%%%%%%%%%%%%%%%%%%%%%%%%%%%%%%%%%%%%%%%%%%%%%%%%%%%%%%%%%%%%%%%%%%%%%%%%%%%%%%%%%%%%%%%%%%%%%%%%%%%%%%%%%%%%%%%%%%%%%%%%%%%%%%%%%%%%%%%%%%%%%%%%%%%%%%%%%%%%%%%%%%%%%%%%%%%%%%%%%%%%%%%%%%%%%%%%%%%%%%%%%%%%%%%%%%%%%%%%%%%%%%%%%%%%%%%%%%%%%%%%%%%%%%%%%%%%%%%%%%%%%%%%%%%%%%%%%%%%%%%%%%%%%%%%%%%%%%%%%%%%%%%%%%%%%%%%%%%%%%%%%%%%%%%%%%%%%%%%%%%%%%%%%%%%%%%%%%%%%%%%%%%%%%%%%%%%%%%%%%%%%%%%%%%%%%%%%%%%%%%%%%%%%%%%%%%%%%%%%%%%%%%%%%%%%%%%%%%%%%%%%%%%%%%%%%%%%%%%%%%%%%%%%%%%%%%%%%%%%%%%%%%%%%%%%%%%%%%%%%%%%%%%%%%%%%%%%%%%%%%%%%%%%%%%%%%%%%%%%%%%%%%%%%%%%%%%%%%%%%%%%%%%%%%%%%%%%%%%%%%%%%%%%%%%%%%%%%%%%%%%%%%%%%%%%%%%%%%%%%%%%%%%%%%%%%%%%%%%%%%%%%%%%%%%%%%%%%%%%%%%%%%%%%%%%%%%%%%%%%%%%%%%%%%%%%%%%%%%%%%%%%%%%%%%%%%%%%%%%%%%%%%%%%%%%%%%%%%%%%%%
\pagenumbering{roman} 

\tableofcontents
\listoffigures



\chapter{Definiciones preeliminares}
\pagenumbering{arabic} 
En este capítulo doy las definiciones y resultados preeliminares para la representación y clasificación de superficies. Me basaré principalmente en los libros de J. M. Lee \cite{lee1} y V. Muñoz - J. J. Madrigal \cite{juanjo}.

\section{Variedades}

Los espacios topológicos de los que nos vamos a ocupar en el siguiente trabajo son las variedades, y en concreto las superficies. Definámoslas.

\begin{defin}%%%%DEF: variedad topológica
Una \textbf{\emph{variedad topológica}} (de ahora en adelante \emph{variedad}) es un espacio topológico Hausdorff, II AN y localmente homeomorfo a $R^n$, para algún $n\geq 0$.
\end{defin}

\begin{obs}%%%%OBS: toda propiedad local de Rn se pasa a las variedades
Como la propiedad ``ser localmente homeomorfo a $\R^n$'' es local, toda propiedad local de $\R^n$ se traslada a una variedad. Así, las variedades son localmente compactas, I AN, localmente conexas, localmente conexas por caminos y localmente simplemente conexas.\\
\end{obs}


%COPIADO lo de abajo, cambiar las palabras!!!!!!!!!!!!

El \textit{teorema de invarianza del dominio} dice que si $W\subset \R^n$ y $W'\subset \R^m$ son abiertos y existe $\phi: W \rightarrow W'$ homeomorfismo, entonces $n=m$. Esto implica que, dado un punto $p\in X$ de una variedad, hay un único $n=n(p)$ tal que un entorno $U^p$ es homeomorfo a un abierto $U'\subset \R^n$. Llamamos $n(p)$ la dimensión en p. Claramente, para todo punto $q\in U$ podemos tomar $U$ como entorno de $q$, y por tanto $n(q)=n(p)$. Luego en toda la componente conexa de $p$, el $n$ que aparece es el mismo, y lo llamaremos dimensión de dicha componente conexa. Nótese que si escribimos $X=\sqcup X_i$, con $X_i$ componentes conexas de $X$, todas las $X_i$ son variedades. Si todas las $X_i$ tienen la misma dimensión $n$, entonces escribimos $n=dim X$ , y decimos que $X$ es una $n$-variedad.\\



\begin{eje}%%%%EJE: variedades.
\begin{itemize}
\item Las 0-variedades son espacios discretos numerables. La única 0-variedad conexa es un punto.
\item Existen dos 1-variedades conexas salvo homeomorfismo: la recta $\R$ y el círculo $\mathbb{S}^1$.
\end{itemize}
\end{eje}

\begin{defin}%%%%DEF: SUPERFICIE
Una \textbf{\emph{superficie}} es una $2$-variedad.
\end{defin}



\begin{eje}%%%%EJE: toro y esfera
\begin{itemize}
\item La esfera $\mathbb{S}^2=\{(x,y,z) \in \R^3\mid  x^2+y^2+z^2=1\}$.

\item El toro $\mathbb{T}^2=\{(x,y,z)\in \R^3\mid  (\sqrt{x^2+y^2}-2)^2+z^2=1\}$.
\end{itemize}

\end{eje}



%%%%%%%%%%%%%
%%%%%%%%%%%%%%
%%%%%%%%%%%%%%
%%%%%%%%%%%%%%%
Para el teorema de clasificación necesitamos un método uniforme de representación de las superficies compactas. Trataremos de dar una forma de representarlas como polígonos, y veremos que toda superficie compacta se puede representar en el plano como el cociente de un polígono por una relación de equivalencia que identifica sus lados dos a dos. Empecemos viendo tres ejemplos elementales: la esfera $\Esfera$, el plano proyectivo $\Proyectivo$ y el toro $\Toro$. Como veremos, estas superficies son fundamentales pues toda superficie compacta se puede construir a partir de ellas. 
Para ello necesitaremos antes la siguiente proposición:





\begin{prop}% TEOREMA 5.1 LEE%%%% CONVEXO HOMEOM ESFERA
\label{teo:convexo_homeom_esfera} 
Si $D\subseteq \R^n$ es un conjunto compacto y convexo con interior no vacío, entonces $D$ es homeomorfo a $\overline{\mathbb{B}}^n.$ De hecho, dado $p\in \mathring{D}$, entonces existe un homeomorfismo $F:\overline{\mathbb{B}}^n\to D$ que envía $0$ a $p$, $\overline{\mathbb{B}}^n$ a $\mathring{D}$, y $\mathbb{S}^{n-1}$ a $\partial D$.
\end{prop}
\begin{proof}
Sea $p\in D$ un punto de su interior. Si reemplazamos $D$ por su imagen mediante la traslación $x\mapsto x-p$, que es un homeomorfismo de $\R^n$ en sí mismo, podemos asumir que $p=0\in \mathring{D}$. Entonces existe un $\varepsilon >0 $ tal que la bola $B_{\varepsilon}(0)$ está contenida en $D$. Usando la dilatación $x\mapsto x/\varepsilon$, podemos asumir que $\mathbb{B}^n = B_ 1(0) \subseteq D$.
La clave de la demostración es la siguiente: \emph{cada semirecta cerrada empezando en el origen interseca $\partial
D$ en exactamente un punto}. Sea $R$ una semirecta así. Dado que $D$ es compacto, su intersección con $R$ es compacta. Por tanto existe un punto $x_0$ en su intersección tal que en él su distancia al origen asume el máximo. Es claro %lo es?
que pertenece a la frontera de $D$. Para ver que el punto es único, veamos que el segmento que une $0$ y $x_0$ está formado enteramente por puntos interiores de $D$ excepto por el $x_0$ mismo. Cualquier punto en este segmento distinto de $x_o$ se puede escribir de la forma $\lambda x_0 $ para $0\leq \lambda <1. $ Supongamos $z\in B_{1-\lambda}(\lambda x_0)$, y sea $y=(z-\lambda x_0)/(1-\lambda )$. Como $|z-\lambda x_0|<|1-\lambda|$ se tiene que $|y|<1$, y por tanto $y\in B_ 1(0)\subseteq D$ (ver \autoref{fig:convexo_esfera}). Como $y$ y $x_0$ están en $D$ y $z=\lambda x_0 + (1-\lambda )y$, se sigue de la convexidad que $\in D$. Por tanto la bola abierta $B_{1-\lambda}(\lambda x_0)$ está contenida en $D$, lo que implica que $\lambda x_0$ es un punto interior.

Definimos ahora la aplicación $f:\partial D \to \mathbb{S}^{n-1}$ por 
$$f(x)=\frac{x}{|x|}$$

$f(x)$ es el punto donde el segmento desde el origen hasta $x$ interseca la esfera unidad. Como $f$ es la restricción de una función continua, es continua, y por el parágrafo anterior es biyectiva. Dado que $\partial D$ es compacta, $f$ es un homeomorfismo por el teorema de la aplicación cerrada (\autoref{teo:aplicac_cerrada}).

Finalmente definimos $F:\overline{\mathbb{S}}^n \to D$ por 
$$F(x)= \left\{\begin{array}{lc}
				|x|f^{-1}\left(\dfrac{x}{|x|}\right), & x\neq 0;
				\\0, & x=0.

\end{array}
\right. $$
$F$ es continua fuera del origen por serlo $f^{-1}$, y en el origen porque por ser $f^{-1}$ acotada $F(x) \to 0$ cuando $x\to 0$. Geometricamente, $F$ manda cada segmento radial que conecta 0 con un punto de $\mathbb{S}^{n-1}$ al segmento radial desde $0$ hasta el punto $f^{-1}(w)\in \partial D$. Por convexidad, $F$ toma valores en $D$. La aplicación $F$ es inyectiva, pues puntos de distintas semirectas van a parar a distintas semirectas, y cada segmento radial va linealmente a su imagen. Es sobreyectiva pues cada punto $y \in D$ está en una semirecta empezando en 0. Por el teorema de la aplicación cerrada, $F$ es un homeomorfismo.
\end{proof}





\


\begin{prop}%%%%PROP: ESFERA COCIENTE DISCO Y CUADRADO%%%%
\label{prop:Esfera como cociente de disco y cuadrado}
La esfera $\mathbb{S}^2$ es homeomorfa a los siguientes espacios cociente: 
%%%%%%%%%%o cocientes??????????%%%%%
\begin{itemize}
\item[(a)] El disco cerrado $\overline{\mathbb{B}}^2\subseteq \mathbb{R}^2$ módulo la relación de equivalencia generada por $(x,y)\sim (-x,y)$, si $(x,y)\in \partial \overline{\mathbb{B}}^2$
\item[(b)] El cuadrado $S=\{(x,y):|x|+|y|\leq 1\}$ módulo la relación de equivalencia generada por $(x,y)\sim(-x,y)$ si $(x,y)\in \partial S$.
\end{itemize}
\end{prop}
\begin{proof}
Para ver que cada espacio es homeomorfo a la esfera, daremos una aplicación cociente desde cada espacio a la esfera que haga las mismas identificaciones que la relación de equivalencia, y entonces apelaremos a la unicidad del espacio cociente. (\autoref{teo:unicidad_espacio_cociente})\\
Para (a), vamos a definir una aplicación que ``envuelve'' cada segmento horizontal del disco en un paralelo de la esfera (ver \autoref{fig:esfera_cociente_circunferencia}).
Formalmente, esta aplicación $\pi:\overline{\mathbb{B}}^2\to \mathbb{S}^2$ vienen dada por 
$$\pi(x,y)=\left\{\begin{array}{lc}
			(-\sqrt{1-y^2} \cos\dfrac{\pi x}{\sqrt{1-y^2}}, -\sqrt{1-y^2}, y), & y\neq \pm 1 \\
			\\(0,0,y), & y=\pm1 

\end{array}
\right.$$
Es claro que $\pi$ es continua y hace las mismas identificaciones que la relación de equivalencia. Por ser sobreyectiva, es una aplicación cociente (\autoref{teo:aplicac_cerrada}).

Para probar (b), sea $\alpha:S\to \overline{\mathbb{B}}^2$ el homeomorfismo construido en la demostración de \autoref{teo:convexo_homeom_esfera} que manda linealmente cada segmento radial entre el origen y la frontera de $S$ al segmento paralelo entre centro del disco y su frontera. Hagamos ahora $\beta=\pi \circ \alpha : S \to \mathbb{S}^2$, donde $\pi$ es la aplicación cociente del parágrafo anterior. Tenemos entonces que $\beta$ identifica $(x,y)$ y $(-x,y)$ cuando $(x,y)\in \partial S$, pero por otro lado es inyectiva, así que hace las mismas identificaciones que la aplicación cociente definida en (b), completando así la demostración (ver \autoref{fig:esfera_cuadrado}). 
\end{proof}


\begin{prop}%%%%PROP: TORO COMO CUADRADO
\label{prop:toro_cuadrado}
El toro $\Toro$ es homeomorfo al espacio cociente resultante de la relación de equivalencia en el cuadrado $I\times I$ que identifica $(x,0)\sim (x,1)$ para todo $x\in I$, y $(0,y)\sim(1,y)$ para todo $y\in I$ (\autoref{fig:toro_cuadrado}). 
\end{prop}
\begin{proof}
Definimos la aplicación $q:I\times I \to \Toro$ que manda $q(u,v)=(e^{2\pi iu},e^{2\pi iv})$. Por el teorema de la aplicación cerrada (\autoref{teo:aplicac_cerrada}), es una aplicación cociente. Al hacer las mismas identificaciones que la relación de equivalencia, por la unicidad del espacio cociente (\autoref{teo:unicidad_espacio_cociente}) se obtiene el resultado.
\end{proof}







\begin{prop}%%%%PROP: PLANO PROYECTIVO COMO CUADRADO
\label{prop:proyectivo_cociente_cuadrado}
El plano proyectivo $\mathbb{P}^2$ es homeomorfo a los siguientes espacios cociente:
\begin{itemize}
\item[(a)] El disco cerrado $\overline{\mathbb{B}}^2$ módulo la relación de equivalencia generada por $(x,y) \sim (-x,-y)$ para cada $(x,y)\in \partial \overline{\mathbb{B}}^2$.
\item[(b)] La región cuadrada $S=\{(x,y):|x|+|y|\leq 1\} $ módulo la relación de equivalencia generada por $(x,y)\sim (-x,-y) $ para todo $(x,y)\in \partial S$.
\end{itemize}
\end{prop}
\begin{proof}
Sea $p:\mathbb{S}^2 \to \mathbb{P}^2$ la aplicación cociente dada por la relación de equivalencia $\sim$ generada por $(x,y) \sim (-x,-y)$ para cada $(x,y)\in \mathbb{S}^2$, que representa $\mathbb{P}^2$ como el cociente de una esfera. %AQUI FALTA UN POCO DE EXPLICACION, EJEMPLO 4.54 DEL LEE
Si $F:\overline{\mathbb{B}}^2 \to \mathbb{S}^2$ es la aplicación que manda el disco al emisferio superior de la esfera por $F(x,y)=(x,y,\sqrt{1-x^2-y^2})$, entonces $p\circ F:\overline{\mathbb{B}}^2 \to \mathbb{S}^2/\sim$ es sobreyectiva (---lo demuestro?) y es así una aplicación cociente por el teorema de la aplicación cerrada (\autoref{teo:aplicac_cerrada}). La aplicación identifica únicamente $(x,y)\in \partial \overline{\mathbb{B}}^2$ con $(-x,-y)\in \partial \overline{\mathbb{B}}^2$, por lo que $\mathbb{P}^2$ es homeomorfo al espacio cociente resultante.
Para la parte (b) hacemos como en la demostración de la \autoref{prop:Esfera como cociente de disco y cuadrado} (b). 
\end{proof}





En las anteriores proposiciones hemos visto una o varias formas de representar superficies dadas ciertas construcciones geométricas. En estos casos hemos dado aplicaciones y demostraciones concretas para validar nuestros argumentos, pero a medida que aumenta la sofisticación es más útil guiarse visualmente por las figuras construidas. Por ello debemos formalizar un método para construir superficies identificando lados de figuras geométricas.
Daremos por sabidas las definiciones básicas de símplices y CW-complejos que dejaremos en el apéndice \autoref{apendice simplices}.

\begin{defin}%%%DEF:Polígono
Un \enfatiza{polígono} es un subconjunto de $\R^2$ que es homeomorfo a $\mathbb{S}^1$ y está formado por un número finito de segmentos, que llamaremos \enfatiza{bordes} y que se intersecan sólo en sus extremos, que llamaremos \enfatiza{vértices}. %Los $0$-símplices y $1$-símplices del poígono son respectivamente sus \enfatiza{vértices} y sus \enfatiza{bordes}. Del lema \autoref{lemma:cw} se sigue que un borde yace exactamente en dos vértices.
\end{defin}

\begin{defin}%%%DEF:Región poligonal

Una \enfatiza{región poligonal} es un subconjunto compacto de $\R^2$ cuyo interior es homeomorfo al disco $\mathbb{B}^2$ y cuya frontera es un polígono.
%Una \enfatiza{región poligonal} es un subconjunto compacto de $\R^2$ cuyo interior es una bola coordenada y cuya frontera es un polígono.
A los vértices y lados del polígono de la frontera también los llamamos vértices y lados de la región poligonal.
\end{defin} 

Veamos pues que identificando bordes de regiones poligonales de par en par obtenemos un espacio cociente que es siempre una superficie:

\begin{prop}%%%PROP: Teorema poligonos
\label{prop:poligonos}
Sean $P_1,\dots, P_k$ regiones poligonales en el plano, y sea $P=P_1\amalg \dots \amalg P_k$, y supongamos dada una relación de equivalencia en $P$ que identifica algunos bordes de los polígonos con otros por homeomorfismos afines. Entonces se tiene:
\begin{itemize}
\item[(a)] El espacio cociente resultante es un CW-complejo $2$-dimensional cuyo $0$-esqueleto es la imagen del conjunto de vértices de $P$ por la aplicación cociente, y cuyo $1$-esqueleto es la imagen de la unión de los bordes de las regiones poligonales.
\item[(b)] Si la relación de equivalencia identifica cada borde de cada $P_i$ con exactamente otro borde de un $P_j$ (no necesariamente $i\neq j$), entonces el espacio cociente resultante es una superficie compacta.
\end{itemize}
\end{prop}

\begin{proof}
%...6.4 del Lee.
\end{proof}

\begin{eje}%%%%EJE: Botella Klein
La \enfatiza{botella de Klein} es la superficie $K$ obtenida identificando los lados del cuadrado $I\times I$ de acuerdo a $(0,t)\sim (1,t)$ y $(t,0)\sim (1-t,1)$ para $0\leq t\leq 1$.  Para visualizar $K$, podemos pensar en pegar los lados izquierdo y derecho creando un cilindro, y luego hacer pasar el extremo superior por la parte inferior del cilindro, para finalmente pegar los dos extremos (ver \autoref{fig:klein}).
\end{eje}



%%%%%%%%%%%%%%%%%%%%%%%%%%%%%%%%%%%%%%%%%%%%%%%%%%%%%%%%%%%%%%%%%%%%%%%%%%%%%%%%%%%%%%%%%%%%%%%%%%%%%%%%%%%%%%%%%%%%%%%%%%%%%%%%%%%%%%%%%%%%%%%%%%%%%%%%%%%%%%%%%%%%%%%%%%%%%%%%%%%%%%%%%%%%%%%%%%%%%%%%%%%%%%%%%%%%%%%%%%%%%%%%%%%%%%%
 
\section{Representación de superficies}



Como ya hemos dicho anteriormente, para el teorema de clasificación necesitamos un método uniforme de representación de las superficies compactas. Representaremos todas las superficies como cocientes de regiones poligonales con $2n$ lados. 


\begin{defin}%%%%DEF: palabra
Sea $S$ un conjunto. Una \textbf{\textit{palabra en $S$}} es una $k$-tupla ordenada de símbolos, cada uno de la forma $a$ o $a^{-1}$, para cierto $a\in S$.
\end{defin}

\begin{defin}%%%%DEF: Representación poligonal
\label{def:rep_pol}
Una \textbf{\textit{representación poligonal}}, que denotaremos por $$\mathcal{P}=\langle S\mid W_1,...,W_k\rangle$$ es un conjunto finito S junto con un número finito de palabras $W_1,..,W_k$ de longitud $3$ o más, tal que para todo $a\in S$ existe un $W_i$ tal que $a\in W_i$. Por cuestiones de notación, cuando el conjunto $S$ esté descrito listando sus elementos, quitaremos los corchetes que rodean los elementos de $S$ y denotaremos las palabras $W_i$ por youxtaposición. Por ejemplo, la presentación con $S=\{a,b\}$ y la palabra $W=(a,b,a^{-1},b^{-1})$ se escribe $\langle a,b\mid  aba^{-1}b^{-1}\rangle$. 

Permitimos el caso especial de que $S=\{a\}$ (u otro símbolo cualquiera) y que $\mathcal{P}$ tenga una sola palabra de longitud $2$, es decir, $\langle a\mid aa\rangle$, $\langle a\mid a^{-1}a^{-1}\rangle$, $\langle a\mid aa^{-1}\rangle$ y $\langle a\mid a^{-1}a\rangle$.
\begin{comment} o $1$ \end{comment} 
\begin{comment}, $\langle a\mid a\rangle$ y $\langle a\mid a^{-1}\rangle$ \end{comment}
\end{defin}

\begin{defin}%%%%DEF: Realización geométrica
Toda representación poligonal $\mathcal{P}$ da lugar a un espacio topológico $|\mathcal{P}|$, llamado \textbf{\textit{realización geométrica de $\mathcal{P}$}} . $|\mathcal{P}|$ se obtiene de la siguiente manera:
\begin{itemize}
\item[1.] Para cada $W_i\in \mathcal{P}$ de longitud $k$, sea $P_i$ la $k$-región poligonal convexa centrada en el origen con lados de longitud 1, ángulos iguales y tal que un vértice yace sobre el eje $OY$.
\item[2.] Se define una correspondencia uno a uno entre los símbolos de $W_ i$ y los bordes de $P_i$ en sentido contrario a las agujas del reloj, empezando por el que yace en el eje $OY$.
\item[3.] Sea $|\mathcal{P}|$ el espacio cociente de $\coprod_i P_i$ determinado identificando bordes que tengan el mismo símbolo, conforme al homeomorfismo afín que hace coincidir los primeros vértices de lo bordes con una etiqueta dada $a$ y los últimos vertices de los que tienen la correspondiente etiqueta $a^{-1}$ (en el sentido contrario a las agujas del reloj).

\end{itemize}


Si $\mathcal{P}$ es una de las representaciones poligonales de un solo elemento, definimos $|\mathcal{P}|$ como la esfera $\mathbb{S}^2$ si la palabra es $aa^{-1}$ o $a^{-1}a$, o como el plano proyectivo $\mathbb{P}^2$ si es $aa$ o $a^{-1}a^{-1}$.% También decimos que $|\mathcal{P}|$ es el disco cerrado $\overline{\mathbb{B}}^2$ si la palabra es $a$ o $a^{-1}$.
\end{defin}

Por notación, dadas dos palabras $W_1$ y $W_2$, $W_1W_2$ representará la palabra formada concatenando $W_1$ y $W_2$. Por otro lado, adoptaremos la convención de que $(a^{-1})^{-1}=a$.

También lo que sigue $S$ denotará una secuencia cualquiera de símbolos, $a,b,c,a_1,a_2,\dots$ símbolos de $S$, $e$ un símbolo que no sea de $S$ y $W_1, W_2, \dots$ palabras formadas por símbolos de $S$.

\begin{defin}%%%%DEF: caras, lados y vértices polígono
Los interiores, los bordes y los vértices de cada región polgonal $P_i$ se llaman \textbf{\emph{caras, bordes y vértices de la presentación}}. El número de caras es el mismo que el número de palabras, y el número de bordes coincide con la suma de la longitud de las palabras.
Para un lado etiquetado $a$, el \textbf{\emph{vértice inicial}} es el primero en el sentido contrario de las agujas del reloj, y el otro es el \textbf{\emph{vértice final}}. Para un lado etiquetado $a^{-1}$, estas definiciones se invierten. 
\end{defin}

\begin{defin}%%%%DEF: representación de una superficie
Una representación poligonal es una \textbf{\emph{representación de una superficie}} si para todo $a\in S$, $a$ ocurre exáctamente dos veces en $W_1,...,W_k$ como $a$ o como $a^{-1}$.
Por la \autoref{prop:poligonos}, la realización geométrica de una representación de una superficie es una superficie compacta.
\end{defin}

\begin{defin}%%%%DEF: representación de un espacio topológico
Si $X$ es un espacio topológico y $\mathcal{P}$ una representación poligonal cuya realización geométrica es homeomorfa a $\mathcal{P}$, decimos que $\mathcal{P}$ es una \textbf{\emph{representación de $X$}}.
\end{defin}

\begin{obs}%%%%OBS: representación de una sola cara implica conexo
Un espacion topológico que admite una representación con una sola cara es conexo, pues es homeomorfo al cociente de una región poligonal conexa. Con más de una cara, puede ser o no conexo.
\end{obs}




\begin{eje}%%%%PROP: representación de superficies importantes
Veamos la representación de algunas superficies importantes (ver \autoref{fig:Representacion_superficies_importantes} y \autoref{fig:representacion_esfera_proyectivo}).
\begin{itemize}
\item[(a)] $\mathbb{S}^2=\langle a\mid aa^{-1}\rangle=\langle a,b\mid  abb^{-1}a^{-1}\rangle$ (\autoref{prop:Esfera como cociente de disco y cuadrado})
\item[(b)] $\mathbb{P}^2= \langle a\mid aa \rangle = \langle a,b\mid abab \rangle$ (\autoref{prop:proyectivo_cociente_cuadrado})
\item[(c)] $\Toro=\langle a,b\mid aba^{-1}b^{-1}\rangle$ (\autoref{prop:toro_cuadrado})
\end{itemize}
\end{eje}





De ahora en adelante visualizaremos el plano proyectivo como un \enfatiza{crosscap}, cuya construcción se sigue en la \autoref{fig:crosscap_paso_a_paso}.


Parece claro que, además de $\Esfera$ y $\Proyectivo$, una superficie pueda tener varias presentaciones poligonales. Sea por ejemplo la presentación del toro $\Toro=\langle a,b\mid aba^{-1}b^{-1}\rangle$. Intuitivamente podemos ver que, subdividiendo los lados etiquetados con $b$ y reetiquetándolos con $c$ y $d$ (ver \autoref{fig:ejemplo_operacion}), la superficie que representa la representación obtenida $\langle a,c,d\mid  acda^{-1}c^{-1}d^{-1}\rangle$ será la misma.
Vamos ahora a desarrollar unas reglas generales de transformación.



\begin{defin}%%%%DEF: Realizaciones geometricas topologicamente equivalentes
Sean $\mathcal{P}_1$ y $\mathcal{P}_2$ dos representaciones tal que sus realizaciones geométricas son equivalentes. Entonces decimos que son \enfatiza{topológicamente equivalentes} y escribimos $\mathcal{P}_1 \approx \mathcal{P}_2$.
\end{defin}

Vamos a definir ahora unas operaciones elementales sobre las representaciones poligonales. Veremos luego que estas dan lugar a representaciones equivalentes.

\begin{defin}%%%%DEF: Operaciones elementales sobre representaciones
Las siguientes operaciones se llaman \enfatiza{transformaciones elementales} de una presentación poligonal:
\begin{itemize}
\item \textit{Reetiquetar:} Cambiar todas las apariciones de un símbolo $a$ por otro símbolo que no está todavía en la representación, intercambiar todas las apariciones de dos símbolos $a$ y $b$ o intercambiar todas las apariciones de $a$ y $a^{-1}$.
\item \textit{Subdividir:} Cambiar todas las apariciones de $a$ por $ae$ y todas las de $a^{-1}$ por $e^{-1}a^{-1}$, donde $e$ es un símbolo que no está todavía en la presentación.
\item \textit{Consolidar:} Si $a$ y $b$ aparecen siempre de forma adyacente, intercambiar $ab$ por $a$ y $b^{-1}a^{-1}$ por $a^{-1}$, siempre que esto de lugar a una o más palabras de longitud al menos $3$ o una sola palabra de longitud $2$.
\item \textit{Reflejar:} $$\langle S\mid   a_1 \dots a_m, W_2,\dots,W_k\rangle \mapsto \langle S\mid   a_m^{-1}\dots a_1^{-1}, W_2, \dots ,W_k\rangle .$$
\item \textit{Rotar:} $$\langle S\mid a_1a_2\dots a_m, W_2,\dots , W_k\rangle \mapsto \langle S\mid   a_2\dots a_ma_1, W_2,\dots , W_k\rangle .$$
\item \textit{Cortar:} Si $W_1$ y $W_2$ tienen longitud al menos $2$, $$\langle S\mid W_1W_2, W_3,\dots , W_k\rangle \mapsto \langle S\mid W_1e, e^{-1}W_2, W_3,\dots W_k\rangle .$$
\item \textit{Pegar:} $$\langle S,e\mid W_1e, e^-1W_2, W_3,\dots , W_k\rangle \mapsto \langle S\mid W_1W_2, W_3,\dots , W_k\rangle .$$
\item \textit{Plegar:} Si $w_1$ tiene longitud al menos $3$, $$\langle S,e\mid W_1ee^{-1}, W_2,\dots W_k\rangle \mapsto \langle S\mid W_1, W_2,\dots , W_k\rangle .$$ Permitimos que $W_1$ tenga longitud $2$, siempre que la representación tenga una sola palabra.
\item \textit{Desplegar:} $$\langle S\mid W_1, W_2,\dots , W_k\rangle \mapsto \langle S,e\mid W_1ee^{-1}, W_2,\dots , W_k\rangle .$$
\end{itemize}
\end{defin}


\begin{prop}%%%%PROP: Operaciones sobre representaciones
Las operaciones elementales sobre representaciones poligonales dan lugar a representaciones poligonales equivalentes.
\end{prop}
\begin{proof}
El caso de reetiquetar es trivial. Los casos subdividir/consolidar, cortar/pegar y plegar/desplegar son inversos, con lo que bastará probar uno de cada par.  
\end{proof}
%%%%%%%%%%%%%%%%%%%%%%%%%%%%%%%%%%%%%%%%%%%%%%%%%%%%%%%%%%%%%%%%%%%%%%%%%%%%%%%%%%%%%%%%%%%%%%%%%%%%%%%%%%%%%%%%%%%%%%%%%%%%%%%%%%%%%%%%%%%%%%%%%%%%%%%%%%%%%%%%%%%%%%%%%%%%%%%%%%%%%%%%%%%%%%%%%%%%%%%%%%%%%%%%%%%%%%%%%%%%%%%%%%%%%%%%%%%%%%%%%%%%%%%%%%%%%%%%%%%%%%%%%%%%%%%%%%%%%%%%%%%%%%%%%%%%%%%%%%%%%%%%%%%%%%%%%%%%%%%%%%%%%%%%%%%%%%%%%%%%%%%%

\section{Suma conexa de variedades}


Sean $V_1$ y $V_2$ dos $n$-variedades conexas. Dados $p_1\in V_1$ y $p_2\in V_2$ sean $U_1^{p_1}\subset V_1$, $U_2^{p_2}\subset V_2$  entornos de $p_1$ y $p_2$ respectivamente, y sean $\phi_1:U_1\to\R^n$ y $\phi_2:U_2\to\R^n$ dos homeomorfismos tales que $\phi_1(p_1)=0$ y $\phi_2(p_2)=0$. Si llamamos $B_1=\phi_1^{-1}(B_1(0))\subset V_1$ y $B_2=\phi_2^{-1}(B_1(0))\subset V_2$, consideremos $V_1^o=V_1-B_1$, $V_2^o=V_2-B_2$ y $V_1^o \sqcup V_2^o$ con la topología unión disjunta.
Se define la relación de equivalencia $\sim$ en la que si $x_1\in S_1=\phi_1^{-1}(\partial B_1(0))$, $x_2\in S_2=\phi_2^{-1}(\partial B_1(0))$, entonces $x_1\sim x_2$ si y sólo si $\phi_1(x_1)=\phi_2(x_2)$, y se considera el cociente 

$$X=\frac{V_1^o\sqcup V_2^o}{\sim}.$$

\begin{defin}%%%% DEF: suma conexa
A $X$ así definido se le llama \textbf{\textit{suma conexa}} de $V_1$ y $V_2$, y se denota por $X=V_1\#V_2$.
\end{defin}







\begin{prop}%%%% PROP: suma conexa es variedad.
Sean $V_1$ y $V_2$ variedades. Entonces $X=V_1\#V_2$ es una variedad.
\end{prop}
\begin{proof}
Denotemos la proyección $\pi:M_1^o\sqcup M_2^o\to X$. Sea $S=\pi (S_1)=\pi (S_2)$. Tenemosdos abiertos $U_j=M_j^o-S_j$, $j=1,2$ saturados. Por tanto, $\pi :U_ j\to \pi (U_j)=U_j'$ es homeomorfismo. Esto implica que $X$ es localmente $\R^n$ en los puntos de $U_1'\cup U_2'$ . Además ahí la topología es Hausdorff y IIAN.
Veamos ahora qué ocurre para un punto $p\in S$. Se tiene que $p=\pi (p_1)=\pi (p_2)$, $p_j\in S_j$, $j=1,2$, y $\varphi_j(p_j)=x_0 \in \partial B_1(0) \in \R^n$. Tomamos un entorno $V\subset \partial B_1(0)$ de $x_0$ en $\partial B_1(0)$, con lo que $\hat{V}=\{rx\mid   r\in (1-\varepsilon , 1+\varepsilon ), x\in V\}$ es entorno de $x_0$ en $\R^n$, y $\hat{V}-B_1(0)=\{rx\mid r\in [1, 1+\varepsilon ), x \in V\}$. Sea $V_j=\varphi_j^{-1}(\hat{V}-B_1(0))\subset M_j^o$, que es entorno de $p_j$. Claramente $V_1\sqcup V_2$ es abierto saturado de $M_1^o\sqcup M_2^o$, luego $\tilde{V}=\pi (V_1\sqcup V_2)$ es entorno de $p$ en $X$. Veamos ahora que es homeomorfo a un abierto de $\R^n$. Sea
\begin{align*}
\Phi : & V_1\sqcup V_2  \to  V\times (1-\varepsilon , 1+ \varepsilon ), \\
& q_1\in V_1  \mapsto  (x,r), r=\norm{\varphi_1(q_1)}, x=\varphi_1(q_1)/r,\\
& q_2\in V_2  \mapsto  (x,2-r), r= \norm{\varphi_2(q_2)}, x= \varphi_2(q_2)/r.
\end{align*}
Por tanto, $\Phi : V_1 \to V \times [1, 1+\varepsilon)$ y $\Phi : V_2 \to V \times (1- \varepsilon, 1]$ son homeomorfismos. Además, $q_1 \sim q_2$ si y sólo si $\Phi(q_1)=\Phi(q_2)$. De este modo, $\Phi$ induce una aplicación continua y biyectiva $$\overline{\Phi} : \tilde{V} \to V \times (1- \varepsilon, 1+ \varepsilon)$$

$\overline{\Phi}$ es abierta: si tomamos un abierto básico saturado de $V_1\sqcup V_2$, o bien está totalmente incluido en $V_1-S_1$ o en $V_2-S_2$, en cuyo caso su imagen es un abierto de $V\times (1-\varepsilon ,1)$ o $V\times (1,1+\varepsilon )$, o bien interseca a $S_1$ y $S_2$. En ese caso se puede asumir que es un abierto de la forma $W_1\sqcup W_2$, construido como antes y donde hemos partido de un $W\subset V \subset \partial B_1(0)$. Entonces $\overline{\Phi} (\tilde{W})= W \times (1-\delta , 1+\delta )$ con $0<\delta \leq \varepsilon$, $\tilde{W} = \pi (W_1 \sqcup W_2)$. Luego $\overline{\Phi}$ es un homeomorfismo, y $\tilde{V}$ es homeomorfo a un abierto de $\R^n$.

Los abiertos construidos, $\tilde{V} \subset X$, se pueden tomar en cantidad numerable para formar una base de la topología, con lo cual $X$ es IIAN. También, dado un $q\in U_j'$, $j=1,2$, y un $p\in S$, se puede tomar un abierto $\tilde{V}$ entorno de $p$ disjunto de un entorno pequeño de $q$. Y si tomamos $p, p' \in S$ distintos, los abiertos $\tilde{V}$, $\tilde{V}'$ construidos partiendo de $V$, $V'\subset \partial B_1(0)$ disjuntos, serán disjuntos. Luego $X$ es Hausdorff.
\end{proof}




\begin{obs}
Sea $S$ una superficie. Entonces la suma conexa $S\# \Esfera$ es homeomorfa a $S$ (\autoref{fig:suma_toro_esfera}).\label{obs:suma_esfera}
\end{obs}




\begin{obs}%%%%EJE: Asas y toros
Dada una superficie $S$, la suma conexa $S\# \Toro$ puede visualizarse como el espacio que se obtiene al ``pegarle'' un \textit{asa} a $S$. De forma más precisa, sea $S_0$ que denota a $S$ con dos perforaciones, es decir la superficie que queda al retirar dos discos cerrados disjuntos de $M$ (se dará una construcción más precisa en \autoref{def:perforacion}). Entonces $S_0$ y $\mathbb{S}^1\times I$ son ambas superficies con borde, y sus bordes son ambos homeomorfos a la union disjunta de dos circunferencias. Sea $\tilde{S}$ el espacio adjunción (\autoref{app:adjuncion}) obtenido pegando $S_0$ y $\mathbb{S}^1\times I$ por sus bordes. Este espacio cociente es homeomorfo a $S\# \Toro$. La razón se puede ver en la \autoref{fig:toro_asa}, y es que podemos obtener un espacio homeomorfo a $S\# \Toro$ primero quitando un disco abierto de $S$, luego pegando un disco cerrado con dos discos abiertos quitados (es decir, la parte gris de la figura), y finalmente pegando a la frontera de la construcción el cilindro $\Esfera \times I$. Dado que la primera operación da resultado a un espacio homeomorfo a $S$ con dos discos abiertos quitados, el resultado es el mismo que si quitamos directamente a $S$ dos discos abiertos y entonces pegamos el cilindro a su frontera. \label{obs:toro_asa}

%%165%%
\end{obs}




\begin{prop}
Sean $M_1$ y $M_2$ superficies que tienen (---admiten mejor?---) respectivamente representaciones $\langle S_1\mid W_1\rangle $ y $\langle S_2\mid W_2\rangle $, donde $S_1$ y $S_2$ son conjuntos disjuntos y tal que cada presentación tiene una sola cara. Entonces $\langle S_1,S_2\mid W_1W_2\rangle$ es una presentación de la suma conexa $M_1 \# M_2$.
\end{prop}


%%%%%%%%%%%%%%%%%%%%%%%%%%%%%%%%%%%%%%%%%%%%%%%%%%%%%%%%%%%%%%%%%%%%%%%%%%%%%%%%%%%%%%%%%%%%%%%%%%%%%%%%%%%%%%%%%%%%%%%%%%%%%%%%%%%%%%%%%%%%%%%%%%%%%%%%%%%%%%%%%%%%%%%%%%%%%%%%%%%%%%%%%%%%%%%%%%%%%%%%%%%%%%%%%%%%%%%%%%%%%%%%%%%%%%%%%%%%%%%%%%%%%%%%%%%%%%%%%%%%%%%%%%%%%%%%%%%%%%%%%%%%%%%%%%%%%%%%%%%%%%%%%%%%%%%%%%%%%%%%%%%%%%%%%%%%%%%%%%%%%%%%%%%%%%%%%%%%%%%%%%%%%%%%%%%%%%%%%%%%%%%%%%%%%%%%%%%%%%%%%%%%%%%%%%%%%%%%%%%%%%%%%%%%%%%%%%%%%%%%%%%%%%%%%%%%%%%%%%%%%%%%%%%%%%%%%%%%%%%%%%%%%%%%%%%%%%%%%%%%%%%%%%%%%%%%%%%%%%%%%%%%%%%%%%%%%%%%%%%%%%%%%%%%%%%%%%%%%%%%%%%%%%%%%%%%%%%%%%%%%%%%%%%%%%%%%%%%%%%%%%%%%%%%%%%%%%%%%%%%%%%%%%%%%%%%%%%%%%%%%%%%%%%%%%%%%%%%%%%%%%%%%%%%%%%%%%%%%%%%%%%%%%%%%%%%%%%%%%%%%%%%%%%%%%%%%%%%%%%%%%%%%%%%%%%%%%%%%%%%%%%%%%%%%%%%%%%%%%%%%%%%%%%%%%%%%%%%%%%%%%%%%%%%%%%%%%%%%%%%%%%%%%%%%%%%%%%%%%%%%%%%%%%%%%%%%%%%%%%%%%%%%%%%%%%%%%%%%%%%%%%%%%%%%%%%%%%%%%%%%%%%%%%%%%%%%%%%%%%%%%%%%%%%%%%%%%%%%%%%%%%%%%%%%%%%%%%%%%%%%%%%%%%%%%%%%%%%%%%%%%%%%%%%%%%%%%%%%%%%%%%%%%%%%%%%%%%%%%%%%%%%%%%%%%%%%%%%%%%%%%%%%%%%%%%%%%%%%%%%%%%%%%%%%%%%%%%%%%%%%%%%%%%%%%%%%%%%%%%%%%%%%%%%%%%%%%%%%%%%%%%%%%%%%%%%%%%%%%%%%%%%%%%%%%%%%%%%%%%%%%%%%%%%%%%%%%%%%%%%%%%%%%%%%%%%%%%%%%%%%%%%%%%%%%%%%%%%%%%%%%%%%%%%%%%%%%%%%%%%%%%%%%%%%%%%%%%%%%%%%%%%%%%%%%%%%%%%%%%%%%%%%%%%%%%%%%%%%%%%%%%%%%%%%%%%%%%%%%%%%%%%%%%%%%%%%%%%%%%%%%%%%%%%%%%%%%%%%%%%%%%%%%%%%%%%%%%%%%%%%%%%%%%%%%%%%%%%%%%%%%%%%%%%%%%%%%%%%%%%%%%%%%%%%%%%%%%%%%%%%%%%%%%%%%%%%%%%%%%%%%%%%%%%%%%%%%%%%%%%%%%%%%%%%%%%%%%%%%%%%%%%%%%%%%%%%%%%%%%%%%%%%%%%%%%%%%%%%%%%%%%%%%%%%%%%%%%%%%%%%%%%%%%%


\chapter{Triangulación de superficies}

Un hecho fundamental para la prueba del teorema de clasificación es que toda superficie es triangulable. La demostración, atribuída a Radó en 1925 \cite{rado}, utiliza el \emph{teorema de Schönflies}, cuya prueba es larga y técnica. Utilizaremos el truco de Kirby para superficies dado por Hatcher \cite{hatcher_torus}.

\

\section{Complejos simpliciales y triangulación}

Para poder dar una definición rigurosa de triangulación de variedades necesitamos la noción de \textit{complejos simpliciales}. Estos son construcciones formadas por \textit{símplices}, que son una generalización de los triángulos. En esta primera parte me baso en las definiciones de Munkres \cite{munkres}.

\begin{defin}%%%%DEF: Posición general
Sean $v_0,\dots v_k$ $k+1$ puntos distintos de $\R^n$. Decimos que $\{ v_0,\dots ,v_k\}$ están en \enfatiza{posición general} si $c_0,\dots c_k$ son números reales tales que  $$\sum_{i=0}^{k}c_iv_i=0 \text{ y } \sum_{i=0}^kc_i=0,$$ entonces $c_0=\dots =c_k=0$.
\end{defin}


\begin{defin}%%%%DEF: Símplice
Sean $\{ v_0,\dots ,v_n\}$ un conjunto de $k+1$ puntos de $\R^n$ en posición general. El \enfatiza{símplice} generado por ellos, que denotamos por $[ \, v_0,\dots ,v_k ] \,$, es el conjunto $$[ \, v_0,\dots ,v_k] \, =\left\{  \sum_{i=0}^{k}t_iv_i \mid t_i\geq 0,\, \sum_{i=0}^{k}t_i=1 \right\}, $$ con la topología heredada. Para todo punto $x=\sum_it_iv_i\in [ \, v_0,\dots ,v_k] \,$, llamamos a los $t_i$ \enfatiza{coordenadas baricéntricas de $x$}. Cada uno de los $v_i$ se llama \enfatiza{vértice} del símplice. Al entero $k$ se le llama \enfatiza{dimensión}, y diremos que $[ \, v_0,\dots ,v_k] \,$ es un \enfatiza{$k$-símplice}. 
\end{defin}

\begin{eje}
Un $0$-símplice es un punto, un 1-símplice es un segmento, un $2$-símplice es un triángulo junto a su interior, un $3$-símplice es un tetraedro sólido, y así sucesivamente (\autoref{fig:simplices}).
\end{eje}





Sea $\sigma$ un $k$-símplice. Cada símplice generado por un subconjunto no vacío de vértices de $\sigma$ se llama \enfatiza{cara de $\sigma$}. Las caras que no son iguales a $\sigma$ se llaman \enfatiza{caras propias}. Las caras $0$-dimensionales de $\sigma$ son sus vértices, y a las caras $1$-dimensionales se les llama \enfatiza{lados}. Las caras $(k-1)$-dimensionales de un $k$-símplice se llaman bordes, y a su unión la llamamos \enfatiza{frontera}. Definimos el \enfatiza{interior} como $\sigma$ menos su frontera. 


\begin{defin}%%%%DEF: Complejo simplicial
Un \enfatiza{complejo simplicial} es una colección $K$ de símplices en algún espacio euclídeo $\R ^n$, que satisface las siguientes condiciones:
\begin{itemize}
\item[(i)] Si $\sigma \in K $, entonces toda cara de $\sigma$ está en $K$
\item[(ii)] La intersección de dos símplices cualesquiera en $K$ es o bien vacía o bien una cara de ambos.
%\item[(iii)] $K$ es una colección finitamente local.
\end{itemize}
\label{def:complex}
\end{defin}
%La tercera condición implica que $K$ es numerable, pues todo punto de $\R ^n$ tiene un entorno intersecando al menos un número finito de símplices de $K$, y este recubrimiento abierto de $\R ^n$ tiene un subrecubrimiento numerable. A nosotros los símplices que más nos interesan son los \enfatiza{complejos simpliciales finitos}, que son los que contienen únicamente un número finito de símplices. Para estos complejos, la condición (iii) es redundante.

Si $K$ un complejo simplicial en $\R ^n$, llamamos \enfatiza{dimensión de $K$} a la dimensión máxima de los símplices en $K$. Esta no es mayor que $n$.
Un subconjunto $K'\subseteq K$ se dice que es un \enfatiza{subcomlejo de $K$} si para todo $\sigma \in K'$, toda cara de $\sigma$ está en $K'$. Un subcomplejo es un complejo simplicial en sí.
Para todo $k\leq n$, el conjunto de todos los símplices de $K$ de dimensión menor o igual que $k$ es un subcomplejo llamado \enfatiza{$k$-esqueleto de $K$}.

La \autoref{fig:complex} muestra un complejo simplicial en $\R^2$. En cambio en la \autoref{fig:not_complex} los símplices representados no forman un complejo, pues no se respeta la condición (ii) de la \autoref{def:complex}.

%%
%%
%%

%Aquí se puede coger la definición de polítopo de munkres, pero quizas es un poco inutil
%%
%%
%%


\begin{defin}%%%%DEF: Poliedro
Sea un complejo simplicial $K$ en $\R ^n$. La unión de todos los símplices en $K$ junto con la topología heredada de $\R ^n$ es un espacio topológico que denotamos por $|K|$ y que llamamos \enfatiza{poliedro de K}.
\end{defin}





\begin{defin}%%%%DEF: Triangulación
Sea $X$ un espacio topológico. Llamamos \enfatiza{triangulación de $X$} a un homeomorfismo entre $X$ y el poliedro de algún complejo simplicial.
\end{defin}

\begin{defin}%%%%DEF: Superficie triangulable
Toda superficie que admita una triangulación se dice \enfatiza{triangulable}.
\end{defin}



\section{Teorema de Radó}

\begin{tma}[Teorema de Radó]
Toda superficie es triangulable por un poliedro de un complejo simplicial 2-dimensional, en donde cada $1$-símplice es una cara de exáctamente dos $2$-símplices.\label{teo:rado}
\end{tma}



\chapter{La prueba ZIP de Conway}

\section{Cremalleras}

\begin{defin}%M%%%DEF: perforación
\label{def:perforacion}
Sea $S$ una superficie (con o sin borde). Sea $p\in \interior (S)$ y $U$ un entorno abierto de $p$ en $S$. Sea $\phi :U\to \R^n$ un homeomorfismo tal que $\phi (p)=0$. Sea $B=\phi ^{-1}(B_1(0))$. Decimos que la nueva superficie con borde $S^o=S\setminus B$ es $S$ \enfatiza{$1$-perforada}, y a $\partial B\subset S^o$ la llamamos \enfatiza{perforación}. Podemos repetir el proceso sobre $S^o$ sucesivamente, obteniendo $S$ $n$-perforada con un número finito $n\in \N$ de perforaciones.
\end{defin}

\begin{obs}%M%%%OBS: Disco como esfera perforada
Dado que los dos espacios son homeomorfos, podemos visualizar el disco cerrado $\overline{\mathbb{B}}^2$ como una esfera con una perforación (\autoref{fig:disco_esfera_perforada}).
\end{obs}
 




Conway utiliza las cremalleras (\textit{zips} en inglés) para describir cómo actúan las identificaciones topológicas. Cada cremallera actúa sobre una o dos perforaciones de una superficie. Están formadas por dos \textit{zips} (dos partes dentadas) fijadas la/s perforación/es y un \textit{zipper} (el deslizador). Al cerrar el \textit{zipper}, las \textit{zips} se juntan identificándose. Trato de dar una definición rigurosa:


\begin{defin}%%%DEF: ZIP
Sea $S$ una superficie compacta. Una \enfatiza{cremallera} es una identificación entre dos subconjuntos (abtos, cerrados??) de la frontera de $S$. A este par lo llamamos \enfatiza{par-zip}.
\end{defin}


En la \textit{prueba ZIP}, Conway nos explica gráficamente las posibles formas de unir cremalleras. 

\begin{defin}
Sea $S$ una superficie. Definimos cuatro formas elementales de identificar pares-\textit{zip} en perforaciones de 
$S$ $1$ o $2$-perforada:

\begin{itemize}
\item[1.] \enfatiza{Cap}: Los pares zip yacen cada uno sobre la mitad de una misma perforación con orientaciones opuestas (\autoref{fig:cap}).
\item[2.] \enfatiza{Crosscap}: Los pares zip yacen cada uno sobre la mitad de una misma perforación con la misma orientación (\autoref{fig:crosscap}).
\item[3.] \enfatiza{Handle}: Los pares zip yacen cada uno sobre una perforación distinta de S con orientaciones opuestas (\autoref{fig:handle}).
\item[4.] \enfatiza{Crosshandle}: Los pares zip yacen cada uno sobre una perforación distinta de S con la misma orientación (\autoref{fig:crosshandle}).
\end{itemize}
\end{defin}

Sea $S$ una superficie conexa tal que admite una representación poligonal de una sola cara $P=\langle A\mid W\rangle$, y sea $|\mathcal{P}|$ su realización geométrica. Sea $\partial B$ una perforación sobre $\Esfera$, y sea $\phi$ un homeomorfismo entre los lados de $|\mathcal{P}|$ y $\partial B$. Si identificamos ahora los pares de segmentos sobre $\partial B$ de la imagen de $\phi$, obtenemos la suma conexa $S\# \Esfera$, es decir, $S$ (\autoref{obs:suma_esfera}). 

Sean ahora $S$ y $S'$ superficies conexas. Entonces, hacer una perforación sobre $S'$ es lo mismo que hacer la suma conexa de una esfera $\Esfera$ con una perforación $S'$. Por tanto, hacer una perforación sobre $S'$ asociada a $S$ da lugar a la suma conexa $S\# S'$.
 

\begin{prop}
Sea $S$ una superficie. Los siguientes espacios son homeomorfos:
\begin{itemize}
\item[a)] $S$ con un cap y $S$.
\item[b)] $S$ con un crosscap y $S\# \Proyectivo$.
\item[c)] $S$ con un handle y $S\# \Toro$.
\item[d)] $S$ con un crosshandle y $S\# K$ (siendo $K$ la botella de Klein). 
\end{itemize}
\end{prop}
\begin{proof}
a) y b) son consecuencia directa de lo anterior. Para c), utilizamos la \autoref{obs:toro_asa}, y para d) utilizar una construcción parecida a c) ((habría que especificar más?)).
\end{proof}
















\section{Teorema de Clasificación}

\begin{defin}%%%%DEF: Ordinaria
Una superficie se dice ordinaria si es homeomorfa a una colección finita de esferas cada una con un número finito de \textit{handles}, \textit{crosshandles}, \textit{crosscaps} y perforaciones.
\end{defin}


\begin{lema}%%%%LEMA: Superficie ordinaria zips
Sea $S$ una superficie con borde con un par-zip tal que cada cremallera está en una parte de su borde. Entonces, si $S$ es ordinaria antes de identificar las cremalleras, es ordinaria también después.\label{lema:superficie_ordinaria}
\end{lema}
\begin{proof}
Consideramos el caso en que las dos cremalleras ocupan cada una una perforación. Entonces al identificarlas se tiene un \textit{handle} (\autoref{fig:handle}) o un \textit{crosshandle} (\autoref{fig:crosshandle}), dependiendo de sus respectivas orientaciones. Si las dos perforaciones pertenecen a componentes conexas distintas de $S$, entonces identificando obtenemos el espacio adjunción de las dos componentes. (MEJORAR). 

Consideramos ahora el caso en el que las dos cremalleras yacen sobre la misma perforación y la cubren totalmente. Identificándolas nos da o bien un \textit{cap} (\autoref{fig:cap}) o bien un \textit{crosscap} (\autoref{fig:crosscap}), dependiendo de sus respectivas orientaciones.

Finalmente, consideramos los varios casos en que las cremalleras no ocupan perforaciones en su totalidad. (A PARTIR DE AQUI NO SE MUY BIEN COMO ORIENTARLO... CON OPERACIONES ELEMENTALES O COMO HACE EL???)
\end{proof}

\begin{tma}[Teorema de clasificación, versión preeliminar] %M%%%TEO: Clasificación preeliminar

Toda superficie compacta es ordinaria.

\end{tma}

\begin{proof}
Sea $S$ una superficie compacta. Sabemos, por el \nameref{teo:rado}, que $S$ está triangulada por un poliedro $|K|$ asociado a un complejo simplicial $K$ tal que cada 1-símplice que contiene puntos interiores de $S$ es una cara de exáctamente dos 2-símplices, y cada 1-símplice que contiene puntos del borde de $S$ es cara de exáctamente un 2-símplice. Si sobre los primeros 1-símplices ponemos una cremallera distinta, en los 2-símplices habrá algunos 1-símplices que se identifiquen. Llamemos $K_2=\left\{\sigma_1,\dots ,\sigma_j\right\}$, donde cada $\sigma_i \text{ es un 2-símplice para todo } i=1\dots ,j$. $K_2$ es una superficie ordinaria, pues cada $\sigma_i$ es homeomorfo a una esfera perforada. Si identificamos ahora las cremalleras una a una, por el \autoref{lema:superficie_ordinaria} y por inducción, la superficie resultante es ordinaria.
\end{proof}



%\begin{prop}
%Si $S$ es una superficie compacta y conexa, $S$ admite una representación de una sola cara.
%\end{prop}


%A la perforación $\partial B_i$ la llamamos \enfatiza{perforación} asociada a $W_i$, y a la imagen por $\phi _i$ de cada par de segmentos $a$ la llamamos \enfatiza{cremallera} asociada a $a\in W_i$.


%\begin{defin}%M%%%DEF: zips asociadas a una representación poligonal de una superficie
%Sea $S$ una superficie. Sea $|\mathcal{P}|$ la realización geométrica de una representación poligonal $P=\langle A\mid W_1,\dots W_n\rangle$ de la superficie. Para cada $W_i$, con $i=1,\dots n$, sea $\partial B_i$ una perforación sobre una esfera $\Esfera_i$, y sea $\phi _i$ un homeomorfismo entre los lados de $|\mathcal{P}|$ y $\partial B_i$. A la perforación $\partial B_i$ la llamamos \enfatiza{perforación} asociada a $W_i$, y a la imagen de $\phi _i$ la llamamos \enfatiza{cremallera} asociada a $W_i$.
%\end{defin}









\appendix
\chapter{Teoremas Usados}

\begin{tma}[Lema de la aplicación cerrada]
\label{teo:aplicac_cerrada}
Sea $F$ una aplicación continua de un espacio topológico compacto en un espacio topológico Hausdorff. Entonces:
\begin{itemize}
\item[(a)]$F$ es una aplicación cerrada.
\item[(b)] Si $F$ es sobreyectiva, entonces es una aplicación cociente.
\item[(c)] Si $F$ es inyectiva, entonces es una inmersión topológica. %?????
\item[(d)] Si $F$ es biyectiva, entonces es un homeomorfismo.
\end{itemize}
\end{tma}

\begin{tma}[Unicidad de espacios cociente]
\label{teo:unicidad_espacio_cociente}
Supongamos $q_1:X\to Y_1$ y $q_2:X\to Y_2$ son aplicaciones cociente que hacen las mismas identificaciones, es decir, tales que $q_1(x)=q_1(x')$ si y solo si $q_2(x)=q_2(x')$. Entonces existe un único homeomorfismo $\phi:Y_1\to Y_2$ tal que $\phi \circ q_1=q_2$.
\end{tma}

\chapter{CW-complejos}

\chapter{Definiciones}

\begin{defin}%%%%DEF: Finitamente local
Una colección de subconjuntos de un espacio topológico $X$ se dice \enfatiza{localmente finita} si cada punto del espacio tiene un entorno que interseca sólo un número finito de subconjuntos de la colección.
\end{defin}
\begin{thebibliography}{9}

\bibitem{lee1}
J. M. Lee.
\textit{Introduction to Topological Manifolds}. Graduate text in mathematics, Springer - Verlag New York, 2011.

\bibitem{juanjo}
V. Muñoz, J. J. Madrigal. 
\textit{Topología Algebráica}. Sanz y Torres, 2015.


\bibitem{hatcher_torus}
A. Hatcher.
\textit{The Kirby Torus Trick for Surfaces}, 2013.
\\\url{http://front.math.ucdavis.edu/1312.3518}

\bibitem{rado}

T. Radó.
\textit{Über den Begriff der Riemannschen Fläche, Acta Sci}. Math. Szeged. 2 1925,
101–121.

\bibitem{munkres}
J. R. Munkres.
\textit{Elements of Agebraic Topology}. Addison-Wesley, 1984.


\end{thebibliography}

\end{document}


