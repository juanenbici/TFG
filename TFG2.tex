\documentclass[10pt]{report}
\usepackage[utf8]{inputenc}

\usepackage[spanish, es-lcroman]{babel} % Idioma
\selectlanguage{spanish}
\usepackage{graphicx} %imagenes
\usepackage{verbatim} % Para comment


\usepackage{appendix} %Para el apéndice


\usepackage{ragged2e} %para alinear texto izqda y derecha



\usepackage[a4paper, total={6in, 8in}]{geometry} %documento estructurado
\usepackage{microtype}%pone bonito el doc a nivel de formato

\setlength{\parindent}{0pt} % Quitar indentado parrafos automatico

\usepackage{amsmath} % Movidas útiles
\usepackage{amssymb} % Simbolos mates
\usepackage{amsthm} % Personalizar teoremas (mas abajo continuacion)
\usepackage{thmtools} % Mas movidas teoremas

\usepackage{tikz} % Para pictures
\usetikzlibrary{decorations.markings,arrows} %decoracion en tikz
\usetikzlibrary{shadows,arrows,positioning,shapes.geometric}
\usetikzlibrary{decorations,decorations.markings}
\usepackage{pgfplots}
\usetikzlibrary{intersections, pgfplots.fillbetween}
\usetikzlibrary {arrows.meta}

\usepackage{etoolbox}
\usetikzlibrary{optics}
\usepackage{pgf,tikz}
\usetikzlibrary{arrows}
\usetikzlibrary{babel}

\usepackage[hidelinks]{hyperref}
\usepackage[all]{hypcap}

\usepackage{tocbibind}%para poner la bibliografía en el toc, poner entre corchetes nottoc o notlof para quitar toc y lof


\usepackage{sectsty} % Personalizar titulos secciones
%\sectionfont{\underline} % Titulo seccion subrayado

% Comandos simbolos utiles 
\newcommand{\C}{\mathbb{C}}
\newcommand{\R}{\mathbb{R}}
\newcommand{\Q}{\mathbb{Q}}
\newcommand{\Z}{\mathbb{Z}}
\newcommand{\N}{\mathbb{N}}
\newcommand{\Epsilon}{\mathcal{E}}
\DeclareMathOperator{\interior}{Int} %interior


\newcommand{\norm}[1]{\left\lVert#1\right\rVert} % Comando para normas
\newcommand{\Esfera}{\mathbb{S}^2}
\newcommand{\Toro}{\mathbb{T}^2}
\newcommand{\Proyectivo}{\mathbb{P}^2}
\newcommand{\enfatiza}[1]{\textbf{\textit{#1}}}

% Estilos teoremas (Mejorable)
\theoremstyle{definition}
\newtheorem{defin}{Definición}[section]
\newtheorem{tma}[defin]{Teorema}
\newtheorem*{tma*}{Teorema}
\newtheorem{corol}[defin]{Corolario}
\newtheorem{prop}[defin]{Proposición}
\newtheorem{lema}[defin]{Lema}
%\newcommand{\demo}{Demostración.\\}
%\newcommand{\ok}{\hfill$\square$}
%\theoremstyle{remark}
\newtheorem{obs}[defin]{Observación}
\newtheorem{eje}[defin]{Ejemplo}
\graphicspath{ {images/} }
\usepackage{afterpage}

\newcommand\blankpage{%
    \null
    \thispagestyle{empty}%
    \newpage}

\usepackage{subcaption} %para hacer subfigures
% Cabeceras
\renewcommand{\title}{ZIP}
\newcommand{\subtitle}{Trabajo fin de grado}
\renewcommand{\maketitle}{{\Large{\textbf{\title}}}\\\\{\Large \subtitle}\\\rule{17cm}{0.4pt}\\}


\pagenumbering{roman}
\setcounter{page}{1}


\begin{document}

%%%%%%%%%%%%%%%%%%%%%%%%%%%%%%%%%%%%%%%%%%%%%%%%%%%%%%%%%%%%%%%%%%%%%%%%%%%%%%%%%%%%%%%%%%%%%%%%%%%%%%%%%%%%%%%%%%%%%%%%%%%%%%%%%%%%%%%%%%%%%%%%%%%%%%%%%%%%%%%%%%%%%%%%%%%%%%%%%%%%%%%%%%%%%%%%%%%%%%%%%%%%%%%%%%%%%%%%%%%%%%%%%%%%%%%%%%%%%%%%%%%%%%%%%%%%%%%%%%%%%%%%%%%%%%%%%%%%%%%%%%%%%%%%%%%%%%%%%%%%%%%%%%%%%%%%%%%%%%%%%%%%%%%%%%%%%%%%%%%%%%%%%%%%%%%%%%%%%%%%%%%%%%%%%%%%%%%%%%%%%%%%%%%%%%%%%%%%%%%%%%%%%%%%%%%%%%%%%%%%%%%%%%%%%%%%%%%%%%%%%%%%%%%%%%%%%%%%%%%%%%%%%%%%%%%%%%%%%%%%%%%%%%%%%%%%%%%%%%%%%%%%%%%%%%%%%%%%%%%%%%%%%%%%%%%%%%%%%%%%%%%%%%%%%%%%%%%%%%%%%%%%%%%%%%%%%%%%%%%%%%%%%%%%%%%%%%%%%%%%%%%%%%%%%%%%%%%%%%%%%%%%%%%%%%%%%%%%%%%%%%%%%%%%%%%%%%%%%%%%%%%%%%%%%%%%%%%%%%%%%%%%%%%%%%%%%%%%%%%%%%%%%%%%%%%%%%%%%%%%%%%%%%%%%%%%%%%%%%%%%%%%%%%%%%%%%%%%%%%%%%%%%%%%%%%%%%%%%%%%%%%%%%%%%%%%%%%%%%%%%%%%%%%%%%%%%%%%%%%%%%%%%%%%%%%%%%%%%%%%%%%%%%%%%%%%%%%%%%%%%%%%%%%%%%%%%%%%%%%%%%%%%%%%%%%%%%%%%%%%%%%%%%%%%%%%%%%%%%%%%%%%%%%%%%%%%%%%%%%%%%%%%%%%%%%%%%%%%%%%%%%%%%%%%%%%%%%%%%%%%%%%%%%%%%%%%%%%%%%%%%%%%%%%%%%%%%%%%%%%%%%%%%%%%%%%%%%%%%%%%%%%%%%%%%%%%%%%%%%%%%%%%%%%%%%%%%%%%%%%%%%%%%%%%%%%%%%%%%%%%%%%%%%%%%%%%%%%%%%%%%%%%%%%%%%%%%%%%%%%%%%%%%%%%%%%%%%%%%%%%%%%%%%%%%%%%%%%%%%%%%%%%%%%%%%%%%%%%%%%%%%%%%%%%%%%%%%%%%%%%%%%%%%%%%%%%%%%%%%%%%%%%%%%%%%%%%%%%%%%%%%%%%%%%%%%%%%%%%%%%%%%%%%%%%%%%%%%%%%%%%%%%%%%%%%%%%%%%%%%%%%%%%%%%%%%%%%%%%%%%%%%%%%%%%%%%%%%%%%%%%%%%%%%%%%%%%%%
%%%%%%%%PORTADA%%%%%%%%%%%

%%% PORTADA%%%%%%
\begin{titlepage} %Creo que esto es para la numeración de páginas
\begin{center} %Que todo quede centradito

% Todo esto de abajo habría que retocarlo pero así sirve de ejemplo
\huge\textsc{Universidad Complutense de Madrid}\\[0.2in]
\includegraphics[scale=0.8]{comlu}\\[0.1in] %Introduce la imagen y la reescala, inserta un pequeño hueco con lo de debajo

\Large{Facultad de Matemáticas}\\[0.5in] %inserta un hueco mayor con lo de abajo
 %la linea horizontal
\Large{Trabajo de Fin de Grado}\\[.1in]
\Huge {Un tratamiento riguroso de la prueba ZIP}\\[0.2in]



\vfill %Llenar verticalmente
\Large {Juan Valero Oliet}\\[0.5in]
\vfill 
Dirigido por:\\
Manuel Alonso Morón\\[.1in]
\Large{Junio de 2020}
\end{center}

\end{titlepage}


\afterpage{\blankpage}

%%%%%%%%%%%%%PORTADA%%%%%%%%%%%%%%%%%%%%%
%%%%%%%%%%%%%%%%%%%%%%%%%%%%%%%%%%%%%%%%%%%%%%%%%%%%%%%%%%%%%%%%%%%%%%%%%%%%%%%%%%%%%%%%%%%%%%%%%%%%%%%%%%%%%%%%%%%%%%%%%%%%%%%%%%%%%%%%%%%%%%%%%%%%%%%%%%%%%%%%%%%%%%%%%%%%%%%%%%%%%%%%%%%%%%%%%%%%%%%%%%%%%%%%%%%%%%%%%%%%%%%%%%%%%%%%%%%%%%%%%%%%%%%%%%%%%%%%%%%%%%%%%%%%%%%%%%%%%%%%%%%%%%%%%%%%%%%%%%%%%%%%%%%%%%%%%%%%%%%%%%%%%%%%%%%%%%%%%%%%%%%%%%%%%%%%%%%%%%%%%%%%%%%%%%%%%%%%%%%%%%%%%%%%%%%%%%%%%%%%%%%%%%%%%%%%%%%%%%%%%%%%%%%%%%%%%%%%%%%%%%%%%%%%%%%%%%%%%%%%%%%%%%%%%%%%%%%%%%%%%%%%%%%%%%%%%%%%%%%%%%%%%%%%%%%%%%%%%%%%%%%%%%%%%%%%%%%%%%%%%%%%%%%%%%%%%%%%%%%%%%%%%%%%%%%%%%%%%%%%%%%%%%%%%%%%%%%%%%%%%%%%%%%%%%%%%%%%%%%%%%%%%%%%%%%%%%%%%%%%%%%%%%%%%%%%%%%%%%%%%%%%%%%%%%%%%%%%%%%%%%%%%%%%%%%%%%%%%%%%%%%%%%%%%%%%%%%%%%%%%%%%%%%%%%%%%%%%%%%%%%%%%%%%%%%%%%%%%%%%%%%%%%%%%%%%%%%%%%%%%%%%%%%%%%%%%%%%%%%%%%%%%%%%%%%%%%%%%%%%%%%%%%%%%%%%%%%%%%%%%%%%%%%%%%%%%%%%%%%%%%%%%%%%%%%%%%%%%%%%%%%%%%%%%%%%%%%%%%%%%%%%%%%%%%%%%%%%%%%%%%%%%%%%%%%%%%%%%%%%%%%%%%%%%%%%%%%%%%%%%%%%%%%%%%%%%%%%%%%%%%%%%%%%%%%%%%%%%%%%%%%%%%%%%%%%%%%%%%%%%%%%%%%%%%%%%%%%%%%%%%%%%%%%%%%%%%%%%%%%%%%%%%%%%%%%%%%%%%%%%%%%%%%%%%%%%%%%%%%%%%%%%%%%%%%%%%%%%%%%%%%%%%%%%%%%%%%%%%%%%%%%%%%%%%%%%%%%%%%%%%%%%%%%%%%%%%%%%%%%%%%%%%%%%%%%%%%%%%%%%%%%%%%%%%%%%%%%%%%%%%%%%%%%%%%%%%%%%%%%%%%%%%%%%%%%%%%%%%%%%%%%%%%%%%%%%%%%%%%%%%%%%%%%%%%%%%%%%%%%%%%%%%%%%%%%%%%%%%%%%%%%%%%%%%%%%%%%%%%%%%%%%%%%%%%%%%%%%%%%%%%%%%%%%%%%%%%%%%%%%%%%%%%%%%%%%%%%%%%%%%%%%%%%%%%%%%%%%%%%%%%%%%%%%%%%%%%%%%%%%%%%%%%%%%%%%%%%%%%%%%%%%%%%%%%%%%%%%%%%%%%%%%%%%%%%%%%%%%%%%%%%%%%%%%%%%%%%%%%%%%%%%%%%%%%%%%%%%%%%%%%%%%%%%%%%%%%%%%%%%%%%%%%%%%%%%%%%%%%%%%%%%%%%%%%%%%%%%%%%%%%%%%%%%%%%%%%%%%%%%%%%%%%%%%%%%%%%%%%%%%%%%%%%%%%%%%%%%%%%%%%%%%%%%%%%%%%%%%%%%%%%%%%%%%%%%%%%%%%%%%%%%%%%%%%%%%%%%%%%
\begin{comment}
\section*{Resumen}\addcontentsline{toc}{chapter}{\numberline{}Resumen}
\clearpage
\section*{Abstract}\addcontentsline{toc}{chapter}{\numberline{}Abstract}
\clearpage
\section*{Introducción}\addcontentsline{toc}{chapter}{\numberline{}Introducción}
\clearpage
\section*{Notación}\addcontentsline{toc}{chapter}{\numberline{}Notación}
$I$ representará el intervalo $\left[ 0,1\right]$.
\clearpage
\end{comment}
\tableofcontents
\clearpage
\listoffigures





\chapter{Variedades y superficies}
\pagenumbering{arabic} 
\setcounter{page}{1}
En este capítulo doy las definiciones y resultados básicos sobre variedades y superficies. Me basaré principalmente en los libros de J. M. Lee \cite{lee1} y V. Muñoz - J. J. Madrigal \cite{juanjo}.\\
\section{Variedades}\label{sec:variedades}
Los espacios topológicos de los que nos vamos a ocupar en el siguiente trabajo son las variedades, y en concreto las superficies. Definámoslas.
\begin{defin}\label{def:variedad}%%%%DEF: variedad topológica
Una \textbf{\emph{variedad topológica}} (de ahora en adelante \emph{variedad}) es un espacio topológico Hausdorff, II AN y localmente homeomorfo a $R^n$, para algún $n\geq 0$.
\end{defin}
Sea $M$ una variedad, y sea $p\in M$. Si $U\subseteq M$ es un entorno de $p$ que es homeomorfo a un abierto $U'$ de $\R^n$, decimos que $U$ es un \enfatiza{dominio coordenado}, llamamos \enfatiza{aplicación coordenada} a cualquier homeomorfismo $\varphi :U\to U'$ y decimos que el par $(U,\varphi )$ es una \enfatiza{carta para $M$}.\\
A un dominio coordenado que es homeomorfo a una bola de $\R^n$ se le llama \enfatiza{bola coordenada} (si $n=2$, decimos \enfatiza{disco coordenado}). Si $p\in M$ y $U$ es un dominio coordenado que contiene a $p$, decimos que $U$ es un \enfatiza{entorno Euclídeo} de $p$.  
\begin{obs}%%%%OBS: toda propiedad local de Rn se pasa a las variedades
Como la propiedad ``ser localmente homeomorfo a $\R^n$'' es local, toda propiedad local de $\R^n$ se traslada a una variedad. Así, las variedades son localmente compactas, I AN, localmente conexas, localmente conexas por caminos y localmente simplemente conexas.
\end{obs}%COPIADO lo de abajo, cambiar las palabras!!!!!!!!!!!!
El \textit{Teorema de Invarianza del Dominio} (\autoref{teo:inv_dom}) dice que si $W\subset \R^n$ y $W'\subset \R^m$ son abiertos y existe $\phi: W \rightarrow W'$ homeomorfismo, entonces $n=m$. Esto implica que, dado un  $p\in M$ de una variedad, hay un único $n=n(p)$ tal que un entorno $U^p$ es homeomorfo a un abierto $U'\subset \R^n$. Llamamos $n(p)$ la \enfatiza{dimensión en p}. Como para todo punto $q\in U$ podemos tomar $U$ como entorno de $q$, se tiene por tanto $n(q)=n(p)$. Luego en toda la componente conexa de $p$, el $n$ que aparece es el mismo, y lo llamaremos \enfatiza{dimensión} de dicha componente conexa. Si escribimos $M=\amalg M_i$, con $M_i$ componentes conexas de $M$, todas las $M_i$ son variedades, y si todas las $M_i$ tienen la misma dimensión $n$, entonces escribimos $n=dim M$ , y decimos que $M$ es una \enfatiza{n-variedad}.\\
El ejemplo más trivial de $n$-variedad es $\R^n$, pero también lo es cualquier abierto suyo. De hecho, podemos generalizar esto:
\begin{prop}%%%%PROP: abierto variedad abierto
Todo subconjunto abierto de una $n$-variedad es una $n$-variedad.
\end{prop}
\begin{proof}
Sea $M$ una $n$-variedad, y sea $V$ un subconjunto abierto de $M$. Para todo $p\in V$, $p$ tiene un entorno $U^p$ en $M$ que es homeomorfo a un subconjunto abierto de $\R^n$. $U^p\cap V$ es también abierto y homeomorfo a un subconjunto abierto de $\R^n$, y está contenido en $V$. Por tanto $V$ es localmente homeomorfo a $R^n$. Por otro lado, todo abierto de un espacio Hausdorff es Hausdorff y todo abierto de un espacio II AN es II AN. Por lo que $M$ es una variedad.
\end{proof}
\begin{eje}%%%%EJE: variedades.
\begin{itemize}
\item Las 0-variedades son espacios discretos numerables. La única 0-variedad conexa es un punto.
\item Existen dos 1-variedades conexas salvo homeomorfismo: la recta $\R$ y el círculo $\mathbb{S}^1\\
=\left\{(x,y)\in \R^2\mid x^2+y^2=1\right\}=\left\{z=e^{2\pi i\theta}\in \C \mid \theta \in \left[0,1\right] \right\}=\left\{ z\in \C \mid |z|=1\right\}$.
\end{itemize}
\end{eje}
\begin{defin}%%%%DEF: SUPERFICIE
Una \textbf{\emph{superficie}} es una $2$-variedad.
\end{defin}
\begin{eje}%%%%EJE: toro y esfera
\begin{itemize}
\item La esfera $\mathbb{S}^2=\{(x,y,z) \in \R^3\mid  x^2+y^2+z^2=1\}$ (\autoref{fig:esfera}).
\item El toro $\mathbb{T}^2=\{(x,y,z)\in \R^3\mid  (\sqrt{x^2+y^2}-2)^2+z^2=1\}$ (\autoref{fig:toro}).
\end{itemize}
\end{eje}
\begin{figure}%%%%FIG: Toro y esfera
\begin{subfigure}{.5\textwidth}
\centering
\begin{tikzpicture}[scale=0.8]
\draw (0,0) circle (2cm);
\draw (2,0) arc[x radius=2, y radius=0.7, start angle=0, end angle=-180];
\draw [dashed] (2,0) arc[x radius=2, y radius=0.7, start angle=0, end angle=180];
\end{tikzpicture}
\caption{$\Esfera$\label{fig:esfera}}
\end{subfigure}
\begin{subfigure}{.5\textwidth}
\centering
\begin{tikzpicture}
\draw (6,-0.1) ellipse (2.9cm and 1.4 cm);
\draw (7.5,0.1) arc[x radius=1.5, y radius=0.4, start angle=0, end angle=-180];
\draw (7.3,-0.1) arc[x radius=1.3, y radius=0.3, start angle=0, end angle=180]; 
\end{tikzpicture}
\caption{$\Toro\label{fig:toro}$}
\end{subfigure}	
\caption{Ejemplos de superficies.}
\end{figure}

\section{Variedades con borde}\label{sec:borde}
Hay una serie de espacios topológicos que no son variedades pero que tienen interés desde el punto de vista geométrico y son útiles a la hora de estudiarlas. Por ejemplo, la bola cerrada $\overline{\mathbb{B}}^n$ no es una variedad, puesto que para los puntos de su frontera no hay entornos homeomorfos a abiertos de $\R^n$. Vamos a definir por tanto una clase de espacios que extienda a las variedades pero que admita la existencia de algún tipo de \textit{bordes}:
\begin{defin}
Una \enfatiza{variedad con borde $n$-dimensional} $M$ (o $n$-variedad con borde) es un espacio topológico  Hausdorff y II AN tal que todo punto $p\in M$ tiene un entorno $U$ homeomorfo a un abierto $U'\subset \mathbb{H}^n=
\left\{(x_1,\dots ,x_n)\in \R^n\mid x_n\geq 0\right\}$.
\end{defin}
Si $M$ una n-variedad con borde, $p\in M$ y $\varphi :U\to U'\subset \mathbb{H}^n$ un homeomorfismo como en la definición. Como en el caso de variedades, decimos que $U$ es un \enfatiza{dominio coordenado}, que $\varphi$ es una \enfatiza{aplicación coordenada} para $M$ y que el par $(U,\varphi )$ es una \enfatiza{carta para $M$}.\\
Si denotamos $\varphi (p)=a=(a_1,\dots ,a_n)$, podemos distinguir entre dos tipos de puntos:
\begin{itemize}

\item[1.] \textit{Puntos interiores:} Si $a_n>0$, entonces podemos tomar una bola $B_{\varepsilon}(a)\subset U'$ y $V=\varphi^{-1}(V')$, por lo que $V$ es un entorno de $p$ homeomorfo a un abierto de $\R^n$. Decimos por tanto que $p$ es un \enfatiza{punto interior de $M$}, y denotamos por $\interior{M}$ al conjunto de puntos interiores de $M$.
\item[2.] \textit{Puntos borde:} Si $a_n=0$ para todo $n$, entonces podemos tomar una semibola $V'=B_{\varepsilon}^{+}=B_{\varepsilon}(a)\cap \mathbb{H}^n$, y $V=\varphi^{-1}(V')$. Decimos que $p$ es un \enfatiza{punto borde de $M$}. Al conjunto de todos los puntos borde de $M$ lo denotamos por $\partial M$.
\end{itemize}
\begin{prop}%%%%PROP: interior y borde de una variedad con borde
Sea $M$ una variedad con borde, entonces $\interior{M}$ es un abierto de M, y es en sí mismo una $n$-variedad.
\end{prop}
\begin{proof}

\end{proof}
\begin{prop}%%%%PROP: invarianza del borde
Si $M$ es una $n$-variedad con borde, entonces un punto $p\in M$ no puede ser un punto interior y un punto borde al mismo tiempo. O lo que es lo mismo, $M=\interior{M} \cup \partial M$, con $\interior{M} \cap \partial M=\emptyset$.
\end{prop}
\begin{proof}

\end{proof}
\begin{corol}
Si $M$ es una $n$-variedad con borde, entonces $\partial M$ es un cerrado en $M$, $\partial M$ es una $(n-1)$-variedad. Por otro lado, $M$ es una variedad si y solo si $\partial M=\emptyset$.
\end{corol}
\begin{proof}

\end{proof}
Hacemos incapié en que los conceptos de \textit{variedad} y \textit{variedad con borde} son distintos. Una variedad con borde puede ser o no una variedad, pues puede tener el borde vacío. En cambio, una variedad es siempre una variedad con borde en la que todo punto es un punto interior. Para evitar confusiones, si el contexto lo pide, utilizaremos \enfatiza{variedad sin borde} para referirnos a una variedad en el sentido de la \autoref{def:variedad}.\\
Veamos algunos ejemplos no triviales de superficies con borde:
\begin{eje}\label{rep:borde}
\begin{itemize}
\item[(1)] El cilindro  (\autoref{fig:cilindro}): $$Cil=\left\{(x,y,z)\in \R^3 \mid x^2+y^2=1,\, z\in \left[ 0,1\right] \right\} \,.$$
\item[(2)] La banda de Möbius (\autoref{fig:mobius}):
\begin{align*}
Mob=
& \left\{ \left(\left(1+\left(y-\frac{1}{2}\right)\cos \left(\pi x\right)\right), \, \left(1+\left(y-\frac{1}{2}\right)\cos \left(\pi x\right)\right)\sin \left( 2 \pi x\right), \right. \right.\\
& \left. \left. \left( y-\frac{1}{2} \right)\sin \left( \pi x\right)\right) \mid x\in \left[0,1\right], \, y\in \left[0,1\right] \right\} \, .
\end{align*}.
\end{itemize}
\end{eje}

\begin{figure}[b]%%%%FIG: Cilindro y Möbius
\begin{subfigure}{.5\textwidth}
\centering
\begin{tikzpicture}[scale=1.5]
\draw (-.5,0) arc [x radius=1, y radius=.3, start angle=180, end angle=360] -- +(0,1) arc[x radius=1, y radius=.3, start angle=0, end angle=180] -- +(0,-1);
\draw [dashed] (-.5,0) arc[x radius=1, y radius=.3, start angle=180, end angle=0];
\draw (1.5,1) arc[x radius=1, y radius=.3, start angle=0, end angle=-180];
\end{tikzpicture}
\caption{Cilindro.\label{fig:cilindro}}
\end{subfigure}
\begin{subfigure}{.5\textwidth}
\centering
\begin{tikzpicture}[scale=1.7]
%\draw [help lines, step=1mm] (-2,-2) grid (2,2);
%\draw [gray](-2,0) -- (2,0);
%\draw [gray](0,-2) -- (0,2);
%\draw[looseness=1] (.65,-.05) [out=135, in=0] to (-.2, .175) [out=180, in=90] to(-.5,-.0) [out=-90, in=180] to (0,-.2) [out=0, in=-90] to (.7,.1) [out=90, in=0] to (-.,.45) [out=180, in=90] to (-.8,0) [out=-90, in=180] to (.2,-.5) [out=0, in=-100] to (.75,-.25) [out=95, in=-80] to (.7,.15) ;
\draw (1,.7)[out=-90, in=45] to (.95,.6) [out=-135,in=0] to (-.5, -.0) [out=180, in=-20] to (-1,.1) -- (-1,.7) arc [x radius=1, y radius=0.2, start angle=180, end angle=0] -- (1,.1) [out=-160, in=-10] to (-.3,0.006);
\draw (-.0,.05) [out=160, in=-45] to (-.95,.6) [out=135, in=-90] to (-1,.7);
\draw (-.7,.39) arc [x radius=.7, y radius=.1, start angle=180, end angle=-4];
\end{tikzpicture}
\caption{Banda de Möbius.\label{fig:mobius}}
\end{subfigure}
\end{figure}
\section{Suma conexa de variedades}

OJO:TAMBIEN SE PUEDE DAR EN FUNCION DE ESPACIOS ADJUNCION
Sean $V_1$ y $V_2$ dos $n$-variedades conexas. Dados $p_1\in V_1$ y $p_2\in V_2$ sean $U_1^{p_1}\subset V_1$, $U_2^{p_2}\subset V_2$  entornos abiertos de $p_1$ y $p_2$ respectivamente, y sean $\phi_1:U_1\to\R^n$ y $\phi_2:U_2\to\R^n$ dos homeomorfismos tales que $\phi_1(p_1)=0$ y $\phi_2(p_2)=0$. Si llamamos $B_1=\phi_1^{-1}(B_1(0))\subset V_1$ y $B_2=\phi_2^{-1}(B_1(0))\subset V_2$, consideremos $V_1^o=V_1-B_1$, $V_2^o=V_2-B_2$ y $V_1^o \amalg V_2^o$ con la topología unión disjunta.
Se define la relación de equivalencia $\sim$ en la que si $x_1\in S_1=\phi_1^{-1}(\partial B_1(0))$, $x_2\in S_2=\phi_2^{-1}(\partial B_1(0))$, entonces $x_1\sim x_2$ si y sólo si $\phi_1(x_1)=\phi_2(x_2)$, y se considera el cociente 

$$M=\frac{V_1^o\amalg V_2^o}{\sim}.$$\\

\begin{defin}%%%% DEF: suma conexa
A $M$ así definida se le llama \textbf{\textit{suma conexa}} de $V_1$ y $V_2$, y se denota por $M=V_1\#V_2$.
\end{defin}


\begin{figure}[t]%%%%FIG: 2-toro
\centering
\begin{tikzpicture}
\draw (0.23,1.85) arc[x radius=2, y radius=1.4, start angle=150, end angle=-150];
\draw (-0.23,1.85) arc[x radius=2, y radius=1.4, start angle=30, end angle=330];
\draw plot [smooth] coordinates { (-0.23,1.85) (0,1.67)(0.23,1.85)};
\draw plot [smooth] coordinates { (-0.23,0.45) (0,0.67)(0.23,0.45)};

\draw (-1.25,1.35) arc[x radius=0.7, y radius=0.5, start angle=0, end angle=-180];
\draw (-1.45,1) arc[x radius=0.5, y radius=0.3, start angle=0, end angle=180]; 
\draw (2.65,1.35) arc[x radius=0.7, y radius=0.5, start angle=0, end angle=-180];
\draw (2.45,1) arc[x radius=0.5, y radius=0.3, start angle=0, end angle=180]; 

\draw (0,1.67) arc[x radius=0.1, y radius=0.5, start angle=90, end angle=270];
\draw [dashed] (0,1.67) arc[x radius=0.1, y radius=0.5, start angle=90, end angle=-90];
\end{tikzpicture}
\caption{Suma conexa de toros.}
\end{figure}






\begin{prop}%%%% PROP: suma conexa es variedad.
Sean $V_1$ y $V_2$ variedades. Entonces $M=V_1\#V_2$ es una variedad.
\end{prop}
\begin{proof}
Denotemos la proyección $\pi:M_1^o\amalg M_2^o\to X$. Sea $S=\pi (S_1)=\pi (S_2)$. Tenemosdos abiertos $U_j=M_j^o-S_j$, $j=1,2$ saturados. Por tanto, $\pi :U_ j\to \pi (U_j)=U_j'$ es homeomorfismo. Esto implica que $X$ es localmente $\R^n$ en los puntos de $U_1'\cup U_2'$ . Además ahí la topología es Hausdorff y IIAN.
Veamos ahora qué ocurre para un punto $p\in S$. Se tiene que $p=\pi (p_1)=\pi (p_2)$, $p_j\in S_j$, $j=1,2$, y $\varphi_j(p_j)=x_0 \in \partial B_1(0) \in \R^n$. Tomamos un entorno $V\subset \partial B_1(0)$ de $x_0$ en $\partial B_1(0)$, con lo que $\hat{V}=\{rx\mid   r\in (1-\varepsilon , 1+\varepsilon ), x\in V\}$ es entorno de $x_0$ en $\R^n$, y $\hat{V}-B_1(0)=\{rx\mid r\in [1, 1+\varepsilon ), x \in V\}$. Sea $V_j=\varphi_j^{-1}(\hat{V}-B_1(0))\subset M_j^o$, que es entorno de $p_j$. Claramente $V_1\amalg V_2$ es abierto saturado de $M_1^o\amalg M_2^o$, luego $\tilde{V}=\pi (V_1\amalg V_2)$ es entorno de $p$ en $X$. Veamos ahora que es homeomorfo a un abierto de $\R^n$. Sea
\begin{align*}
\Phi : & V_1\amalg V_2  \to  V\times (1-\varepsilon , 1+ \varepsilon ), \\
& q_1\in V_1  \mapsto  (x,r), r=\norm{\varphi_1(q_1)}, x=\varphi_1(q_1)/r,\\
& q_2\in V_2  \mapsto  (x,2-r), r= \norm{\varphi_2(q_2)}, x= \varphi_2(q_2)/r.
\end{align*}
Por tanto, $\Phi : V_1 \to V \times [1, 1+\varepsilon)$ y $\Phi : V_2 \to V \times (1- \varepsilon, 1]$ son homeomorfismos. Además, $q_1 \sim q_2$ si y sólo si $\Phi(q_1)=\Phi(q_2)$. De este modo, $\Phi$ induce una aplicación continua y biyectiva $$\overline{\Phi} : \tilde{V} \to V \times (1- \varepsilon, 1+ \varepsilon)$$

$\overline{\Phi}$ es abierta: si tomamos un abierto básico saturado de $V_1\amalg V_2$, o bien está totalmente incluido en $V_1-S_1$ o en $V_2-S_2$, en cuyo caso su imagen es un abierto de $V\times (1-\varepsilon ,1)$ o $V\times (1,1+\varepsilon )$, o bien interseca a $S_1$ y $S_2$. En ese caso se puede asumir que es un abierto de la forma $W_1\amalg W_2$, construido como antes y donde hemos partido de un $W\subset V \subset \partial B_1(0)$. Entonces $\overline{\Phi} (\tilde{W})= W \times (1-\delta , 1+\delta )$ con $0<\delta \leq \varepsilon$, $\tilde{W} = \pi (W_1 \amalg W_2)$. Luego $\overline{\Phi}$ es un homeomorfismo, y $\tilde{V}$ es homeomorfo a un abierto de $\R^n$.

Los abiertos construidos, $\tilde{V} \subset X$, se pueden tomar en cantidad numerable para formar una base de la topología, con lo cual $X$ es IIAN. También, dado un $q\in U_j'$, $j=1,2$, y un $p\in S$, se puede tomar un abierto $\tilde{V}$ entorno de $p$ disjunto de un entorno pequeño de $q$. Y si tomamos $p, p' \in S$ distintos, los abiertos $\tilde{V}$, $\tilde{V}'$ construidos partiendo de $V$, $V'\subset \partial B_1(0)$ disjuntos, serán disjuntos. Luego $X$ es Hausdorff.
\end{proof}




\begin{obs}
Sea $S$ una superficie. Entonces la suma conexa $S\# \Esfera$ es homeomorfa a $S$ (\autoref{fig:suma_toro_esfera}).\label{obs:suma_esfera}
\end{obs}

\begin{figure}[h]%%%% FIG: suma conexa toro esfera
\centering
\begin{tikzpicture}[line cap=round,line join=round,>=triangle 45,x=1cm,y=1cm, scale=0.75]
\fill [black!10] (0.5,0) arc[x radius=0.5, y radius=0.5, start angle=0, end angle=-180];
\fill[color=black!15] (0.5,0) arc[x radius=0.5, y radius=0.23, start angle=0, end angle=360];
\draw [rotate around={0:(-5.5,0)}] (-5.5,0) ellipse (2.5cm and 2cm);
\draw(0,0) circle (0.5cm);
\draw (0.5,0) arc[x radius=0.5, y radius=0.23, start angle=0, end angle=-180];
\draw [dashed] (0.5,0) arc[x radius=0.5, y radius=0.23, start angle=0, end angle=180];
\draw [rotate around={0:(5.7,0)}] (5.7,0) ellipse (2.5cm and 2cm);
\draw [dashed] (6.67,1.32) arc[x radius=0.45, y radius=0.22, start angle=0, end angle=180];
\draw (6.67,1.32) arc[x radius=0.45, y radius=0.22, start angle=0, end angle=-180];
\filldraw [color=black, fill=black!10] [rotate around={0:(-5,1.32)}] (-5.0,1.32) ellipse (0.45cm and 0.22cm);
\draw (5.77,1.32) arc[x radius=0.45, y radius=0.45, start angle=180, end angle=0];
\draw (-6.1,0) arc[x radius=0.4, y radius=0.2, start angle=180, end angle=0];
\draw (-6.4,0.31) arc[x radius=0.7, y radius=0.4, start angle=180, end angle=360];
\draw (-1.8,0.56) node[anchor=north west] {$\#$};
\draw (1.22,0.56) node[anchor=north west] {$=$};
\draw (5.1,0) arc[x radius=0.4, y radius=0.2, start angle=180, end angle=0];
\draw (4.8,0.31) arc[x radius=0.7, y radius=0.4, start angle=180, end angle=360];
\end{tikzpicture}
\caption{Suma conexa de un toro y una esfera.\label{fig:suma_toro_esfera}}
\end{figure}


\begin{obs}%%%%EJE: Asas y toros
Dada una superficie $S$, la suma conexa $S\# \Toro$ puede visualizarse como el espacio que se obtiene al ``pegarle'' un \textit{asa} a $S$. De forma más precisa, sea $S_0$ que denota a $S$ con dos perforaciones, es decir la superficie que queda al retirar dos discos cerrados disjuntos de $M$ (se dará una construcción más precisa en \autoref{def:perforacion}). Entonces $S_0$ y $\mathbb{S}^1\times I$ son ambas superficies con borde, y sus bordes son ambos homeomorfos a la union disjunta de dos circunferencias. Sea $\tilde{S}$ el espacio adjunción (\autoref{app:adjuncion}) obtenido pegando $S_0$ y $\mathbb{S}^1\times I$ por sus bordes. Este espacio cociente es homeomorfo a $S\# \Toro$. La razón se puede ver en la \autoref{fig:toro_asa}, y es que podemos obtener un espacio homeomorfo a $S\# \Toro$ primero quitando un disco abierto de $S$, luego pegando un disco cerrado con dos discos abiertos quitados (es decir, la parte gris de la figura), y finalmente pegando a la frontera de la construcción el cilindro $\Esfera \times I$. Dado que la primera operación da resultado a un espacio homeomorfo a $S$ con dos discos abiertos quitados, el resultado es el mismo que si quitamos directamente a $S$ dos discos abiertos y entonces pegamos el cilindro a su frontera. \label{obs:toro_asa}

%%165%%
\end{obs}

\begin{figure}%%%%FIG: Toro-Asa
\centering
\begin{tikzpicture}
%1


\fill[gray!20] (-3.3, 0.2) -- (-3.3,0) arc[x radius=0.7, y radius=0.2, start angle=0, end angle=-180] (-4.7,0) -- (-4.7, 0.2)  [out=90, in=270] to (-5,1) arc[x radius=0.25, y radius=0.1, start angle=180, end angle=360] arc[x radius=0.5, y radius=0.3, start angle=180, end angle=360]  arc[x radius=0.25, y radius=0.1, start angle=180, end angle=360] [out=270, in=90] to (-3.3,0.2) ;


\draw (-3.3, 0.2) -- (-3.3,0) arc[x radius=0.7, y radius=0.2, start angle=0, end angle=-180] (-4.7,0) -- (-4.7, 0.2)  [out=90, in=270] to (-5,1) arc[x radius=0.25, y radius=0.1, start angle=180, end angle=360] arc[x radius=0.5, y radius=0.3, start angle=180, end angle=360]  arc[x radius=0.25, y radius=0.1, start angle=180, end angle=360] [out=270, in=90] to (-3.3,0.2) ;






\draw (-3,1) arc[x radius=1, y radius=0.8, start angle=0, end angle=180]; \draw [dashed] (-5,1)  arc [x radius=0.25, y radius=0.1, start angle=180, end angle=0];
\draw (-4.5,1) arc [x radius=0.5, y radius=0.3, start angle=180, end angle=0];
\draw (-3.5,1) [dashed] arc [x radius=0.25, y radius=0.1, start angle=180, end angle=0];

\draw (-5.8,-0.5) [out=0, in=270] to (-4.7,0);
\draw [dashed](-3.3,0) arc[x radius=0.7, y radius=0.2, start angle=0, end angle=180];
\draw (-3.3,0) [out=270, in=180] to (-2.2,-0.5);

%2



\draw (1,1) arc[x radius=1, y radius=0.8, start angle=0, end angle=180];
\draw [dashed] (-1,1)  arc [x radius=0.25, y radius=0.1, start angle=180, end angle=0];
\draw (-0.5,1) arc [x radius=0.5, y radius=0.3, start angle=180, end angle=0];
\draw (0.5,1) [dashed] arc [x radius=0.25, y radius=0.1, start angle=180, end angle=0];


\fill[color=gray!20] (-0.5,1) arc[x radius=0.5, y radius=0.3, start angle=180, end angle=360] arc[x radius=0.25, y radius=0.1, start angle=180, end angle=360] [out=270, in=90] to (1.2,0) arc[x radius=1.2, y radius=0.3, start angle=0, end angle=-180] [out=90, in=270] to (-1,1) arc[x radius=0.25, y radius=0.1, start angle=180, end angle=360];
\draw (-0.5,1) arc[x radius=0.5, y radius=0.3, start angle=180, end angle=360] arc[x radius=0.25, y radius=0.1, start angle=180, end angle=360] [out=270, in=90] to (1.2,0) arc[x radius=1.2, y radius=0.3, start angle=0, end angle=-180] [out=90, in=270] to (-1,1) arc[x radius=0.25, y radius=0.1, start angle=180, end angle=360];

\draw (-1.8,-0.5) [out=0, in=270] to (-1.2,0);

\draw [dashed] (1.2,0) arc[x radius=1.2, y radius=0.3, start angle=0, end angle=180];
\draw (1.2,0) [out=270, in=180] to (1.8,-0.5);

%3


\draw (5.2,0) [out=270, in=180] to (5.8, -0.5);
\draw (2.2,-0.5) [out=0, in=270] to (2.8, 0);
\fill[gray!20](2.8,0) arc[x radius=1.2, y radius=0.3, start angle=180, end angle=540];
\draw (2.8,0) arc[x radius=1.2, y radius=0.3, start angle=180, end angle=360];

\fill[white] (3,0) -- (3, 0.5) arc[x radius=1, y radius=.8, start angle=180, end angle=0] -- +(0,-0.5) arc[x radius=0.25, y radius=.1, start angle=0, end angle=-180] -- +(0,.5) arc[x radius=0.5, y radius=0.3, start angle=0, end angle=180] -- +(0,-.5) arc[x radius=0.25, y radius=0.1, start angle=0, end angle=-180];

\draw (3,0) -- (3, 0.5) arc[x radius=1, y radius=.8, start angle=180, end angle=0] -- +(0,-0.5) arc[x radius=0.25, y radius=.1, start angle=0, end angle=-180] -- +(0,.5) arc[x radius=0.5, y radius=0.35, start angle=0, end angle=180] -- +(0,-.5) arc[x radius=0.25, y radius=0.1, start angle=0, end angle=-180];

\draw[dashed] (3,0) arc[x radius=0.25, y radius=0.1, start angle=180, end angle=0] (5,0) arc[x radius=0.25, y radius=0.1, start angle=0, end angle=180];
\draw[dash pattern=on 7.5pt off 2.2pt on 2.2pt off 2.2pt on 2.2pt off 2.2pt on 2.2pt off 1.8 pt on 28.4pt off 2pt on 2pt off 2pt on 2pt off 2pt on 2pt off 2.4pt] (2.8,0) arc[x radius=1.2, y radius=0.3, start angle=180, end angle=0];

\draw (-2,1) node {$ \approx $};
\draw (2,1) node {$ \approx $};


\end{tikzpicture}
\caption{Suma conexa de una superficie y un toro visto como pegar un \textit{asa}. \label{fig:toro_asa}}
\end{figure}




%%%%%%%%%%%%%%%%%%%%%%%%%%%%%%%%%%%%%%%%%%%%%%%%%%%%%%%%%%%%%%%%%%%%%%%%%%%%%%%%%%%%%%%%%%%%%%%%%%%%%%%%%%%%%%%%%%%%%%%%%%%%%%%%%%%%%%%%%%%%%%%%%%%%%%%%%%%%%%%%%%%%%%%%%%%%%%%%%%%%%%%%%%%%%%%%%%%%%%%%%%%%%%%%%%%%%%%%%%%%%%%%%%%%%%%%%%%%%%%%%%%%%%%%%%%%%%%%%%%%%%%%%%%%%%%%%%%%%%%%%%%%%%%%%%%%%%%%%%%%%%%%%%%%%%%%%%%%%%%%%%%%%%%%%%%%%%%%%%%%%%%%%%%%%%%%%%%%%%%%%%%%%%%%%%%%%%%%%%%%%%%%%%%%%%%%%%%%%%%%%%%%%%%%%%%%%%%%%%%%%%%%%%%%%%%%%%%%%%%%%%%%%%%%%%%%%%%%%%%%%%%%%%%%%%%%%%%%%%%%%%%%%%%%%%%%%%%%%%%%%%%%%%%%%%%%%%%%%%%%%%%%%%%%%%%%%%%%%%%%%%%%%%%%%%%%%%%%%%%%%%%%%%%%%%%%%%%%%%%%%%%%%%%%%%%%%%%%%%%%%%%%%%%%%%%%%%%%%%%%%%%%%%%%%%%%%%%%%%%%%%%%%%%%%%%%%%%%%%%%%%%%%%%%%%%%%%%%%%%%%%%%%%%%%%%%%%%%%%%%%%%%%%%%%%%%%%%%%%%%%%%%%%%%%%%%%%%%%%%%%%%%%%%%%%%%%%%%%%%%%%%%%%%%%%%%%%%%%%%%%%%%%%%%%%%%%%%%%%%%%%%%%%%%%%%%%%%%%%%%%%%%%%%%%%%%%%%%%%%%%%%%%%%%%%%%%%%%%%%%%%%%%%%%%%%%%%%%%%%%%%%%%%%%%%%%%%%%%%%%%%%%%%%%%%%%%%%%%%%%%%%%%%%%%%%%%%%%%%%%%%%%%%%%%%%%%%%%%%%%%%%%%%%%%%%%%%%%%%%%%%%%%%%%%%%%%%%%%%%%%%%%%%%%%%%%%%%%%%%%%%%%%%%%%%%%%%%%%%%%%%%%%%%%%%%%%%%%%%%%%%%%%%%%%%%%%%%%%%%%%%%%%%%%%%%%%%%%%%%%%%%%%%%%%%%%%%%%%%%%%%%%%%%%%%%%%%%%%%%%%%%%%%%%%%%%%%%%%%%%%%%%%%%%%%%%%%%%%%%%%%%%%%%%%%%%%%%%%%%%%%%%%%%%%%%%%%%%%%%%%%%%%%%%%%%%%%%%%%%%%%%%%%%%%%%%%%%%%%%%%%%%%%%%%%%%%%%%%%%%%%%%%%%%%%%%%%%%%%%%%%%%%%%%%%%%%%%%%%%%%%%%%%%%%%%%%%%%%%%%%%%%%%%%%%%%%%%%%%%%%%%%%%%%%%%%%%%%%%%%%%%%%%%%%%%%%%%%%%%%%%%%%%%%%%%%%%%%%%%%%%%%%%%%%%%%%%%%%%%%%%%%%%%%%%%%%%%%%%%%%%%%%%%%%%%%%%%%%%%%%%%%%%%%%%%%%%%%%%%%%%%%%%%%%%%%%%%%%%%%%%%%%%%%%%%%%%%%%%%%%%%%%%%%


\chapter{Triangulación de superficies}

Un hecho fundamental para la prueba del teorema de clasificación es que toda superficie es triangulable. La demostración, atribuída a Radó en 1925 \cite{rado}, utiliza el \emph{teorema de Schönflies}, cuya prueba es larga y técnica. Utilizaremos el truco de Kirby para superficies dado por Hatcher \cite{hatcher_torus}.

\

\section{Complejos simpliciales y triangulación}\label{sec:simplices}

Para poder dar una definición rigurosa de triangulación de variedades necesitamos la noción de \textit{complejos simpliciales}. Estos son construcciones formadas por \textit{símplices}, que son una generalización de los triángulos. En esta primera parte me baso en las definiciones de Munkres \cite{munkres}.

\begin{defin}%%%%DEF: Posición general
Sean $v_0,\dots v_k$ $k+1$ puntos distintos de $\R^n$. Decimos que $\{ v_0,\dots ,v_k\}$ están en \enfatiza{posición general} si $c_0,\dots c_k$ son números reales tales que  $$\sum_{i=0}^{k}c_iv_i=0 \text{ y } \sum_{i=0}^kc_i=0,$$ entonces $c_0=\dots =c_k=0$.
\end{defin}


\begin{defin}%%%%DEF: Símplice
Sean $\{ v_0,\dots ,v_n\}$ un conjunto de $k+1$ puntos de $\R^n$ en posición general. El \enfatiza{símplice} generado por ellos, que denotamos por $[ \, v_0,\dots ,v_k ] \,$, es el conjunto $$[ \, v_0,\dots ,v_k] \, =\left\{  \sum_{i=0}^{k}t_iv_i \mid t_i\geq 0,\, \sum_{i=0}^{k}t_i=1 \right\}, $$ con la topología heredada. Para todo punto $x=\sum_it_iv_i\in [ \, v_0,\dots ,v_k] \,$, llamamos a los $t_i$ \enfatiza{coordenadas baricéntricas de $x$}. Cada uno de los $v_i$ se llama \enfatiza{vértice} del símplice. Al entero $k$ se le llama \enfatiza{dimensión}, y diremos que $[ \, v_0,\dots ,v_k] \,$ es un \enfatiza{$k$-símplice}. 
\end{defin}

\begin{eje}
Un $0$-símplice es un punto, un 1-símplice es un segmento, un $2$-símplice es un triángulo junto a su interior, un $3$-símplice es un tetraedro sólido, y así sucesivamente (\autoref{fig:simplices}).
\end{eje}

\begin{figure}
\centering
\begin{tikzpicture}[scale=1.4]

\fill[gray!20] (1,0) -- (1.5,0.87) -- (2,0) --cycle;
\draw (1,0) -- (1.5,0.87) -- (2,0) -- cycle;
\draw (-1,0.2) -- (0,0.5);
\fill[gray!20] (3,0.5) -- (3.5,0.9)--(4,0.5)--(3.5,0)--cycle;
\draw (3,0.5) -- (3.5,0.9)--(4,0.5)--(3.5,0)--cycle;
\draw (3.5,0.9)--+(0,-.9);
\draw [dashed](3,.5)--+(1,0);
\fill (3,.5) circle (1.5pt);
\fill (4,.5) circle (1.5pt);
\fill (3.5,.9) circle (1.5pt);
\fill (3.5,0) circle (1.5pt);
\fill (-1,0.2) circle (1.5pt);
\fill (0,0.5) circle (1.5pt);
\fill (-2,0.6) circle (1.5pt);
\fill (1,0) circle (1.5pt);
\fill (1.5, 0.87) circle (1.5pt);
\fill (2,0) circle (1.5pt);
\end{tikzpicture}
\caption{$k$-símplices, $k=0,\dots ,3$.\label{fig:simplices}}
\end{figure}



Sea $\sigma$ un $k$-símplice. Cada símplice generado por un subconjunto no vacío de vértices de $\sigma$ se llama \enfatiza{cara de $\sigma$}. Las caras que no son iguales a $\sigma$ se llaman \enfatiza{caras propias}. Las caras $0$-dimensionales de $\sigma$ son sus vértices, y a las caras $1$-dimensionales se les llama \enfatiza{lados}. Las caras $(k-1)$-dimensionales de un $k$-símplice se llaman bordes, y a su unión la llamamos \enfatiza{frontera}. Definimos el \enfatiza{interior} como $\sigma$ menos su frontera. 


\begin{defin}%%%%DEF: Complejo simplicial
Un \enfatiza{complejo simplicial} es una colección $K$ de símplices en algún espacio euclídeo $\R ^n$, que satisface las siguientes condiciones:
\begin{itemize}
\item[(i)] Si $\sigma \in K $, entonces toda cara de $\sigma$ está en $K$
\item[(ii)] La intersección de dos símplices cualesquiera en $K$ es o bien vacía o bien una cara de ambos.
%\item[(iii)] $K$ es una colección finitamente local.
\end{itemize}
\label{def:complex}
\end{defin}
%La tercera condición implica que $K$ es numerable, pues todo punto de $\R ^n$ tiene un entorno intersecando al menos un número finito de símplices de $K$, y este recubrimiento abierto de $\R ^n$ tiene un subrecubrimiento numerable. A nosotros los símplices que más nos interesan son los \enfatiza{complejos simpliciales finitos}, que son los que contienen únicamente un número finito de símplices. Para estos complejos, la condición (iii) es redundante.

Si $K$ un complejo simplicial en $\R ^n$, llamamos \enfatiza{dimensión de $K$} a la dimensión máxima de los símplices en $K$. Esta no es mayor que $n$.
Un subconjunto $K'\subseteq K$ se dice que es un \enfatiza{subcomlejo de $K$} si para todo $\sigma \in K'$, toda cara de $\sigma$ está en $K'$. Un subcomplejo es un complejo simplicial en sí.
Para todo $k\leq n$, el conjunto de todos los símplices de $K$ de dimensión menor o igual que $k$ es un subcomplejo llamado \enfatiza{$k$-esqueleto de $K$}.

La \autoref{fig:complex} muestra un complejo simplicial en $\R^2$. En cambio en la \autoref{fig:not_complex} los símplices representados no forman un complejo, pues no se respeta la condición (ii) de la \autoref{def:complex}.

%%
%%
%%

%Aquí se puede coger la definición de polítopo de munkres, pero quizas es un poco inutil
%%
%%
%%


\begin{defin}%%%%DEF: Poliedro
Sea un complejo simplicial $K$ en $\R ^n$. La unión de todos los símplices en $K$ junto con la topología heredada de $\R ^n$ es un espacio topológico que denotamos por $|K|$ y que llamamos \enfatiza{poliedro de K}.
\end{defin}



\begin{figure}%%%%FIG: Complejos y no complejos
   \begin{minipage}{0.48\textwidth}
   \centering
	\begin{tikzpicture}
	\fill[gray!20](-2.96,0)--(-2.46,1.19)--(-1,1.32)--(-1.45,0)--cycle;
	\fill[gray!20] (-0.05,0.66)--(1.03,1.41)--(1.55,0.64)--cycle;
	\draw (-3.98,-.3)--(-2.96,0)--(-2.46,1.19)--(-1.45,0)--(-1,1.32)--(-2.46,1.19);
	\draw(-2.96,0)--(-1.45,0);
	\draw (-1,1.32)--(-0.05,0.66)--(1.03,1.41)--(1.55,0.64)--(-0.05,0.66)--(-0.22,-0.45)--(1.14,-0.28)--(1.55,0.64);
	\draw (-1.11,-0.46)--(-0.56,-0.17);
	\fill [color=black] (-3.98,-0.3) circle (1.5pt);
	\fill [color=black] (-2.96,0) circle (1.5pt);
	\fill [color=black] (-2.46,1.19) circle (1.5pt);
	\fill [color=black] (-1.45,0) circle (1.5pt);
	\fill [color=black] (-1,1.32) circle (1.5pt);
	\fill [color=black] (-0.05,0.66) circle (1.5pt);
	\fill [color=black] (1.03,1.41) circle (1.5pt);
	\fill [color=black] (1.55,0.64) circle (1.5pt);
	\fill [color=black] (1.14,-0.28) circle (1.5pt);
	\fill [color=black] (-0.22,-0.45) circle (1.5pt);
	\fill [color=black] (-1.11,-0.46) circle (1.5pt);
	\fill [color=black] (-0.56,-0.17) circle (1.5pt);

	\end{tikzpicture}
	\caption{Un complejo simplicial en $\R^2$\label{fig:complex}}
   \end{minipage}\hfill
   \begin{minipage}{0.48\textwidth}
     \centering
    \begin{tikzpicture}[scale=0.5]
	\fill[gray!20] (-1.82,0.46)--(1.04,3.14)--(2.33,0.67)--cycle;
	\fill[gray!20] (1.67,1.94)--(2.94,-0.5)--(5,3)--cycle;
	\draw (1.04,3.14)-- (2.94,-0.5);
	\draw (1.04,3.14)-- (-1.82,0.46);
	\draw (-1.82,0.46)-- (2.33,0.67);
	\draw (2.94,-0.5)-- (5,3);
	\draw (5,3)-- (1.67,1.94);
	\fill [color=black] (1.04,3.14) circle (3pt);
	\fill [color=black] (2.94,-0.5) circle (3pt);
	\fill [color=black] (-1.82,0.46) circle (3pt);
	\fill [color=black] (2.33,0.67) circle (3pt);
	\fill [color=black] (5,3) circle (3pt);
	\fill [color=black] (1.67,1.94) circle (3pt);
	\end{tikzpicture}
	\caption{Símplices que no forman un complejo.\label{fig:not_complex}}
    
   \end{minipage}
\end{figure}

\begin{defin}%%%%DEF: Triangulación
Sea $X$ un espacio topológico. Llamamos \enfatiza{triangulación de $X$} a un homeomorfismo entre $X$ y el poliedro de algún complejo simplicial.
\end{defin}

\begin{defin}%%%%DEF: Superficie triangulable
Toda superficie que admita una triangulación se dice \enfatiza{triangulable}.
\end{defin}



\section{Teorema de Radó}

\begin{tma}[Teorema de Radó]
Toda superficie es triangulable por un poliedro de un complejo simplicial 2-dimensional, en donde cada $1$-símplice es una cara de exáctamente dos $2$-símplices.\label{teo:rado}
\end{tma}


\chapter{Teorema de Clasificación}
En esta sección daremos una demostración clásica del teorema de clasificación de superficies compactas, que fue probado por primera vez en 1907 por Max Dehn y Poul Heegaard \cite{dehn}.  En las primeras dos secciones utilizo las nociones que da Lee \cite{lee1} sobre la representación de superficies, y las amplío a la representación de superficies con borde. En la tercera sección demuestro el teorema basándome en la prueba de Lee \cite{lee1}, que a su vez se basa en la demostración clásica de Seifert y Threlfall \cite{seifert}. Finalmente clasifico las superficies con borde, siguiendo la idea de Massey \cite{massey}.


\section{Superficies como cocientes}

Para el teorema de clasificación necesitamos un método uniforme de representación de las superficies compactas. Trataremos de dar una forma de representarlas como polígonos, y veremos que toda superficie compacta se puede representar en el plano como el cociente de un polígono por una relación de equivalencia que identifica sus lados dos a dos. Empecemos viendo tres ejemplos elementales: la esfera $\Esfera$, el plano proyectivo $\Proyectivo$ y el toro $\Toro$. Como veremos, estas superficies son fundamentales pues toda superficie compacta se puede construir a partir de ellas. 

\begin{prop}%%%%PROP: ESFERA COCIENTE DISCO Y CUADRADO%%%%
\label{prop:Esfera como cociente de disco y cuadrado}
La esfera $\mathbb{S}^2$ es homeomorfa a los siguientes espacios cociente: 
%%%%%%%%%%o cocientes??????????%%%%%
\begin{itemize}
\item[(a)] El disco cerrado $\overline{\mathbb{B}}^2\subseteq \mathbb{R}^2$ módulo la relación de equivalencia generada por $(x,y)\sim (-x,y)$, si $(x,y)\in \partial \overline{\mathbb{B}}^2$
\item[(b)] El cuadrado $S=\{(x,y):|x|+|y|\leq 1\}$ módulo la relación de equivalencia generada por $(x,y)\sim(-x,y)$ si $(x,y)\in \partial S$.
\end{itemize}
\end{prop}
\begin{proof}
Para ver que cada espacio es homeomorfo a la esfera, daremos una aplicación cociente desde cada espacio a la esfera que haga las mismas identificaciones que la relación de equivalencia, y entonces apelaremos a la unicidad del espacio cociente. (\autoref{teo:unicidad_espacio_cociente})\\
Para (a), vamos a definir una aplicación que ``envuelve'' cada segmento horizontal del disco en un paralelo de la esfera (ver \autoref{fig:esfera_cociente_circunferencia}).
Formalmente, esta aplicación $\pi:\overline{\mathbb{B}}^2\to \mathbb{S}^2$ vienen dada por 
$$\pi(x,y)=\left\{\begin{array}{lc}
			(-\sqrt{1-y^2} \cos\dfrac{\pi x}{\sqrt{1-y^2}}, -\sqrt{1-y^2}, y), & y\neq \pm 1 \\
			\\(0,0,y), & y=\pm1 

\end{array}
\right.$$
Es claro que $\pi$ es continua y hace las mismas identificaciones que la relación de equivalencia. Por ser sobreyectiva, es una aplicación cociente (\autoref{teo:aplicac_cerrada}).

Para probar (b), sea $\alpha:S\to \overline{\mathbb{B}}^2$ el homeomorfismo construido en la demostración de \autoref{teo:convexo_homeom_esfera} que manda linealmente cada segmento radial entre el origen y la frontera de $S$ al segmento paralelo entre centro del disco y su frontera. Hagamos ahora $\beta=\pi \circ \alpha : S \to \mathbb{S}^2$, donde $\pi$ es la aplicación cociente del parágrafo anterior. Tenemos entonces que $\beta$ identifica $(x,y)$ y $(-x,y)$ cuando $(x,y)\in \partial S$, pero por otro lado es inyectiva, así que hace las mismas identificaciones que la aplicación cociente definida en (b), completando así la demostración (ver \autoref{fig:esfera_cuadrado}). 
\end{proof}
\begin{figure}%%%%%%% FIGURA: ESFERA COCIENTE CIRCUNFERENCIA%%%%
\begin{center}
\begin{tikzpicture}[line cap=round,line join=round,>=triangle 45,x=1.5cm,y=1.5cm, scale=0.7]
\draw [<-][shift={(0,1.25)}] plot[domain=0.64:2.5,variable=\t]({1*1.25*cos(\t r)+0*1.25*sin(\t r)},{0*1.25*cos(\t r)+1*1.25*sin(\t r)});
\draw (0,3) node[anchor=north] {$\pi$};
\draw (2,1.5) arc[x radius=0.7, y radius=1.5, start angle=90, end angle=270];
\draw [->] (2,0) -- (0.87,-1.68);
\draw(2,0) circle (2.25cm);
\draw(-2,0) circle (2.25cm);
\draw (-3.12,1)-- (-0.88,1);
\draw (-3.41,0.5)-- (-0.59,0.5);
\draw (-3.41,-0.5)-- (-0.59,-0.5);
\draw (-3.12,-1)-- (-0.88,-1);
\draw [->] (2,0) -- (2,2);
\draw [dashed] (3.1,1) arc[x radius=1.1, y radius=0.2, start angle=0, end angle=180];
\draw (3.1,1) arc[x radius=1.1, y radius=0.2, start angle=0, end angle=-180];
\draw [dashed] (3.4,0.5) arc[x radius=1.4, y radius=0.2, start angle=0, end angle=180];
\draw (3.4,0.5) arc[x radius=1.4, y radius=0.2, start angle=0, end angle=-180];
\draw [dashed] (3.5,0) arc[x radius=1.5, y radius=0.2, start angle=0, end angle=180];
\draw (3.5,0) arc[x radius=1.5, y radius=0.2, start angle=0, end angle=-180];
\draw [dashed] (3.4,-0.5) arc[x radius=1.4, y radius=0.2, start angle=0, end angle=180];
\draw (3.4,-0.5) arc[x radius=1.4, y radius=0.2, start angle=0, end angle=-180];
\draw [dashed] (3.10,-1) arc[x radius=1.10, y radius=0.2, start angle=0, end angle=180];
\draw (3.10,-1) arc[x radius=1.10, y radius=0.2, start angle=0, end angle=-180];
\draw (-3.5,0)-- (-0.5,0);
\draw (-0.6,0.1)--(-0.5,0)--(-0.4,0.1);
\draw (-3.6,0.13)--(-3.5,0)--(-3.4,0.13);
\draw [->] (2,0) -- (4,0);
\begin{scriptsize}
\fill [color=black] (-2,1.5) circle (2.0pt);
\fill [color=black] (-2,-1.5) circle (2.0pt);
\fill [color=black] (2,1.5) circle (2.0pt);
\fill [color=black] (2,-1.5) circle (2.0pt);
\fill [color=black] (-2,1.5) circle (2.0pt);
%\fill [color=black] (1.3,-0.17) circle (2.0pt);
\draw (1.2,-0.07)--(1.3,-0.17)--(1.4,-0.07);
\end{scriptsize}
\end{tikzpicture}
\end{center}

\caption{La esfera como cociente del disco $\overline{\mathbb{B}}^2$.\label{fig:esfera_cociente_circunferencia}}
\end{figure}

\begin{figure}%%%%FIGURA: ESFERA CUADRADO
\centering

\begin{tikzpicture}[use optics, line cap=round,line join=round,>=triangle 45,x=1cm,y=1cm, scale=0.7]
\draw[-<-={at=0.125},->-={at=0.375}, -<<-={at=0.625}, ->>-={at=0.875} ](0,0) circle (2cm);
\draw [-<-={at=0.5}] (-8,0)-- (-6,2);
\draw [-<-={at=0.5}](-4,0)-- (-6,2);
\draw [-<<-={at=0.5}](-8,0)-- (-6,-2);
\draw [->>-={at=0.5}](-6,-2)-- (-4,0);
\draw(6,0) circle (2cm);
\draw (8,0) arc[x radius=2, y radius=0.7, start angle=0, end angle=-180];
\draw [dashed] (8,0) arc[x radius=2, y radius=0.7, start angle=0, end angle=180];
\draw [->] (-8.5,0) -- (-3.5,0);
\draw [->] (-6,-2.5) -- (-6,2.5);
\draw [->] (-2.5,0) -- (2.5,0);
\draw [->] (0,-2.5) -- (0,2.5);
\draw [->] (3.5,0) -- (8.5,0);
\draw [->] (6,0) -- (6,2.5);
\draw [->] (6,0) -- (4.51,-2.5);
\draw [<-][shift={(-3,1.25)}] plot[domain=0.64:2.5,variable=\t]({1*1.25*cos(\t r)+0*1.25*sin(\t r)},{0*1.25*cos(\t r)+1*1.25*sin(\t r)});
\draw [<-] [shift={(3,1.25)}] plot[domain=0.64:2.5,variable=\t]({1*1.25*cos(\t r)+0*1.25*sin(\t r)},{0*1.25*cos(\t r)+1*1.25*sin(\t r)});
\draw (-3.1,3.21) node[anchor=north west] {$ \alpha $};
\draw (2.89,3.23) node[anchor=north west] {$ \pi $};
\draw [->-={at=0.45}, -<<-={at=0.82}](6,2) arc[x radius=0.7, y radius=2, start angle=90, end angle=270];
\begin{scriptsize}
\fill [color=black] (0,2) circle (1.5pt);
\fill [color=black] (-8,0) circle (1.5pt);
\fill [color=black] (-4,0) circle (1.5pt);
\fill [color=black] (-6,-2) circle (1.5pt);
\fill [color=black] (-6,2) circle (1.5pt);
\fill [color=black] (-2,0) circle (1.5pt);
\fill [color=black] (2,0) circle (1.5pt);
\fill [color=black] (0,-2) circle (1.5pt);
\fill [color=black] (0,2) circle (1.5pt);
\fill [color=black] (5.34,-0.66) circle (1.5pt);
\end{scriptsize}
\end{tikzpicture}

\caption{La esfera como cociente de un cuadrado.\label{fig:esfera_cuadrado}}


\end{figure}

\begin{prop}%%%%PROP: TORO COMO CUADRADO
\label{prop:toro_cuadrado}
El toro $\Toro$ es homeomorfo al espacio cociente resultante de la relación de equivalencia en el cuadrado $I\times I$ que identifica $(x,0)\sim (x,1)$ para todo $x\in I$, y $(0,y)\sim(1,y)$ para todo $y\in I$ (\autoref{fig:toro_cuadrado}). 
\end{prop}
\begin{proof}
Definimos la aplicación $q:I\times I \to \Toro$ que manda $q(u,v)=(e^{2\pi iu},e^{2\pi iv})$. Por el teorema de la aplicación cerrada (\autoref{teo:aplicac_cerrada}), es una aplicación cociente. Al hacer las mismas identificaciones que la relación de equivalencia, por la unicidad del espacio cociente (\autoref{teo:unicidad_espacio_cociente}) se obtiene el resultado.
\end{proof}



\begin{figure}%%%%FIGURA: TORO COMO CUADRADO
\begin{center}
\begin{tikzpicture}[use optics, line cap=round,line join=round,>=triangle 45,x=1cm, y=1cm, scale=0.7]

\draw [->-={at=0.5}](-8,-2)-- (-8,2);
\draw [->>-={at=0.5}](-8,2)-- (-4,2);
\draw [-<-={at=0.5}](-4,2)-- (-4,-2);
\draw [->>-={at=0.5}](-8,-2)-- (-4,-2);
\draw (-2,1)-- (2,1);
%\draw (-2,1)-- (-2,-1);
%\draw (2,1)-- (2,-1);
\draw (2,-1)-- (-2,-1);
%\draw [-<-={at=0.6]}](2,1) arc[x radius=0.5, y radius=1, start angle=90, end angle=270];
%\draw (2,1) arc[x radius=0.5, y radius=1, start angle=90, end angle=-90];
\draw [-<-={at=0.6]}](-2,1) arc[x radius=0.5, y radius=1, start angle=90, end angle=270];
\draw [dashed] (-2,1) arc[x radius=0.5, y radius=1, start angle=90, end angle=-90];
\filldraw [-<-={at=0.55]}][color=black, fill=black!0] [rotate around={0:(2,0)}] (2,0) ellipse (0.5cm and 1cm);
\draw [->>-={at=0.5}](-2.5,0)-- (1.5,0);
\draw [rotate around={0:(6,0)}] (6,0) ellipse (3cm and 1.5cm);
\draw (7.4,0.1) arc[x radius=1.5, y radius=0.4, start angle=0, end angle=-180];
\draw (7.2,-0.1) arc[x radius=1.3, y radius=0.3, start angle=0, end angle=180]; 
\draw [dashed](6,-1.5) arc[x radius=0.2, y radius=0.6, start angle=-90, end angle=90];
\draw [-<-={at=0.63}] (6,-0.3) arc[x radius=0.2, y radius=0.6, start angle=90, end angle=270];
\begin{scriptsize}
\fill [color=black] (-8,-2) circle (1.5pt);
\fill [color=black] (-8,2) circle (1.5pt);
\fill [color=black] (-4,2) circle (1.5pt);
\fill [color=black] (-4,-2) circle (1.5pt);
\fill [color=black] (-2.5,0) circle (1.5pt);
\fill [color=black] (1.5,0) circle (1.5pt);
\fill [color=black] (5.8,-0.9) circle (1.5pt);
\end{scriptsize}
\end{tikzpicture}
\end{center}
\caption{El toro como cociente de un cuadrado.\label{fig:toro_cuadrado}}
\end{figure}


\begin{prop}%%%%PROP: PLANO PROYECTIVO COMO CUADRADO
\label{prop:proyectivo_cociente_cuadrado}
El plano proyectivo $\mathbb{P}^2$ es homeomorfo a los siguientes espacios cociente:
\begin{itemize}
\item[(a)] El disco cerrado $\overline{\mathbb{B}}^2$ módulo la relación de equivalencia generada por $(x,y) \sim (-x,-y)$ para cada $(x,y)\in \partial \overline{\mathbb{B}}^2$.
\item[(b)] La región cuadrada $S=\{(x,y):|x|+|y|\leq 1\} $ módulo la relación de equivalencia generada por $(x,y)\sim (-x,-y) $ para todo $(x,y)\in \partial S$.
\end{itemize}
\end{prop}
\begin{proof}
Sea $p:\mathbb{S}^2 \to \mathbb{P}^2$ la aplicación cociente dada por la relación de equivalencia $\sim$ generada por $(x,y) \sim (-x,-y)$ para cada $(x,y)\in \mathbb{S}^2$, que representa $\mathbb{P}^2$ como el cociente de una esfera. %AQUI FALTA UN POCO DE EXPLICACION, EJEMPLO 4.54 DEL LEE
Si $F:\overline{\mathbb{B}}^2 \to \mathbb{S}^2$ es la aplicación que manda el disco al emisferio superior de la esfera por $F(x,y)=(x,y,\sqrt{1-x^2-y^2})$, entonces $p\circ F:\overline{\mathbb{B}}^2 \to \mathbb{S}^2/\sim$ es sobreyectiva (---lo demuestro?) y es así una aplicación cociente por el teorema de la aplicación cerrada (\autoref{teo:aplicac_cerrada}). La aplicación identifica únicamente $(x,y)\in \partial \overline{\mathbb{B}}^2$ con $(-x,-y)\in \partial \overline{\mathbb{B}}^2$, por lo que $\mathbb{P}^2$ es homeomorfo al espacio cociente resultante.
Para la parte (b) hacemos como en la demostración de la \autoref{prop:Esfera como cociente de disco y cuadrado} (b). 
\end{proof}


\begin{figure}[h]%%%%FIGURA: PLANO PROYECTIVO COMO CUADRADO
\begin{center}
\begin{tikzpicture}[use optics, line cap=round,line join=round,>=triangle 45,x=1.0cm,y=1.0cm, scale=0.6]
\draw[-<-={at=0.5}, -<-={at=1}](-3,0) circle (2cm);
\draw [-<-={at=0.5}] (4,2)-- (6,0);
\draw [-<<-={at=0.5}](6,0)-- (4,-2);
\draw [-<-={at=0.5}](4,-2)-- (2,0);
\draw [-<<-={at=0.5}](2,0)-- (4,2);
\begin{scriptsize}
\fill [color=black] (6,0) circle (1.5pt);
\fill [color=black] (4,2) circle (1.5pt);
\fill [color=black] (2,0) circle (1.5pt);
\fill [color=black] (4,-2) circle (1.5pt);
\fill [color=black] (-3,-2) circle (1.5pt);
\fill [color=black] (-3,2) circle (1.5pt);
\end{scriptsize}
\end{tikzpicture}
\caption{Representación de $\mathbb{P}^2$ como un espacio cociente.\label{fig:plano_proyectivo_cuadrado}}
\end{center}
\end{figure}


En las anteriores proposiciones hemos visto una o varias formas de representar superficies dadas ciertas construcciones geométricas. En estos casos hemos dado aplicaciones y demostraciones concretas para validar nuestros argumentos, pero a medida que aumenta la sofisticación es más útil guiarse visualmente por las figuras construidas. Por ello debemos formalizar un método para construir superficies identificando lados de figuras geométricas.
Daremos por sabidas las definiciones básicas de símplices (\autoref{sec:simplices}) y CW-complejos (\autoref{sec:CW}).

\begin{defin}%%%DEF:Polígono
Un \enfatiza{polígono} es un subconjunto de $\R^2$ que es homeomorfo a $\mathbb{S}^1$ y está formado por un número finito de segmentos, que llamaremos \enfatiza{bordes} y que se intersecan sólo en sus extremos, que llamaremos \enfatiza{vértices}. %Los $0$-símplices y $1$-símplices del poígono son respectivamente sus \enfatiza{vértices} y sus \enfatiza{bordes}. Del lema \autoref{lemma:cw} se sigue que un borde yace exactamente en dos vértices.
\end{defin}

\begin{defin}%%%DEF:Región poligonal

Una \enfatiza{región poligonal} es un subconjunto compacto de $\R^2$ cuyo interior es homeomorfo al disco $\mathbb{B}^2$ y cuya frontera es un polígono.
%Una \enfatiza{región poligonal} es un subconjunto compacto de $\R^2$ cuyo interior es una bola coordenada y cuya frontera es un polígono.
A los vértices y lados del polígono de la frontera también los llamamos vértices y lados de la región poligonal.
\end{defin} 

Veamos pues que identificando bordes de regiones poligonales de par en par obtenemos un espacio cociente que es siempre una superficie:

\begin{prop}%%%PROP: Teorema poligonos
\label{prop:poligonos}
Sean $P_1,\dots, P_k$ regiones poligonales en el plano, y sea $P=P_1,\dots ,P_k$, y supongamos dada una relación de equivalencia en $P$ que identifica algunos bordes de los polígonos con otros por homeomorfismos afines. Entonces se tiene:
\begin{itemize}
\item[(a)] El espacio cociente resultante es un CW-complejo $2$-dimensional cuyo $0$-esqueleto es la imagen del conjunto de vértices de $P$ por la aplicación cociente, y cuyo $1$-esqueleto es la imagen de la unión de los bordes de las regiones poligonales.
\item[(b)] Si la relación de equivalencia identifica cada borde de cada $P_i$ con exactamente otro borde de un $P_j$ (no necesariamente $i\neq j$), entonces el espacio cociente resultante es una superficie compacta.
\item[(c)] Si para algunos $P_i$ la relación de equivalencia identifica algún borde suyo con exáctamente otro borde de un $P_j$ (no necesariamente $i\neq j$), y para los bordes restantes no hay ninguna identificación, entonces el espacio cociente resultante es una superficie con borde.
\end{itemize}
\end{prop}

\begin{proof}
%...6.4 del Lee.
\end{proof}

\begin{eje}%%%%EJE: Botella Klein
La \enfatiza{botella de Klein} es la superficie $K$ obtenida identificando los lados del cuadrado $I\times I$ de acuerdo a $(0,t)\sim (1,t)$ y $(t,0)\sim (1-t,1)$ para $0\leq t\leq 1$.  Para visualizar $K$, podemos pensar en pegar los lados izquierdo y derecho creando un cilindro, y luego hacer pasar el extremo superior por la parte inferior del cilindro, para finalmente pegar los dos extremos (ver \autoref{fig:klein}).
\end{eje}

\begin{figure}[b]%%%%FIG: Klein
\centering
\begin{tikzpicture} [use optics]
%\draw [help lines] (-6,-2) grid (6,2);
%1
\draw[-<<-={at=0.125}, -<-={at=0.375}, ->>-={at=0.625}, -<-={at=0.875}] (-5,-1) -- (-5,1) -- (-3,1) -- (-3, -1) -- (-5,-1);
%2
\draw[-<<-] (-1.7, -1.085)--+(0,2);
\draw [-<-={at=0.43}, -<-={at=0.897}] (-2,-1) -- +(0,2) arc[x radius=0.5, y radius=0.1, start angle=180, end angle=360] -- +(0,-2) arc[x radius=0.5, y radius=0.1, start angle=0, end angle=-180];
\draw [dashed] (-2,-1) arc[x radius=0.5, y radius=0.1, start angle=180, end angle=0];
\draw (-2,1) arc[x radius=0.5, y radius=0.1, start angle=180, end angle=0];
%3
\draw [->-={at=0.535}, ->-={at=0.974}](0.2,-0.5) [out=80, in=180] to (1.64,1.5) [out=0, in=100] to (3,-1) arc [x radius=0.35, y radius=0.1, start angle=0, end angle=-180] [out=85, in=0] to (1.64,1) [out=180, in=95] to (0.75, -0.5) arc[x radius=0.275, y radius=0.1, start angle=0, end angle=-180];
\draw [dashed] (0.2,-0.5) arc[x radius=0.275, y radius=0.1, start angle=180, end angle=0] (3,-1) arc[x radius=0.35, y radius=0.1, start angle=0, end angle=180];

%4
\draw [->-={at=0.07}][dash pattern= on 52.5pt off 2pt on 2pt off 2pt on 2pt off 2pt on 79.5pt off 2pt on 2pt off 2pt on 2pt off 2pt on 2pt off 2pt on 2pt off 2pt on 2.2pt off 2pt on 2pt off 2pt on 2pt off 2pt on 2pt off 2pt on 2pt off 2pt on 2pt off 2pt on 200pt]
	(5,-1) arc[x radius=0.4, y radius=0.1, start angle=0, end angle=-180] [out=135, in= 270] to +(-0.2,0.6) [out=90, in=270] to +(0.5,1) [out=90, in=0] to (4.3,1) [out=180, in=90] to +(-0.2,-0.3) [out=270, in=135] to +(0.5, -0.8) [out=-65, in=-65] to +(-0.3,-0.2) [out=-80, in=80] to (4.2,-1) arc [x radius=0.4, y radius=0.1, start angle=180, end angle=0] ;

\draw [dash pattern= on 2pt off 2pt on 2pt off 2pt on 2pt off 2pt on 2pt off 2pt on 2pt off 2pt on 2pt off 2pt on 2pt off 2pt on 2pt off 2pt on 2pt off 2pt on 2pt off 2pt on 180pt ](5,-1)[out=100, in=-45] to (4.6,-0.1) [out=115, in=115] to +(-0.3,-0.2) [out=135, in=270] to +(-0.5,1) [out=90, in=180] to (4.4,1.3) [out=0, in=90] to +(0.4, -0.7) [out=-90, in=90] to +(0.4,-1) [out=-90, in=45] to (5,-1);


\draw [-Stealth](-.4,1.5) arc[x radius=0.4, y radius=0.2, start angle=150, end angle=30];
\draw [-Stealth](-2.8,1.5) arc[x radius=0.4, y radius=0.2, start angle=150, end angle=30];
\draw [-Stealth](3,1.5) arc[x radius=0.4, y radius=0.2, start angle=150, end angle=30];
\end{tikzpicture}
\caption{Construcción de la botella de Klein.\label{fig:klein}}
\end{figure}
\begin{eje}
Veamos ahora como obtener las figuras del \autoref{eje:rep_borde}. El cilindro $Cil$ es la superficie con borde que se obtiene al identificar los lados de un cuadrado $I\times I$ dada la relación de equivalencia $(0,y)\sim (1,y),$ con $y\in I$ (\autoref{fig:ident_cil}). La banda de Möbius $Mob$ es la superficie con borde que se obtiene identificando los lados del cuadrado $I\times I$ de acuerdo a la relación $(0,y)\sim (1,1-y)$, $y\in I$ (\autoref{fig:ident_mob}).
\end{eje}

\section{Representación de superficies}
Vamos ahora a dar un método uniforme para representar todas las superficies y las superficies con borde como regiones poligonales del plano. A su vez asociaremos a cada región poligonal una secuencia de símbolos llamada palabra.


\begin{defin}%%%%DEF: palabra
Sea $S$ un conjunto. Una \textbf{\textit{palabra en $S$}} es una $k$-tupla ordenada de símbolos, cada uno de la forma $a$ o $a^{-1}$, para cierto $a\in S$.
\end{defin}

\begin{defin}%%%%DEF: Representación poligonal
\label{def:rep_pol}
Una \textbf{\textit{representación poligonal}}, que denotaremos por $$\mathcal{P}=\langle S\mid W_1,\dots ,W_k\rangle$$ es un conjunto finito S junto con un número finito de palabras $W_1,..,W_k$ de longitud $3$ o más, tal que para todo $a\in S$ existe un $W_i$ tal que $a\in W_i$. Por cuestiones de notación, cuando el conjunto $S$ esté descrito listando sus elementos, quitaremos los corchetes que rodean los elementos de $S$ y denotaremos las palabras $W_i$ por youxtaposición. Por ejemplo, la presentación con $S=\{a,b\}$ y la palabra $W=(a,b,a^{-1},b^{-1})$ se escribe $\langle a,b\mid  aba^{-1}b^{-1}\rangle$. 

Permitimos el caso especial de que $S=\{a\}$ (u otro símbolo cualquiera) y que $\mathcal{P}$ tenga una sola palabra de longitud $2$, es decir, $\langle a\mid aa\rangle$, $\langle a\mid a^{-1}a^{-1}\rangle$, $\langle a\mid aa^{-1}\rangle$ y $\langle a\mid a^{-1}a\rangle$.
\begin{comment} o $1$ \end{comment} 
\begin{comment}, $\langle a\mid a\rangle$ y $\langle a\mid a^{-1}\rangle$ \end{comment}
\end{defin}

\begin{defin}%%%%DEF: Realización geométrica
Toda representación poligonal $\mathcal{P}$ da lugar a un espacio topológico $|\mathcal{P}|$, llamado \textbf{\textit{realización geométrica de $\mathcal{P}$}} . $|\mathcal{P}|$ se obtiene de la siguiente manera:
\begin{itemize}
\item[1.] Para cada $W_i\in \mathcal{P}$ de longitud $k$, sea $P_i$ la $k$-región poligonal convexa centrada en el origen con lados de longitud 1, ángulos iguales y tal que un vértice yace sobre el eje $OY$.
\item[2.] Se define una correspondencia uno a uno (que llamaremos \textit{etiquetado}) entre los símbolos de $W_ i$ y los bordes de $P_i$ en sentido contrario a las agujas del reloj, empezando por el que yace en el eje $OY$.
\item[3.] Sea $|\mathcal{P}|$ el espacio cociente de $\coprod_i P_i$ determinado identificando bordes que tengan el mismo símbolo, conforme al homeomorfismo afín que hace coincidir los primeros vértices de lo bordes con una etiqueta dada $a$ y los últimos vertices de los que tienen la correspondiente etiqueta $a^{-1}$ (en el sentido contrario a las agujas del reloj).

\end{itemize}


Si $\mathcal{P}$ es una de las representaciones poligonales de un solo elemento, definimos $|\mathcal{P}|$ como la esfera $\mathbb{S}^2$ si la palabra es $aa^{-1}$ o $a^{-1}a$, o como el plano proyectivo $\mathbb{P}^2$ si es $aa$ o $a^{-1}a^{-1}$.% También decimos que $|\mathcal{P}|$ es el disco cerrado $\overline{\mathbb{B}}^2$ si la palabra es $a$ o $a^{-1}$.
\end{defin}

Por notación, dadas dos palabras $W_1$ y $W_2$, $W_1W_2$ representará la palabra formada concatenando $W_1$ y $W_2$. Por otro lado, adoptaremos la convención de que $(a^{-1})^{-1}=a$.

También lo que sigue $S$ denotará una secuencia cualquiera de símbolos, $a,b,c,a_1,a_2,\dots$ símbolos de $S$, $e$ un símbolo que no sea de $S$ y $W_1, W_2, \dots$ palabras formadas por símbolos de $S$.

\begin{defin}%%%%DEF: caras, lados y vértices polígono
Los interiores, los bordes y los vértices de cada región polgonal $P_i$ se llaman \textbf{\emph{caras, bordes y vértices de la presentación}}. El número de caras es el mismo que el número de palabras, y el número de bordes coincide con la suma de la longitud de las palabras.
Para un lado etiquetado $a$, el \textbf{\emph{vértice inicial}} es el primero en el sentido contrario de las agujas del reloj, y el otro es el \textbf{\emph{vértice final}}. Para un lado etiquetado $a^{-1}$, estas definiciones se invierten. 
\end{defin}

\begin{defin}%%%%DEF: representación de una superficie
Sea $\mathcal{P} =\langle S\mid W_1,\dots ,W_k\rangle$ una representación poligonal. 
\begin{itemize}
\item[(i)] Decimos que $\mathcal{P}$ es una \enfatiza{representación de una superficie} si para todo $a\in S$, $a$ ocurre exáctamente dos veces en $W_1,...,W_k$ como $a$ o como $a^{-1}$.
\item[(ii)] Si en cambio en la representación poligonal cada símbolo $a\in S$ ocurre una o dos veces en $W_1,\dots ,W_k$ como $a$ o $a^{-1}$, diremos que $\mathcal{P}$ es una \enfatiza{representación de una superficie con borde}. 
\end{itemize}
\end{defin}
\begin{obs}%M%%%:OBS: Representación superficie con/sin borde
Por la \autoref{prop:poligonos}, la realización geométrica de una representación de una superficie es una superficie compacta, y la realización geométrica de una representación de una superficie con borde es una superficie con borde.\\
En el apartado (ii) de la definición anterior dejamos la posibilidad de que cada símbolo aparezca exáctamente dos veces, como en (i). Esto no es ninguna ambigüedad pues como se expuso en la \autoref{sec:borde} una superficie sin borde es siempre una superficie con borde, y una superficie con borde tal que tiene borde vacío es una superficie sin borde.
\end{obs}
\begin{defin}%%%%DEF: representación de un espacio topológico
Si $X$ es un espacio topológico y $\mathcal{P}$ una representación poligonal cuya realización geométrica es homeomorfa a $\mathcal{P}$, decimos que $\mathcal{P}$ es una \textbf{\emph{representación de $X$}}.
\end{defin}
\begin{obs}%%%%OBS: representación de una sola cara implica conexo
Un espacion topológico que admite una representación con una sola cara es conexo, pues es homeomorfo al cociente de una región poligonal conexa. Con más de una cara, puede ser o no conexo.
\end{obs}
\begin{eje}%%%%PROP: representación de superficies importantes
Veamos las representaciones de algunas superficies importantes (ver \autoref{fig:Representacion_superficies_importantes} y \autoref{fig:representacion_esfera_proyectivo}).
\begin{itemize}
\item[(a)] $\mathbb{S}^2=\langle a\mid aa^{-1}\rangle=\langle a,b\mid  abb^{-1}a^{-1}\rangle$ (\autoref{prop:Esfera como cociente de disco y cuadrado})
\item[(b)] $\mathbb{P}^2= \langle a\mid aa \rangle = \langle a,b\mid abab \rangle$ (\autoref{prop:proyectivo_cociente_cuadrado})
\item[(c)] $\Toro=\langle a,b\mid aba^{-1}b^{-1}\rangle$ (\autoref{prop:toro_cuadrado})
\end{itemize}
\end{eje}

\begin{eje}%M%%%:PROP: representación de superficies con borde
Veamos las representaciones de algunas superficies con borde (\autoref{fig:representacion_borde}):
\begin{itemize}
\item[(a)] $\overline{\mathbb{B}}^2=\langle a,b,c\mid abc\rangle$ (\autoref{prop:rep_disco}).
\item[(b)] $Mob=\langle a,b,c\mid abac\rangle$ (\autoref{prop:rep_moebius}).
\item[(c)] $Cil=\langle a,b,c\mid aba^{-1}c\rangle$ (\autoref{prop:rep_mobius}).
\end{itemize}
\end{eje}
\begin{figure}[h]%%%%FIG: Representación de superficies impoirtantes
\centering
\begin{tikzpicture}[use optics, line cap=round,line join=round,>=triangle 45,x=1.0cm,y=1.0cm, scale=0.8]
\draw [-<-](0,2)-- (2,0);
\draw [-<-](2,0)-- (0,-2);
\draw [-<-](0,-2)-- (-2,0);
\draw [-<-](-2,0)-- (0,2);

\draw [-<-](3,0)-- (5,2);
\draw [->-](5,2)-- (7,0);
\draw [->-](7,0)-- (5,-2);
\draw [-<-](5,-2)-- (3,0);

\draw [-<-](-3,0)-- (-5,2);
\draw [-<-](-5,-2)-- (-3,0);
\draw [-<-](-5,-2)-- (-7,0);
\draw [-<-](-7,0)-- (-5,2);

\draw (-4,1) node[anchor=south west] {$ a $};
\draw (-4,-1) node[anchor=north west] {$ b $};
\draw (-6,1) node[anchor=south east] {$ a $};
\draw (-6,-1) node[anchor=north east] {$ b $};

\draw (1,1) node[anchor=south west] {$ b $};
\draw (1,-1) node[anchor=north west] {$ a $};
\draw (-1,1) node[anchor=south east] {$ a $};
\draw (-1,-1) node[anchor=north east] {$ b $};

\draw (6,1) node[anchor=south west] {$ b $};
\draw (6,-1) node[anchor=north west] {$ a $};
\draw (4,1) node[anchor=south east] {$ a $};
\draw (4,-1) node[anchor=north east] {$ b $};

\draw (5.1,-2.6) node[anchor=south] {$ \Toro $};
\draw (0.1,-2.6) node[anchor=south] {$ \Proyectivo $};
\draw (-4.9,-2.6) node[anchor=south] {$ \Esfera $};

\end{tikzpicture}
\caption{Representación de superficies importantes.\label{fig:Representacion_superficies_importantes}}
\end{figure}

\begin{figure}[h]%%%%FIG: Representacion esfera proyectivo
\centering
\begin{tikzpicture}[use optics, line cap=round,line join=round,>=triangle 45,x=1cm,y=1cm, scale=0.8]
\draw [-<-={at=0}, ->-={at=0.5}](-5,0) circle (2cm);
\draw [->-={at=0}, ->-={at=0.5}](5,0) circle (2cm);
\begin{scriptsize}
\fill [color=black] (-5,2) circle (1.5pt);
\fill [color=black] (-5,-2) circle (1.5pt);
\fill [color=black] (5,2) circle (1.5pt);
\fill [color=black] (5,-2) circle (1.5pt);
\end{scriptsize}
\draw (-7.1,0) node[anchor=east] {$ a $};
\draw (-2.9,0) node[anchor=west] {$ a $};
\draw (2.9,0) node[anchor=east] {$ a $};
\draw (7.1,0) node[anchor=west] {$ a $};
\draw (-4.9,-2.9) node[anchor=south] {$ \Esfera $};
\draw (5.1,-2.9) node[anchor=south] {$ \Proyectivo $};
\end{tikzpicture}
\caption{Representaciones de la esfera y el plano proyectivo. \label{fig:representacion_esfera_proyectivo}}
\end{figure}


De ahora en adelante visualizaremos el plano proyectivo como un \enfatiza{crosscap}, cuya construcción se sigue en la \autoref{fig:crosscap_paso_a_paso}.
\begin{figure}%%%FIG: CROSSCAP
\centering

\begin{tikzpicture}[use optics][line cap=round,line join=round,>=triangle 45,x=1.0cm,y=1.0cm]
%1
\fill[gray!10] (-5,5) -- (-7,3) -- (-5,1) -- (-3,3) -- cycle;
\draw [->-](-5,5)-- (-7,3);
\draw [-<<-](-5,5)-- (-3,3);
\draw [-<-](-3,3)-- (-5,1);
\draw [->>-](-7,3)-- (-5,1);

%2
\fill[gray!10] (-2,3.5) arc[ x radius=2, y radius=0.5, start angle=180, end angle=540];
%\draw(0,3) circle (2cm);
\draw [->>-={at=0.125},->-={at=0.375}, ->>-={at=0.625}, ->-={at=0.875}] (-2,3.5) arc[x radius=2, y radius=0.5, start angle=180, end angle=540];
\draw [name path=C] (-2,3.5) arc[x radius=2, y radius=2.5, start angle=180, end angle=360];

%3
\fill [gray!10] (3, 3.5) [out=270, in=225, looseness=0.7] to (5,3.5) [out=45, in=90, looseness=0.7] to (7, 3.5) [out=-90, in=-45, looseness=0.7] to (5,3.5) [out=135, in=90, looseness=0.7] to (3, 3.5);
%\draw[name path=A] (3,3.5) [-<-={at=0.25}, -<<-={at=0.75}](3,3.5) arc[x radius=1, y radius=0.5, start angle=180, end angle=-180];
%\draw [name path=B] (5,3.5) [-<<-={at=0.25}, -<-={at=0.75}] arc[x radius=1, y radius=0.5, start angle=180, end angle=-180];
\draw [name path=C] (3,3.5) arc[x radius=2, y radius=2.5, start angle=180, end angle=360];
\draw [->>-][out=270, in=225, looseness=0.7] (3, 3.5) to (5,3.5);
\draw [-<<-][out=45, in=90, looseness=0.7] (5,3.5) to (7,3.5);
\draw [-<-][out=90, in=135, looseness=0.7] (3, 3.5) to (5,3.5);
\draw [->-][out=-45, in=-90, looseness=0.7] (5,3.5) to (7,3.5);

%3B
\fill[gray!20] (-7,-3) [out=90, in=135, looseness=2.3] to (-5,-3) [out=135, in=90, looseness=0.7] to (-7,-3);
\fill[gray!10] (-5,-3) [out=225, in=270, looseness=0.7] to (-7,-3) [out=90, in=135] to (-5,-3) ;

\draw [name path=C] (-7,-3) arc[x radius=2, y radius=2.5, start angle=180, end angle=360];
\draw [->>-][out=270, in=225, looseness=0.7] (-7, -3) to (-5,-3);
\draw [-<<-, dashed][out=45, in=90, looseness=0.7] (-5,-3) to (-3,-3);
\draw [-<-][out=90, in=135, looseness=2.3] (-7, -3) to (-5,-3);
\draw [->-][out=45, in=90, looseness=2.3] (-5,-3) to (-3,-3);
\draw [color=gray, dashed][out=90, in=135, looseness=0.7] (-7, -3) to (-5,-3);
\draw [color=gray][out=-45, in=-90, looseness=0.7] (-5, -3) to (-3,-3);

%4
\fill[gray!10] (0,-3) [out=225, in=270, looseness=0.7] to (-2,-3) [out=90, in=135] to (0,-3) ;
\fill[gray!20] (-2,-3) [out=90, in=180] to (0,-1.5)  -- (0, -3) [out=135, in=90, looseness=0.7] to (-2,-3);
\draw [out=90,in=360] (2,-3) to (0,-1.5);
\draw [out=180, in=90] (0, -1.5) to (-2,-3);
\draw [->>-][out=270, in=225, looseness=0.7] (-2, -3) to (0,-3);
\draw [dashed] [-<<-][out=45, in=90, looseness=0.7] (0, -3) to (2,-3);
%\draw (-5,-3) [->>-] arc[x radius=1, y radius=0.5, start angle=180, end angle=360];
\draw (-2,-3) arc[x radius=2, y radius=2.5, start angle=180, end angle=360];
\draw [->-](0,-3)-- (0,-1.5);
\draw [color=gray, dashed][out=90, in=135, looseness=0.7] (-2, -3) to (0,-3);
\draw [color=gray][out=-45, in=-90, looseness=0.7] (0, -3) to (2,-3);




%5
\draw [-<<-] (5,-3)-- (5,-1.5);
\draw (3,-3) arc[x radius=2, y radius=2.5, start angle=180, end angle=360];
\draw [out=90,in=360] (7,-3) to (5,-1.5);
\draw [out=180, in=90] (5, -1.5) to (3,-3);
\draw [color=gray][out=270, in=225, looseness=0.7] (3,-3) to (5,-3);
\draw [color=gray][dashed][out=45, in=90, looseness=0.7] (5,-3) to (7,-3);
\draw [color=gray][dashed][out=90, in=135, looseness=0.7] (3,-3) to (5,-3);
\draw [color=gray][out=-45, in=-90, looseness=0.7] (5,-3) to (7,-3);

\begin{scriptsize}
\fill [color=black] (-7,3) circle (1.5pt);
\fill [color=black] (-5,5) circle (1.5pt);
\fill [color=black] (-3,3) circle (1.5pt);
\fill [color=black] (-5,1) circle (1.5pt);
\fill [color=black] (-2,3.5) circle (1.5pt);
\fill [color=black] (3,3.5) circle (1.5pt);
\fill [color=black] (7,3.5) circle (1.5pt);
\fill [color=black] (5,3.5) circle (1.5pt);
\fill [color=black] (2,3.5) circle (1.5pt);
\fill [color=black] (0,4) circle (1.5pt);
\fill [color=black] (0,3) circle (1.5pt);
\fill [color=black] (2,-3) circle (1.5pt);
\fill [color=black] (-2,-3) circle (1.5pt);
\fill [color=black] (0,-3) circle (1.5pt);
\fill [color=black] (5,-1.5) circle (1.5pt);
\fill [color=black] (5,-3) circle (1.5pt);
\fill [color=black] (-7,-3) circle (1.5pt);
\fill [color=black] (-5,-3) circle (1.5pt);
\fill [color=black] (-3,-3) circle (1.5pt);
\end{scriptsize}
\end{tikzpicture}
\caption{Construcción del crosscap a partir de la representación pologonal del plano proyectivo. \label{fig:crosscap_paso_a_paso}}
\end{figure}

Parece claro que, además de $\Esfera$ y $\Proyectivo$, una superficie pueda tener varias presentaciones poligonales. Sea por ejemplo la presentación del toro $\Toro=\langle a,b\mid aba^{-1}b^{-1}\rangle$. Intuitivamente podemos ver que, subdividiendo los lados etiquetados con $b$ y reetiquetándolos con $c$ y $d$ (ver \autoref{fig:ejemplo_operacion}), la superficie que representa la representación obtenida $\langle a,c,d\mid  acda^{-1}c^{-1}d^{-1}\rangle$ será la misma.
Vamos ahora a desarrollar unas reglas generales de transformación.

\begin{figure}%%%%FIG: ejemplo operacion
\centering
\begin{tikzpicture}[use optics, x=1cm, y=1cm, scale=1.1]
\draw [->-={at=0.125}, ->-={at=0.375}, -<-={at=0.625}, -<-={at=0.875}](-1,1) -- (-1,-1) -- (-3,-1) -- (-3, 1) -- (-1,1); 
\draw [->-] (1,0)-- (1.57,-1);
\draw [-<-] (1.57,-1)-- (2.73,-1);
\draw [-<-] (2.73,-1)-- (3.31,0);
\draw [-<-] (3.31,0)-- (2.73,1);
\draw [->-] (2.73,1)-- (1.58,1);
\draw [->-] (1.58,1)-- (1,0);

\draw [<->] (-.1,0)--+(0.5,0);
\draw (-0.9,0) node[anchor= west] {$ b $};
\draw (-3.1,0) node[anchor= east] {$ b $};
\draw (-2,1.1) node[anchor= south] {$ a $};
\draw (-2,-1.1) node[anchor= north] {$ a $};
\draw (2.16,1.1) node[anchor= south] {$ a $};

\draw (3.02,0.5) node[anchor= south west] {$ c $};
\draw (3.02,-0.5) node[anchor= north west] {$ d $};

\draw (2.15, -1.1) node[anchor= north] {$ a $};

\draw (1.29,-0.5) node[anchor= north east] {$ d $};
\draw (1.29,0.5) node[anchor= south east] {$ c $};
\end{tikzpicture}

\caption{Dos representaciones equivalentes de $\Toro$ mediante la operación subdividir/consolidar. \label{fig:ejemplo_operacion}}
\end{figure}

\begin{defin}%%%%DEF: Realizaciones geometricas topologicamente equivalentes
Sean $\mathcal{P}_1$ y $\mathcal{P}_2$ dos representaciones tal que sus realizaciones geométricas son equivalentes. Entonces decimos que son \enfatiza{topológicamente equivalentes} y escribimos $\mathcal{P}_1 \approx \mathcal{P}_2$.
\end{defin}

\begin{figure}
\begin{subfigure}{0.5\textwidth}
\centering
\begin{tikzpicture}[use optics]
\draw [->-] (-2.15,0.84) -- (-2.71,0.47);
\draw [->-] (-2.71,0.47)--(-3.03,-.35);
\draw [-<-] (-3.03,-.35) -- (-1.67, -0.76);
\draw [-<-] (-1.67, -0.76) -- (-1.46,0.29);
\draw [->-] (-1.46,0.29) -- (-2.15,0.84);

\draw [->-] (2.15,0.84) -- (2.71,0.47);
\draw [->-] (2.71,0.47)--(3.03,-.35);
\draw [-<-] (3.03,-.35) -- (1.67, -0.76);
\draw [-<-] (1.67, -0.76) -- (1.46,0.29);
\draw [->-] (1.46,0.29) -- (2.15,0.84);

\draw (-2.42,0.66) node[anchor=south east] {$a$};
\draw (-2.87,0.06) node [anchor=east] {$b$};
\draw (-2.35,-0.56) node[anchor= north east] {$c$};
\draw (-1.56,-.24) node [anchor= west] {$d$};
\draw (-1.81,.57) node [anchor= south west] {$e$};
\draw (2.42,0.66) node[anchor=south west] {$a$};
\draw (2.87,0.06) node [anchor=west] {$b$};
\draw (2.35,-0.56) node[anchor= north west] {$c$};
\draw (1.56,-.24) node [anchor= east] {$d$};
\draw (1.81,.57) node [anchor= south east] {$e$};

\draw [<->] (-0.25,0)--+(.5,0);
\end{tikzpicture}
\caption{Reflejar.\label{fig:reflejar}}
\end{subfigure} 
\begin{subfigure}{.5\textwidth}
\centering
\begin{tikzpicture}[use optics]
\draw [->-] (-2.15,0.84) -- (-2.71,0.47);
\draw [->-] (-2.71,0.47)--(-3.03,-.35);
\draw [-<-] (-3.03,-.35) -- (-1.67, -0.76);
\draw [-<-] (-1.67, -0.76) -- (-1.46,0.29);
\draw [->-] (-1.46,0.29) -- (-2.15,0.84);

\draw [-<-](1.67,.76)--(1.46,-.29);
\draw [->-](1.46,-.29) -- (2.15,-.84);
\draw [->-](2.15,-.84)--(2.71,-.47);
\draw [->-](2.71,-.47)--(3.03,.35);
\draw [-<-](3.03,.35)--(1.67,.76);

\draw [<->] (-0.25,0)--+(.5,0);

\draw (-2.42,0.66) node[anchor=south east] {$a$};
\draw (-2.87,0.06) node [anchor=east] {$b$};
\draw (-2.35,-0.56) node[anchor= north east] {$c$};
\draw (-1.56,-.24) node [anchor= west] {$d$};
\draw (-1.81,.57) node [anchor= south west] {$e$};

\draw (2.43035, -0.65644) node[anchor=north west] {$a$};
\draw (2.86796, -0.05807) node [anchor=west] {$b$};
\draw (2.34997, 0.55816) node[anchor= south west] {$c$};
\draw (1.56406, 0.23665) node [anchor= east] {$d$};
\draw (1.80519, -0.56713) node [anchor= north east] {$e$};



\end{tikzpicture}
\caption{Rotar.\label{fig:rotar}}
\end{subfigure}
\end{figure}

Vamos a definir ahora unas operaciones elementales sobre las representaciones poligonales. Veremos luego que estas dan lugar a representaciones equivalentes.

\begin{defin}%%%%DEF: Operaciones elementales sobre representaciones
Las siguientes operaciones se llaman \enfatiza{transformaciones elementales} de una presentación poligonal:
\begin{itemize}
\item \enfatiza{Reetiquetar}: Cambiar todas las apariciones de un símbolo $a$ por otro símbolo que no está todavía en la representación, intercambiar todas las apariciones de dos símbolos $a$ y $b$ o intercambiar todas las apariciones de $a$ y $a^{-1}$.
\item \enfatiza{Subdividir} (\autoref{fig:ejemplo_operacion}): Cambiar todas las apariciones de $a$ por $ae$ y todas las de $a^{-1}$ por $e^{-1}a^{-1}$, donde $e$ es un símbolo que no está todavía en la presentación.
\item \enfatiza{Consolidar} (\autoref{fig:ejemplo_operacion}): Si $a$ y $b$ aparecen siempre de forma adyacente, intercambiar $ab$ por $a$ y $b^{-1}a^{-1}$ por $a^{-1}$, siempre que esto de lugar a una o más palabras de longitud al menos $3$ o una sola palabra de longitud $2$.
\item \enfatiza{Reflejar} (\autoref{fig:reflejar}): $$\langle S\mid   a_1 \dots a_m, W_2,\dots,W_k\rangle \mapsto \langle S\mid   a_m^{-1}\dots a_1^{-1}, W_2, \dots ,W_k\rangle .$$
\item \enfatiza{Rotar} (\autoref{fig:rotar}): $$\langle S\mid a_1a_2\dots a_m, W_2,\dots , W_k\rangle \mapsto \langle S\mid   a_2\dots a_ma_1, W_2,\dots , W_k\rangle .$$
\item \enfatiza{Cortar} (\autoref{fig:cortar}): Si $W_1$ y $W_2$ tienen longitud al menos $2$, $$\langle S\mid W_1W_2, W_3,\dots , W_k\rangle \mapsto \langle S\mid W_1e, e^{-1}W_2, W_3,\dots W_k\rangle .$$ 
\item \enfatiza{Pegar} (\autoref{fig:cortar}): $$\langle S,e\mid W_1e, e^-1W_2, W_3,\dots , W_k\rangle \mapsto \langle S\mid W_1W_2, W_3,\dots , W_k\rangle .$$
\item \enfatiza{Plegar} (\autoref{fig:plegar}): Si $w_1$ tiene longitud al menos $3$, $$\langle S,e\mid W_1ee^{-1}, W_2,\dots W_k\rangle \mapsto \langle S\mid W_1, W_2,\dots , W_k\rangle .$$ Permitimos que $W_1$ tenga longitud $2$, siempre que la representación tenga una sola palabra.
\item \enfatiza{Desplegar} (\autoref{fig:plegar}): $$\langle S\mid W_1, W_2,\dots , W_k\rangle \mapsto \langle S,e\mid W_1ee^{-1}, W_2,\dots , W_k\rangle .$$
\end{itemize}
\end{defin}


\begin{prop}%%%%PROP: Operaciones sobre representaciones
Las operaciones elementales sobre representaciones poligonales dan lugar a representaciones poligonales equivalentes.
\end{prop}
\begin{proof}
Los casos de reetiquetar, subdividir y rotar son cambios puramente formales que no afectan al espacio cociente. Los casos subdividir/consolidar, cortar/pegar y plegar/desplegar son inversos, con lo que basta probar uno de cada par. 
Empecemos con la operación de cortar. Sean $P_1, P_2$ dos regiones poligonales convexas etiquetadas por las palabras $W_1e$ y $e^{-1}W_2$ y sea $P'$ la region poligonal etiquetada por $W_1 W_2$. Supongamos de momento que no hay más palabras en las respectivas presentaciones. Si $\pi :P_1\amalg P_2\to M$ y $\pi' :P'\to M'$ representan respectivamente las aplicaciones cocientes. El segmento que va desde el vértice final de $W_1$ en $P'$ a su vértice inicial yace en $P'$ por convexidad, y lo etiquetamos con $e$. Por el \autoref{lema:celdas}, existe una aplicación continua $f:P_1\amalg P_2\to P'$ que lleva cada lado de $P_1$ o $P_2$ a su correspondiente en $P'$, y cuya restricción a cada $P_i$ es un homeomorfismo sobre su imagen. Por el \autoref{teo:aplicac_cerrada}, $f$ es una aplicación cociente. Dado que $f$ identifica únicamente los lados $e$ y $e^{-1}$, la aplicación cociente $\pi '\circ f$ y $\pi$ hacen exactamente las mismas identificaciones, y por tanto los espacios cociente $M$ y $M'$ son homeomorfos. 
En el caso de que hubiese otras palabras $W_3,\dots ,W_k$, simplemente extendemos $f$ como la identidad en sus respectivas presentaciones poligonales y procedemos como antes.

Para el plegado, ignoramos como antes palabras adicionales $W_2,\dots W_k$. Supongamos para empezar que $W_1=abcd$ tiene longitud exáctamente 4. Sea $P$ una región poligonal convexa con lados etiquetados por $abee^{-1}cd$. Si cortamos a lo largo de el segmento $f$ que une el vértice inicial de $a$ con el vértice final de $e$ y obtenemos así $abef$ y $f^{-1}e^{-1}cd$. Consolidamos $ef$ en $h$, obteniendo así $ah^{-1}$ y $bch$. Pegamos por $h$ y obtenemos así $abcd$. ((Lo he cambiado para evitar hablar de simplex que los introduzco más tarde. Vale igualmente?))
Si $W_1$ tiene longitud 2 o 3, podemos subdividir para alargar la longitud de la palabra a 4, plegar, y finalmente consolidar de nuevo.
\end{proof}

\begin{figure}
\begin{subfigure}{0.5\textwidth}
\centering
\begin{tikzpicture}[use optics, scale=0.3]
\draw [<->] (-.5,0)--(.5,0);
\draw (-4.79,4.04)--(-6.79,3.04)--(-6.79,-2.96)--(-4.79,-3.96)--(-3,0)--cycle;
\draw[dashed] (-4.79,-3.96)--(-4.79, 4.04);
\draw[->-={at=0.775}] (5,4)--(3,3)--(3,-3)--(5,-4)-- cycle; 
\draw[-<-={at=0.249}] (7,4)--(7,-4)--(8.79,-0) --cycle;
\draw (-6.79,0) node[anchor=east] {$W_1$}; 
\draw (-3,0) node[anchor=west]{$W_2$};
\draw (3,0) node[anchor=east] {$W_1$};
\draw (5,0.8) node[anchor=west] {$e$};
\draw (7,-1) node[anchor=east] {$e$};
\draw (8.79,0) node[anchor=west] {$W_2$};
\end{tikzpicture}
\caption{Cortar/Pegar\label{fig:cortar}}
\end{subfigure}
\begin{subfigure}{0.5\textwidth}
\centering
\begin{tikzpicture}[use optics, scale=1.2]
\draw [->-](0.5, 1.53884)--(-0.30902, 0.95106);
\draw [->-](-0.30902, 0.95106)--(0,0);
\draw [->-](0,0)--(1,0);
\draw [->-](1,0)--(1.30902, 0.95106);
\draw [-<-](1.30902, 0.95106)--(0.5, 1.53884);
\draw [->-](3.11256, 0)--(4.88896, 0.000);
\draw [->-](4.88896, 0.000)--(4, 1.53884);
\draw [->-](4, 1.53884)--(3.11256, 0);
\draw [->-](4,1.54)--(4,.5);
\draw (0.1,1.24) node[anchor=south east] {$a$};
\draw (-.15, 0.48) node[anchor=north east] {$b$};
\draw (.5,0) node[anchor=north] {$c$};
\draw (1.15,0.48) node [anchor=north west] {$e$};
\draw (.9, 1.24) node [anchor=south west] {$e$};
\draw (3.56, 0.77) node[anchor=south east] {$a$};
\draw (4,0) node [anchor=north] {$b$};
\draw (4.44,.77) node[anchor=south west] {$c$};
\draw (4,.8) node[anchor=west] {$e$};
\filldraw (4,.5) circle (1pt);
\filldraw (1.30902, 0.95106) circle (1pt);
\end{tikzpicture}
\caption{Plegar/Desplegar\label{fig:plegar}}
\end{subfigure}
\end{figure}





\begin{prop}%%%%PROP: representación suma conexa
Sean $M_1$ y $M_2$ superficies que admiten respectivamente representaciones $\langle S_1\mid W_1\rangle $ y $\langle S_2\mid W_2\rangle $, donde $S_1$ y $S_2$ son conjuntos disjuntos y tal que cada presentación tiene una sola cara. Entonces $\langle S_1,S_2\mid W_1W_2\rangle$ es una presentación de la suma conexa $M_1 \# M_2$.
\end{prop}

\begin{proof}
Consideramos la representación $\mathcal{P}_1=\langle S_1,a,b,c\mid W_1c^{-1}b^{-1}a^{-1},abc\rangle$ (\autoref{fig:suma_conexa_rep_pol_a}). Pegando a lo largo de $a$ y doblando dos veces, tenemos que $\mathcal{P}_1$ es una representación equivalente a $\langle S_1\mid W_1\rangle$ y por tanto es una representación de $M_1$. Llamemos $B_1$ a la imagen en $M_1$ del interior de la región poligonal cuyo borde es $abc$. Veamos que existe un entorno $B'$ de $B_1$ en $M_1$ y un homeomorfismo $\phi :B'\to \R^2$ que manda $B_1$ a $\mathbb{B}^2$ y $\overline{B}$ a $\overline{\mathbb{B}}^2$. Una forma de verlo lo sugiere la \autoref{fig:suma_conexa_rep_pol}: sean $P_1$, $Q$ y $P_1'$ regiones poligonales convexas etiquetadas respectivamente por las palabras $W_1c^{-1}b^{-1}a^{-1}$, $abc$ y $W_1$. Si triangulamos las regiones poligonales como en la  \autoref{fig:suma_conexa_rep_pol}, obtenemos una aplicación simplicial $f:P_1\amalg Q\to P_1'$ que lleva $Q$ a el triángulo $Q'\subseteq P_1'$, que comparte una vértice $v$ con $P_1'$. La composición $P_1\amalg Q\to P_1'\to M_1$ respeta las identificaciones hechas por la aplicación cociente $P_1'\to M_1$, por lo que se tiene un homeomorfismo de $M_1$ llevando $B_1$ a la imagen de $Q'$.\\
Ahora, fijémonos en la demostración de la \autoref{prop:poligonos} (b). Cuando construimos un entorno Euclídeo de un vértice, ensamblamos ``wedges'' a los varios vértices en un disco coordenado. Aplicando esa construcción al vértice $v$, llevamos $Q'$ a un conjunto que es homeomorfo a un disco cerrado en el plano (\autoref{fig:disco_regular}), y  entonces extendemos ese homeomorfismo a un disco abierto que lo contiene.
\end{proof}


%%%%%
%%%%
%%%%
%%%%%




\section{Demostración clásica}

\chapter{La prueba ZIP de Conway}

\section{Cremalleras}

\begin{defin}%M%%%DEF: perforación
\label{def:perforacion}
Sea $S$ una superficie (con o sin borde). Sea $p\in \interior (S)$ y $U$ un entorno abierto de $p$ en $S$. Sea $\phi :U\to \R^n$ un homeomorfismo tal que $\phi (p)=0$. Sea $B=\phi ^{-1}(B_1(0))$. Decimos que la nueva superficie con borde $S^o=S\setminus B$ es $S$ \enfatiza{$1$-perforada}, y a $\partial B\subset S^o$ la llamamos \enfatiza{perforación}. Podemos repetir el proceso sobre $S^o$ sucesivamente, obteniendo $S$ $n$-perforada con un número finito $n\in \N$ de perforaciones.
\end{defin}

\begin{obs}%M%%%OBS: Disco como esfera perforada
Dado que los dos espacios son homeomorfos, podemos visualizar el disco cerrado $\overline{\mathbb{B}}^2$ como una esfera con una perforación (\autoref{fig:disco_esfera_perforada}).
\end{obs}
 

\begin{figure}[h]%%%%FIG: Disco esfera perforada
\centering
\begin{tikzpicture}[line cap=round,line join=round,>=triangle 45,x=1.0cm,y=1.0cm]
\fill[gray!20](3,0) circle (2cm);
\fill[color=gray!20][rotate around={-14.04:(-1.34,0.02)}] (-1.34,0.02) ellipse (0.32cm and 0.24cm);
\fill[color=white][rotate around={-14.04:(3.33,1.31)}] (3.33,1.31) ellipse (0.32cm and 0.24cm);
\draw[rotate around={-14.04:(-1.34,0.02)}] (-1.34,0.02) ellipse (0.32cm and 0.24cm);
\draw[rotate around={-14.04:(3.33,1.31)}] (3.33,1.31) ellipse (0.32cm and 0.24cm);
\draw(3,0) circle (2cm);
\draw(1,0) arc[x radius=2, y radius=0.5, start angle=180, end angle=360];
\draw[dashed](5,0) arc[x radius=2, y radius=0.5, start angle=0, end angle=180];
\draw(0,0) node[anchor=north] {$ \approx $};
%\draw [out=-14.04, in=0, looseness=2] (-1.28,0.26) to (-1.5,-2);
%\draw [out=180, in=165.96, looseness=4] (-1.5,-2) to (-1.28,0.26);
\end{tikzpicture}
\caption{El disco cerrado como una esfera perforada.\label{fig:disco_esfera_perforada}}
\end{figure}


Conway utiliza las cremalleras (\textit{zips} en inglés) para describir cómo actúan las identificaciones topológicas. Cada cremallera actúa sobre una o dos perforaciones de una superficie. Están formadas por dos \textit{zips} (dos partes dentadas) fijadas la/s perforación/es y un \textit{zipper} (el deslizador). Al cerrar el \textit{zipper}, las \textit{zips} se juntan identificándose. Trato de dar una definición rigurosa:


\begin{defin}%%%DEF: ZIP
Sea $S$ una superficie compacta. Una \enfatiza{cremallera} es una identificación entre dos subconjuntos (cerrados??) de la frontera de $S$. A este par lo llamamos \enfatiza{par-zip}.
\end{defin}


En la \textit{prueba ZIP}, Conway nos explica gráficamente las posibles formas de unir cremalleras. 

\begin{defin}
Sea $S$ una superficie. Definimos cuatro formas elementales de identificar pares-\textit{zip} en perforaciones de 
$S$ $1$ o $2$-perforada:

\begin{itemize}
\item[1.] \enfatiza{Cap}: Los pares zip yacen cada uno sobre la mitad de una misma perforación con orientaciones opuestas (\autoref{fig:cap}).
\item[2.] \enfatiza{Crosscap}: Los pares zip yacen cada uno sobre la mitad de una misma perforación con la misma orientación (\autoref{fig:crosscap}).
\item[3.] \enfatiza{Handle}: Los pares zip yacen cada uno sobre una perforación distinta de S con orientaciones opuestas (\autoref{fig:handle}).
\item[4.] \enfatiza{Crosshandle}: Los pares zip yacen cada uno sobre una perforación distinta de S con la misma orientación (\autoref{fig:crosshandle}).
\end{itemize}
\end{defin}

Sea $S$ una superficie conexa tal que admite una representación poligonal de una sola cara $P=\langle A\mid W\rangle$, y sea $|\mathcal{P}|$ su realización geométrica. Sea $\partial B$ una perforación sobre $\Esfera$, y sea $\phi$ un homeomorfismo entre los lados de $|\mathcal{P}|$ y $\partial B$. Si identificamos ahora los pares de segmentos sobre $\partial B$ de la imagen de $\phi$, obtenemos la suma conexa $S\# \Esfera$, es decir, $S$ (\autoref{obs:suma_esfera}). 

Sean ahora $S$ y $S'$ superficies conexas. Entonces, hacer una perforación sobre $S'$ es lo mismo que hacer la suma conexa de una esfera $\Esfera$ con una perforación $S'$. Por tanto, hacer una perforación sobre $S'$ asociada a $S$ da lugar a la suma conexa $S\# S'$.
 

\begin{prop}
Sea $S$ una superficie. Los siguientes espacios son homeomorfos:
\begin{itemize}
\item[a)] $S$ con un cap y $S$.
\item[b)] $S$ con un crosscap y $S\# \Proyectivo$.
\item[c)] $S$ con un handle y $S\# \Toro$.
\item[d)] $S$ con un crosshandle y $S\# K$ (siendo $K$ la botella de Klein). 
\end{itemize}
\end{prop}
\begin{proof}
a) y b) son consecuencia directa del parrafo que precede la preposición. Para c), utilizamos la \autoref{obs:toro_asa}, y para d) utilizar una construcción parecida a c) ((habría que especificar más?)).
\end{proof}





\begin{figure}

\begin{subfigure}{1\textwidth} %%%%FIG: Cap
\centering
\begin{tikzpicture} [use optics]
%\draw [help lines, step=1mm](0,-2) grid (6,2);
\fill [color=gray!10] (-3,-1.5) -- (-5,0) -- (-3,1) -- (-1,0) -- cycle;
\fill [gray!10](3,-1.5) -- (5,0) -- (3,1) -- (1,0) -- cycle;
\fill [gray!20] (2,-.1) [out=80, in=180] to (3,0.7) [out=0, in=100] to (4,-.1) arc[x radius=1, y radius=0.5, start angle=0, end angle=-180];
\fill [white] (-4,-.1) arc[x radius=1, y radius=0.5, start angle=180, end angle=-180];
\draw (-3,-1.5) -- (-5,0) -- (-3,1) -- (-1,0) -- cycle;
\draw (3,-1.5) -- (5,0) -- (3,1) -- (1,0) -- cycle;
\draw [-<-={at=0.1}, ->-={at=0.6}] (-4,-.1) arc[x radius=1, y radius=0.5, start angle=180, end angle=-180];
\draw [dashed](2,-.1) arc[x radius=1, y radius=0.5, start angle=180, end angle=0];
\draw (2,-.1) arc[x radius=1, y radius=0.5, start angle=180, end angle=360];
\draw[-<-={at=0.48}] [dash pattern= on 50pt off 2pt on 2pt off 2pt on 2pt off 2pt on 2pt off 2pt on 2pt off 2pt] (2.4,-.5) [out=90, in=180] to (3.1,.685) [out=0, in=90] to (3.6,.3);
\draw (2,-.1) [out=80, in=180] to (3,0.7) [out=0, in=100] to (4,-.1);
%\draw (2.4,-.5)-- (3.6,.3);

\fill [color=black] (-3.6,-.5) circle (1.pt);
\fill [color=black] (-2.4,.3) circle (1.pt);
\fill [color=black] (3.6,.3) circle (1.pt);
\fill [color=black] (2.4,-.5) circle (1.pt);

\end{tikzpicture}
\caption{Construcción del \textit{cap}.\label{fig:cap}}
\end{subfigure}



\begin{subfigure}{1\textwidth} %%%%FIG: Crosscap
\centering
\begin{tikzpicture} [use optics]
%\draw [help lines, step=1mm](0,-2) grid (6,2);
\fill [color=gray!10] (-3,-1.5) -- (-5,0) -- (-3,1) -- (-1,0) -- cycle;
\fill [gray!10](3,-1.5) -- (5,0) -- (3,1) -- (1,0) -- cycle;
\fill [white] (-4,-.1) arc[x radius=1, y radius=0.5, start angle=180, end angle=-180];
\fill [gray!20](2,-.1) [out=90,in=180] to (3.1,2)[out=0, in=90] to (4,-.1)  arc[x radius=1, y radius=0.5, start angle=0, end angle=-180];
\draw (-3,-1.5) -- (-5,0) -- (-3,1) -- (-1,0) -- cycle;
\draw[dash pattern= on 103pt off 2pt on 2pt off 2pt on 2pt off 2pt on 2pt off 2pt on 2pt off 2pt on 2pt off 2pt on 2pt off 2pt on 2pt off 2pt on 2pt off 2pt on 2pt off 2pt on 2pt off 2pt on 2pt off 2pt on 2pt off 2pt on 2pt off 2pt  on 2pt off 2pt  on 2pt off 3pt  on 150pt] (3,-1.5) -- (5,0) -- (3,1) -- (1,0);
\draw (1,0) -- (3,-1.5);
\draw [-<-={at=0.4}, -<-={at=0.9}] (-4,-.1) arc[x radius=1, y radius=0.5, start angle=180, end angle=-180];
\draw [dashed](2,-.1) arc[x radius=1, y radius=0.5, start angle=180, end angle=0];
\draw (2,-.1) arc[x radius=1, y radius=0.5, start angle=180, end angle=360];
\draw (2,-.1) [out=90,in=180] to (3.1,2)[out=0, in=90] to (4,-.1);
\draw (3.1,2) -- (3.1,0.7);

\fill [color=black] (-3.6,.3) circle (1.pt);
\fill [color=black] (-2.4,-.5) circle (1.pt);

%\fill [color=black] (-4,-.1) circle (1.pt);
%\fill [color=black] (-2,-.1) circle (1.pt);

%\fill [color=black] (-3.6,-.5) circle (1.pt);
%\fill [color=black] (-2.4,.3) circle (1.pt);


\draw (2.17,1.15) [out=-80, in=-135, looseness=0.6] to (3.1,1.2) [out=-45, in=-90, looseness=0.6] to (3.9,1.2);
\draw [dashed] (2.17,1.15) [out=90, in=135, looseness=0.6] to (3.1,1.2) [out=45, in=90, looseness=0.6] to (3.9,1.2);
\end{tikzpicture}
\caption{Construcción del \textit{crosscap}.\label{fig:crosscap}}
\end{subfigure}


\begin{subfigure}{1\textwidth} %%%%FIG: Handle
\centering

\begin{tikzpicture} [use optics]
%\draw [help lines, step=1mm](-6,-2) grid (6,2);

\fill [gray!10] (-3,-1.5) -- (-5,0) -- (-3,1) -- (-1,0) -- cycle;
\fill [white](-4.4,-.05) arc[x radius=0.5, y radius=0.25, start angle=180, end angle=540];
\fill [white](-1.6,-.05) arc [x radius=0.5, y radius=0.25, start angle=0, end angle=360];
\fill [gray!10](3,-1.5) -- (5,0) -- (3,1) -- (1,0) -- cycle;
\fill [gray!20](1.6,-.05) [out=90, in=180] to (3,2) [out=0, in=90] to (4.4,-.05) arc [x radius=0.5, y radius=0.25, start angle=0, end angle=-180]  [out=90, in=0] to (3,1.2) [out=180, in=90] to (2.6,-.05) arc [x radius=0.5, y radius=0.25, start angle=0, end angle=-180] ;

\draw (-3,-1.5) -- (-5,0) -- (-3,1) -- (-1,0) -- cycle;
\draw [dash pattern= on 19pt off 2 pt on 2pt off 2pt on 2pt off 2pt on 2pt off 2pt on 2pt off 2pt on 2pt off 2pt  on 2pt off 2pt on 2pt off 2pt  on 1pt off 1pt on 24pt](5,0) -- (3,1);
\draw [dash pattern= on 12pt off 2pt on 2pt off 2pt on 2pt off 2pt on 2pt off 2pt on 2pt off 2pt on 2pt off 2pt on 2pt off 2pt on 2pt off 2pt on 50pt ](3,1) -- (1,0);
\draw (1,0) -- (3,-1.5) -- (5,0);
\draw [-<-={at=0}](-4.4,-.05) arc[x radius=0.5, y radius=0.25, start angle=180, end angle=540];
\draw [->-={at=0}](-1.6,-.05) arc [x radius=0.5, y radius=0.25, start angle=0, end angle=360];

\draw (1.6,-.05) arc[x radius=0.5, y radius=0.25, start angle=180, end angle=360];
\draw  (4.4,-.05)arc [x radius=0.5, y radius=0.25, start angle=0, end angle=-180];
\draw [dashed](1.6,-.05) arc[x radius=0.5, y radius=0.25, start angle=180, end angle=0];
\draw  [dashed](4.4,-.05)arc [x radius=0.5, y radius=0.25, start angle=0, end angle=180];

\draw (1.6,-.05) [out=90, in=180] to (3,2) [out=0, in=90] to (4.4,-.05);
\draw (2.6,-.05) [out=90, in=180] to (3,1.2) [out=0, in=90]to (3.4,-.05);
\draw [->-](3,1.2) [out=180, in=180, looseness=0.7] to (3,2);
\draw [dashed] (3,1.2)[out=0, in=0, looseness=0.7] to (3,2);
\fill (-3.4,-.05) circle (1pt);
\fill (-2.6,-.05) circle (1pt);
\end{tikzpicture}

\caption{Construcción del \textit{handle}. \label{fig:handle}}
\end{subfigure}


\begin{subfigure}{1\textwidth} 
\centering
\begin{tikzpicture}[use optics]
%\draw [help lines, step=1mm](-6,-2) grid (6,2);

\fill [gray!10] (-3,-1.5) -- (-5,0) -- (-3,1) -- (-1,0) -- cycle;
\fill [white](-4.4,-.05) arc[x radius=0.5, y radius=0.25, start angle=180, end angle=540];
\fill [white](-1.6,-.05) arc [x radius=0.5, y radius=0.25, start angle=0, end angle=360];
\fill [gray!10](3,-1.5) -- (5,0) -- (3,1) -- (1,0) -- cycle;
\fill [gray!20](1.6,-.05) [out=90, in=180] to (3,2) [out=0, in=90] to (4.4,-.05) arc [x radius=0.5, y radius=0.25, start angle=0, end angle=-180]  [out=90, in=0] to (3,1.2) [out=180, in=90] to (2.6,-.05) arc [x radius=0.5, y radius=0.25, start angle=0, end angle=-180] ;

\draw (-3,-1.5) -- (-5,0) -- (-3,1) -- (-1,0) -- cycle;
\draw [dash pattern= on 19pt off 2 pt on 2pt off 2pt on 2pt off 2pt on 2pt off 2pt on 2pt off 2pt on 2pt off 2pt  on 2pt off 2pt on 2pt off 2pt  on 1pt off 1pt on 24pt](5,0) -- (3,1);
\draw [dash pattern= on 12pt off 2pt on 2pt off 2pt on 2pt off 2pt on 2pt off 2pt on 2pt off 2pt on 2pt off 2pt on 2pt off 2pt on 2pt off 2pt on 50pt ](3,1) -- (1,0);
\draw (1,0) -- (3,-1.5) -- (5,0);
\draw [-<-={at=0}](-4.4,-.05) arc[x radius=0.5, y radius=0.25, start angle=180, end angle=540];
\draw [-<-={at=0}](-1.6,-.05) arc [x radius=0.5, y radius=0.25, start angle=0, end angle=360];

\draw (1.6,-.05) arc[x radius=0.5, y radius=0.25, start angle=180, end angle=360];
\draw  (4.4,-.05)arc [x radius=0.5, y radius=0.25, start angle=0, end angle=-180];
\draw [dashed](1.6,-.05) arc[x radius=0.5, y radius=0.25, start angle=180, end angle=0];
\draw  [dashed](4.4,-.05)arc [x radius=0.5, y radius=0.25, start angle=0, end angle=180];

\draw (2.1,1.56) [out=-90, in=-135, looseness=0.5] to (3, 1.56) [out=-45, in=-90, looseness=0.5] to (3.9, 1.56);
\draw [dashed] (2.1,1.56) [out=90, in=135, looseness=0.5] to (3, 1.56) [out=45, in=90, looseness=0.5] to (3.9, 1.56);
\draw (1.6,-.05) [out=90, in=180] to (3,2) [out=0, in=90] to (4.4,-.05);
\draw (2.6,-.05) [out=90, in=180] to (3,1.2) [out=0, in=90]to (3.4,-.05);
%\draw (3,1.2) [out=180, in=180, looseness=0.2] to (3,2);
\draw (3,1.2)-- (3,2);
%\draw [dashed] (3,1.2)[out=0, in=0, looseness=0.7] to (3,2);
\fill (-3.4,-.05) circle (1pt);
\fill (-2.6,-.05) circle (1pt);

\end{tikzpicture}
\caption{Construcción del \textit{crosshandle}.\label{fig:crosshandle}}
\end{subfigure}
\caption{Construcciones elementales con cremalleras.}
\end{figure}



\section{Representación de superficies con cremalleras}
Veamos un método para representar las perforaciones y cremalleras sobre colecciones de esferas, y definamos unas operaciones sobre estas representaciones que nos den espacios topológicos equivalentes, tal como hicimos en el Capítulo 1. Formalizo la idea dada por Wildberger \cite{wildberger}.

Sean $\Esfera_1,\dots ,\Esfera_k$ esferas con perforaciones, algunas con pares-zip. Para cada esfera, etiquetamos las cremalleras asociadas a un mismo par-zip con el mismo símbolo. Para las partes de perforaciones que no tengan cremallera, las etiquetamos cada una con un símbolo distinto. A cada perforación le asociaremos una palabra, formada a partir de las etiquetas que acabamos de crear. Empezamos arbitrariamente por una, digamos $a$, y si esta está asociada a una cremallera, escribiremos $a$ o $a^{-1}$ dependiendo de si su orientación se corresponde con la de las agujas del reloj; si no está asociada a una cremallera escribiremos simplemente $a$. Recorreremos la perforación en orden en sentido contrario de las agujas del reloj reptiendo el proceso anterior hasta llegar a la anterior etiqueta de la que hemos partido.
Cada esfera la representaremos por la yuxtaposición de las palabras entre paréntesis asociadas a las perforaciones, y una usperficie formada por varias esferas la escribiremos como la suma de sus representaciones.

\begin{eje}\label{eje:rep_perf}
Sea una esfera perforada como la de la izquierda de la \autoref{fig:rep_perf}. Hay dos pares-zip, uno de ellos con las cremalleras en la misma perforación y el otro en perforaciones distintas. Etiquetamos las primeras por $a$ y las segundas por $b$ y etiquetamos el resto de segmentos de las perforaciones sin cremallera por $x,y,z$. Por tanto, una representación de nuestra esfera sería $$(a^{-1}xab)(by)(z)$$
\end{eje}

\begin{figure}%%%%FIG: Representación esferas
\centering
\begin{tikzpicture}[use optics, line cap=round,line join=round,>=triangle 45,x=1.0cm,y=1.0cm, scale=0.8]

\fill[gray!10](-3,0) circle (3cm);
\fill[gray!10] (5,0) circle (3cm);

\fill[white][rotate around={-41.01:(-1.54,1.6)}](-1.54,1.6) ellipse (0.8cm and 0.51cm);
\fill[white](-4.14,-1.36) circle (0.4cm);
\fill[white][rotate around={33.69:(-1.79,-1.86)}] (-1.79,-1.86) ellipse (0.4cm and 0.31cm);
\fill[white][rotate around={-41.01:(6.46,1.6)}] (6.46,1.6) ellipse (0.8cm and 0.51cm);
\fill[white](3.86,-1.36) circle (0.4cm);
\fill[white][rotate around={33.69:(6.21,-1.86)}] (6.21,-1.86) ellipse (0.4cm and 0.31cm);

\draw (0,0) arc[x radius=3, y radius=.9, start angle=0, end angle=-180] (8,0) arc[x radius=3, y radius=.9, start angle=0, end angle=-180] ;
\draw [dashed] (0,0) arc[x radius=3, y radius=.9, start angle=0, end angle=180] (8,0) arc[x radius=3, y radius=0.9, start angle=0, end angle=180] ;

\draw(-3,0) circle (3cm);
\draw [rotate around={-41.01:(-1.54,1.6)}, -<<-={at=0.39}, ->-={at=0.2}, ->>-={at=0.9}] (-1.54,1.6) ellipse (0.8cm and 0.51cm);
\draw  [->-={at=0.45}](-4.14,-1.36) circle (0.4cm);
\draw [rotate around={33.69:(-1.79,-1.86)}] (-1.79,-1.86) ellipse (0.4cm and 0.31cm);

\draw (5,0) circle (3cm);
\draw [rotate around={-41.01:(6.46,1.6)}, -<-={at=0.39}, ->-={at=0.2}, ->-={at=0.9}, ->-={at=0.6}] (6.46,1.6) ellipse (0.8cm and 0.51cm);
\draw[->-={at=0.45}, ->-={at=0.91}](3.86,-1.36) circle (0.4cm);
\draw [rotate around={33.69:(6.21,-1.86)}, ->-={at=0.8}] (6.21,-1.86) ellipse (0.4cm and 0.31cm);

\fill [color=black] (-2.1,2.17) circle (1.5pt);
\fill [color=black] (-0.86,1.45) circle (1.5pt);
\fill [color=black] (-1.31,2.07) circle (1.5pt);
\fill [color=black] (-1.98,1.31) circle (1.5pt);
\fill [color=black] (-4.35,-1.7) circle (1.5pt);
\fill [color=black] (-3.99,-0.98) circle (1.5pt);
\fill [color=black] (5.9,2.17) circle (1.5pt);
\fill [color=black] (7.14,1.45) circle (1.5pt);
\fill [color=black] (6.69,2.07) circle (1.5pt);
\fill [color=black] (6.02,1.31) circle (1.5pt);
\fill [color=black] (3.65,-1.7) circle (1.5pt);
\fill [color=black] (4.01,-0.98) circle (1.5pt);
\fill [color=black] (5.97,-1.67) circle (1.5pt);

\draw[anchor=east] (3.47,-1.29) node {$ b $};
\draw[anchor=south west] (6.97,1.71) node {$ b $};
\draw[anchor=north] (6.68,0.89) node {$ a $};
\draw[anchor=south] (6.29,2.24) node {$ a $};
\draw[anchor=east] (5.79,1.72) node {$ x $};
\draw[anchor=north west] (4.23,-1.51) node {$ y $};
\draw[anchor=north west] (6.46,-2.05) node {$ z $};


\end{tikzpicture}
\caption{Representación de una esfera con perforaciones y cremalleras.	 \label{fig:rep_perf}}
\end{figure}


\section{Teorema de Clasificación}

\begin{defin}%%%%DEF: Ordinaria
Una superficie se dice ordinaria si es homeomorfa a una colección finita de esferas cada una con un número finito de \textit{handles}, \textit{crosshandles}, \textit{crosscaps} y perforaciones.
\end{defin}


\begin{lema}%%%%LEMA: Superficie ordinaria zips
Sea $S$ una superficie con borde con un par-zip tal que cada cremallera está en una parte de su borde. Entonces, si $S$ es ordinaria antes de identificar las cremalleras, es ordinaria también después.\label{lema:superficie_ordinaria}
\end{lema}
\begin{proof}
Consideramos el caso en que las dos cremalleras ocupan cada una una perforación en su totailidad. Entonces al identificarlas se tiene un \textit{handle} (\autoref{fig:handle}) o un \textit{crosshandle} (\autoref{fig:crosshandle}), dependiendo de sus respectivas orientaciones. Si las dos perforaciones pertenecen a componentes conexas distintas de $S$, entonces identificando obtenemos el espacio adjunción de las dos componentes. (MEJORAR). 

Consideramos ahora el caso en el que las dos cremalleras yacen sobre la misma perforación y la cubren totalmente. Identificándolas nos da o bien un \textit{cap} (\autoref{fig:cap}) o bien un \textit{crosscap} (\autoref{fig:crosscap}), dependiendo de sus respectivas orientaciones.

Finalmente, consideramos los varios casos en que las cremalleras no ocupan perforaciones en su totalidad. (A PARTIR DE AQUI NO SE MUY BIEN COMO ORIENTARLO... CON OPERACIONES ELEMENTALES O COMO HACE EL???) 
\end{proof}

\begin{tma}[Teorema de clasificación, versión preeliminar] %M%%%TEO: Clasificación preeliminar
Toda superficie compacta es ordinaria.
\end{tma}
\begin{proof}
Sea $S$ una superficie compacta. Sabemos, por el \nameref{teo:rado}, que $S$ está triangulada por un poliedro $|K|$ asociado a un complejo simplicial $K$ tal que cada 1-símplice que contiene puntos interiores de $S$ es una cara de exáctamente dos 2-símplices, y cada 1-símplice que contiene puntos del borde de $S$ es cara de exáctamente un 2-símplice. Si sobre los primeros 1-símplices ponemos una cremallera distinta, en los 2-símplices habrá algunos 1-símplices que se identifiquen. Llamemos $K_2=\left\{\sigma_1,\dots ,\sigma_j\right\}$, donde cada $\sigma_i \text{ es un 2-símplice para todo } i=1\dots ,j$. $K_2$ es una superficie ordinaria, pues cada $\sigma_i$ es homeomorfo a una esfera perforada. Si identificamos ahora las cremalleras una a una, por el \autoref{lema:superficie_ordinaria} y por inducción, la superficie resultante es ordinaria.
\end{proof}



%\begin{prop}
%Si $S$ es una superficie compacta y conexa, $S$ admite una representación de una sola cara.
%\end{prop}


%A la perforación $\partial B_i$ la llamamos \enfatiza{perforación} asociada a $W_i$, y a la imagen por $\phi _i$ de cada par de segmentos $a$ la llamamos \enfatiza{cremallera} asociada a $a\in W_i$.


%\begin{defin}%M%%%DEF: zips asociadas a una representación poligonal de una superficie
%Sea $S$ una superficie. Sea $|\mathcal{P}|$ la realización geométrica de una representación poligonal $P=\langle A\mid W_1,\dots W_n\rangle$ de la superficie. Para cada $W_i$, con $i=1,\dots n$, sea $\partial B_i$ una perforación sobre una esfera $\Esfera_i$, y sea $\phi _i$ un homeomorfismo entre los lados de $|\mathcal{P}|$ y $\partial B_i$. A la perforación $\partial B_i$ la llamamos \enfatiza{perforación} asociada a $W_i$, y a la imagen de $\phi _i$ la llamamos \enfatiza{cremallera} asociada a $W_i$.
%\end{defin}









\appendix
\chapter{Algunos teoremas de topología importantes}

\begin{tma}[Lema de la aplicación cerrada]
\label{teo:aplicac_cerrada}
Sea $F$ una aplicación continua de un espacio topológico compacto en un espacio topológico Hausdorff. Entonces:
\begin{itemize}
\item[(a)]$F$ es una aplicación cerrada.
\item[(b)] Si $F$ es sobreyectiva, entonces es una aplicación cociente.
\item[(c)] Si $F$ es inyectiva, entonces es una inmersión topológica. %?????
\item[(d)] Si $F$ es biyectiva, entonces es un homeomorfismo.
\end{itemize}
\end{tma}

\begin{tma}[Unicidad de espacios cociente]
\label{teo:unicidad_espacio_cociente}
Supongamos $q_1:X\to Y_1$ y $q_2:X\to Y_2$ son aplicaciones cociente que hacen las mismas identificaciones, es decir, tales que $q_1(x)=q_1(x')$ si y solo si $q_2(x)=q_2(x')$. Entonces existe un único homeomorfismo $\phi:Y_1\to Y_2$ tal que $\phi \circ q_1=q_2$.
\end{tma}

\begin{tma}[Invarianza del dominio]\label{teo:inv_dom}
Sean $U\subset \R^n$ y $V\subset \R^m$ abiertos homeomorfos. Entonces $n=m$.
\end{tma}


\chapter{CW-complejos}\label{sec:CW}
Utilizo la definición inductiva de CW-complejo dada por Hatcher \cite{hatcher} y algunas propiedades expuestas por Lee \cite{lee1}.


\begin{defin}
Una \enfatiza{n-celda abierta} es un espacio topológico homeomorfo a la bola abierta unidad $\mathbb{B}^n$, y una \enfatiza{n-celda} cerrada es un espacio homemorfo a $\overline{\mathbb{B}}^n$. 
\end{defin}

Toda bola abierta o cerrada en $\R^n$ es claramente una $n$-celda. El siguiente teorema nos proporciona más ejemplos:

\begin{prop}% TEOREMA 5.1 LEE%%%% CONVEXO HOMEOM ESFERA
\label{teo:convexo_homeom_esfera} 
Si $D\subseteq \R^n$ es un conjunto compacto y convexo con interior no vacío, entonces $D$ es una $n$-celda cerrada y su interior es una $n$-celda abierta. De hecho, dado $p\in \mathring{D}$, entonces existe un homeomorfismo $F:\overline{\mathbb{B}}^n\to D$ que envía $0$ a $p$, $\overline{\mathbb{B}}^n$ a $\mathring{D}$, y $\mathbb{S}^{n-1}$ a $\partial D$.
\end{prop}
\begin{proof}
Sea $p\in D$ un punto de su interior. Si reemplazamos $D$ por su imagen mediante la traslación $x\mapsto x-p$, que es un homeomorfismo de $\R^n$ en sí mismo, podemos asumir que $p=0\in \mathring{D}$. Entonces existe un $\varepsilon >0 $ tal que la bola $B_{\varepsilon}(0)$ está contenida en $D$. Usando la dilatación $x\mapsto x/\varepsilon$, podemos asumir que $\mathbb{B}^n = B_ 1(0) \subseteq D$.
La clave de la demostración es la siguiente: \emph{cada semirecta cerrada empezando en el origen interseca $\partial
D$ en exactamente un punto}. Sea $R$ una semirecta así. Dado que $D$ es compacto, su intersección con $R$ es compacta. Por tanto existe un punto $x_0$ en su intersección tal que en él su distancia al origen asume el máximo. Es claro %lo es?
que pertenece a la frontera de $D$. Para ver que el punto es único, veamos que el segmento que une $0$ y $x_0$ está formado enteramente por puntos interiores de $D$ excepto por el $x_0$ mismo. Cualquier punto en este segmento distinto de $x_o$ se puede escribir de la forma $\lambda x_0 $ para $0\leq \lambda <1. $ Supongamos $z\in B_{1-\lambda}(\lambda x_0)$, y sea $y=(z-\lambda x_0)/(1-\lambda )$. Como $|z-\lambda x_0|<|1-\lambda|$ se tiene que $|y|<1$, y por tanto $y\in B_ 1(0)\subseteq D$ (ver \autoref{fig:convexo_esfera}). Como $y$ y $x_0$ están en $D$ y $z=\lambda x_0 + (1-\lambda )y$, se sigue de la convexidad que $\in D$. Por tanto la bola abierta $B_{1-\lambda}(\lambda x_0)$ está contenida en $D$, lo que implica que $\lambda x_0$ es un punto interior.

Definimos ahora la aplicación $f:\partial D \to \mathbb{S}^{n-1}$ por 
$$f(x)=\frac{x}{|x|}$$

$f(x)$ es el punto donde el segmento desde el origen hasta $x$ interseca la esfera unidad. Como $f$ es la restricción de una función continua, es continua, y por el parágrafo anterior es biyectiva. Dado que $\partial D$ es compacta, $f$ es un homeomorfismo por el teorema de la aplicación cerrada (\autoref{teo:aplicac_cerrada}).

Finalmente definimos $F:\overline{\mathbb{S}}^n \to D$ por 
$$F(x)= \left\{\begin{array}{lc}
				|x|f^{-1}\left(\dfrac{x}{|x|}\right), & x\neq 0;
				\\0, & x=0.

\end{array}
\right. $$
$F$ es continua fuera del origen por serlo $f^{-1}$, y en el origen porque por ser $f^{-1}$ acotada $F(x) \to 0$ cuando $x\to 0$. Geometricamente, $F$ manda cada segmento radial que conecta 0 con un punto de $\mathbb{S}^{n-1}$ al segmento radial desde $0$ hasta el punto $f^{-1}(w)\in \partial D$. Por convexidad, $F$ toma valores en $D$. La aplicación $F$ es inyectiva, pues puntos de distintas semirectas van a parar a distintas semirectas, y cada segmento radial va linealmente a su imagen. Es sobreyectiva pues cada punto $y \in D$ está en una semirecta empezando en 0. Por el teorema de la aplicación cerrada, $F$ es un homeomorfismo.
\end{proof}


\begin{figure}%%%%FIG: CONVEXO ESFERA%

\begin{center}

\begin{tikzpicture}[line cap=round,line join=round,>=triangle 45,x=0.65cm,y=0.65cm, scale=0.8]
\draw [shift={(7.74,-8.14)}] plot[domain=1.68:2.51,variable=\t]({1*14.71*cos(\t r)+0*14.71*sin(\t r)},{0*14.71*cos(\t r)+1*14.71*sin(\t r)});
\draw [shift={(-0.34,-1.89)}] plot[domain=2.56:4.7,variable=\t]({1*4.5*cos(\t r)+0*4.5*sin(\t r)},{0*4.5*cos(\t r)+1*4.5*sin(\t r)});
\draw [shift={(6.31,-14.62)}] plot[domain=1.46:2.25,variable=\t]({1*10.61*cos(\t r)+0*10.61*sin(\t r)},{0*10.61*cos(\t r)+1*10.61*sin(\t r)});
\draw [shift={(5.58,1.03)}] plot[domain=-1.21:1.47,variable=\t]({1*5.47*cos(\t r)+0*5.47*sin(\t r)},{0*5.47*cos(\t r)+1*5.47*sin(\t r)});
\draw (0,0)-- (10.97,0.11);
\draw(0,0) circle (1.65cm);
\draw (0,0)-- (3.74,0.04);
\draw(3.74,0.04) circle (0.66cm);
\draw (-0.1,-0.01) node[anchor=north ] {$0 $};
\draw (10.97,0.11)-- (0.08,0.79);
\draw (-0.37,1.65) node[anchor=north west] {$y $};
\draw (3.6,1.1) node[anchor=north west] {$z $};
\draw (3.3,-0.15) node[anchor=north west] {$ \lambda x_0 $};
\draw (11.17,0.53) node[anchor=north west] {$ x_0 $};
\draw (-3.78,2.92) node[anchor=north west] {$D$};
\draw (-1.10,3.5) node[anchor=north west] {$B_1(0)$};
\draw (2.85,2.2) node[anchor=north west] {$B_{1-\lambda}(\lambda x_0)$};
\begin{scriptsize}
\fill [color=black] (10.97,0.11) circle (1.5pt);
\fill [color=black] (0,0) circle (1.5pt);
\fill [color=black] (3.74,0.04) circle (1.5pt);
\fill [color=black] (0.08,0.79) circle (1.5pt);
\fill [color=black] (3.8,0.56) circle (1.5pt);
\end{scriptsize}
\end{tikzpicture}

\end{center}

\caption{Demostración de que sólo hay un punto de la frontera en la semirecta.\label{fig:convexo_esfera}}

\end{figure}


Esto nos muestra que un intervalo cerrado es una 1-celda cerrada, toda región poligonal es una 2-celda cerrada, y un tetraedro sólido es una 3-celda cerrada. Por convención, los conjuntos unitarios son 0-celdas abiertas y cerradas a la vez.

\begin{obs}
Sea $D$ una $n$-celda. Entonces $D$ es una variedad con borde por serlo $\overline{\mathbb{B}}^n$. Denotamos por $\partial D$ e $\interior{D}$ respectivamente a las imágenes de $\mathbb{S}^{n-1}$ y $\mathbb{B}^n$ por un homeomorfismo $F:\overline{\mathbb{B}}^n\to D$, tal que $\partial D$ es homeomorfo a $\mathbb{S}^{n-1}$ e $\interior{D}$ es una $n$-celda abierta.
\end{obs}

\begin{defin}
Un \enfatiza{CW-complejo} es un espacio topológico $X$ construido de la siguiente manera:
\begin{itemize}
\item[(1)] Empezamos con un espacio discreto $X^0$, cuyos puntos consideramos 0-celdas.
\item[(2)] Inductivamente, formamos el \enfatiza{$n$-esqueleto} $X^n$ a partir de $X^{n-1}$ pegando una colección (que puede ser vacía) de $n$-celdas.
\item[(3)] Definimos $X=\bigcup_n X^n$, y definimos la siguiente topología, que es coherente con la familia $\left\{X^n\right\}$: un conjunto $A\subset X$ es abierto (o cerrado) si y solo si $A\cap X^n$ es abierto (o cerrado) en $X^n$, para todo $n$. 
\end{itemize}
\end{defin}

\begin{defin}
Si $X=X^n$ para algún $n$, entonces se dice que $X$ es de dimensión finita, y decimos que la dimensión de $X$ es $n$.
\end{defin}

\begin{prop}
Sea $X$ un CW-complejo con un conjunto numerable de celdas. Si $X$ es localmente euclídeo, entonces es una variedad.
\end{prop}

\begin{prop}
Si $M$ es una $n$-variedad no vacía y un CW-complejo, entonces $n$ es también la dimensión de $M$ como CW-complejo.
\end{prop}


\begin{comment}
\chapter{Espacios adjunción}

\chapter{Homología}

\chapter{Orientabilidad}

\chapter{Espacios cociente}

\chapter{Definiciones}

\begin{defin}%%%%DEF: Finitamente local
Una colección de subconjuntos de un espacio topológico $X$ se dice \enfatiza{localmente finita} si cada punto del espacio tiene un entorno que interseca sólo un número finito de subconjuntos de la colección.
\end{defin}
\end{comment}

\begin{thebibliography}{99}

\bibitem{lee1}
J. M. Lee.
\textit{Introduction to Topological Manifolds}. Graduate text in mathematics, Springer - Verlag New York, 2011.

\bibitem{juanjo}
V. Muñoz, J. J. Madrigal. 
\textit{Topología Algebráica}. Sanz y Torres, 2015.


\bibitem{hatcher_torus}
A. Hatcher.
\textit{The Kirby Torus Trick for Surfaces}, 2013.
\\\url{http://front.math.ucdavis.edu/1312.3518}

\bibitem{rado}

T. Radó.
\textit{Über den Begriff der Riemannschen Fläche, Acta Sci}. Math. Szeged. 2 1925,
101–121.

\bibitem{munkres}
J. R. Munkres.
\textit{Elements of Agebraic Topology}. Addison-Wesley, 1984.

\bibitem{hatcher}
A. Hatcher.
\textit{Algebraic Topology}. 2001. 
\\\url{http://pi.math.cornell.edu/~hatcher/AT/ATpage.html}

\bibitem{dehn}
M. Dehn, P. Heegard.
\textit{Analysis situs}, Enzyklopädie der Math. Wiss. 1907, 153-220.


\bibitem{seifert}
H. Seifert, W. Threlfall.
\textit{A Textbook of Topology}, 1era edición. Academic, New York. 1980.

\bibitem{massey}
W. S. Massey. 
\textit{Algebraic Topology, an introduction}, Springer - Verlag. 1970.

\bibitem{wildberger}
N. J. Wildberger.
\textit{Algebraic Topolgy: A Begginer's Course}. Lecture 19.
\\\url{http://www.wildegg.com/youtube-algebraic-topology.html}

\end{thebibliography}

\end{document}


