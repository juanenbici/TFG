\documentclass[10pt]{report}
\usepackage[utf8]{inputenc}

\usepackage[spanish, es-lcroman]{babel} % Idioma
\selectlanguage{spanish}
\usepackage{graphicx} %imagenes
\usepackage{verbatim} % Para comment


\usepackage{appendix} %Para el apéndice


\usepackage{ragged2e} %para alinear texto izqda y derecha



\usepackage[a4paper, total={6in, 8in}]{geometry} %documento estructurado
\usepackage{microtype}%pone bonito el doc a nivel de formato

\setlength{\parindent}{0pt} % Quitar indentado parrafos automatico

\usepackage{amsmath} % Movidas útiles
\usepackage{amssymb} % Simbolos mates
\usepackage{amsthm} % Personalizar teoremas (mas abajo continuacion)
\usepackage{thmtools} % Mas movidas teoremas

\usepackage{tikz} % Para pictures
\usetikzlibrary{decorations.markings,arrows} %decoracion en tikz
\usetikzlibrary{shadows,arrows,positioning,shapes.geometric}
\usetikzlibrary{decorations,decorations.markings}
\usepackage{pgfplots}
\usetikzlibrary{intersections, pgfplots.fillbetween}
\usetikzlibrary {arrows.meta}

\usepackage{etoolbox}
\usetikzlibrary{optics}
\usepackage{pgf,tikz}
\usetikzlibrary{arrows}
\usetikzlibrary{babel}

\usepackage[hidelinks]{hyperref}
\usepackage[all]{hypcap}

\usepackage{tocbibind}%para poner la bibliografía en el toc, poner entre corchetes nottoc o notlof para quitar toc y lof


\usepackage{sectsty} % Personalizar titulos secciones
%\sectionfont{\underline} % Titulo seccion subrayado

% Comandos simbolos utiles 
\newcommand{\C}{\mathbb{C}}
\newcommand{\R}{\mathbb{R}}
\newcommand{\Q}{\mathbb{Q}}
\newcommand{\Z}{\mathbb{Z}}
\newcommand{\N}{\mathbb{N}}
\newcommand{\Epsilon}{\mathcal{E}}
\DeclareMathOperator{\interior}{Int} %interior
\DeclareMathOperator{\dimension}{dim} %dimension
\DeclareMathOperator{\Ker}{Ker}
\DeclareMathOperator{\Ab}{Ab}


\newcommand{\norm}[1]{\left\lVert#1\right\rVert} % Comando para normas
\newcommand{\Esfera}{\mathbb{S}^2}
\newcommand{\Toro}{\mathbb{T}^2}
\newcommand{\Proyectivo}{\mathbb{P}^2}
\newcommand{\enfatiza}[1]{\textbf{\textit{#1}}}

% Estilos teoremas (Mejorable)
\theoremstyle{definition}
\newtheorem{defin}{Definición}[section]
\newtheorem{tma}[defin]{Teorema}
\newtheorem*{tma*}{Teorema}
\newtheorem{corol}[defin]{Corolario}
\newtheorem{prop}[defin]{Proposición}
\newtheorem{lema}[defin]{Lema}
%\newcommand{\demo}{Demostración.\\}
%\newcommand{\ok}{\hfill$\square$}
%\theoremstyle{remark}
\newtheorem{obs}[defin]{Observación}
\newtheorem{eje}[defin]{Ejemplo}
\graphicspath{ {images/} }
\usepackage{afterpage}

\newcommand\blankpage{%
    \null
    \thispagestyle{empty}%
    \newpage}

\usepackage{subcaption} %para hacer subfigures
% Cabeceras
\renewcommand{\title}{ZIP}
\newcommand{\subtitle}{Trabajo fin de grado}
\renewcommand{\maketitle}{{\Large{\textbf{\title}}}\\\\{\Large \subtitle}\\\rule{17cm}{0.4pt}\\}


\pagenumbering{roman}
\setcounter{page}{1}


\begin{document}

%%%%%%%%%%%%%%%%%%%%%%%%%%%%%%%%%%%%%%%%%%%%%%%%%%%%%%%%%%%%%%%%%%%%%%%%%%%%%%%%%%%%%%%%%%%%%%%%%%%%%%%%%%%%%%%%%%%%%%%%%%%%%%%%%%%%%%%%%%%%%%%%%%%%%%%%%%%%%%%%%%%%%%%%%%%%%%%%%%%%%%%%%%%%%%%%%%%%%%%%%%%%%%%%%%%%%%%%%%%%%%%%%%%%%%%%%%%%%%%%%%%%%%%%%%%%%%%%%%%%%%%%%%%%%%%%%%%%%%%%%%%%%%%%%%%%%%%%%%%%%%%%%%%%%%%%%%%%%%%%%%%%%%%%%%%%%%%%%%%%%%%%%%%%%%%%%%%%%%%%%%%%%%%%%%%%%%%%%%%%%%%%%%%%%%%%%%%%%%%%%%%%%%%%%%%%%%%%%%%%%%%%%%%%%%%%%%%%%%%%%%%%%%%%%%%%%%%%%%%%%%%%%%%%%%%%%%%%%%%%%%%%%%%%%%%%%%%%%%%%%%%%%%%%%%%%%%%%%%%%%%%%%%%%%%%%%%%%%%%%%%%%%%%%%%%%%%%%%%%%%%%%%%%%%%%%%%%%%%%%%%%%%%%%%%%%%%%%%%%%%%%%%%%%%%%%%%%%%%%%%%%%%%%%%%%%%%%%%%%%%%%%%%%%%%%%%%%%%%%%%%%%%%%%%%%%%%%%%%%%%%%%%%%%%%%%%%%%%%%%%%%%%%%%%%%%%%%%%%%%%%%%%%%%%%%%%%%%%%%%%%%%%%%%%%%%%%%%%%%%%%%%%%%%%%%%%%%%%%%%%%%%%%%%%%%%%%%%%%%%%%%%%%%%%%%%%%%%%%%%%%%%%%%%%%%%%%%%%%%%%%%%%%%%%%%%%%%%%%%%%%%%%%%%%%%%%%%%%%%%%%%%%%%%%%%%%%%%%%%%%%%%%%%%%%%%%%%%%%%%%%%%%%%%%%%%%%%%%%%%%%%%%%%%%%%%%%%%%%%%%%%%%%%%%%%%%%%%%%%%%%%%%%%%%%%%%%%%%%%%%%%%%%%%%%%%%%%%%%%%%%%%%%%%%%%%%%%%%%%%%%%%%%%%%%%%%%%%%%%%%%%%%%%%%%%%%%%%%%%%%%%%%%%%%%%%%%%%%%%%%%%%%%%%%%%%%%%%%%%%%%%%%%%%%%%%%%%%%%%%%%%%%%%%%%%%%%%%%%%%%%%%%%%%%%%%%%%%%%%%%%%%%%%%%%%%%%%%%%%%%%%%%%%%%%%%%%%%%%%%%%%%%%%%%%%%%%%%%%%%%%%%%%%%%%%%%%%%%%%%%%%%%%%%%%%%%%%%%%%%%%%%%%%%%%%%%%%%%%%%%%%%%%%%%%%%%%%%%%%%%%%%%%%%%%%%%%%%%%%%%%%%%%%%%%%%%%%%%%%%%%%%%%%%%%%%%%%%
%%%%%%%%PORTADA%%%%%%%%%%%

%%% PORTADA%%%%%%
\begin{titlepage} %Creo que esto es para la numeración de páginas
\begin{center} %Que todo quede centradito

% Todo esto de abajo habría que retocarlo pero así sirve de ejemplo
\huge\textsc{Universidad Complutense de Madrid}\\[0.2in]
\includegraphics[scale=0.8]{comlu}\\[0.1in] %Introduce la imagen y la reescala, inserta un pequeño hueco con lo de debajo

\Large{Facultad de Matemáticas}\\[0.5in] %inserta un hueco mayor con lo de abajo
 %la linea horizontal
\Large{Trabajo de Fin de Grado}\\[.1in]
\Huge {Un tratamiento riguroso de la prueba ZIP}\\[0.2in]



\vfill %Llenar verticalmente
\Large {Juan Valero Oliet}\\[0.5in]
\vfill 
Dirigido por:\\
Manuel Alonso Morón\\[.1in]
\Large{}%junio de 2020
\end{center}

\end{titlepage}


\afterpage{\blankpage}

%%%%%%%%%%%%%PORTADA%%%%%%%%%%%%%%%%%%%%%
%%%%%%%%%%%%%%%%%%%%%%%%%%%%%%%%%%%%%%%%%%%%%%%%%%%%%%%%%%%%%%%%%%%%%%%%%%%%%%%%%%%%%%%%%%%%%%%%%%%%%%%%%%%%%%%%%%%%%%%%%%%%%%%%%%%%%%%%%%%%%%%%%%%%%%%%%%%%%%%%%%%%%%%%%%%%%%%%%%%%%%%%%%%%%%%%%%%%%%%%%%%%%%%%%%%%%%%%%%%%%%%%%%%%%%%%%%%%%%%%%%%%%%%%%%%%%%%%%%%%%%%%%%%%%%%%%%%%%%%%%%%%%%%%%%%%%%%%%%%%%%%%%%%%%%%%%%%%%%%%%%%%%%%%%%%%%%%%%%%%%%%%%%%%%%%%%%%%%%%%%%%%%%%%%%%%%%%%%%%%%%%%%%%%%%%%%%%%%%%%%%%%%%%%%%%%%%%%%%%%%%%%%%%%%%%%%%%%%%%%%%%%%%%%%%%%%%%%%%%%%%%%%%%%%%%%%%%%%%%%%%%%%%%%%%%%%%%%%%%%%%%%%%%%%%%%%%%%%%%%%%%%%%%%%%%%%%%%%%%%%%%%%%%%%%%%%%%%%%%%%%%%%%%%%%%%%%%%%%%%%%%%%%%%%%%%%%%%%%%%%%%%%%%%%%%%%%%%%%%%%%%%%%%%%%%%%%%%%%%%%%%%%%%%%%%%%%%%%%%%%%%%%%%%%%%%%%%%%%%%%%%%%%%%%%%%%%%%%%%%%%%%%%%%%%%%%%%%%%%%%%%%%%%%%%%%%%%%%%%%%%%%%%%%%%%%%%%%%%%%%%%%%%%%%%%%%%%%%%%%%%%%%%%%%%%%%%%%%%%%%%%%%%%%%%%%%%%%%%%%%%%%%%%%%%%%%%%%%%%%%%%%%%%%%%%%%%%%%%%%%%%%%%%%%%%%%%%%%%%%%%%%%%%%%%%%%%%%%%%%%%%%%%%%%%%%%%%%%%%%%%%%%%%%%%%%%%%%%%%%%%%%%%%%%%%%%%%%%%%%%%%%%%%%%%%%%%%%%%%%%%%%%%%%%%%%%%%%%%%%%%%%%%%%%%%%%%%%%%%%%%%%%%%%%%%%%%%%%%%%%%%%%%%%%%%%%%%%%%%%%%%%%%%%%%%%%%%%%%%%%%%%%%%%%%%%%%%%%%%%%%%%%%%%%%%%%%%%%%%%%%%%%%%%%%%%%%%%%%%%%%%%%%%%%%%%%%%%%%%%%%%%%%%%%%%%%%%%%%%%%%%%%%%%%%%%%%%%%%%%%%%%%%%%%%%%%%%%%%%%%%%%%%%%%%%%%%%%%%%%%%%%%%%%%%%%%%%%%%%%%%%%%%%%%%%%%%%%%%%%%%%%%%%%%%%%%%%%%%%%%%%%%%%%%%%%%%%%%%%%%%%%%%%%%%%%%%%%%%%%%%%%%%%%%%%%%%%%%%%%%%%%%%%%%%%%%%%%%%%%%%%%%%%%%%%%%%%%%%%%%%%%%%%%%%%%%%%%%%%%%%%%%%%%%%%%%%%%%%%%%%%%%%%%%%%%%%%%%%%%%%%%%%%%%%%%%%%%%%%%%%%%%%%%%%%%%%%%%%%%%%%%%%%%%%%%%%%%%%%%%%%%%%%%%%%%%%%%%%%%%%%%%%%%%%%%%%%%%%%%%%%%%%%%%%%%%%%%%%%%%%%%%%%%%%%%%%%%%%%%%%%%%%%%%%%%%%%%%%%%%%%%%%%%%%%%%%%%%%%%%%%%%%%%%%%%%%%%%%%%%%%%%%%%%%%%%%%%%%%%%%%%%%%%%%%%%%%%%%%%%%%%%%%%%%%%%%%%%%%%

\section*{Resumen}\addcontentsline{toc}{chapter}{\numberline{}Resumen}
El presente trabajo tiene como objetivo dar un tratamiento riguroso a la denominada ``prueba ZIP'' de John H. Conway, que trata de forma intuitiva el Teorema de Clasificación de Superficies. Este asegura que toda superficie con borde compacta es homeomorfa a una suma conexa de esferas, a una suma conexa de toros o a una suma conexa de planos proyectivos, junto con una serie de discos perforados. Con este fin se dará una introducción básica al estudio de las variedades topológicas, se tratará el teorema de triangulación de Radó y se dará un método de representación de las superficies con borde como palabras. 

\section*{Abstract}\addcontentsline{toc}{chapter}{\numberline{}Abstract}
\clearpage

\section*{Introducción}\addcontentsline{toc}{chapter}{\numberline{}Introducción}
\clearpage
\begin{comment}
\section*{Notación}\addcontentsline{toc}{chapter}{\numberline{}Notación}

$I$ representará el intervalo $\left[ 0,1\right]$.
\clearpage
\end{comment}
\tableofcontents
\clearpage
\listoffigures





\chapter{Variedades y superficies}
\pagenumbering{arabic} 
\setcounter{page}{1}
En este capítulo doy las definiciones y resultados básicos sobre variedades y superficies. Me basaré principalmente en los libros de J. M. Lee \cite{lee1} y V. Muñoz - J. J. Madrigal \cite{juanjo}.\\

\section{Variedades}\label{sec:variedades}
Los espacios topológicos de los que nos vamos a ocupar en el siguiente trabajo son las variedades, y en concreto las superficies. Definámoslas.
\begin{defin}\label{def:variedad}%%%%DEF: variedad topológica
Una \textbf{\emph{variedad topológica}} (de ahora en adelante \emph{variedad}) es un espacio topológico Hausdorff, II AN y localmente homeomorfo a $R^n$, para algún $n\geq 0$.
\end{defin}
Sea $M$ una variedad, y sea $p\in M$. Si $U\subseteq M$ es un entorno de $p$ que es homeomorfo a un abierto $U'$ de $\R^n$, decimos que $U$ es un \enfatiza{dominio coordenado}, y llamamos \enfatiza{aplicación coordenada} a cualquier homeomorfismo $\varphi :U\to U'$. Decimos que el par $(U,\varphi )$ es una \enfatiza{carta para $M$}.\\
A un dominio coordenado que es homeomorfo a una bola de $\R^n$ se le llama \enfatiza{bola coordenada} (si $n=2$, decimos \enfatiza{disco coordenado}). Si $p\in M$ y $U$ es un dominio coordenado que contiene a $p$, decimos que $U$ es un \enfatiza{entorno Euclídeo} de $p$.\\
No siempre se tiene que la adherencia de una bola coordenada sea homeomorfa a una bola cerrada Euclidea. Por tanto, decimos que una bola coordenada $B\subseteq M$ es una \enfatiza{bola coordenada regular} si existe un entorno $B'$ de $\overline{B}$ y un homeomorfismo $\varphi:B'\to B_{r'}(x)\subseteq \R^n$ que lleva $B$ a $B_r(x)$ y $\overline{B}$ a $\overline{B}_r(x)$ para algún $r'>r>0$ y para algún $x\in \R^n$.
\begin{obs}%%%%OBS: toda propiedad local de Rn se pasa a las variedades
Ser localmente homeomorfo a $\R^n$ es una propiedad local, y por tanto las propiedades locales de $\R^n$ se trasladan a una variedad. Así pues, las variedades son localmente compactas, I AN, localmente conexas, localmente conexas por caminos y localmente simplemente conexas.
\end{obs}
Vamos a definir ahora la dimensión de una variedad. El \textit{Teorema de Invarianza del Dominio} dice que si $W\subset \R^n$ y $W'\subset \R^m$ son abiertos y existe un homeomorfismo $\phi: W \rightarrow W'$, entonces $n=m$. Así pues, sea $M$ una variedad, y sea $p\in M$. Entonces hay un único $n=n_p$ tal que un entorno $U^p$ de $p$ en $M$ es homeomorfo a un abierto $U'\subset \R^n$. Decimos que $n_p$ es la \enfatiza{dimensión en p}. Si para todo punto $q\in U^p$ tomamos $U^p$ como entorno de $q$, tenemos que $n_q=n_p$. Luego en toda la componente conexa de $p$, el $n$ que aparece es el mismo, y lo llamaremos \enfatiza{dimensión} de dicha componente conexa. Si escribimos $M=\amalg M_i$ como la unión disjunta de sus componentes conexas $M_i$, tenemos que todas las $M_i$ son variedades, y si todas las $M_i$ tienen la misma dimensión $n$, entonces escribimos $n=\dimension{M}$ , y decimos que $M$ es una \enfatiza{n-variedad}.\\
El ejemplo más trivial de $n$-variedad es $\R^n$, pero también lo es cualquier abierto suyo. De hecho, esto se puede generalizar:
\begin{prop}%%%%PROP: abierto variedad abierto
Todo subconjunto abierto de una $n$-variedad es una $n$-variedad.
\end{prop}
\begin{proof}
Sea $M$ una $n$-variedad, y sea $V$ un subconjunto abierto de $M$. Para todo $p\in V$, $p$ tiene un entorno $U^p$ en $M$ que es homeomorfo a un subconjunto abierto de $\R^n$. $U^p\cap V$ es también abierto y homeomorfo a un subconjunto abierto de $\R^n$, y está contenido en $V$. Por tanto $V$ es localmente homeomorfo a $R^n$. Por otro lado, todo abierto de un espacio Hausdorff es Hausdorff y todo abierto de un espacio II AN es II AN. Por lo que $M$ es una variedad.
\end{proof}
\begin{eje}%%%%EJE: variedades.
\begin{itemize}
\item Las 0-variedades son espacios discretos numerables. La única 0-variedad conexa es un punto.
\item Existen dos 1-variedades conexas salvo homeomorfismo: la recta $\R$ y el círculo $\mathbb{S}^1\\
=\left\{(x,y)\in \R^2\mid x^2+y^2=1\right\}=\left\{z=e^{2\pi i\theta}\in \C \mid \theta \in \left[0,1\right] \right\}=\left\{ z\in \C \mid |z|=1\right\}$.
\end{itemize}
\end{eje}
\begin{defin}%%%%DEF: SUPERFICIE
Una \textbf{\emph{superficie}} es una $2$-variedad.
\end{defin}
\begin{eje}%%%%EJE: toro y esfera
Son superficies:
\begin{itemize}
\item La esfera $\mathbb{S}^2=\{(x,y,z) \in \R^3\mid  x^2+y^2+z^2=1\}$ (\autoref{fig:esfera}).
\item El toro $\mathbb{T}^2=\{(x,y,z)\in \R^3\mid  (\sqrt{x^2+y^2}-2)^2+z^2=1\}$ (\autoref{fig:toro}).
\end{itemize}
\end{eje}
\begin{figure}%%%%FIG: Toro y esfera
\begin{subfigure}{.5\textwidth}
\centering
\begin{tikzpicture}[scale=0.8]
\draw (0,0) circle (2cm);
\draw (2,0) arc[x radius=2, y radius=0.7, start angle=0, end angle=-180];
\draw [dashed] (2,0) arc[x radius=2, y radius=0.7, start angle=0, end angle=180];
\end{tikzpicture}
\caption{$\Esfera$\label{fig:esfera}}
\end{subfigure}
\begin{subfigure}{.5\textwidth}
\centering
\begin{tikzpicture}
\draw (6,-0.1) ellipse (2.9cm and 1.4 cm);
\draw (7.5,0.1) arc[x radius=1.5, y radius=0.4, start angle=0, end angle=-180];
\draw (7.3,-0.1) arc[x radius=1.3, y radius=0.3, start angle=0, end angle=180]; 
\end{tikzpicture}
\caption{$\Toro\label{fig:toro}$}
\end{subfigure}	
\caption{Ejemplos de superficies.}
\end{figure}


\section{Variedades con borde}\label{sec:borde}
Hay una serie de espacios topológicos que no son variedades pero que tienen interés desde el punto de vista geométrico y son útiles a la hora de estudiarlas. Por ejemplo, la bola cerrada $\overline{\mathbb{B}}^n$ no es una variedad, puesto que para los puntos de su frontera no hay entornos homeomorfos a abiertos de $\R^n$. Vamos a definir por tanto una clase de espacios que extienda a las variedades pero que admita la existencia de algún tipo de \textit{bordes}. Para ello, indicaremos con $\mathbb{H}^n$ al \enfatiza{semiplano superior cerrado n-dimensional} $$\mathbb{H}^n=
\left\{(x_1,\dots ,x_n)\in \R^n\mid x_n\geq 0\right\}.$$
\begin{defin}
Una \enfatiza{variedad con borde n-dimensional} $M$ (o $n$-variedad con borde) es un espacio topológico  Hausdorff y II AN tal que todo punto $p\in M$ tiene un entorno $U$ homeomorfo a un abierto $U'\subset \mathbb{H}^n$.
\end{defin}
Sea $M$ una n-variedad con borde, $p\in M$ y $\varphi :U\to U'\subset \mathbb{H}^n$ un homeomorfismo como en la definición. Como en el caso de variedades, decimos que $U$ es un \enfatiza{dominio coordenado}, que $\varphi$ es una \enfatiza{aplicación coordenada} para $M$ y que el par $(U,\varphi )$ es una \enfatiza{carta para $M$}.\\
Como en la sección anterior, decimos que $B\subseteq M$ es una \enfatiza{semibola coordenada regular} si existe un abierto $B'$ que contiene a $\overline{B}$ y y un homeomorfismo de $B'$ a $B_{r'}(0)\cap \mathbb{H}^n$ que lleva $B$ a $B_r(0)\cap \mathbb{H}^n$ y $\overline{B}$ a $\overline{B}_r(0)\cap \overline{H}^n$ para algún $0<r<r'$.
Si denotamos $\varphi (p)=a=(a_1,\dots ,a_n)$, podemos distinguir entre dos tipos de puntos:
\begin{itemize}

\item[1.] \textit{Puntos interiores:} Si $a_n>0$, entonces podemos tomar una bola $B_{\varepsilon}(a)\subset U'$ y $V=\varphi^{-1}(V')$, por lo que $V$ es un entorno de $p$ homeomorfo a un abierto de $\R^n$. Decimos por tanto que $p$ es un \enfatiza{punto interior de $M$}, y denotamos por $\interior{M}$ al conjunto de puntos interiores de $M$.
\item[2.] \textit{Puntos borde:} Si $a_n=0$ para todo $n$, entonces podemos tomar una semibola $V'=B_{\varepsilon}^{+}=B_{\varepsilon}(a)\cap \mathbb{H}^n$, y $V=\varphi^{-1}(V')$. Decimos que $p$ es un \enfatiza{punto borde de $M$}. Al conjunto de todos los puntos borde de $M$ lo denotamos por $\partial M$.
\end{itemize}
\begin{prop}\label{prop:int_variedad}%%%%PROP: interior y borde de una variedad con borde
Sea $M$ una $n$-variedad con borde, entonces $\interior{M}$ es un abierto de $M$, y es en sí mismo una $n$-variedad.
\end{prop}
\begin{proof}
Sea, para cada $x\in \interior{M}$, $(U_x,\psi_x)$ una carta para $x$, donde $U_x$ es un abierto homeomorfo a $\R^n$. Se tiene que $\interior{M}=\bigcup_{x\in M} U_x$, por lo que es un abierto, y además localmente Euclídeo, y por tanto variedad.
\end{proof}
\begin{prop}[Invarianza del borde]%%%%PROP: invarianza del borde
Si $M$ es una $n$-variedad con borde, entonces un punto $p\in M$ no puede ser un punto interior y un punto borde al mismo tiempo. O lo que es lo mismo, $M=\interior{M} \cup \partial M$, con $\interior{M} \cap \partial M=\emptyset$.
\end{prop}
\begin{proof}
Dado que las variedades que trato en el trabajo son las superficies, voy a dar la demostración en el caso de que $n=2$. El caso general incluye conceptos de homología que no voy a tratar.\\
Supongamos que $p\in M$ es tanto un punto interior como un punto borde de $M$. Entonces existen $(U,\varphi)$ tal que $\varphi (U)\subseteq \interior{\mathbb{H}^2}$ y $(V,\psi)$ tal que $\psi (V)\subseteq \mathbb{H}^2$, donde $\psi (p)=s\in \partial \mathbb{H}^2$. Llamando $W=U\cap V$, se tiene que $\varphi (W)$ es homeomorfo a $\psi (V)$. \\
Ahora bien, podemos elegir un $\varepsilon >0$ tal que $B_{\varepsilon}(s)\cap \mathbb{H}^2 \subseteq \psi (W)$. Sea $U'=\psi^{-1}\left( B_{\varepsilon}(s) \cap \mathbb{H}^2\right)$; utilizando que un abierto de $\R^2$ menos un punto suyo no es simplemente conexo, tenemos que $\varphi (U')\setminus \{\varphi (p) \}$ es no simplemente conexo. Por otro lado, $U'\setminus {p}$ es homeomorfo a $X:=B_{\varepsilon}(s)\cap \mathbb{H}^2 \setminus \{ s\}$.\\
Sea $x_0\in B_{\varepsilon}(s)\cap \interior{\mathbb{H}^2}$, definimos $F: X\times \left[ 0,1\right] \to X$ por $F(x,t)=x_0 + (1-t)(x-x_0)$, que nos indica que $\{x_0\}$ es un retracto por deformación fuerte de $X$. Por tanto, $X$ es simplemente conexo, lo que implica que también lo son $U'\setminus \{p\}$ y $\varphi(U')\setminus \{ \varphi (p)\}$, pero esto es una contradicción.
\end{proof}

\begin{corol}\label{corol:borde_n-1_variedad}
Si $M$ es una $n$-variedad con borde, entonces:
\begin{itemize}
\item[(i)]  $\partial M$ es un cerrado en $M$.
\item[(ii)] $\partial M$ es una $(n-1)$-variedad. 
\item[(iii)] $M$ es una variedad si y solo si $\partial M=\emptyset$.
\end{itemize}
\end{corol}

\begin{proof}
Por el teorema de la invarianza del borde, $\partial M= M\setminus \interior{M}$, y siendo $\interior{M}$ abierto, $\partial M$ es por tanto cerrado, lo que demuestra (i). 
%((AQUÍ FALTA REVISARLO))
Para (ii), sea $p\in \partial M$. Tomamos $U$ un entorno de $p$ en $M$ y $\varphi : U\to U'\subset \mathbb{H}^n$ un homeomorfismo. Sea $V'=B_{\varepsilon}(a)\cap \mathbb{H}^n$ tal que $\varphi(p)=a$, y $V=\varphi^{-1} (V)$.
%Tomamos $U$ un entorno suyo en $M$, y sea $\varphi :V\to V'=B_{\varepsilon}(a)\cap \mathbb{H}^n$ tal que $\varphi(p)=a$. 
Así pues, $\varphi^{-1} (\{(x_1,\dots ,x_n)\in V' \mid x_{n}>0\})\subseteq \interior M$  y $\varphi^{-1} (\{(x_1,\dots ,x_{n-1},0)\in V'\})\subseteq \partial M$. Por lo tanto $\partial M\cap V=\varphi^{-1} (\{(x_1,\dots ,x_{n-1},0)\in V'\})=W$ es un abierto de $\partial M$, y $\varphi : W\to W'=B_{\varepsilon}(a)\cap (\R^{n-1} \times \{ 0\} )$ es un homeomorfismo con un abierto de $\R^{n-1}$, lo que implica que $\partial M$ es una $(n-1)$-variedad.
Finalmente para probar (iii), si suponemos que $M$ es una variedad, entonces todo punto tiene un entorno homeomorfo a un abierto de $\R^n$, por lo que todo punto es interior y $M=\interior{M}$, y se sigue de la invarianza del borde que $\partial M=\emptyset$. Si suponemos ahora que $\partial M=\emptyset$, entonces $M=\interior{M}$, que es una variedad por la \autoref{prop:int_variedad}.
\end{proof}
\begin{figure}[t]%%%%FIG: Cilindro y Möbius
\begin{subfigure}{.5\textwidth}
\centering
\begin{tikzpicture}[scale=1.5]
\draw (-.5,0) arc [x radius=1, y radius=.3, start angle=180, end angle=360] -- +(0,1) arc[x radius=1, y radius=.3, start angle=0, end angle=180] -- +(0,-1);
\draw [dashed] (-.5,0) arc[x radius=1, y radius=.3, start angle=180, end angle=0];
\draw (1.5,1) arc[x radius=1, y radius=.3, start angle=0, end angle=-180];
\end{tikzpicture}
\caption{Cilindro.\label{fig:cilindro}}
\end{subfigure}
\begin{subfigure}{.5\textwidth}
\centering
\begin{tikzpicture}[scale=1.7]
%\draw [help lines, step=1mm] (-2,-2) grid (2,2);
%\draw [gray](-2,0) -- (2,0);
%\draw [gray](0,-2) -- (0,2);
%\draw[looseness=1] (.65,-.05) [out=135, in=0] to (-.2, .175) [out=180, in=90] to(-.5,-.0) [out=-90, in=180] to (0,-.2) [out=0, in=-90] to (.7,.1) [out=90, in=0] to (-.,.45) [out=180, in=90] to (-.8,0) [out=-90, in=180] to (.2,-.5) [out=0, in=-100] to (.75,-.25) [out=95, in=-80] to (.7,.15) ;
\draw (1,.7)[out=-90, in=45] to (.95,.6) [out=-135,in=0] to (-.5, -.0) [out=180, in=-20] to (-1,.1) -- (-1,.7) arc [x radius=1, y radius=0.2, start angle=180, end angle=0] -- (1,.1) [out=-160, in=-10] to (-.3,0.006);
\draw (-.0,.05) [out=160, in=-45] to (-.95,.6) [out=135, in=-90] to (-1,.7);
\draw (-.7,.39) arc [x radius=.7, y radius=.1, start angle=180, end angle=-4];
\end{tikzpicture}
\caption{Banda de Möbius.\label{fig:mobius}}
\end{subfigure}
\end{figure}
Hacemos incapié en que los conceptos de \textit{variedad} y \textit{variedad con borde} son distintos. Una variedad con borde puede ser o no una variedad, pues puede tener el borde vacío. En cambio, una variedad es siempre una variedad con borde en la que todo punto es un punto interior. Para evitar confusiones, si el contexto lo pide, utilizaremos \enfatiza{variedad sin borde} para referirnos a una variedad en el sentido de la \autoref{def:variedad}.\\
Veamos algunos ejemplos no triviales de superficies con borde:
\begin{eje}\label{eje:rep_borde}
\begin{itemize}
\item[(1)] El cilindro  (\autoref{fig:cilindro}): $$Cil=\left\{(x,y,z)\in \R^3 \mid x^2+y^2=1,\, z\in \left[ 0,1\right] \right\} \,.$$
\item[(2)] La banda de Möbius (\autoref{fig:mobius}):
\begin{align*}
Mob=
& \left\{ \left(\left(1+\left(y-\frac{1}{2}\right)\cos \left(\pi x\right)\right), \, \left(1+\left(y-\frac{1}{2}\right)\cos \left(\pi x\right)\right)\sin \left( 2 \pi x\right), \right. \right.\\
& \left. \left. \left( y-\frac{1}{2} \right)\sin \left( \pi x\right)\right) \mid x\in \left[0,1\right], \, y\in \left[0,1\right] \right\} \, .
\end{align*}.
%\item[(3)] Un toro con dos ``perforaciones''.
\end{itemize}
\end{eje}

\section{Suma conexa de variedades}
Sean $M_1$ y $M_2$ dos $n$-variedades conexas. Dados $p_1\in M_1$ y $p_2\in M_2$ sean $U_1\subset M_1$, $U_2\subset M_2$  entornos abiertos de $p_1$ y $p_2$ respectivamente, y sean $\phi_1:U_1\to\R^n$ y $\phi_2:U_2\to\R^n$ dos homeomorfismos tales que $\phi_1(p_1)=0$ y $\phi_2(p_2)=0$. Si llamamos $B_1=\phi_1^{-1}(B_1(0))\subset M_1$ y $B_2=\phi_2^{-1}(B_1(0))\subset M_2$, consideremos $M_1^o=M_1\setminus B_1$, $M_2^o=M_2\setminus B_2$ y $M_1^o \amalg M_2^o$ con la topología unión disjunta.
Se define la relación de equivalencia $\sim$ en la que si $x_1\in S_1=\phi_1^{-1}(\partial B_1(0))$, $x_2\in S_2=\phi_2^{-1}(\partial B_1(0))$, entonces $x_1\sim x_2$ si y sólo si $\phi_1(x_1)=\phi_2(x_2)$, y se considera el cociente 

$$M=\frac{M_1^o\amalg M_2^o}{\sim}.$$

\begin{defin}%%%% DEF: suma conexa
A $M$ así definida se le llama \textbf{\textit{suma conexa}} de $M_1$ y $M_2$, y se denota por $M=M_1\#M_2$.
\end{defin}
\begin{figure}[t]%%%%FIG: 2-toro
\centering
\begin{tikzpicture}
\draw (0.23,1.85) arc[x radius=2, y radius=1.4, start angle=150, end angle=-150];
\draw (-0.23,1.85) arc[x radius=2, y radius=1.4, start angle=30, end angle=330];
\draw plot [smooth] coordinates { (-0.23,1.85) (0,1.67)(0.23,1.85)};
\draw plot [smooth] coordinates { (-0.23,0.45) (0,0.67)(0.23,0.45)};

\draw (-1.25,1.35) arc[x radius=0.7, y radius=0.5, start angle=0, end angle=-180];
\draw (-1.45,1) arc[x radius=0.5, y radius=0.3, start angle=0, end angle=180]; 
\draw (2.65,1.35) arc[x radius=0.7, y radius=0.5, start angle=0, end angle=-180];
\draw (2.45,1) arc[x radius=0.5, y radius=0.3, start angle=0, end angle=180]; 

\draw (0,1.67) arc[x radius=0.1, y radius=0.5, start angle=90, end angle=270];
\draw [dashed] (0,1.67) arc[x radius=0.1, y radius=0.5, start angle=90, end angle=-90];
\end{tikzpicture}
\caption{Suma conexa de toros.}
\end{figure}

\begin{prop}\label{prop:suma_conexa}%%%% PROP: suma conexa es variedad.
Sean $M_1$ y $M_2$ variedades. Entonces $M=M_1\#M_2$ es una variedad.
\end{prop}
\begin{proof}

Sea $\pi:M_1^o\amalg M_2^o\to M$ la proyección al cociente y sea $S=\pi (S_1)=\pi (S_2)$.  $U_j=M_j^o-S_j$, $j=1,2$ son dos abiertos saturados respecto de $\pi$, por lo que $\pi :U_ j\to \pi (U_j)=U_j'$ es también una aplicación cociente, que además es inyectiva y por lo tanto un homeomorfismo. Tenemos así que $M$ es localmente euclídeo en $U_1'\cup U_2'$, y además es Hausdorff y IIAN. Nos quedan por tanto verificar los puntos $p\in S$.
Este $p\in S$ verifica que $p=\pi (p_1)=\pi (p_2)$, con $p_j\in S_j$, $j=1,2$, y que  $\varphi_j(p_j)=x_0 \in \partial B_1(0) \subset \R^n$. Si tomamos $V$ un entorno de $x_0$ en $\partial B_1(0)$, tenemos que dado un $\varepsilon >0$, $\hat{V}=\{rx\mid   r\in (1-\varepsilon , 1+\varepsilon ), x\in V\} $ es entorno de $x_0$ en $\R^n$, y que $\hat{V}-B_1(0)=\{rx\mid r\in [1, 1+\varepsilon ), x \in V\}$. Sea $V_j=\varphi_j^{-1}(\hat{V}-B_1(0))$, que es entorno de $p_j$ en $M_j^o$. Como $ V_1\amalg V_2$ es abierto saturado de $M_1^o\amalg M_2^o$ respecto de $\pi$, entonces $\tilde{V}=\pi (V _1\amalg V_2)$ es entorno de $p$ en $M$. Veamos que es localmente Euclídeo. Si construimos $\Phi$ de la siguiente manera
\begin{align*}
\Phi : &  V_1\amalg V_2  \to  V\times (1-\varepsilon , 1+ \varepsilon ), \\
& q_1\in  V_1  \mapsto  (x,r), r=\norm{\varphi_1(q_1)}, x=\varphi_1(q_1)/r,\\
& q_2\in V_2  \mapsto  (x,2-r), r= \norm{\varphi_2(q_2)}, x= \varphi_2(q_2)/r.
\end{align*}
obtenemos que $\Phi :  V_1 \to V \times [1, 1+\varepsilon)$ y $\Phi : V_2 \to V \times (1- \varepsilon, 1]$ son homeomorfismos. Además se tiene que $q_1 \sim q_2$ si y sólo si $\Phi(q_1)=\Phi(q_2)$.  De este modo, $\Phi$ induce una aplicación continua y biyectiva $$\overline{\Phi}=\Phi \circ \pi^{-1}: \tilde{V} \to V \times (1- \varepsilon, 1+ \varepsilon).$$
Si comprobamos que $\overline{\Phi}$ es abierta, tendremos que es homeomorfismo. Tomamos un abierto básico saturado de $V_1\amalg V_2$, entonces o bien está totalmente incluido en $V_1-S_1$ o en $V_2-S_2$, en cuyo caso la imagen por $\overline{\Phi}$ es un abierto de $V\times (1-\varepsilon ,1)$ o de $V\times (1,1+\varepsilon )$, o bien interseca a $S_1$ y $S_2$. En este caso el abierto es de la forma $W_1\amalg W_2$, construido como hicimos con $V_1\amalg V_2$ a partir de un $W\subset V \subset \partial B_1(0)$. Entonces $\overline{\Phi} (\tilde{W})= W \times (1-\delta , 1+\delta )$ con $0<\delta \leq \varepsilon$, $\tilde{W} = \pi (W_1 \amalg W_2)$. Luego $\overline{\Phi}$ es abierta, y por tanto un homeomorfismo, y así finalmente $\tilde{V}$ es un entorno euclídeo de $p$. 

Por construcción podemos tomar los abiertos $\tilde{V} \subset  M$ en cantidad numerable para dormar una base de la topología, y por lo tanto $M$ es IIAN. Nos queda sólo ver que es Hausdorff. Sea un $q\in U_j'$, $j=1,2$, y  un $p\in S$. Podemos tomar un entorno abierto $\tilde{V}$ de $p$ disjunto de un entorno pequeño de $q$. Y si tomamos $p, p' \in S$ distintos, los abiertos $\tilde{V}$, $\tilde{V}'$ construidos partiendo de $V$, $V'\subset \partial B_1(0)$ disjuntos, son disjuntos. Por lo tanto, $M$ es Hausdorff, y finalmente, variedad.

\end{proof}

FALTA COMENTARIO DE QUE LA CONSTRUCCIÓN RESULTANTE NO DEPENDE DE LAS ELECCIONES HECHAS----ANNULUS THEOREM... ETC

\clearpage
%%%%%%%%%%%%%%%%%%%%%%%%%%%%%%%%%%%%%%%%%%%%%%%%%%%%%%%%%%%%%%%%%%%%%%%%%%%%%%%%%%%%%%%%%%%%%%%%%%%%%%%%%%%%%%%%%%%%%%%%%%%%%%%%%%%%%%%%%%%%%%%%%%%%%%%%%%%%%%%%%%%%%%%%%%%%%%%%%%%%%%%%%%%%%%%%%%%%%%%%%%%%%%%%%%%%%%%%%%%%%%%%%%%%%%%%%%%%%%%%%%%%%%%%%%%%%%%%%%%%%%%%%%%%%%%%%%%%%%%%%%%%%%%%%%%%%%%%%%%%%%%%%%%%%%%%%%%%%%%%%%%%%%%%%%%%%%%%%%%%%%%%%%%%%%%%%%%%%%%%%%%%%%%%%%%%%%%%%%%%%%%%%%%%%%%%%%%%%%%%%%%%%%%%%%%%%%%%%%%%%%%%%%%%%%%%%%%%%%%%%%%%%%%%%%%%%%%%%%%%%%%%%%%%%%%%%%%%%%%%%%%%%%%%%%%%%%%%%%%%%%%%%%%%%%%%%%%%%%%%%%%%%%%%%%%%%%%%%%%%%%%%%%%%%%%%%%%%%%%%%%%%%%%%%%%%%%%%%%%%%%%%%%%%%%%%%%%%%%%%%%%%%%%%%%%%%%%%%%%%%%%%%%%%%%%%%%%%%%%%%%%%%%%%%%%%%%%%%%%%%%%%%%%%%%%%%%%%%%%%%%%%%%%%%%%%%%%%%%%%%%%%%%%%%%%%%%%%%%%%%%%%%%%%%%%%%%%%%%%%%%%%%%%%%%%%%%%%%%%%%%%%%%%%%%%%%%%%%%%%%%%%%%%%%%%%%%%%%%%%%%%%%%%%%%%%%%%%%%%%%%%%%%%%%%%%%%%%%%%%%%%%%%%%%%%%%%%%%%%%%%%%%%%%%%%%%%%%%%%%%%%%%%%%%%%%%%%%%%%%%%%%%%%%%%%%%%%%%%%%%%%%%%%%%%%%%%%%%%%%%%%%%%%%%%%%%%%%%%%%%%%%%%%%%%%%%%%%%%%%%%%%%%%%%%%%%%%%%%%%%%%%%%%%%%%%%%%%%%%%%%%%%%%%%%%%%%%%%%%%%%%%%%%%%%%%%%%%%%%%%%%%%%%%%%%%%%%%%%%%%%%%%%%%%%%%%%%%%%%%%%%%%%%%%%%%%%%%%%%%%%%%%%%%%%%%%%%%%%%%%%%%%%%%%%%%%%%%%%%%%%%%%%%%%%%%%%%%%%%%%%%%%%%%%%%%%%%%%%%%%%%%%%%%%%%%%%%%%%%%%%%%%%%%%%%%%%%%%%%%%%%%%%%%%%%%%%%%%%%%%%%%%%%%%%%%%%%%%%%%%%%%%%%%%%%%%%%%%%%%%%%%%%%%%%%%%%%%%%%%%%%%%%%%%%%%%%%%%%%%%%%%%%%%%%%%%%%%%%%%%%%%%%%%%%%%%%%%%%%%%%%%%%%%%%%%%%%%%%%%%%%%%%%%%%%%%%%%%%%%%%%%%%%%%%%%%%%%%%%%%%%%%%%%%%%%%%%%%%%%%%%%%%%%%%%%%%%%%%%%%%%%%%%%%%%%%%%%%%%%%%%%%%%%%%%%%%%%%%%%%%%%%%%%%%%%%%%%%%%%%%%%%%%%


\chapter{Triangulación de superficies}

Un hecho fundamental para la prueba del teorema de clasificación es que toda superficie es triangulable. La demostración, atribuída a Radó en 1925 \cite{rado}, utiliza el \emph{teorema de Schönflies}, cuya prueba es larga y técnica. ......... \begin{comment} Utilizaremos el truco de Kirby para superficies dado por Hatcher \cite{hatcher_torus}.
\end{comment}
\

\section{Complejos simpliciales y triangulación}\label{sec:simplices}

Para poder dar una definición rigurosa de triangulación de variedades necesitamos la noción de \textit{complejos simpliciales}. Estos son construcciones formadas por \textit{símplices}, que son una generalización de los triángulos. En esta primera parte me baso en las definiciones de Munkres \cite{munkres}.

\begin{defin}%%%%DEF: Posición general
Sean $v_0,\dots v_k$ $k+1$ puntos distintos de $\R^n$. Decimos que $\{ v_0,\dots ,v_k\}$ están en \enfatiza{posición general} si $c_0,\dots c_k$ son números reales tales que  $$\sum_{i=0}^{k}c_iv_i=0 \text{ y } \sum_{i=0}^kc_i=0,$$ entonces $c_0=\dots =c_k=0$.
\end{defin}


\begin{defin}%%%%DEF: Símplice
Sean $\{ v_0,\dots ,v_n\}$ un conjunto de $k+1$ puntos de $\R^n$ en posición general. El \enfatiza{símplice} generado por ellos, que denotamos por $[ \, v_0,\dots ,v_k ] \,$, es el conjunto $$[ \, v_0,\dots ,v_k] \, =\left\{  \sum_{i=0}^{k}t_iv_i \mid t_i\geq 0,\, \sum_{i=0}^{k}t_i=1 \right\}, $$ con la topología heredada de $\R^n$. Para todo punto $x=\sum_it_iv_i\in [ \, v_0,\dots ,v_k] \,$, llamamos a los $t_i$ \enfatiza{coordenadas baricéntricas de $x$}. Cada uno de los $v_i$ se llama \enfatiza{vértice} del símplice. Al entero $k$ se le llama \enfatiza{dimensión}, y diremos que $[ \, v_0,\dots ,v_k] \,$ es un \enfatiza{$k$-símplice}. 
\end{defin}

\begin{eje}
Un $0$-símplice es un punto, un 1-símplice es un segmento, un $2$-símplice es un triángulo junto a su interior, un $3$-símplice es un tetraedro sólido, y así sucesivamente (\autoref{fig:simplices}).
\end{eje}

\begin{figure}
\centering
\begin{tikzpicture}[scale=1.4]

\fill[gray!20] (1,0) -- (1.5,0.87) -- (2,0) --cycle;
\draw (1,0) -- (1.5,0.87) -- (2,0) -- cycle;
\draw (-1,0.2) -- (0,0.5);
\fill[gray!20] (3,0.5) -- (3.5,0.9)--(4,0.5)--(3.5,0)--cycle;
\draw (3,0.5) -- (3.5,0.9)--(4,0.5)--(3.5,0)--cycle;
\draw (3.5,0.9)--+(0,-.9);
\draw [dashed](3,.5)--+(1,0);
\fill (3,.5) circle (1.5pt);
\fill (4,.5) circle (1.5pt);
\fill (3.5,.9) circle (1.5pt);
\fill (3.5,0) circle (1.5pt);
\fill (-1,0.2) circle (1.5pt);
\fill (0,0.5) circle (1.5pt);
\fill (-2,0.6) circle (1.5pt);
\fill (1,0) circle (1.5pt);
\fill (1.5, 0.87) circle (1.5pt);
\fill (2,0) circle (1.5pt);
\end{tikzpicture}
\caption{$k$-símplices, $k=0,\dots ,3$.\label{fig:simplices}}
\end{figure}



Sea $\sigma$ un $k$-símplice. Cada símplice generado por un subconjunto no vacío de vértices de $\sigma$ se llama \enfatiza{cara de $\sigma$}. Las caras que no son iguales a $\sigma$ se llaman \enfatiza{caras propias}. Las caras $0$-dimensionales de $\sigma$ son sus vértices, y a las caras $1$-dimensionales se les llama \enfatiza{aristas}. Las caras $(k-1)$-dimensionales de un $k$-símplice se llaman \enfatiza{caras fronterizas}, y a su unión la llamamos \enfatiza{frontera}. Definimos el \enfatiza{interior} como $\sigma$ menos su frontera. 


\begin{defin}%%%%DEF: Complejo simplicial
Un \enfatiza{complejo simplicial} es una colección $K$ de símplices en un espacio euclídeo $\R ^n$, que satisface las siguientes condiciones:
\begin{itemize}
\item[(i)] Si $\sigma \in K $, entonces toda cara de $\sigma$ está en $K$
\item[(ii)] La intersección de dos símplices cualesquiera en $K$ es o bien vacía o bien una cara de ambos.
%\item[(iii)] $K$ es una colección finitamente local.
\end{itemize}
\label{def:complex}
\end{defin}
%La tercera condición implica que $K$ es numerable, pues todo punto de $\R ^n$ tiene un entorno intersecando al menos un número finito de símplices de $K$, y este recubrimiento abierto de $\R ^n$ tiene un subrecubrimiento numerable. A nosotros los símplices que más nos interesan son los \enfatiza{complejos simpliciales finitos}, que son los que contienen únicamente un número finito de símplices. Para estos complejos, la condición (iii) es redundante.

La \autoref{fig:complex} muestra un complejo simplicial en $\R^2$. En cambio en la \autoref{fig:not_complex} los símplices representados no forman un complejo, pues no se respeta la condición (ii) de la \autoref{def:complex}.

Si $K$ un complejo simplicial en $\R ^n$, llamamos \enfatiza{dimensión de $K$} a la dimensión máxima de los símplices en $K$. Esta no es mayor que $n$.
Un subconjunto $K'\subseteq K$ se dice que es un \enfatiza{subcomlejo de $K$} si para todo $\sigma \in K'$, toda cara de $\sigma$ está en $K'$. Un subcomplejo es un complejo simplicial en sí mismo.
Para todo $k\leq n$, el conjunto de todos los símplices de $K$ de dimensión menor o igual que $k$ es un subcomplejo llamado \enfatiza{k-esqueleto de $K$}.



%%
%%
%%

%Aquí se puede coger la definición de polítopo de munkres, pero quizas es un poco inutil
%%
%%
%%


\begin{defin}%%%%DEF: Poliedro
Sea un complejo simplicial $K$ en $\R ^n$. La unión de todos los símplices en $K$ junto con la topología heredada de $\R ^n$ es un espacio topológico que denotamos por $|K|$ y que llamamos \enfatiza{poliedro de K}.
\end{defin}



\begin{figure}%%%%FIG: Complejos y no complejos
   \begin{minipage}{0.48\textwidth}
   \centering
	\begin{tikzpicture}
	\fill[gray!20](-2.96,0)--(-2.46,1.19)--(-1,1.32)--(-1.45,0)--cycle;
	\fill[gray!20] (-0.05,0.66)--(1.03,1.41)--(1.55,0.64)--cycle;
	\draw (-3.98,-.3)--(-2.96,0)--(-2.46,1.19)--(-1.45,0)--(-1,1.32)--(-2.46,1.19);
	\draw(-2.96,0)--(-1.45,0);
	\draw (-1,1.32)--(-0.05,0.66)--(1.03,1.41)--(1.55,0.64)--(-0.05,0.66)--(-0.22,-0.45)--(1.14,-0.28)--(1.55,0.64);
	\draw (-1.11,-0.46)--(-0.56,-0.17);
	\fill [color=black] (-3.98,-0.3) circle (1.5pt);
	\fill [color=black] (-2.96,0) circle (1.5pt);
	\fill [color=black] (-2.46,1.19) circle (1.5pt);
	\fill [color=black] (-1.45,0) circle (1.5pt);
	\fill [color=black] (-1,1.32) circle (1.5pt);
	\fill [color=black] (-0.05,0.66) circle (1.5pt);
	\fill [color=black] (1.03,1.41) circle (1.5pt);
	\fill [color=black] (1.55,0.64) circle (1.5pt);
	\fill [color=black] (1.14,-0.28) circle (1.5pt);
	\fill [color=black] (-0.22,-0.45) circle (1.5pt);
	\fill [color=black] (-1.11,-0.46) circle (1.5pt);
	\fill [color=black] (-0.56,-0.17) circle (1.5pt);

	\end{tikzpicture}
	\caption{Un complejo simplicial en $\R^2$\label{fig:complex}}
   \end{minipage}\hfill
   \begin{minipage}{0.48\textwidth}
     \centering
    \begin{tikzpicture}[scale=0.5]
	\fill[gray!20] (-1.82,0.46)--(1.04,3.14)--(2.33,0.67)--cycle;
	\fill[gray!20] (1.67,1.94)--(2.94,-0.5)--(5,3)--cycle;
	\draw (1.04,3.14)-- (2.94,-0.5);
	\draw (1.04,3.14)-- (-1.82,0.46);
	\draw (-1.82,0.46)-- (2.33,0.67);
	\draw (2.94,-0.5)-- (5,3);
	\draw (5,3)-- (1.67,1.94);
	\fill [color=black] (1.04,3.14) circle (3pt);
	\fill [color=black] (2.94,-0.5) circle (3pt);
	\fill [color=black] (-1.82,0.46) circle (3pt);
	\fill [color=black] (2.33,0.67) circle (3pt);
	\fill [color=black] (5,3) circle (3pt);
	\fill [color=black] (1.67,1.94) circle (3pt);
	\end{tikzpicture}
	\caption{Símplices que no forman un complejo.\label{fig:not_complex}}
    
   \end{minipage}
\end{figure}

\begin{defin}
Una \enfatiza{aplicación afín} es una aplicación $F:\R^n \to \R^m$ tal que $F(x)=c+A(x)$, donde $c\in \R^m$ es un vector fijo y $A(x)$ es una aplicación lineal.
\end{defin}
\begin{defin}
Sean $K, \, L$ complejos simpliciales. Una \enfatiza{aplicación simplicial} es una aplicación continua $f:|K|\to |L|$ cuya restricción a cada símplice $\theta \in K$ coincide con una aplicación afín que lleva $\theta$ a algún símplice en $L$.
\end{defin}
\begin{defin}%%%%DEF: Triangulación
Sea $X$ un espacio topológico. Llamamos \enfatiza{triangulación de $X$} a un homeomorfismo entre $X$ y el poliedro de algún complejo simplicial.
\end{defin}

\begin{defin}%%%%DEF: Superficie triangulable
Toda variedad (con borde y sin borde) que admita una triangulación se dice \enfatiza{triangulable}.
\end{defin}



\section{Teorema de Radó}

\begin{tma}[Teorema de Radó]
Toda superficie es triangulable por un poliedro de un complejo simplicial 2-dimensional, en donde cada $1$-símplice es una cara de exáctamente dos $2$-símplices.\label{teo:rado}
\end{tma}

\begin{tma}
Toda superficie con borde es triangulable por un poliedro de un complejo simplicial 2-dimensional, en donde hay dos tipos de 1-símplices: los que están totalmente contenidos en el borde, y los que se corresponden con puntos interiores. Los primeros son cara de exactamente un 2-símplice, y los últimos exactamente de dos 2-símplices.
\end{tma}

\clearpage

\chapter{Teorema de Clasificación, primera parte}
En esta sección daremos una demostración clásica de la primera parte del teorema de clasificación de superficies compactas, que fue probado por primera vez en 1907 por Max Dehn y Poul Heegaard \cite{dehn}. En las primeras dos secciones utilizo las nociones que da Lee \cite{lee1} sobre la representación de superficies, y las amplío a la representación de superficies con borde. En la tercera sección demuestro el teorema basándome en la prueba de Lee \cite{lee1}, que a su vez se basa en la demostración clásica de Seifert y Threlfall \cite{seifert}. Finalmente clasifico las superficies con borde, siguiendo las ideas de Massey \cite{massey}.


\section{Superficies como cocientes}

Para el teorema de clasificación necesitamos un método uniforme de representación de las superficies compactas. Trataremos de dar una forma de representarlas como polígonos, y veremos que toda superficie compacta se puede representar en el plano como el cociente de un polígono por una relación de equivalencia que identifica sus aristas dos a dos.\\
Veamos tres ejemplos elementales: la esfera $\Esfera$, el plano proyectivo $\Proyectivo$ y el toro $\Toro$. Como veremos, estas superficies son fundamentales pues toda superficie compacta se puede construir a partir de ellas. 

\begin{prop}%%%%PROP: ESFERA COCIENTE DISCO Y CUADRADO%%%%
\label{prop:Esfera como cociente de disco y cuadrado}
La esfera $\mathbb{S}^2$ es homeomorfa a los siguientes espacios cociente: 
%%%%%%%%%%o cocientes??????????%%%%%
\begin{itemize}
\item[(a)] El disco cerrado $\overline{\mathbb{B}}^2\subseteq \mathbb{R}^2$ módulo la relación de equivalencia generada por $(x,y)\sim (-x,y)$, si $(x,y)\in \partial \overline{\mathbb{B}}^2$
\item[(b)] El cuadrado $S=\{(x,y):|x|+|y|\leq 1\}$ módulo la relación de equivalencia generada por $(x,y)\sim(-x,y)$ si $(x,y)\in \partial S$.
\end{itemize}
\end{prop}
\begin{proof}
Para ver que cada espacio es homeomorfo a la esfera, daremos una aplicación cociente desde cada espacio a la esfera que haga las mismas identificaciones que la relación de equivalencia, y entonces apelaremos a la unicidad del espacio cociente. (\autoref{teo:unicidad_espacio_cociente})\\
Para (a), vamos a definir una aplicación que ``envuelve'' cada paralelo de la esfera con un segmento horizontal del disco (ver \autoref{fig:esfera_cociente_circunferencia}).
Formalmente, esta aplicación $\pi:\overline{\mathbb{B}}^2\to \mathbb{S}^2$ vienen dada por 
$$\pi(x,y)=\left\{\begin{array}{lc}
			(-\sqrt{1-y^2} \cos\dfrac{\pi x}{\sqrt{1-y^2}}, -\sqrt{1-y^2}, y), & y\neq \pm 1 \\
			\\(0,0,y), & y=\pm1 

\end{array}
\right.$$
Es claro que $\pi$ es continua y hace las mismas identificaciones que la relación de equivalencia. Como es sobreyectiva, por el teorema de la aplicación cerrada es una aplicación cociente. % (\autoref{teo:aplicac_cerrada}).
ESTO HAY QUE ARREGLARLO PARA INTENTAR NO APELAR AL B.0.2
Para probar (b), sea $\alpha:S\to \overline{\mathbb{B}}^2$ el homeomorfismo construido a partir de la aplicación dada en la demostración de \autoref{teo:convexo_homeom_esfera} que manda linealmente cada segmento radial entre el origen y la frontera de $S$ a un segmento paralelo a este, que une el centro del disco y su frontera. Hagamos ahora $\beta=\pi \circ \alpha : S \to \mathbb{S}^2$, donde $\pi$ es la aplicación cociente del parágrafo anterior. Tenemos entonces que $\beta$ identifica $(x,y)$ y $(-x,y)$ cuando $(x,y)\in \partial S$, y por otro lado es inyectiva, así que hace las mismas identificaciones que la aplicación cociente definida en (b), completando así la demostración (ver \autoref{fig:esfera_cuadrado}). 
\end{proof}
\begin{figure}%%%%%%% FIGURA: ESFERA COCIENTE CIRCUNFERENCIA%%%%
\begin{center}
\begin{tikzpicture}[line cap=round,line join=round,>=triangle 45,x=1.5cm,y=1.5cm, scale=0.7]
\draw [<-][shift={(0,1.25)}] plot[domain=0.64:2.5,variable=\t]({1*1.25*cos(\t r)+0*1.25*sin(\t r)},{0*1.25*cos(\t r)+1*1.25*sin(\t r)});
\draw (0,3) node[anchor=north] {$\pi$};
\draw (2,1.5) arc[x radius=0.7, y radius=1.5, start angle=90, end angle=270];
\draw [->] (2,0) -- (0.87,-1.68);
\draw(2,0) circle (2.25cm);
\draw(-2,0) circle (2.25cm);
\draw (-3.12,1)-- (-0.88,1);
\draw (-3.41,0.5)-- (-0.59,0.5);
\draw (-3.41,-0.5)-- (-0.59,-0.5);
\draw (-3.12,-1)-- (-0.88,-1);
\draw [->] (2,0) -- (2,2);
\draw [dashed] (3.1,1) arc[x radius=1.1, y radius=0.2, start angle=0, end angle=180];
\draw (3.1,1) arc[x radius=1.1, y radius=0.2, start angle=0, end angle=-180];
\draw [dashed] (3.4,0.5) arc[x radius=1.4, y radius=0.2, start angle=0, end angle=180];
\draw (3.4,0.5) arc[x radius=1.4, y radius=0.2, start angle=0, end angle=-180];
\draw [dashed] (3.5,0) arc[x radius=1.5, y radius=0.2, start angle=0, end angle=180];
\draw (3.5,0) arc[x radius=1.5, y radius=0.2, start angle=0, end angle=-180];
\draw [dashed] (3.4,-0.5) arc[x radius=1.4, y radius=0.2, start angle=0, end angle=180];
\draw (3.4,-0.5) arc[x radius=1.4, y radius=0.2, start angle=0, end angle=-180];
\draw [dashed] (3.10,-1) arc[x radius=1.10, y radius=0.2, start angle=0, end angle=180];
\draw (3.10,-1) arc[x radius=1.10, y radius=0.2, start angle=0, end angle=-180];
\draw (-3.5,0)-- (-0.5,0);
\draw (-0.6,0.1)--(-0.5,0)--(-0.4,0.1);
\draw (-3.6,0.13)--(-3.5,0)--(-3.4,0.13);
\draw [->] (2,0) -- (4,0);
\begin{scriptsize}
\fill [color=black] (-2,1.5) circle (2.0pt);
\fill [color=black] (-2,-1.5) circle (2.0pt);
\fill [color=black] (2,1.5) circle (2.0pt);
\fill [color=black] (2,-1.5) circle (2.0pt);
\fill [color=black] (-2,1.5) circle (2.0pt);
%\fill [color=black] (1.3,-0.17) circle (2.0pt);
\draw (1.2,-0.07)--(1.3,-0.17)--(1.4,-0.07);
\end{scriptsize}
\end{tikzpicture}
\end{center}

\caption{La esfera como cociente del disco $\overline{\mathbb{B}}^2$.\label{fig:esfera_cociente_circunferencia}}
\end{figure}

\begin{figure}%%%%FIGURA: ESFERA CUADRADO
\centering

\begin{tikzpicture}[use optics, line cap=round,line join=round,>=triangle 45,x=1cm,y=1cm, scale=0.7]
\draw[-<-={at=0.125},->-={at=0.375}, -<<-={at=0.625}, ->>-={at=0.875} ](0,0) circle (2cm);
\draw [-<-={at=0.5}] (-8,0)-- (-6,2);
\draw [-<-={at=0.5}](-4,0)-- (-6,2);
\draw [-<<-={at=0.5}](-8,0)-- (-6,-2);
\draw [->>-={at=0.5}](-6,-2)-- (-4,0);
\draw(6,0) circle (2cm);
\draw (8,0) arc[x radius=2, y radius=0.7, start angle=0, end angle=-180];
\draw [dashed] (8,0) arc[x radius=2, y radius=0.7, start angle=0, end angle=180];
\draw [->] (-8.5,0) -- (-3.5,0);
\draw [->] (-6,-2.5) -- (-6,2.5);
\draw [->] (-2.5,0) -- (2.5,0);
\draw [->] (0,-2.5) -- (0,2.5);
\draw [->] (3.5,0) -- (8.5,0);
\draw [->] (6,0) -- (6,2.5);
\draw [->] (6,0) -- (4.51,-2.5);
\draw [<-][shift={(-3,1.25)}] plot[domain=0.64:2.5,variable=\t]({1*1.25*cos(\t r)+0*1.25*sin(\t r)},{0*1.25*cos(\t r)+1*1.25*sin(\t r)});
\draw [<-] [shift={(3,1.25)}] plot[domain=0.64:2.5,variable=\t]({1*1.25*cos(\t r)+0*1.25*sin(\t r)},{0*1.25*cos(\t r)+1*1.25*sin(\t r)});
\draw (-3.1,3.21) node[anchor=north west] {$ \alpha $};
\draw (2.89,3.23) node[anchor=north west] {$ \pi $};
\draw [->-={at=0.45}, -<<-={at=0.82}](6,2) arc[x radius=0.7, y radius=2, start angle=90, end angle=270];
\begin{scriptsize}
\fill [color=black] (0,2) circle (1.5pt);
\fill [color=black] (-8,0) circle (1.5pt);
\fill [color=black] (-4,0) circle (1.5pt);
\fill [color=black] (-6,-2) circle (1.5pt);
\fill [color=black] (-6,2) circle (1.5pt);
\fill [color=black] (-2,0) circle (1.5pt);
\fill [color=black] (2,0) circle (1.5pt);
\fill [color=black] (0,-2) circle (1.5pt);
\fill [color=black] (0,2) circle (1.5pt);
\fill [color=black] (5.34,-0.66) circle (1.5pt);
\end{scriptsize}
\end{tikzpicture}

\caption{La esfera como cociente de un cuadrado.\label{fig:esfera_cuadrado}}


\end{figure}

\begin{prop}%%%%PROP: TORO COMO CUADRADO
\label{prop:toro_cuadrado}
El toro $\Toro$ es homeomorfo al espacio cociente resultante de la relación de equivalencia en el cuadrado $I\times I$ que identifica $(x,0)\sim (x,1)$ para todo $x\in I$, y $(0,y)\sim(1,y)$ para todo $y\in I$ (\autoref{fig:toro_cuadrado}). 
\end{prop}
\begin{proof}
Definimos la aplicación $q:I\times I \to \Toro$ que manda $q(u,v)=(e^{2\pi iu},e^{2\pi iv})$. Por el teorema de la aplicación cerrada (\autoref{teo:aplicac_cerrada}), es una aplicación cociente. Al hacer las mismas identificaciones que la relación de equivalencia, por la unicidad del espacio cociente (\autoref{teo:unicidad_espacio_cociente}) se obtiene el resultado.
\end{proof}

\clearpage

\begin{figure}%%%%FIGURA: TORO COMO CUADRADO
\begin{center}
\begin{tikzpicture}[use optics, line cap=round,line join=round,>=triangle 45,x=1cm, y=1cm, scale=0.7]

\draw [->-={at=0.5}](-8,-2)-- (-8,2);
\draw [->>-={at=0.5}](-8,2)-- (-4,2);
\draw [-<-={at=0.5}](-4,2)-- (-4,-2);
\draw [->>-={at=0.5}](-8,-2)-- (-4,-2);
\draw (-2,1)-- (2,1);
%\draw (-2,1)-- (-2,-1);
%\draw (2,1)-- (2,-1);
\draw (2,-1)-- (-2,-1);
%\draw [-<-={at=0.6]}](2,1) arc[x radius=0.5, y radius=1, start angle=90, end angle=270];
%\draw (2,1) arc[x radius=0.5, y radius=1, start angle=90, end angle=-90];
\draw [-<-={at=0.6]}](-2,1) arc[x radius=0.5, y radius=1, start angle=90, end angle=270];
\draw [dashed] (-2,1) arc[x radius=0.5, y radius=1, start angle=90, end angle=-90];
\filldraw [-<-={at=0.55]}][color=black, fill=black!0] [rotate around={0:(2,0)}] (2,0) ellipse (0.5cm and 1cm);
\draw [->>-={at=0.5}](-2.5,0)-- (1.5,0);
\draw [rotate around={0:(6,0)}] (6,0) ellipse (3cm and 1.5cm);
\draw (7.4,0.1) arc[x radius=1.5, y radius=0.4, start angle=0, end angle=-180];
\draw (7.2,-0.1) arc[x radius=1.3, y radius=0.3, start angle=0, end angle=180]; 
\draw [dashed](6,-1.5) arc[x radius=0.2, y radius=0.6, start angle=-90, end angle=90];
\draw [-<-={at=0.63}] (6,-0.3) arc[x radius=0.2, y radius=0.6, start angle=90, end angle=270];
\begin{scriptsize}
\fill [color=black] (-8,-2) circle (1.5pt);
\fill [color=black] (-8,2) circle (1.5pt);
\fill [color=black] (-4,2) circle (1.5pt);
\fill [color=black] (-4,-2) circle (1.5pt);
\fill [color=black] (-2.5,0) circle (1.5pt);
\fill [color=black] (1.5,0) circle (1.5pt);
\fill [color=black] (5.8,-0.9) circle (1.5pt);
\end{scriptsize}
\end{tikzpicture}
\end{center}
\caption{El toro como cociente de un cuadrado.\label{fig:toro_cuadrado}}
\end{figure}


\begin{prop}%%%%PROP: PLANO PROYECTIVO COMO CUADRADO
\label{prop:proyectivo_cociente_cuadrado}
El plano proyectivo $\mathbb{P}^2$ es homeomorfo a los siguientes espacios cociente:
\begin{itemize}
\item[(a)] El disco cerrado $\overline{\mathbb{B}}^2$ módulo la relación de equivalencia generada por $(x,y) \sim (-x,-y)$ para cada $(x,y)\in \partial \overline{\mathbb{B}}^2$.
\item[(b)] El cuadrado $S=\{(x,y):|x|+|y|\leq 1\} $ módulo la relación de equivalencia generada por $(x,y)\sim (-x,-y) $ para todo $(x,y)\in \partial S$.
\end{itemize}
\end{prop}
\begin{proof}
Sea la relación de equivalencia $\sim$ generada por $(x,y) \sim (-x,-y)$ para cada $(x,y)\in \mathbb{S}^2$, que representa $\mathbb{P}^2$ como el cociente de una esfera en la cual se identifican polos opuestos, y sea $p:\mathbb{S}^2 \to \mathbb{P}^2$ su aplicación cociente. %AQUI FALTA UN POCO DE EXPLICACION, EJEMPLO 4.54 DEL LEE
Si $F:\overline{\mathbb{B}}^2 \to \mathbb{S}^2$ es la aplicación que manda el disco al emisferio norte de la esfera mediante la aplicación $F(x,y)=(x,y,\sqrt{1-x^2-y^2})$, entonces $p\circ F:\overline{\mathbb{B}}^2 \to \mathbb{S}^2/\sim$ es sobreyectiva por serlo $p$ y $F$, y es por tanto una aplicación cociente por el teorema de la aplicación cerrada (\autoref{teo:aplicac_cerrada}). La aplicación identifica únicamente $(x,y)\in \partial \overline{\mathbb{B}}^2$ con $(-x,-y)\in \partial \overline{\mathbb{B}}^2$, por lo que $\mathbb{P}^2$ es homeomorfo al espacio cociente resultante.
Para la parte (b) utilizamos un argumento análogo al de la demostración de la \autoref{prop:Esfera como cociente de disco y cuadrado} (b) 
\end{proof}


\begin{figure}[h]%%%%FIGURA: PLANO PROYECTIVO COMO CUADRADO
\begin{center}
\begin{tikzpicture}[use optics, line cap=round,line join=round,>=triangle 45,x=1.0cm,y=1.0cm, scale=0.6]
\draw[-<-={at=0.5}, -<-={at=1}](-3,0) circle (2cm);
\draw [-<-={at=0.5}] (4,2)-- (6,0);
\draw [-<<-={at=0.5}](6,0)-- (4,-2);
\draw [-<-={at=0.5}](4,-2)-- (2,0);
\draw [-<<-={at=0.5}](2,0)-- (4,2);
\begin{scriptsize}
\fill [color=black] (6,0) circle (1.5pt);
\fill [color=black] (4,2) circle (1.5pt);
\fill [color=black] (2,0) circle (1.5pt);
\fill [color=black] (4,-2) circle (1.5pt);
\fill [color=black] (-3,-2) circle (1.5pt);
\fill [color=black] (-3,2) circle (1.5pt);
\end{scriptsize}
\end{tikzpicture}
\caption{Representación de $\mathbb{P}^2$ como un espacio cociente.\label{fig:plano_proyectivo_cuadrado}}
\end{center}
\end{figure}

De ahora en adelante visualizaremos el plano proyectivo como el \enfatiza{crosscap}, que construmos a partir de la proposición anterior, tal como se sigue en la siguiente figura.




\begin{figure}[h]%%%FIG: CROSSCAP
\centering

\begin{tikzpicture}[use optics, scale=0.7][line cap=round,line join=round,>=triangle 45,x=1.0cm,y=1.0cm]
%1
\fill[gray!10] (-5,5) -- (-7,3) -- (-5,1) -- (-3,3) -- cycle;
\draw [->-](-5,5)-- (-7,3);
\draw [-<<-](-5,5)-- (-3,3);
\draw [-<-](-3,3)-- (-5,1);
\draw [->>-](-7,3)-- (-5,1);

%2
\fill[gray!10] (-2,3.5) arc[ x radius=2, y radius=0.5, start angle=180, end angle=540];
%\draw(0,3) circle (2cm);
\draw [->>-={at=0.125},->-={at=0.375}, ->>-={at=0.625}, ->-={at=0.875}] (-2,3.5) arc[x radius=2, y radius=0.5, start angle=180, end angle=540];
\draw [name path=C] (-2,3.5) arc[x radius=2, y radius=2.5, start angle=180, end angle=360];

%3
\fill [gray!10] (3, 3.5) [out=270, in=225, looseness=0.7] to (5,3.5) [out=45, in=90, looseness=0.7] to (7, 3.5) [out=-90, in=-45, looseness=0.7] to (5,3.5) [out=135, in=90, looseness=0.7] to (3, 3.5);
%\draw[name path=A] (3,3.5) [-<-={at=0.25}, -<<-={at=0.75}](3,3.5) arc[x radius=1, y radius=0.5, start angle=180, end angle=-180];
%\draw [name path=B] (5,3.5) [-<<-={at=0.25}, -<-={at=0.75}] arc[x radius=1, y radius=0.5, start angle=180, end angle=-180];
\draw [name path=C] (3,3.5) arc[x radius=2, y radius=2.5, start angle=180, end angle=360];
\draw [->>-][out=270, in=225, looseness=0.7] (3, 3.5) to (5,3.5);
\draw [-<<-][out=45, in=90, looseness=0.7] (5,3.5) to (7,3.5);
\draw [-<-][out=90, in=135, looseness=0.7] (3, 3.5) to (5,3.5);
\draw [->-][out=-45, in=-90, looseness=0.7] (5,3.5) to (7,3.5);

%3B
\fill[gray!20] (-7,-3) [out=90, in=135, looseness=2.3] to (-5,-3) [out=135, in=90, looseness=0.7] to (-7,-3);
\fill[gray!10] (-5,-3) [out=225, in=270, looseness=0.7] to (-7,-3) [out=90, in=135] to (-5,-3) ;

\draw [name path=C] (-7,-3) arc[x radius=2, y radius=2.5, start angle=180, end angle=360];
\draw [->>-][out=270, in=225, looseness=0.7] (-7, -3) to (-5,-3);
\draw [-<<-, dashed][out=45, in=90, looseness=0.7] (-5,-3) to (-3,-3);
\draw [-<-][out=90, in=135, looseness=2.3] (-7, -3) to (-5,-3);
\draw [->-][out=45, in=90, looseness=2.3] (-5,-3) to (-3,-3);
\draw [color=gray, dashed][out=90, in=135, looseness=0.7] (-7, -3) to (-5,-3);
\draw [color=gray][out=-45, in=-90, looseness=0.7] (-5, -3) to (-3,-3);

%4
\fill[gray!10] (0,-3) [out=225, in=270, looseness=0.7] to (-2,-3) [out=90, in=135] to (0,-3) ;
\fill[gray!20] (-2,-3) [out=90, in=180] to (0,-1.5)  -- (0, -3) [out=135, in=90, looseness=0.7] to (-2,-3);
\draw [out=90,in=360] (2,-3) to (0,-1.5);
\draw [out=180, in=90] (0, -1.5) to (-2,-3);
\draw [->>-][out=270, in=225, looseness=0.7] (-2, -3) to (0,-3);
\draw [dashed] [-<<-][out=45, in=90, looseness=0.7] (0, -3) to (2,-3);
%\draw (-5,-3) [->>-] arc[x radius=1, y radius=0.5, start angle=180, end angle=360];
\draw (-2,-3) arc[x radius=2, y radius=2.5, start angle=180, end angle=360];
\draw [->-](0,-3)-- (0,-1.5);
\draw [color=gray, dashed][out=90, in=135, looseness=0.7] (-2, -3) to (0,-3);
\draw [color=gray][out=-45, in=-90, looseness=0.7] (0, -3) to (2,-3);




%5
\draw [-<<-] (5,-3)-- (5,-1.5);
\draw (3,-3) arc[x radius=2, y radius=2.5, start angle=180, end angle=360];
\draw [out=90,in=360] (7,-3) to (5,-1.5);
\draw [out=180, in=90] (5, -1.5) to (3,-3);
\draw [color=gray][out=270, in=225, looseness=0.7] (3,-3) to (5,-3);
\draw [color=gray][dashed][out=45, in=90, looseness=0.7] (5,-3) to (7,-3);
\draw [color=gray][dashed][out=90, in=135, looseness=0.7] (3,-3) to (5,-3);
\draw [color=gray][out=-45, in=-90, looseness=0.7] (5,-3) to (7,-3);

\begin{scriptsize}
\fill [color=black] (-7,3) circle (1.5pt);
\fill [color=black] (-5,5) circle (1.5pt);
\fill [color=black] (-3,3) circle (1.5pt);
\fill [color=black] (-5,1) circle (1.5pt);
\fill [color=black] (-2,3.5) circle (1.5pt);
\fill [color=black] (3,3.5) circle (1.5pt);
\fill [color=black] (7,3.5) circle (1.5pt);
\fill [color=black] (5,3.5) circle (1.5pt);
\fill [color=black] (2,3.5) circle (1.5pt);
\fill [color=black] (0,4) circle (1.5pt);
\fill [color=black] (0,3) circle (1.5pt);
\fill [color=black] (2,-3) circle (1.5pt);
\fill [color=black] (-2,-3) circle (1.5pt);
\fill [color=black] (0,-3) circle (1.5pt);
\fill [color=black] (5,-1.5) circle (1.5pt);
\fill [color=black] (5,-3) circle (1.5pt);
\fill [color=black] (-7,-3) circle (1.5pt);
\fill [color=black] (-5,-3) circle (1.5pt);
\fill [color=black] (-3,-3) circle (1.5pt);
\end{scriptsize}
\end{tikzpicture}
\caption{Crosscap.\label{fig:crosscap_paso_a_paso}}
\end{figure}



En las anteriores proposiciones hemos visto una o varias formas de representar superficies dadas ciertas construcciones geométricas. En estos casos hemos dado aplicaciones y demostraciones concretas para validar nuestros argumentos, pero a medida que aumenta la sofisticación es más útil guiarse visualmente por las figuras construidas. Por ello debemos formalizar un método para construir superficies identificando aristas de figuras geométricas del plano.
Daremos por sabidas las definiciones básicas de símplices CW-complejos, que dejamos en el \autoref{sec:CW}.

\begin{defin}%%%DEF:Polígono
Un \enfatiza{polígono} es un subconjunto de $\R^2$ que es homeomorfo a $\mathbb{S}^1$ y está formado por un número finito de segmentos, que llamaremos \enfatiza{aristas} y que se intersecan sólo en sus extremos, que llamaremos \enfatiza{vértices}. %Los $0$-símplices y $1$-símplices del poígono son respectivamente sus \enfatiza{vértices} y sus \enfatiza{bordes}. Del lema \autoref{lemma:cw} se sigue que un borde yace exactamente en dos vértices.
\end{defin}

\begin{defin}%%%DEF:Región poligonal

Una \enfatiza{región poligonal} es un subconjunto compacto de $\R^2$ cuyo interior es homeomorfo al disco $\mathbb{B}^2$ y cuya frontera es un polígono.
%Una \enfatiza{región poligonal} es un subconjunto compacto de $\R^2$ cuyo interior es una bola coordenada y cuya frontera es un polígono.
A los vértices y aristas del polígono de la frontera también los llamamos vértices y aristas de la región poligonal.
\end{defin} 

Veamos pues que identificando aristas de regiones poligonales de par en par obtenemos un espacio cociente que es siempre una superficie:

\begin{prop}%%%PROP: Teorema poligonos
\label{prop:poligonos}
Sean $P_1,\dots, P_k$ regiones poligonales en el plano, y sea $P=P_1\amalg \dots \amalg P_k$, y supongamos dada una relación de equivalencia en $P$ que identifica algunas aristas de los polígonos con otros por homeomorfismos afines. Entonces se tiene:
\begin{itemize}
%\item[(a)] El espacio cociente resultante es un CW-complejo $2$-dimensional cuyo $0$-esqueleto es la imagen del conjunto de vértices de $P$ por la aplicación cociente, y cuyo $1$-esqueleto es la imagen de la unión de las aristas de las regiones poligonales.
\item[(a)] Si la relación de equivalencia identifica cada arista de cada $P_i$ con exactamente otra arista de un $P_j$ (no necesariamente $i\neq j$), entonces el espacio cociente resultante es una superficie compacta.
\item[(b)] Si para algunos $P_i$ la relación de equivalencia identifica alguna arista suya con exáctamente otra arista de un $P_j$ (no necesariamente $i\neq j$), y para las aristas restantes no hay ninguna identificación, entonces el espacio cociente resultante es una superficie con borde compacta.
\end{itemize}
\end{prop}

\begin{proof} 
%Sea $M$ el espacio cociente, sea $\pi :P\to M$ la aplicación cociente y sean $M_0, \, M_1, \, M_2=M$ respectivamente las imágenes por $\pi$ de los vértices, las aristas y las regiones poligonales. Por la propia definición de $P$, $M_0$ es un espacio discreto, y para $k=1,2$, $M_k$ se obtiene a partir de $M_{k-1}$ pegando un número finito de $k$-celdas. Por tanto $M$ es un CW-complejo (\autoref{def:cw}).\\
Para demostrar esta propiedad seguiremos una idea similar a la de la \autoref{prop:suma_conexa}. Para probar (a), denotemos por $M$ el espacio cociente, y sea $\pi :P\to M$ la proyección.  %, entonces, por la \autoref{prop:cw_variedad}, $M$ será una superficie.\\
Por un lado, los puntos que provienen del interior de cada región poligonal son abiertos en $M$ por definición, y por lo tanto son entornos euclideos de cada uno de sus puntos. Así, tenemos que aquí $M$ es localmente euclídea, y además Hausdorff y IIAN \\
Sea ahora $D$ la imagen por $\pi$ del interior de una arista de una región poligonal, y sea $d\in D$. Veamos que $d$ tiene un entorno Euclideo. Por un lado, $d$ tiene exáctamente dos preimágenes $x$ e $y$, cada una en el interior de una arista distinta $D_1$ y $D_2$. Supongamos sin pérdida de generalidad que estas dos aristas pertenecen respectivamente a $P_1$ y $P_2$ ($P_1$ puede ser igual a $P_2$), y denotemos por $h:D_2\to D_1$ un homeomorfismo tal que $h(y)=x$. Dado que cada $P_i$ es una variedad con borde, y $x,y$ son puntos borde, podemos elegir cartas coordenadas $(U,\varphi)$  para $P_1$ y $(V,\psi)$ para $P_2$ tal que $x\in U$, $y\in V$. Denotamos $\widehat{U}=\varphi(U),\widehat{V}=\psi(V)\subseteq \mathbb{H}^2$ y podemos asumir, contrayendo $U$ y $V$ si es necesario, que $h(V\cap D_2)=U\cap D_1$, y que $\widehat{U}=U_0\times [0,\varepsilon)$, $\widehat{V}=V_0\times [0,\varepsilon)$  para un $\varepsilon >0$ y unos subconjuntos $U_0, V_0 \subset \R$ como se muestra en la \autoref{fig:6a4a} AUNPORHACEr. Así pues podemos escribir las aplicaciones coordenadas como $\varphi(x)=(\varphi_0(x),\varphi_1(x))$, $\psi(x)=(\psi_0(y),\psi_1(y))$ con $\varphi_0:U\to U_0$, $\varphi_1:U\to [0,\varepsilon)$, $\psi_0:V\to V_0$ y $\psi_1:V\to [0,\varepsilon)$ aplicaciones continuas. Que $x$ e $y$ sean puntos borde significa que $\varphi_1(x)=\psi_1(y)=0$.\\
Queremos ensamblar estas dos cartas con una aplicación cuya imagen sea un abierto de $\R^2$, pegándolas por los puntos que se corresponden en $D_1$ y $D_2$. En la demostración de la \autoref{prop:suma_conexa} teníamos un homeomorfismo entre los puntos borde de los entornos, pero en este caso el problema es que las aplicaciones $\varphi$ y $\psi$ no tienen por qué llevar puntos borde que se correspondan al mismo punto imagen, y por tanto tenemos que ajustarlo. Se tiene que las dos restricciones 
$$\begin{array}{lr}
\varphi_0|_{U\cap D_1}: U\cap D_1\to U_0  & \psi_0|_{V\cap D_2}:V\cap D_2\to V_0
\end{array}$$
son homeomorfismos, y definimos así el homeomorfismo $\beta:V_0\to U_0$ por $$\beta=(\varphi_0|_{U\cap D_1})\circ h\circ (\psi_0|_{V\cap D_2})^{-1}\, .$$ 
Sea ahora $B:\widehat{V}\to \R^2$ la aplicación $$B(y_1,y_2)=\left(\beta (y_1), -y_2\right)\, .$$
Geométricamente, $B$ actúa juntando cada punto borde de acuerdo con $\beta$, y dando la vuelta al segmento que está por encima de él, llevándolo a un segmento por debajo (como en la Figura AUN POR HACER \autoref{fig:6a4a.}). Esta construcción nos asegura que 
\begin{equation}\label{ec:1}
B\circ \psi(y)=\left(\beta \circ \psi_0(y),0\right)=\left(\varphi_0 \circ h(y),0\right)=\varphi \circ h(y).
\end{equation}
Definimos ahora $\tilde{\Phi}:U\amalg V\to \R^2$ por $$\tilde{\Phi}(y)=\begin{cases}
\varphi(y), & y\in U,\\
B\circ \psi (y), & y\in V.\\
\end{cases}$$
Dado que $U\amalg V$ es un abierto saturado de $M$, la restricción $\pi|_{U\amalg V}:U\amalg V\to \pi (U\amalg V)$ es una aplicación cociente en el entorno $\pi (U\amalg V)$ de $d$, y por (\ref{ec:1}) $\tilde{\Phi}$ pasa al cociente y define una aplicación continua e inyectiva $\Phi :\pi (U\amalg V)\to \R^2$. Como $\varphi$, $\psi$ y $B$ son homeomorfismos entre sus dominios y sus imágenes, podemos definir la inversa de $\Phi$ de la siguiente manera:
$$\Phi ^{-1}(s)= \begin{cases} 
\pi \circ \varphi ^{-1}(s), & s_2\geq 0, \\
\pi \circ \psi ^{-1} \circ B^{-1}(s), & s_2\leq 0.\\
 \end{cases}$$
Dado que las dos partes de la función son iguales donde coinciden, la aplicación es continua, y por tanto $\Phi$ es un homeomorfismo. Así pues, $U\amalg V$ es un entorno Euclideo de $d$. Un argumento análogo al de la demostración de la Proposición \ref{prop:suma_conexa} nos indica que es Hausdorff y IIAN.\\
Para finalizar, sea $v\in M$ tal que su preimagen por $\pi$ es un conjunto de vértices $\left\{v_1,\dots ,v_k\right\} \subseteq P$. Para cada uno de estos vértices podemos elegir un $\varepsilon>0$ tal que el disco $B_{\varepsilon}(v_i)$ no contenga ningun vértice de la región poligonal $P_j$ a la que pertenece $v_i$ a parte de sí mismo, y tal que no interseca más aristas que las que lo contienen. $B_{\varepsilon}(v_i)\cap P_j$ es homeomorfo a un subconjunto $\Lambda_i$ de $\R^2$ definido por la intersección de dos semiplanos cuyas fronteras coinciden en un único punto y tal que el ángulo que forman es de $2\pi /k$, y que descrito en coordenadas polares queda $\Lambda_i =\left\{ (r,\theta ) \mid \theta_0 \leq \theta \leq \theta_0 + 2k\pi \right\}$. El homeomorfismo viene dado por una traslación del vértice $v_i$ al origen junto con una aplicación que en coordenadas polares es de la forma $(r,\theta ) \mapsto (r,\theta_0 +c\theta )$, con $c$ y $\theta_0$ constantes. Como cada arista está asociada exáctamente a otra arista, podemos ahora construir una aplicación que envíe cada una de las $\Lambda_i$ a un conjunto que contenga un entorno del origen y que, despues de reescalar si es necesario, respete las identificaciones entre las aristas de los $P_j$ (ver \autoref{fig:6.4}). Esta aplicación está definida sobre un conjunto saturado de $P$, que al pasar al cociente determina un homeomorfismo de $v$ a un entorno del origen de $\R^2$. HAUSDORFF Y IIAN?

Finalmente, para probar (b), volvemos a ver cada polígono como una superficie con borde. Las imágenes por $\pi$ de los puntos del interior de aristas que se identifican son localmente euclídeos por la demostración del apartado anterior. Por otro lado, para los puntos del borde que provengan de una arista y que tengan sólo una preimagen podemos encontrar como semibolas coordenadas las mismas de antes de pasar al cociente. Finalmente para las imágenes por $\pi$ de los vértices podemos encontrar una bola o una semibola coordenada con una construcción similar a la de la demostración anterior, dependiendo de si estas imágenes están contenidas en el borde o no. Por tanto se tiene que el espacio cociente resultante es una superficie con borde. HAUSDORFF Y IIAN?
\end{proof}
\begin{figure}
\centering
\begin{tikzpicture}[use optics, line cap=round,line join=round,>=triangle 45,x=1cm,y=1cm, scale=0.3]
%\clip(-15.330160097387678,-13.834559140936781) rectangle (29.900136203459667,16.920192433770143);
\draw [dashed, shift={(-8.120466412995146,8.354581757803452)},fill=black,fill opacity=0.10000000149011612] (0,0) -- (-117.75854060106003:1.387432401866483) arc (-117.75854060106003:11.093723011557817:1.387432401866483) -- cycle;
\draw [dashed, shift={(-1.7579751439008828,5.15614020090742)},fill=black,fill opacity=0.10000000149011612] (0,0) -- (126.29777621005324:1.387432401866483) arc (126.29777621005324:175.30713749441017:1.387432401866483) -- cycle;
\draw [dashed, shift={(-0.7749302637196742,-4.954712812271268)}, fill=black,fill opacity=0.10000000149011612] (0,0) -- (90.70690247273468:1.387432401866483) arc (90.70690247273468:130.45547367349133:1.387432401866483) -- cycle;
\draw [dashed, shift={(-5,0)},fill=black,fill opacity=0.10000000149011612] (0,0) -- (-49.54452632650868:1.387432401866483) arc (-49.54452632650868:25.10774294664457:1.387432401866483) -- cycle;
\draw   (-5,0)-- (-0.7749302637196742,-4.954712812271268);
\draw   (-0.7749302637196742,-4.954712812271268)-- (-0.86,1.94);
\draw   (-0.86,1.94)-- (-5,0);
\draw   (-6.3664823334069975,5.534450492583295)-- (-1.7579751439008828,5.15614020090742);
\draw   (-1.7579751439008828,5.15614020090742)-- (-4.612498253818849,9.042418651759586);
\draw   (-4.612498253818849,9.042418651759586)-- (-8.120466412995146,8.354581757803452);
\draw   (-8.120466412995146,8.354581757803452)-- (-9.840058647885487,5.087356511511806);
\draw   (-9.840058647885487,5.087356511511806)-- (-8.670735928160054,1.8889149546157749);
\draw   (-8.670735928160054,1.8889149546157749)-- (-6.3664823334069975,5.534450492583295);
\draw   (7.949566411767295,8.090769939874141)-- (7.949566411767295,6.090769939874141);
\draw   (7.949566411767295,6.090769939874141)-- (9.949566411767293,6.090769939874141);
\draw   (7.949566411767295,4.090769939874141)-- (7.949566411767295,2.090769939874141);
\draw   (7.949566411767295,2.090769939874141)-- (9.949566411767293,2.090769939874141);
\draw   (7.949566411767295,0)-- (7.949566411767295,-1.909230060125859);
\draw   (7.949566411767295,-1.909230060125859)-- (9.949566411767293,-1.909230060125859);
\draw   (7.949566411767295,-3.909230060125859)-- (7.949566411767295,-5.909230060125859);
\draw   (7.949566411767295,-5.909230060125859)-- (9.949566411767293,-5.909230060125859);
\draw [shift={(7.949566411767295,6.090769939874141)}, fill=black, opacity=0.10000000149011612]  plot[domain=0:1.5707963267948966,variable=\t]({1*1*cos(\t r)+0*1*sin(\t r)},{0*1*cos(\t r)+1*1*sin(\t r)})--(0,0);
\draw [shift={(7.949566411767295,2.090769939874141)}, fill=black, opacity=0.10000000149011612]  plot[domain=0:1.5707963267948966,variable=\t]({1*1*cos(\t r)+0*1*sin(\t r)},{0*1*cos(\t r)+1*1*sin(\t r)})--(0,0);
\draw [shift={(7.949566411767295,-5.909230060125859)}, fill=black, opacity=0.10000000149011612]  plot[domain=0:1.5707963267948966,variable=\t]({1*1*cos(\t r)+0*1*sin(\t r)},{0*1*cos(\t r)+1*1*sin(\t r)})--(0,0);
\draw [shift={(7.949566411767295,-1.909230060125859)}, fill=black, opacity=0.10000000149011612]  plot[domain=0:1.5707963267948966,variable=\t]({1*1*cos(\t r)+0*1*sin(\t r)},{0*1*cos(\t r)+1*1*sin(\t r)})--(0,0);


\draw [shift={(7.949566411767295,6.090769939874141)}, dashed]  plot[domain=0:1.5707963267948966,variable=\t]({1*1*cos(\t r)+0*1*sin(\t r)},{0*1*cos(\t r)+1*1*sin(\t r)});
\draw [shift={(7.949566411767295,2.090769939874141)}, dashed]  plot[domain=0:1.5707963267948966,variable=\t]({1*1*cos(\t r)+0*1*sin(\t r)},{0*1*cos(\t r)+1*1*sin(\t r)});
\draw [shift={(7.949566411767295,-5.909230060125859)},dashed]  plot[domain=0:1.5707963267948966,variable=\t]({1*1*cos(\t r)+0*1*sin(\t r)},{0*1*cos(\t r)+1*1*sin(\t r)});
\draw [shift={(7.949566411767295,-1.909230060125859)}, dashed]  plot[domain=0:1.5707963267948966,variable=\t]({1*1*cos(\t r)+0*1*sin(\t r)},{0*1*cos(\t r)+1*1*sin(\t r)});

\draw  [fill=black,opacity=0.10000000149011612]  (16,0) circle (1.2148083407586547cm);
\draw   (14,0)-- (18,0);
\draw   (16,2)-- (16,-2);
\draw   [dashed] (16,0) circle (1.2148083407586547cm);
\draw [fill=black] (-5,0) circle ( 4pt);
\draw [fill=black] (-0.7749302637196742,-4.954712812271268) circle ( 4pt);
\draw [fill=black] (-0.86,1.94) circle ( 4pt);
\draw [fill=black] (-6.3664823334069975,5.534450492583295) circle ( 4pt);
\draw [fill=black] (-1.7579751439008828,5.15614020090742) circle ( 4pt);
\draw [fill=black] (-4.612498253818849,9.042418651759586) circle ( 4pt);
\draw [fill=black] (-8.120466412995146,8.354581757803452) circle ( 4pt);
\draw [fill=black] (-9.840058647885487,5.087356511511806) circle ( 4pt);
\draw [fill=black] (-8.670735928160054,1.8889149546157749) circle ( 4pt);
%\draw [fill=black] (7.949566411767295,8.090769939874141) circle ( 4pt);
\draw [fill=black] (7.949566411767295,6.090769939874141) circle ( 4pt);
%\draw [fill=black] (9.949566411767293,6.090769939874141) circle ( 4pt);
%\draw [fill=black] (7.949566411767295,4.090769939874141) circle ( 4pt);
\draw [fill=black] (7.949566411767295,2.090769939874141) circle ( 4pt);
%\draw [fill=black] (9.949566411767293,2.090769939874141) circle ( 4pt);
%\draw [fill=black] (7.949566411767295,0) circle ( 4pt);
\draw [fill=black] (7.949566411767295,-1.909230060125859) circle ( 4pt);
%\draw [fill=black] (9.949566411767293,-1.909230060125859) circle ( 4pt);
%\draw [fill=black] (7.949566411767295,-3.909230060125859) circle ( 4pt);
\draw [fill=black] (7.949566411767295,-5.909230060125859) circle ( 4pt);
%\draw [fill=black] (9.949566411767293,-5.909230060125859) circle ( 4pt);
\draw [fill=black] (16,0) circle (4pt);


\draw [->] (-7.2,9.27) [out=45, in=150] to (-1.3,9) [out=-30,in=180] to (6,6.8); 
\draw [->] (0,5) [out=0, in= 180] to (6,2.4);
\draw [->] (-4,2) [out=45, in=180] to (-.4, 3.5) [out=0, in=180] to (6,-1.6);
\draw [->] (0,-4.5) [out=0, in=180] to (6,-5.6);
\draw [->] (11,0) [out=45, in=135] to (14,1);
\end{tikzpicture}
\caption{Entorno euclídeo de un vértice.\label{fig:6.4}}
\end{figure}

\begin{eje}%%%%EJE: Botella Klein
La \enfatiza{botella de Klein} es la superficie $K$ obtenida identificando las aristas del cuadrado $I\times I$ de acuerdo a $(0,t)\sim (1,t)$ y $(t,0)\sim (1-t,1)$ para $0\leq t\leq 1$.  Para visualizar $K$, podemos pensar en pegar las aristas izquierda y derecha creando un cilindro, y luego hacer pasar el extremo superior por la parte inferior del cilindro, para finalmente pegar los dos extremos (ver \autoref{fig:klein}).
\end{eje}

\begin{eje}\label{eje:rep_mobius}
Veamos ahora como obtener las figuras del \autoref{eje:rep_borde}. El cilindro $Cil$ es la superficie con borde que se obtiene al identificar las aristas de un cuadrado $I\times I$ dada la relación de equivalencia $(0,y)\sim (1,y),$ con $y\in I$ (\autoref{fig:ident_cil}). La banda de Möbius $Mob$ es la superficie con borde que se obtiene identificando las aristas del cuadrado $I\times I$ de acuerdo a la relación $(0,y)\sim (1,1-y)$, $y\in I$ (\autoref{fig:ident_mob}).
\end{eje}

\begin{figure}[h]%%%%FIG: Klein
\centering
\begin{tikzpicture} [use optics]
%\draw [help lines] (-6,-2) grid (6,2);
%1
\draw[-<<-={at=0.125}, -<-={at=0.375}, ->>-={at=0.625}, -<-={at=0.875}] (-5,-1) -- (-5,1) -- (-3,1) -- (-3, -1) -- (-5,-1);
%2
\draw[-<<-] (-1.7, -1.085)--+(0,2);
\draw [-<-={at=0.43}, -<-={at=0.897}] (-2,-1) -- +(0,2) arc[x radius=0.5, y radius=0.1, start angle=180, end angle=360] -- +(0,-2) arc[x radius=0.5, y radius=0.1, start angle=0, end angle=-180];
\draw [dashed] (-2,-1) arc[x radius=0.5, y radius=0.1, start angle=180, end angle=0];
\draw (-2,1) arc[x radius=0.5, y radius=0.1, start angle=180, end angle=0];
%3
\draw [->-={at=0.535}, ->-={at=0.974}](0.2,-0.5) [out=80, in=180] to (1.64,1.5) [out=0, in=100] to (3,-1) arc [x radius=0.35, y radius=0.1, start angle=0, end angle=-180] [out=85, in=0] to (1.64,1) [out=180, in=95] to (0.75, -0.5) arc[x radius=0.275, y radius=0.1, start angle=0, end angle=-180];
\draw [dashed] (0.2,-0.5) arc[x radius=0.275, y radius=0.1, start angle=180, end angle=0] (3,-1) arc[x radius=0.35, y radius=0.1, start angle=0, end angle=180];

%4
\draw [->-={at=0.07}][dash pattern= on 52.5pt off 2pt on 2pt off 2pt on 2pt off 2pt on 79.5pt off 2pt on 2pt off 2pt on 2pt off 2pt on 2pt off 2pt on 2pt off 2pt on 2.2pt off 2pt on 2pt off 2pt on 2pt off 2pt on 2pt off 2pt on 2pt off 2pt on 2pt off 2pt on 200pt]
	(5,-1) arc[x radius=0.4, y radius=0.1, start angle=0, end angle=-180] [out=135, in= 270] to +(-0.2,0.6) [out=90, in=270] to +(0.5,1) [out=90, in=0] to (4.3,1) [out=180, in=90] to +(-0.2,-0.3) [out=270, in=135] to +(0.5, -0.8) [out=-65, in=-65] to +(-0.3,-0.2) [out=-80, in=80] to (4.2,-1) arc [x radius=0.4, y radius=0.1, start angle=180, end angle=0] ;

\draw [dash pattern= on 2pt off 2pt on 2pt off 2pt on 2pt off 2pt on 2pt off 2pt on 2pt off 2pt on 2pt off 2pt on 2pt off 2pt on 2pt off 2pt on 2pt off 2pt on 2pt off 2pt on 180pt ](5,-1)[out=100, in=-45] to (4.6,-0.1) [out=115, in=115] to +(-0.3,-0.2) [out=135, in=270] to +(-0.5,1) [out=90, in=180] to (4.4,1.3) [out=0, in=90] to +(0.4, -0.7) [out=-90, in=90] to +(0.4,-1) [out=-90, in=45] to (5,-1);


\draw [-Stealth](-.4,1.5) arc[x radius=0.4, y radius=0.2, start angle=150, end angle=30];
\draw [-Stealth](-2.8,1.5) arc[x radius=0.4, y radius=0.2, start angle=150, end angle=30];
\draw [-Stealth](3,1.5) arc[x radius=0.4, y radius=0.2, start angle=150, end angle=30];
\end{tikzpicture}
\caption{Construcción de la botella de Klein.\label{fig:klein}}
\end{figure}
\begin{figure}[h]
\centering
\begin{subfigure}{0.5\textwidth}
\centering

\begin{tikzpicture}[use optics, scale=.9]
\draw (-3,1)--(-1,1);
\draw [->-](-1,1)--(-1,-1);
\draw  (-1,-1)--(-3,-1);
\draw [-<-] (-3,-1)--(-3,1);
\draw (1,1) arc [x radius=0.8, y radius=0.2, start angle=180, end angle=0];
\draw [dashed](1,-1) arc [x radius= 0.8, y radius= 0.2, start angle=180, end angle=0];
\draw (1,1) arc [x radius=0.8, y radius=0.2, start angle= 180, end angle= 360] -- +(0,-2) arc [x radius=0.8, y radius=0.2, start angle=0, end angle=-180] -- cycle;
\draw [->-] (2.04,0.81) -- +(0,-2);
\end{tikzpicture}
\caption{El cilindro $Cil$.\label{fig:ident_cil}}
\end{subfigure}
\begin{subfigure}{.5\textwidth}
\centering
\begin{tikzpicture}[use optics, scale=1.8]
\draw [->-](-2,1)--(-2,0);
\draw (-2,0)--(-3,0);
\draw [->-](-3,0)--(-3,1);
\draw (-3,1)--(-2,1);


\draw (1,.7)[out=-90, in=45] to (.95,.6) [out=-135,in=0] to (-.5, -.0) [out=180, in=-20] to (-1,.1) -- (-1,.7) arc [x radius=1, y radius=0.2, start angle=180, end angle=0] -- (1,.1) [out=-160, in=-10] to (-.3,0.006);
\draw (-.0,.05) [out=160, in=-45] to (-.95,.6) [out=135, in=-90] to (-1,.7);
\draw (-.7,.39) arc [x radius=.7, y radius=.1, start angle=180, end angle=-4];
\draw [-<-] (0,0.9)--(0,.49);

\end{tikzpicture}
\caption{La banda de Möbius $Mob$.\label{fig:ident_mob}}
\end{subfigure}
\caption{Superficies con borde como cocientes topológicos.}
\end{figure}

\section{Representación de superficies}
Vamos ahora a dar un método uniforme para representar todas las superficies compactas (con y sin borde) como regiones poligonales del plano. A su vez asociaremos a cada región poligonal, y por tanto a cada superficie compacta, una secuencia de símbolos llamada palabra. 


\begin{defin}%%%%DEF: palabra
Sea $S$ un conjunto. Una \textbf{\textit{palabra en $S$}} es una $k$-tupla ordenada de símbolos, cada uno de los cuales aparece de la forma $a$ o $a^{-1}$, para cierto $a\in S$.
\end{defin}

\begin{defin}%%%%DEF: Representación poligonal
\label{def:rep_pol}
Una \textbf{\textit{representación poligonal}}, que denotaremos por $$\mathcal{P}=\langle S\mid W_1,\dots ,W_k\rangle$$ está formada por un conjunto finito S junto con un número finito de palabras $W_1,\dots ,W_k$ en $S$ de longitud mayor o igual que $3$, tal que que para todo $a\in S$ existe un $W_i$ tal que $a\in W_i$. Cuando el conjunto $S$ esté descrito listando sus elementos, quitaremos los corchetes que rodean los elementos de $S$ y denotaremos las palabras $W_i$ por yuxtaposición. Por ejemplo, la presentación con $S=\{a,b\}$ y la palabra $W=(a,b,a^{-1},b^{-1})$ la escribiremos $\langle a,b\mid  aba^{-1}b^{-1}\rangle$. 

Permitimos también el caso en que $S=\{a\}$ (u otro símbolo cualquiera) y que $\mathcal{P}$ tenga una sola palabra de longitud $2$, es decir, $\langle a\mid aa\rangle$, $\langle a\mid a^{-1}a^{-1}\rangle$, $\langle a\mid aa^{-1}\rangle$ y $\langle a\mid a^{-1}a\rangle$.
\begin{comment} o $1$ \end{comment} 
\begin{comment}, $\langle a\mid a\rangle$ y $\langle a\mid a^{-1}\rangle$ \end{comment}
\end{defin}

\begin{defin}%%%%DEF: Realización geométrica
Una representación poligonal $\mathcal{P}$ da lugar a un espacio topológico $|\mathcal{P}|$, llamado \textbf{\textit{realización geométrica de $\mathcal{P}$}}, que se obtiene de la siguiente manera:
\begin{itemize}
\item[1.] Para cada $W_i\in \mathcal{P}$ de longitud $k$, sea $P_i$ una región poligonal convexa con $k$ aristas, centrada en el origen, con aristas de longitud 1, ángulos iguales y tal que un vértice yaczca sobre el eje $OY$.
\item[2.] Definimos una biyección (que llamaremos \textit{etiquetado}) entre los símbolos de $W_ i$ y las aristas de $P_i$, en sentido contrario a las agujas del reloj y empezando por la que yace en el eje $OY$.
\item[3.] Sea $|\mathcal{P}|$ el espacio cociente de $\coprod_i P_i$ determinado por la identificación de las aristas que tengan el mismo símbolo, conforme al homeomorfismo afín que hace coincidir los primeros vértices de las aristas con una etiqueta dada $a$ y los últimos vertices de las que tienen la correspondiente etiqueta $a^{-1}$ (en el sentido contrario a las agujas del reloj).

\end{itemize}


Si $\mathcal{P}$ es una de las representaciones poligonales de un solo elemento, definimos $|\mathcal{P}|$ como la esfera $\mathbb{S}^2$ si la palabra es $aa^{-1}$ o $a^{-1}a$, o como el plano proyectivo $\mathbb{P}^2$ si la palabra es $aa$ o $a^{-1}a^{-1}$.% También decimos que $|\mathcal{P}|$ es el disco cerrado $\overline{\mathbb{B}}^2$ si la palabra es $a$ o $a^{-1}$.
\end{defin}

Dadas dos palabras $W_1$ y $W_2$, $W_1W_2$ representará la palabra formada concatenando los símbolos de $W_1$ y $W_2$. Por otro lado, adoptaremos la convención de que $(a^{-1})^{-1}=a$.

También, en lo que sigue, $S$ denotará una secuencia cualquiera de símbolos, $a,b,c,a_1,a_2,\dots$ símbolos de $S$, $e$ un símbolo que no sea de $S$ y $W_1, W_2, \dots$ palabras formadas por símbolos de $S$.

\begin{defin}%%%%DEF: caras, lados y vértices polígono
Los interiores, las aristas y los vértices de cada región polgonal $P_i$ se llaman \textbf{\emph{caras, aristas y vértices de la representación}}. El número de caras es el mismo que el número de palabras, y el número de aristas coincide con la suma de la longitud de las palabras.
Para una arista etiquetada con $a$, el \textbf{\emph{vértice inicial}} es el primero en el sentido contrario de las agujas del reloj, y el otro es el \textbf{\emph{vértice final}}. Para una arista etiquetada $a^{-1}$, estas definiciones se invierten. 
\end{defin}

\begin{defin}%%%%DEF: representación de una superficie
Sea $\mathcal{P} =\langle S\mid W_1,\dots ,W_k\rangle$ una representación poligonal. 
\begin{itemize}
\item[(i)] Decimos que $\mathcal{P}$ es una \enfatiza{representación de una superficie} si para todo $a\in S$, $a$ ocurre exáctamente dos veces en $W_1,...,W_k$ como $a$ o como $a^{-1}$.
\item[(ii)] Si en cambio en la representación poligonal cada símbolo $a\in S$ ocurre una o dos veces en $W_1,\dots ,W_k$ como $a$ o $a^{-1}$, diremos que $\mathcal{P}$ es una \enfatiza{representación de una superficie con borde}. 
\end{itemize}
\end{defin}
\begin{obs}%M%%%:OBS: Representación superficie con/sin borde
Por la \autoref{prop:poligonos}, la realización geométrica de una representación de una superficie es una superficie compacta, y la realización geométrica de una representación de una superficie con borde es una superficie con borde compacta.\\
En el apartado (ii) de la definición anterior dejamos la posibilidad de que cada símbolo aparezca exáctamente dos veces, como en (i). Esto no es ninguna ambigüedad pues como se expuso en la \autoref{sec:borde} una superficie sin borde es siempre una superficie con borde, y una superficie con borde tal que tiene borde vacío es una superficie sin borde.
\end{obs}
\begin{defin}%%%%DEF: representación de un espacio topológico
Si $X$ es un espacio topológico y $\mathcal{P}$ una representación poligonal cuya realización geométrica es homeomorfa a $\mathcal{P}$, decimos que $\mathcal{P}$ es una \textbf{\emph{representación de $X$}}.
\end{defin}
\begin{obs}%%%%OBS: representación de una sola cara implica conexo
Un espacion topológico que admite una representación con una sola cara es conexo, pues es homeomorfo al cociente de una región poligonal conexa. Con más de una cara, puede ser o no conexo.
\end{obs}

\begin{eje}\label{eje:rep_sup_importantes}%%%%PROP: representación de superficies importantes
Veamos las representaciones de algunas superficies importantes (ver \autoref{fig:Representacion_superficies_importantes} y \autoref{fig:representacion_esfera_proyectivo}).
\begin{itemize}
\item[(a)] $\mathbb{S}^2=\langle a\mid aa^{-1}\rangle=\langle a,b\mid  abb^{-1}a^{-1}\rangle$ (\autoref{prop:Esfera como cociente de disco y cuadrado})
\item[(b)] $\mathbb{P}^2= \langle a\mid aa \rangle = \langle a,b\mid abab \rangle$ (\autoref{prop:proyectivo_cociente_cuadrado})
\item[(c)] $\Toro=\langle a,b\mid aba^{-1}b^{-1}\rangle$ (\autoref{prop:toro_cuadrado})
\end{itemize}
\end{eje}

\begin{eje}%M%%%:PROP: representación de superficies con borde
Podemos dar también las representaciones de algunas superficies con borde (\autoref{fig:rep_sup_borde}):
\begin{itemize}
\item[(a)] $\overline{\mathbb{B}}^2=\langle a,b,c\mid abc\rangle$.
\item[(b)] $Mob=\langle a,b,c\mid abac\rangle$ (\autoref{eje:rep_mobius}).
\item[(c)] $Cil=\langle a,b,c\mid aba^{-1}c\rangle$ (\autoref{eje:rep_mobius}).
\end{itemize}
\end{eje}
\clearpage




\begin{figure}[t]%%%%FIG: Representación de superficies impoirtantes
\centering
\begin{subfigure}{0.3\textwidth}
\centering
\begin{tikzpicture}[use optics, line cap=round,line join=round,>=triangle 45,x=1.0cm,y=1.0cm, scale=0.8]
\draw [-<-](0,2)-- (2,0);
\draw [-<-](2,0)-- (0,-2);
\draw [-<-](0,-2)-- (-2,0);
\draw [-<-](-2,0)-- (0,2);
\draw (1,1) node[anchor=south west] {$ b $};
\draw (1,-1) node[anchor=north west] {$ a $};
\draw (-1,1) node[anchor=south east] {$ a $};
\draw (-1,-1) node[anchor=north east] {$ b $};
\end{tikzpicture}
\caption{El plano proyectivo $\Proyectivo$.}
\end{subfigure}
\begin{subfigure}{0.3\textwidth}
\centering
\begin{tikzpicture}[use optics, line cap=round,line join=round,>=triangle 45,x=1.0cm,y=1.0cm, scale=0.8]
\draw [-<-](3,0)-- (5,2);
\draw [->-](5,2)-- (7,0);
\draw [->-](7,0)-- (5,-2);
\draw [-<-](5,-2)-- (3,0);
\draw (6,1) node[anchor=south west] {$ b $};
\draw (6,-1) node[anchor=north west] {$ a $};
\draw (4,1) node[anchor=south east] {$ a $};
\draw (4,-1) node[anchor=north east] {$ b $};
\end{tikzpicture}
\caption{El toro $\Toro$.}
\end{subfigure}
\begin{subfigure}{0.3\textwidth}
\centering
\begin{tikzpicture}[use optics, line cap=round,line join=round,>=triangle 45,x=1.0cm,y=1.0cm, scale=0.8]
\draw [-<-](-3,0)-- (-5,2);
\draw [-<-](-5,-2)-- (-3,0);
\draw [-<-](-5,-2)-- (-7,0);
\draw [-<-](-7,0)-- (-5,2);
\draw (-4,1) node[anchor=south west] {$ a $};
\draw (-4,-1) node[anchor=north west] {$ b $};
\draw (-6,1) node[anchor=south east] {$ a $};
\draw (-6,-1) node[anchor=north east] {$ b $};
\end{tikzpicture}
\caption{La esfera $\Esfera$.}
\end{subfigure}
\caption{Representación de superficies importantes.\label{fig:Representacion_superficies_importantes}}
\end{figure}

\begin{figure}%%%%FIG: Representacion esfera proyectivo
\centering
\begin{tikzpicture}[use optics, line cap=round,line join=round,>=triangle 45,x=1cm,y=1cm, scale=0.8]
\draw [-<-={at=0}, ->-={at=0.5}](-5,0) circle (2cm);
\draw [->-={at=0}, ->-={at=0.5}](5,0) circle (2cm);
\begin{scriptsize}
\fill [color=black] (-5,2) circle (1.5pt);
\fill [color=black] (-5,-2) circle (1.5pt);
\fill [color=black] (5,2) circle (1.5pt);
\fill [color=black] (5,-2) circle (1.5pt);
\end{scriptsize}
\draw (-7.1,0) node[anchor=east] {$ a $};
\draw (-2.9,0) node[anchor=west] {$ a $};
\draw (2.9,0) node[anchor=east] {$ a $};
\draw (7.1,0) node[anchor=west] {$ a $};
\draw (-4.9,-2.9) node[anchor=south] {$ \Esfera $};
\draw (5.1,-2.9) node[anchor=south] {$ \Proyectivo $};
\end{tikzpicture}
\caption{Representaciones de la esfera y el plano proyectivo. \label{fig:representacion_esfera_proyectivo}}
\end{figure}

\begin{figure}%%%%FIG: Representación superficies con borde
\centering
\begin{subfigure}{0.3\textwidth}
\centering
\begin{tikzpicture}[use optics, line cap=round,line join=round,>=triangle 45,x=1.0cm,y=1.0cm, scale=0.8]
\draw [->-] (-2,0)-- (2,0);
\draw [->-] (2,0)-- (0,3.46);
\draw [->-] (0,3.46)-- (-2,0);
\draw (-1,1.73) node[anchor=south east] {$a$};
\draw (0,0) node[anchor=north] {$b$};
\draw (1,1.73) node[anchor=south west] {$c$};
\end{tikzpicture}
\caption{El disco $\overline{\mathbb{B}}^2$.}
\end{subfigure}
\begin{subfigure}{0.3\textwidth}
\centering
\begin{tikzpicture}[use optics, line cap=round,line join=round,>=triangle 45,x=1.0cm,y=1.0cm, scale=0.8]
\draw [-<-](0,2)-- (2,0);
\draw [-<-](2,0)-- (0,-2);
\draw [-<-](0,-2)-- (-2,0);
\draw [-<-](-2,0)-- (0,2);
\draw (1,1) node[anchor=south west] {$ c $};
\draw (1,-1) node[anchor=north west] {$ a $};
\draw (-1,1) node[anchor=south east] {$ a $};
\draw (-1,-1) node[anchor=north east] {$ b $};
\end{tikzpicture}
\caption{La banda de Möbius $Mob$.}
\end{subfigure}
\begin{subfigure}{0.3\textwidth}
\centering
\begin{tikzpicture}[use optics, line cap=round,line join=round,>=triangle 45,x=1.0cm,y=1.0cm, scale=0.8]
\draw [-<-](0,2)-- (2,0);
\draw [-<-](2,0)-- (0,-2);
\draw [-<-](0,-2)-- (-2,0);
\draw [-<-](-2,0)-- (0,2);
\draw (1,1) node[anchor=south west] {$ c $};
\draw (1,-1) node[anchor=north west] {$ a $};
\draw (-1,1) node[anchor=south east] {$ a $};
\draw (-1,-1) node[anchor=north east] {$ b $};
\end{tikzpicture}
\caption{El cilindro $Cil$.}
\end{subfigure}
\caption{Representación de superficies con borde.}\label{fig:rep_sup_borde}
\end{figure}
\clearpage

\begin{figure}[h]%%%%FIG: ejemplo operacion
\centering
\begin{tikzpicture}[use optics, x=1cm, y=1cm, scale=0.8]
\draw [->-={at=0.125}, ->-={at=0.375}, -<-={at=0.625}, -<-={at=0.875}](-1,1) -- (-1,-1) -- (-3,-1) -- (-3, 1) -- (-1,1); 
\draw [->-] (1,0)-- (1.57,-1);
\draw [-<-] (1.57,-1)-- (2.73,-1);
\draw [-<-] (2.73,-1)-- (3.31,0);
\draw [-<-] (3.31,0)-- (2.73,1);
\draw [->-] (2.73,1)-- (1.58,1);
\draw [->-] (1.58,1)-- (1,0);

\draw [<->] (-.1,0)--+(0.5,0);
\draw (-0.9,0) node[anchor= west] {$ a $};
\draw (-3.1,0) node[anchor= east] {$ a $};
\draw (-2,1.1) node[anchor= south] {$ b $};
\draw (-2,-1.1) node[anchor= north] {$ b $};
\draw (2.16,1.1) node[anchor= south] {$ b $};

\draw (3.02,0.5) node[anchor= south west] {$ a $};
\draw (3.02,-0.5) node[anchor= north west] {$ e $};

\draw (2.15, -1.1) node[anchor= north] {$ b $};

\draw (1.29,-0.5) node[anchor= north east] {$ e $};
\draw (1.29,0.5) node[anchor= south east] {$ a $};
\end{tikzpicture}

\caption{Subdividir/consolidar. \label{fig:ejemplo_operacion}}
\end{figure}


Parece claro que, además de $\Esfera$ y $\Proyectivo$, una superficie pueda tener varias presentaciones poligonales. Sea por ejemplo la presentación del toro $\Toro=\langle a,b\mid aba^{-1}b^{-1}\rangle$. Intuitivamente podemos ver que, subdividiendo las aristas etiquetadas con $a$ y reetiquetándolas con $e$ y $a$ (ver \autoref{fig:ejemplo_operacion}), la superficie que representa la representación obtenida $\langle a,e,b\mid  aeb^{-1}e^{-1}a^{-1}b\rangle$ será la misma.
Vamos ahora a desarrollar unas reglas generales de transformación de representaciones de superficies que den lugar a superficies homeomorfas.



\begin{defin}%%%%DEF: Realizaciones geometricas topologicamente equivalentes
Sean $\mathcal{P}_1$ y $\mathcal{P}_2$ dos representaciones tal que sus realizaciones geométricas son homeomorfas. Entonces decimos que son \enfatiza{topológicamente equivalentes} y escribimos $\mathcal{P}_1 \approx \mathcal{P}_2$.
\end{defin}




Vamos a definir ahora unas operaciones elementales sobre las representaciones poligonales. Veremos luego que estas dan lugar a representaciones equivalentes.

\begin{defin}%%%%DEF: Operaciones elementales sobre representaciones
A las siguientes operaciones las llamamos \enfatiza{transformaciones elementales} de una representación poligonal:
\begin{itemize}
\item \enfatiza{Reetiquetar}: Se refiere a tres cambios distintos que se pueden dar sobre la representación: cada vez que aparece un símbolo $a$, cambiarlo por otro símbolo que no esté todavía en la representación; intercambiar entre ellos dos símbolos $a$ y $b$ de la representación cada vez que aparezcan; intercambiar todas las $a$ por $a^{-1}$ y las $a^{-1}$ por $a$.
\item \enfatiza{Subdividir} (\autoref{fig:ejemplo_operacion}): Cada vez que aparezca $a$, sustituirlo por $ae$, y cada vez que aparezca $a^{-1}$ sustituirlo por $e^{-1}a^{-1}$, donde $e$ es un símbolo que no está antes en la presentación.
\item \enfatiza{Consolidar} (\autoref{fig:ejemplo_operacion}): Si $a$ y $b$ aparecen siempre de forma adyacente, intercambiar $ab$ por $a$ y $b^{-1}a^{-1}$ por $a^{-1}$, siempre que esto de lugar a una o más palabras de longitud al menos $3$ o a una sola palabra de longitud $2$.
\item \enfatiza{Reflejar} (\autoref{fig:reflejar}): $$\langle S\mid   a_1 \dots a_m, W_2,\dots,W_k\rangle \mapsto \langle S\mid   a_m^{-1}\dots a_1^{-1}, W_2, \dots ,W_k\rangle .$$
\item \enfatiza{Rotar} (\autoref{fig:rotar}): $$\langle S\mid a_1a_2\dots a_m, W_2,\dots , W_k\rangle \mapsto \langle S\mid   a_2\dots a_ma_1, W_2,\dots , W_k\rangle .$$
\item \enfatiza{Cortar} (\autoref{fig:cortar}): Si $W_1$ y $W_2$ tienen longitud al menos $2$, $$\langle S\mid W_1W_2, W_3,\dots , W_k\rangle \mapsto \langle S\mid W_1e, e^{-1}W_2, W_3,\dots W_k\rangle .$$ 
\item \enfatiza{Pegar} (\autoref{fig:cortar}): $$\langle S,e\mid W_1e, e^-1W_2, W_3,\dots , W_k\rangle \mapsto \langle S\mid W_1W_2, W_3,\dots , W_k\rangle .$$
\item \enfatiza{Plegar} (\autoref{fig:plegar}): Si $W_1$ tiene longitud al menos $3$, $$\langle S,e\mid W_1ee^{-1}, W_2,\dots W_k\rangle \mapsto \langle S\mid W_1, W_2,\dots , W_k\rangle .$$ También podemos utilizar esta transformación en el caso de que $W_1$ tenga longitud $2$, siempre que la representación tenga una única palabra.
\item \enfatiza{Desplegar} (\autoref{fig:plegar}): $$\langle S\mid W_1, W_2,\dots , W_k\rangle \mapsto \langle S,e\mid W_1ee^{-1}, W_2,\dots , W_k\rangle .$$
\end{itemize}
\end{defin}


\begin{figure}
\begin{subfigure}{0.5\textwidth}
\centering
\begin{tikzpicture}[use optics]
\draw [->-] (-2.15,0.84) -- (-2.71,0.47);
\draw [->-] (-2.71,0.47)--(-3.03,-.35);
\draw [-<-] (-3.03,-.35) -- (-1.67, -0.76);
\draw [-<-] (-1.67, -0.76) -- (-1.46,0.29);
\draw [->-] (-1.46,0.29) -- (-2.15,0.84);

\draw [->-] (2.15,0.84) -- (2.71,0.47);
\draw [->-] (2.71,0.47)--(3.03,-.35);
\draw [-<-] (3.03,-.35) -- (1.67, -0.76);
\draw [-<-] (1.67, -0.76) -- (1.46,0.29);
\draw [->-] (1.46,0.29) -- (2.15,0.84);

\draw (-2.42,0.66) node[anchor=south east] {$a$};
\draw (-2.87,0.06) node [anchor=east] {$b$};
\draw (-2.35,-0.56) node[anchor= north east] {$c$};
\draw (-1.56,-.24) node [anchor= west] {$d$};
\draw (-1.81,.57) node [anchor= south west] {$e$};
\draw (2.42,0.66) node[anchor=south west] {$a$};
\draw (2.87,0.06) node [anchor=west] {$b$};
\draw (2.35,-0.56) node[anchor= north west] {$c$};
\draw (1.56,-.24) node [anchor= east] {$d$};
\draw (1.81,.57) node [anchor= south east] {$e$};

\draw [<->] (-0.25,0)--+(.5,0);
\end{tikzpicture}
\caption{Reflejar.\label{fig:reflejar}}
\end{subfigure} 
\begin{subfigure}{.5\textwidth}
\centering
\begin{tikzpicture}[use optics]
\draw [->-] (-2.15,0.84) -- (-2.71,0.47);
\draw [->-] (-2.71,0.47)--(-3.03,-.35);
\draw [-<-] (-3.03,-.35) -- (-1.67, -0.76);
\draw [-<-] (-1.67, -0.76) -- (-1.46,0.29);
\draw [->-] (-1.46,0.29) -- (-2.15,0.84);

\draw [-<-](1.67,.76)--(1.46,-.29);
\draw [->-](1.46,-.29) -- (2.15,-.84);
\draw [->-](2.15,-.84)--(2.71,-.47);
\draw [->-](2.71,-.47)--(3.03,.35);
\draw [-<-](3.03,.35)--(1.67,.76);

\draw [<->] (-0.25,0)--+(.5,0);

\draw (-2.42,0.66) node[anchor=south east] {$a$};
\draw (-2.87,0.06) node [anchor=east] {$b$};
\draw (-2.35,-0.56) node[anchor= north east] {$c$};
\draw (-1.56,-.24) node [anchor= west] {$d$};
\draw (-1.81,.57) node [anchor= south west] {$e$};

\draw (2.43035, -0.65644) node[anchor=north west] {$a$};
\draw (2.86796, -0.05807) node [anchor=west] {$b$};
\draw (2.34997, 0.55816) node[anchor= south west] {$c$};
\draw (1.56406, 0.23665) node [anchor= east] {$d$};
\draw (1.80519, -0.56713) node [anchor= north east] {$e$};

\end{tikzpicture}
\caption{Rotar.\label{fig:rotar}}
\end{subfigure}
\end{figure}


\begin{figure}
\begin{subfigure}{0.5\textwidth}
\centering
\begin{tikzpicture}[use optics, scale=0.3]
\draw [<->] (-.5,0)--(.5,0);
\draw (-4.79,4.04)--(-6.79,3.04)--(-6.79,-2.96)--(-4.79,-3.96)--(-3,0)--cycle;
\draw[dashed] (-4.79,-3.96)--(-4.79, 4.04);
\draw[->-={at=0.775}] (5,4)--(3,3)--(3,-3)--(5,-4)-- cycle; 
\draw[-<-={at=0.249}] (7,4)--(7,-4)--(8.79,-0) --cycle;
\draw (-6.79,0) node[anchor=east] {$W_1$}; 
\draw (-3,0) node[anchor=west]{$W_2$};
\draw (3,0) node[anchor=east] {$W_1$};
\draw (5,0.8) node[anchor=west] {$e$};
\draw (7,-1) node[anchor=east] {$e$};
\draw (8.79,0) node[anchor=west] {$W_2$};
\end{tikzpicture}
\caption{Cortar/Pegar.\label{fig:cortar}}
\end{subfigure}
\begin{subfigure}{0.5\textwidth}
\centering
\begin{tikzpicture}[use optics, scale=1.2]
\draw [->-](0.5, 1.53884)--(-0.30902, 0.95106);
\draw [->-](-0.30902, 0.95106)--(0,0);
\draw [->-](0,0)--(1,0);
\draw [->-](1,0)--(1.30902, 0.95106);
\draw [-<-](1.30902, 0.95106)--(0.5, 1.53884);
\draw [->-](3.11256, 0)--(4.88896, 0.000);
\draw [->-](4.88896, 0.000)--(4, 1.53884);
\draw [->-](4, 1.53884)--(3.11256, 0);
\draw [->-](4,1.54)--(4,.5);
\draw (0.1,1.24) node[anchor=south east] {$a$};
\draw (-.15, 0.48) node[anchor=north east] {$b$};
\draw (.5,0) node[anchor=north] {$c$};
\draw (1.15,0.48) node [anchor=north west] {$e$};
\draw (.9, 1.24) node [anchor=south west] {$e$};
\draw (3.56, 0.77) node[anchor=south east] {$a$};
\draw (4,0) node [anchor=north] {$b$};
\draw (4.44,.77) node[anchor=south west] {$c$};
\draw (4,.8) node[anchor=west] {$e$};
\filldraw (4,.5) circle (1pt);
\filldraw (1.30902, 0.95106) circle (1pt);
\end{tikzpicture}
\caption{Plegar/Desplegar.\label{fig:plegar}}
\end{subfigure}
\end{figure}

\begin{prop}%%%%PROP: Operaciones sobre representaciones
Las operaciones elementales sobre representaciones poligonales dan lugar a representaciones poligonales topológicamente equivalentes.
\end{prop}
\begin{proof}
Los pares subdividir/consolidar, cortar/pegar y plegar/desplegar son inversos, con lo que basta probar la proposición para una transformación de cada par. 
Los casos de reetiquetar, reflejar y rotar son cambios puramente formales que no afectan al espacio cociente. 
Empecemos con la operación de cortar. Sean $P_1$ y $P_2$ dos regiones poligonales convexas etiquetadas por las palabras $W_1e$ y $e^{-1}W_2$ y sea $P'$ la region poligonal etiquetada por $W_1 W_2$. Supongamos de momento que no hay más palabras en las respectivas representaciones. Sean también $\pi :P_1\amalg P_2\to M$ y $\pi' :P'\to M'$ respectivamente las aplicaciones cocientes. El segmento que va desde el vértice final de $W_1$ en $P'$ a su vértice inicial yace en $P'$ por convexidad, y lo etiquetamos con $e$. La aplicación $\tilde{f} :\partial P_1\amalg \partial P_2\to \partial P'$ que lleva cada arista de $P_1$ o $P_2$ a su correspondiente arista en $P'$ es continua y se puede extender a una aplicación continua $f:P_1\amalg P_2\to P'$ cuya restricción a cada $P_i$ sea un homeomorfismo sobre su imagen. Por el teorema de la aplicación cerrada, $f$ es una aplicación cociente. Dado que $f$ identifica únicamente las aristas $e$ y $e^{-1}$, las aplicaciones cociente $\pi '\circ f$ y $\pi$ hacen exactamente las mismas identificaciones, y por tanto los espacios cociente $M$ y $M'$ son homeomorfos. 
En el caso de que hubiese otras palabras $W_3,\dots ,W_k$, simplemente extendemos $f$ sobre sus respectivas representaciones poligonales como la identidad, y procedemos como antes.

Para la operación de consolidar enmpezamos considerando las palabras de longitud $3$ o más. Para ello, sean $P_1$ y $P_2$ las regiones poligonales etiquetadas respectivamente por $aeWe^{-1}a^{-1}W'$ y $aWa^{-1}W'$ (pudiendo ser $W'$ vacía), sean $M$ y $M'$ los espacios cociente y sean $\pi :P_1\to M$ y $\pi' :P_2\to M'$ las respectivas aplicaciones cociente. Podemos construir un homeomorfismo $f:\partial P_1\to \partial P_2$, que lleve la arista $a$ en $P_1$ a la primera mitad de $a$ en $P_2$, la arista $e$ a la segunda mitad de la arista $a$ en $P_2$, que lleve de forma análoga $e^{-1}$ y $a^{-1}$ en $P_1$ a la arista $a^{-1}$ en $P_2$, y tal que sea la identidad en $W$ y $W'$. Podemos extender este homeomorfismo a un homeomorfismo de $P_1$ en $P_2$, y se tiene entonces que las aplicaciones cociente $\pi \circ f$ y $\pi'$ hacen las mismas identificaciones, por lo que $M$ y $M'$ son homeomorfos. El caso de que la palabra que nos quede al aplicar la consolidación tenga longitud $2$, puede provenir sólo de una representación de longitud $4$ de $\Esfera$ o $\Proyectivo$, quedándonos pues una de las representaciones equivalentes de longitud $2$ que definimos.

Para el plegado, ignoramos como antes palabras adicionales $W_2,\dots W_k$. Supongamos para empezar que $W_1=abcd$ tiene longitud exáctamente 4. Sea $P$ una región poligonal convexa con aristas etiquetadas por $abee^{-1}cd$. Si cortamos a lo largo de el segmento $f$ que une el vértice inicial de $a$ con el vértice final de $e$ y obtenemos así $abef$ y $f^{-1}e^{-1}cd$. Consolidamos $ef$ en $h$, obteniendo así $ah^{-1}$ y $bch$. Pegamos por $h$ y obtenemos así $abcd$.
Si $W_1$ tiene longitud 2 o 3, podemos subdividir para alargar la longitud de la palabra a 4, plegar, y finalmente consolidar de nuevo.
\end{proof}

\begin{prop}
Sea $M$ una superficie que admite como representación $\langle S\mid W\rangle$. Sea $B$ un disco coordenado regular en $M$. Entonces $\langle S,a,b,c\mid Wc^{-1}b^{-1}a^{-1}\rangle$ es una representación de $M\setminus B$.
\end{prop}
\begin{proof}
Consideramos la representación $\mathcal{P}=\langle S,a,b,c\mid Wc^{-1}b^{-1}a^{-1},abc\rangle$ (\autoref{fig:suma_conexa_rep_pol}). Pegando a lo largo de $a$ y doblando dos veces, tenemos que $\mathcal{P}$ es una representación equivalente a $\langle S\mid W\rangle$ y por tanto es una representación de $M$. Llamemos $B$ a la imagen por la proyección $\pi$ en $M$ del interior de la región poligonal cuya frontera es $abc$. Veamos que $B$ es un disco coordenado regular, es decir, que existe un entorno $B'$ de $B$ en $M$ y un homeomorfismo $\phi :B'\to \R^2$ que manda $B$ a $\mathbb{B}^2$ y $\overline{B}$ a $\overline{\mathbb{B}}^2$. Siguiendo la figura \autoref{fig:suma_conexa_rep_pol}, sean $P_1$, $Q$ y $P_1'$ regiones poligonales convexas etiquetadas respectivamente por las palabras $Wc^{-1}b^{-1}a^{-1}$, $abc$ y $W$. Si triangulamos las regiones poligonales como en la  imagen, obtenemos una aplicación simplicial $f:P_1\amalg Q\to P_1'$ que lleva $Q$ a el triángulo $Q'\subseteq P_1'$, que comparte un vértice $v$ con $P_1'$. La composición $\pi \circ f: P_1\amalg Q\to \to M$ respeta las identificaciones hechas por la aplicación cociente $\pi :P_1'\to M$, por lo que se tiene un homeomorfismo de $M$, que lleva $B$ a la imagen de $Q'$.\\
Ahora, fijémonos en la demostración de la \autoref{prop:poligonos} (a). Cuando construimos el entorno Euclídeo de un vértice, ensamblábamos las regiones del plano $\Lambda$ que se correspondían con los varios vértices en un mismo disco coordenado. Aplicando esa construcción al vértice $v$, llevamos $Q'$ a un conjunto que es homeomorfo a un disco cerrado en el plano, y  entonces extendemos ese homeomorfismo a un disco abierto que lo contiene.\\
Se tiene por tanto que la realización geométrica de $\langle S,a,b,c\mid Wc^{-1}b^{-1}a^{-1}\rangle$ es homeomorfa a $M\setminus B$, y que $\partial B$ es la imagen de las aristas $c^{-1}b^{-1}a^{-1}$
\end{proof}

\begin{prop}%%%%PROP: representación suma conexa
Sean $M_1$ y $M_2$ superficies que admiten respectivamente representaciones $\langle S_1\mid W_1\rangle $ y $\langle S_2\mid W_2\rangle $, donde $S_1$ y $S_2$ son conjuntos disjuntos y tal que cada presentación tiene una sola cara. Entonces $\langle S_1,S_2\mid W_1W_2\rangle$ es una representación de la suma conexa $M_1 \# M_2$.

\end{prop}

\begin{proof}
Se sigue de la proposición anterior que la realización geométrica de $\langle S_1,a,b,c\mid W_1c^{-1}b^{-1}a^{-1}\rangle$ es homeomorfa a $M_1'=M_1\setminus B_1$ y un argumento análogo nos indica que $\langle S_2,a,b,c\mid abcW_2\rangle$ es una presentación de $M_2$ menos un disco coordenado, que denotamos por $M_2'=M_2\setminus B_1$. Por tanto se tiene que $\langle S_1,S_2,a,b,c\mid W_1,c^{-1}b^{-1}a^{-1},abcW_2\rangle$ es una presentación de $M_1'\amalg M_2'$ con las fronteras de los respectivos discos identificados, es decir, $M_1\# M_2$. Finalmente, pegando por $a$ y doblando dos veces, se obtiene la presentación $\langle S_1,S_2\mid W_1W_2\rangle$.
\end{proof}

\begin{figure}[t]
\centering
\begin{tikzpicture}[use optics, scale=0.5]
\draw [dashed] (-6,-3) [out=135, in=-90] to (-8,0) [out=90, in=-135] to (-6,3);
\draw [->-={at=0.25}, ->-={at=0.5}, ->-={at=0.75}] (-6,3) -- (-4,4)--(-2,2)--(-2,-2)--(-4,-4)--(-6,-3);
\draw [dashed] (-6,3) -- (-2,2) -- (-6,0) -- (-2,-2) -- (-6,-3) -- cycle;

\filldraw [black!10] (3.5,1) -- (2,-3) -- (5,-3) -- cycle;
\draw [->-] (3.5,1) -- (2,-3);
\draw [->-] (2,-3) -- (5,-3);
\draw [->-] (5,-3) -- (3.5,1);

\draw [dashed] (14,4) [out=-155, in=90] to (10,0) [out=-90, in=155] to (14,-4) -- cycle;
\draw [dashed] (14,4) -- (18,1) -- (14,0) -- (18,-1) -- (14,-4);
\filldraw [black!10] (18,1) -- (22,0) -- (18,-1) -- cycle;
\draw [-<-](18,1) -- (22,0);
\draw [-<-] (22,0) -- (18,-1);
\draw [-<-](18, -1) -- (18,1);
\draw (14,4) --(22,0) -- (14,-4);
 
\filldraw (-4,4) circle (2pt);
\filldraw (3.5,1) circle (2pt);
\filldraw (22,0) circle (2pt);
\filldraw (-4,-4) circle (2pt);

\draw (-4,4) node[anchor=south] {$v$};
\draw (-4,-4) node[anchor=north] {$v$};
\draw (3.5,1) node [anchor=south] {$v$};
\draw (22,0) node [anchor=west] {$v$};

\draw (-3,3) node [anchor=south west] {$a$};
\draw (-2,0) node [anchor=west] {$b$};
\draw (-3,-3) node [anchor= north west ] {$c$};

\draw (2.75,-1) node [anchor=south east] {$a$};
\draw (3.5,-3) node[anchor=north] {$b$};
\draw (4.25, -1) node[anchor=south west] {$c$};

\draw (18,0) node [anchor=east] {$b$};
\draw (20,-.5) node [anchor=north east] {$c$};
\draw (20,.5) node [anchor = south east] {$a$};

\draw[->] (6,0) -- (8,0);

\draw [->] (20,-3) [out=60, in=-30] to (19,0);

\draw (20,-3) node [anchor=north east] {$Q'$};
\draw (14,-6) node [anchor=north]{$P_1'$};
\draw (3.5,-5) node [anchor=north] {$Q$};
\draw (-4,-6) node [anchor=north]{$P_1$};
\draw (7,0) node [anchor=south]{$f$};
\draw (11,3) node [anchor=south east]{$W$};
\draw (-6,3) node [anchor=south east]{$W$};

\end{tikzpicture}
\caption{La representación $\mathcal{P}_1=\langle S_1,a,b,c\mid W_1c^{-1}b^{-1}a^{-1},abc\rangle$.\label{fig:suma_conexa_rep_pol}}
\end{figure}

\begin{eje}
Con la proposición anterior podemos aumentar nuestra lista de superficies conocidas. A estas representaciones las llamaremos representaciones \enfatiza{estándar}.
\begin{itemize}
\item [(a)] \textit{Esfera} 
$$\langle a \mid aa^{-1} \rangle$$
\item[(b)] \textit{Suma conexa de n toros.}
$$\langle a_1,\dots ,a_n, b_1,\dots ,b_n\mid a_1b_1a_1^{-1}b_1^{-1}\dots a_nb_na_n^{-1}b_n^{-1}\rangle$$ 
\item[(c)] \textit{Suma conexa de n planos proyectivos.} 
$$\langle a_1,\dots ,a_n\mid a_1a_1\dots a_na_n\rangle$$
\end{itemize}
\end{eje}

Hemos visto que toda representación poligonal da lugar a una superficie compacta. Veamos que el recíproco también es cierto.

\begin{prop}\label{prop:superficie_tiene_representacion}%%%PROP: representacion poligonal
Sea $M$ una superficie compacta. Entonces tiene una representación poligonal.
\end{prop}
\begin{proof}
Por el Teorema de Radó, $M$ es homeomorfa al poliedro de un complejo simplicial $K$ de dimensión 2, que denotaremos por $|K|$, en donde todo 1-símplice es una cara de exáctamente dos 2-símplices.
Sea $\mathcal{P}$ la representación poligonal de una superficie que consta de una palabra de longitud 3 por cada 2-símplice, y cuyas aristas tienen las mismas etiquetas si y solo si corresponden al mismo 1-símplice. Veamos que la realización geométrica de $\mathcal{P}$ es homeomorfa al poliedro $|K|$.
Si $P=P_1\amalg \dots \amalg P_k$ denota la unión disjunta de los 2-símplices de $K$, entonces tenemos dos aplicaciones cociente $\pi_K :P\to |K|$ y $\pi_{\mathcal{P}} :P\to |\mathcal{P}|$. Por la unicidad del espacio cociente, hay que ver, por tanto, que hacen las mimas identificaciones. Sabemos que ambas aplicaciones son inyectivas en el interior de los 2-símplices, que identifican vértices únicamente con vertices y que identifican las aristas de la misma manera. Falta por tanto demostrar que $\pi_K$ y $\pi_{\mathcal{P}}$ identifican los vértices sólo de acuerdo a la relación generada por la identificación de las aristas.\\
Supongamos que $v\in K$ es un vértice. Se tiene que $v$ pertenece a un 1-símplice, pues si no sería un punto aislado de $|K|$, lo que contradice que $|K|$ sea una variedad. El Teorema de Radó nos garantiza que este 1-símplice es una cara de exáctamente dos 2-símplices. Definimos una relación de equivalencia sobre los dos 2-símplices que contienen a $v$: si $\sigma , \sigma '$ son dos 2-símplices que contienen a $v$, decimos que están \enfatiza{contectados por arista en v} si hay una secuencia $\sigma =\sigma_1 ,\dots ,\sigma_k =\sigma '$ de 2-símplices que contienen a $v$ talque cada $\sigma_i$ comparte una arista con $\sigma_{i+1}$,  para todo $i=1,\dots , k-1$. Vamos a comprobar por tanto que sólo hay una relación de equivalencia. Supongamos, por reducción al absurdo, que podemos agrupar los 2-símplices que contienen a $v$ en dos conjuntos disjuntos $\{ \sigma_1 ,\dots ,\sigma_k \}$ y $\{ \tau_1 ,\dots ,\tau_m \}$ tales que todo $\sigma_i$ está conectado por arista con todo $\sigma_j$ pero ningún $\tau_i$ está conectado con ningún $\sigma_j$. Sea un $\varepsilon >0$ tal que la bola $B_{\varepsilon}(v)$ interseca sólo los símplices que contienen a $v$.  Se tiene entonces que $B_{\varepsilon}(v)\cap |K|$ es un abierto de $|K|$ y por tanto una superficie, por lo que $v$ tiene un entorno $W\subseteq B_{\varepsilon}(v)\cap |K|$ homeomorfo a $\R^2$, y así pues $W\setminus \{ v \}$ es conexo. 
Pero si denotamos ahora los conjuntos $$U=W\cap (\sigma_1 \cup \dots \cup \sigma_k )\setminus \{ v \}$$
$$V=W\cap (\tau_1 \cup \dots \cup \tau_m )\setminus \{ v \}$$ entonces $U$ y $V$ son dos abiertos disjuntos de $|K|$ cuya intersección con cada símplice es abierta en el símplice, y así, $W=U\cap V$ es una desconexión de $W$. Pero esto es una contradicción.
\end{proof}


\section{Teorema de clasificación de superficies compactas}

Ya tenemos la maquinaria necesaria para hacer una primera demostración de la primera parte del Teorema de Clasificación.
Como venimos anunciando desde el principio del presente trabajo nuestras superficies fundamentales son el toro $\Toro$, la esfera $ \Esfera$, y el plano proyectivo  $\Proyectivo$. ¿Qué pasa con las otras superficies que hemos visto, como la botella de Klein o la suma conexa de un toro y un plano proyectivo? La respuesta la obtenemos de los dos siguientes lemas.

\begin{lema}\label{lema:klein_proyectivoproyectivo}
La botella de Klein es homeomorfa a $\Proyectivo \# \Proyectivo$.
\end{lema}
\begin{proof}
Vamos a demostrarlo haciendo transformaciones elementales sobre la representación de la botella de Klein vista en el \autoref{eje:rep_sup_importantes}. 
Siguiendo la figura \autoref{fig:klein_proyectivo}, tenemos
\begin{align*}
\langle a,b\mid abab^{-1}\rangle & & \\
\approx & \, \langle a,b,c\mid abc, c^{-1}ab^{-1}\rangle &\text{(cortar por } c\text{)}\\
\approx & \, \langle a,b,c\mid bca,a^{-1}cb\rangle &\text{(rotar y reflejar)}\\
\approx & \, \langle b,c\mid bbcc\rangle &\text{(pegar por } a \text{ y rotar)}
\end{align*}
Y esto es la presentación poligonal de la suma conexa de dos planos proyectivos.
\end{proof}

\begin{figure}
\centering

\begin{tikzpicture}[use optics,scale=1.5, >=angle 60]
\draw [->-] (0,0)--(1,0);
\draw [-<-] (1,0)--(1,1);
\draw [->-] (1,1)--(0,1);
\draw [->-] (0,1) -- (0,0);
\draw [dashed] (0,0)--(1,1);

\draw [->-] (2,0) -- (3,1);
\draw [->-] (3,1)--(2,1);
\draw [->-] (2,1)--(2,0);

\draw [->-] (2.5,0)--(3.5,0);
\draw [-<-] (3.5,0)--(3.5,1);
\draw [-<-] (3.5,1) -- (2.5,0);

\draw [->-] (4.5,0)--(5.5,0);
\draw [->-] (5.5,0) -- (5,.87);
\draw [->-] (5,.87)--(4.5,0);

\draw [->-] (6,0)--(7,0);
\draw [->-] (7,0) -- (6.5,.87);
\draw [->-] (6.5,.87)--(6,0);

\draw [->-](8,0) -- (9,0);
\draw [->-] (9,0)--(9,1);
\draw [->-] (9,1) -- (8,1);
\draw [->-](8,1)--(8,0);
\draw [-<-] (8,1) -- (9,0);

\draw [->] (1.25,0.75) [out=45,in=135] to (1.75,0.75);
\draw [->] (3.85,.75) [out=45,in=135] to (4.35,.75);
\draw [->] (7.2,.75) [out=45,in=135] to (7.7,.75);

\draw (0,.5) node [anchor=east] {$b$};
\draw (0.5,0) node [anchor=north]{$a$};
\draw (1,.5) node [anchor=west] {$b$};
\draw (.5,1) node [anchor=south]{$a$};

\draw (2,.5) node[anchor=east]{$b$};
\draw (2.3,.43) node [anchor=north west] {$c$};
\draw (2.5,1) node [anchor=south] {$a$};

\draw (3,0) node [anchor=north]{$a$};
\draw (3.5,.55) node [anchor=west] {$b$};
\draw (3,.43) node [anchor=south east] {$c$};

\draw (4.75,.43) node [anchor=south east] {$b$};
\draw (5,0) node[anchor= north] {$c$};
\draw (5.25,.43)node[anchor=south west] {$a$};

\draw (6.25,.43) node[anchor=south east] {$a$};
\draw (6.5,0) node[anchor=north] {$c$};
\draw (6.75,.43) node [anchor=south west] {$b$};

\draw (8,.5) node[anchor=east] {$b$};
\draw (8.5,0)node[anchor=north] {$c$};
\draw (9,.5) node [anchor=west] {$c$};
\draw (8.5,1) node[anchor=south] {$b$};

\draw (8.5,.5)node [anchor=south west] {$a$};

\end{tikzpicture}
\caption{Transformación de la bottella de Klein en $\Proyectivo$.\label{fig:klein_proyectivo}}
\end{figure}

\begin{lema}\label{lema:toro_proyectivo}
La suma conexa $\Toro \# \Proyectivo$ es homeomorfa a $\Proyectivo \# \Proyectivo \# \Proyectivo$.
\end{lema}
\begin{proof}
Partimos de una representación poligonal de $K\# \Proyectivo$, que es, por el lema anterior, una presentación de $\Proyectivo \# \Proyectivo \# \Proyectivo$ y procedemos haciendo transformaciones elementales:
\begin{align*}
\langle a,b,c \mid abab^{-1}cc \rangle &&\\
\approx &\, \langle a,b,c,d \mid cabd^{-1}, c^{-1}ba^{-1}d^{-1} \rangle &\text{(rotar, cortar por } d\text{ y reflejar)}\\
\approx &\, \langle a,b,d \mid abd^{-1}ba^{-1}d^{-1}\rangle & \text{(rotar y pegar por } c\text{)}\\
\approx &\, \langle a,b,d,e \mid a^{-1}d^{-1}abe, b^{-1}de\rangle & \text{(cortar por }e\text{ y reflejar)}\\
\approx &\, \langle a,d,e\mid a^{-1}d^{-1}adee\rangle & \text{(rotar y pegar por } b \text{)} 
\end{align*}
Esta última es, tal como queríamos, una representación de $\Toro \# \Proyectivo$.
\end{proof}

\begin{tma}[Clasificación de Superficies Compactas]\label{teo:csc}
Toda superficie compacta y conexa no vacía es homeomorfa a una de las siguientes superficies:
\begin{itemize}
\item[(a)] La esfera $\Esfera$.
\item[(b)] Una suma conexa $\Toro \# \dots \# \Toro$ de copias del toro.
\item[(c)] Una suma conexa $\Proyectivo \,  \# \dots\, \# \Proyectivo$ de copias del plano proyectivo. 
\end{itemize}
A estas superficies las llamamos \enfatiza{superficies estándar}.
\end{tma}

\begin{proof}
Sea $M$ una superficie junto con una representación poligonal suya, que existe por la \autoref{prop:superficie_tiene_representacion}. Probaremos el teorema haciendo transformaciones elementales sobre esta representación hasta llegar a una de las estándar.
Diremos que dos aristas de una representación poligonal con una misma etiqueta son \enfatiza{complementarias} si aparecen tanto en la forma $a$ como en la de $a^{-1}$, y diremos que son \enfatiza{torcidas} si aparecen únicamente como $a,\dots ,a$ o como $a^{-1},\dots a^{-1}$. 
\underline{PASO 1}: \textit{M admite una representación con una sola cara.} Si en la representación poligonal de $M$ aparecen dos o más caras, por ser $M$ conexa algua arista de una cara de $M$ tiene que estar identificada con otra arista de otra cara de $M$. Si no fuese así, $M$ sería la unión de los cocientes de sus caras, y dado que cada uno de esos cocientes es abierto y cerrado, desconectarían $M$. Por tanto, a través de sucesivos pegados, con rotaciones y reflexiones si es necesario, podemos reducir el número de caras hasta una.

\underline{PASO 2}: \textit{O bien M es homeomorfa a la esfera, o bien M admite una presentación en la cual no hay pares complementarios adyacentes.} Podemos ir eliminando los pares complementarios adyacentes haciendo sucesivos pegados. O bien los eliminamos todos, o bien nos queda $\langle a\mid aa^{-1}\rangle$, lo que implica que $M$ es homeomorfa a la esfera.

\underline{PASO 3}: \textit{M admite una representación en la que todos los pares torcidos son adyacentes.} Si todos los pares torcidos son adyacentes, procedemos al Paso 4. Si por el contrario hay un par torcido no adyacente, podemos suponer que la representación de $M$ es de la forma $VaWa$, donde $V$ y $W$ son palabras no vacías formadas por símbolos de un conjunto no vacío $S$. Entonces, mediante las siguientes transformaciones (que se pueden seguir en la \autoref{fig:PASO3})
\begin{align*}
\langle a,S \mid VaWa \rangle  &&\\
\approx &\, \langle a,b,S \mid bVa, b^{-1}Wa \rangle &\text{(cortar por } b\text{ y rotar)}\\
\approx &\, \langle a,b,S \mid bVa, a^{-1}W^{-1}b \rangle & \text{(reflexión)} \\
\approx &\, \langle b,S \mid bbVW^{-1}\rangle & \text{(pegar por } a\text{)}
\end{align*}
podemos sustituir el par torcido $a,a$ por uno $b,b$, que es adyacente ($W^{-1}$ representa la reflexión de $W$). Como no hemos cortado en medio de $V$ o $W$, no hemos separado otros pares adyacentes. Por otro lado, si bien al reflejar $W$ podemos haber creado un nuevo par torcido, al haber reducido en una unidad el número total de pares torcidos no adyacentes, después de un número finito de pasos los habremos eliminado todos. Finalmente, si en este paso hemos creado nuevos pares complementarios adyacentes, repetimos el Paso 2, que tampoco incrementa el número de pares no adyacentes.


%Si en la representación de $M$ hay un par torcido no adyacente, podemos transformar la representación con rotaciones hasta una de la forma $VaWa$, donde $V$ y $W$ son palabras no vacías. Guiándonos por la \autoref{fig:PASO3} podemos ahora efectuar sobre $VaWa$ sucesivas transformaciones (cortar a lo largo de $b$, reflejar, pegar a lo largo de $a$) hasta llegar a la representación $VW^{-1}bb$, donde $W^{-1}$ denota la reflexión de $W$. Hemos por tanto sustituido el par torcido $a$,$a$ por el par torcido $b$,$b$, que es adyacente. Al hacer esta operación no hemos separado pares adyacentes. Se pueden haber creado nuevos pares torcidos al reflejar $W$, pero hemos reducido el número total de pares no adyacentes en al menos una unidad. Por tanto, al cabo de un número finito de pasos, no hay más pares torcidos no adyacentes. Si se han creado nuevos pares adyacentes complementarios, repetimos el Paso 2, que no aumenta el número de pares no adyacentes.
\begin{figure}
\centering
\begin{tikzpicture}[use optics,scale=1.5, >= angle 60]
\draw [->-](0,0)--(1,0);
\draw [-<-](0,1)--(1,1);
\draw [dashed] (0,1) [out=-120, in= 120] to (0,0);
\draw [dashed] (1,1) [out=-60, in=60] to (1,0);
\draw [dash pattern=on 2pt off 2pt] (1,0)--(0,1);

\draw [->-](2,0) -- (3,0);
\draw [->-](3,0)-- (2,1);
\draw [dashed] (2,1) [out=-120, in=120] to (2,0);

\draw [->-] (3.5,0) -- (2.5,1);
\draw [-<-] (2.5,1) -- (3.5,1);
\draw [dashed] (3.5,1) [out=-60, in=60] to (3.5,0);


\draw [->-] (4.5,1) -- (4.5,0);
\draw [-<-] (4.5,1) -- (5.5,0);
\draw [dashed] (5.5, 0) [out=-120, in=-60] to (4.5,0);
\draw [->-] (6,0) -- (5,1);
\draw [-<-] (5,1) -- (6,1);
\draw [dashed] (6,1) [out=-60, in=60] to (6,0);

\draw [->-] (8,1) -- (7,1);
\draw [->-] (7,1) -- (7,0);
\draw [dashed] (7,0) [out=0, in=-90] to (8,1);
\draw [dash pattern= on 2pt off 2pt] (7,1) -- (7.71,.3);

\draw (0.5,0) node [anchor=north] {$a$};
\draw (.5,1) node [anchor=south] {$a$};
\draw (-.1,.8) node [anchor=east] {$V$};
\draw (1.1,.2) node [anchor=west] {$W$};

\draw (2.5,0) node [anchor=north] {$a$};
\draw (3,1) node [anchor=south] {$a$};
\draw (1.9,.8) node [anchor=east] {$V$};
\draw (3.6,.2) node [anchor=west] {$W$};
\draw (3,.5) node [anchor=north] {$b$};
\draw (2.5,.5) node [anchor=south]{$b$};

\draw (4.5,.5) node [anchor=east] {$b$};
\draw (5,0.5) node [anchor=south] {$a$};
\draw (5.5,.5) node [anchor=north] {$a$};
\draw (5.5,1) node [anchor=south] {$b$};
\draw (5.2,-.25) node [anchor=north] {$V$};
\draw (6.1,.8) node [anchor=west] {$W^{-1}$};

\draw (7,.5) node [anchor=east] {$b$};
\draw (7.5,1) node [anchor=south] {$b$};
\draw (7.71, .3) node [anchor= north west] {$VW^{-1}$};

\draw [->] (1.2,1.1) [out=45, in=135] to (1.8,1.1);
\draw [->] (3.7,1.1) [out=45, in=135] to (4.3,1.1);
\draw [->] (6.2,1.1) [out=45, in=135] to (6.8,1.1);

\end{tikzpicture}
\caption{Haciendo que los pares torcidos sean adyacentes.\label{fig:PASO3}}
\end{figure}

\underline{PASO 4}: \textit{M admite una representación en la que todos los vértices se identifican en un único punto.} Supongamos que existen al menos dos clases de equivalencia de vértices distintas. Entonces la representación poligonal tendrá dos vertices de dos clases de equivalencia distintas, llamémoslos $v$ y $w$, conectados por una arista $a$. La otra arista que tiene a $v$ como vértice no puede estar etiquetada con $a$, pues si formasen un par complementario lo habríamos eliminado en el Paso 2, y si formasen un par torcido, entonces $v$ y $w$ se identificarían, cuando estamos suponiendo lo contrario. 
Por lo tanto etiquetamos esta arista com $b$, y a su otro vértice lo llamamos $x$, el cual puede estar en la misma clase de equivalencia de $v$, de $w$, o en otra distinta.

En la representación poligonal hay otra arista que se identifica con $b$, y puede estar etiquetada por $b$ o por $b^{-1}$. Supongamos que la etiqueta es $b^{-1}$ (el otro caso implica sólo una reflexión adicional, que indicaremos cuándo ha de hacerse). Podemos escribir entonces la palabra que describe la representación poligonal como $baXb^{-1}Y$. Ahora, tal como se muestra en la \autoref{fig:PASO4a}, si cortamos a lo largo de $c$, rotamos y pegamos por $b$ (efectuando aquí la reflexión necesaria si la arista al principio aparecía como $b$) obtenemos una representación equivalente en la cual el número de vértices de la clase de equivalencia de $v$ ha disminuído, mientras que el número de vértices de la clase de $w$ ha aumentado. Si se han creado nuevos pares complementarios adyacentes, volvemos a aplicar el Paso 2 para eliminarlos. En esta última operación, si plegamos la representación para eliminar un par complementario $aa^{-1}$ que tiene un vértice $v$ entre medias, podemos hacer disminuir los vértices de la clase de $v$, pero nunca aumentarlos. Iterando el proceso para cada clase de equivalencia, podemos quedarnos finalmente con una sola clase de vértices.
 

%Sea $v$ una clase de vértices. Si existe otra clase de vértices que no se identifican con $v$, entonces existirá una arista que una $v$ y la otra clase de vértices. Etiquetémoslos por $a$ y $w$ respectivamente, tal como se muestra en la \autoref{fig:PASO4a}. La otra arista que junta con $a$ en su vértice $v$ no puede estar identificado con $a$, pues si fuese su complementario, lo habríamos eliminado en el Paso 2, y si formase con $a$ un par torcido, entonces la aplicación cociente identificaría también los otros dos vértices de cada $a$, y estamos asumiendo lo contrario. Por lo tanto llamamos a esta otra arista $b$, y a su otro vértice $x$, el cual puede estar identificado con $v$, con $w$ o con ninguno de los dos.
%
%En algún lugar de la representación poligonal hay una arista que se identifica con $b$, de la forma $b$ o $b^{-1}$. Supongamos que aparece en la forma $b^{-1}$ (el otro caso implica como diferencia sólo una reflexión, que indicaremos cuándo ha de hacerse). Podemos escribir la representación como $baXb^{-1}Y$, donde $X$ e $Y$ son palabras desconocidas y al menos una es no vacía. Ahora, cortando a lo largo de $c$ y pegando a lo largo de $b$ (efectuando aquí la reflexión necesaria si la arista aparecía como $b$) como se muestra en la \autoref{fig:PASO4b}, llegamos a una representación en la cual el número de vértices con etiqueta $v$ ha disminuído, mientras que en cambio el número de vértices con la etiqueta $w$ ha aumentado. Si se han introducido nuevos pares complementarios adyacentes, volvemos a aplicar el Paso 2 de nuevo para quitarlos. Esta operación puede hacer disminuir los vértices etiquetados por $v$ (por ejemplo por la operación de plegado) pero nunca aumentarlos. Repitiendo la secuencia un número finito de veces, podemos finalmente eliminar totalmente la clase de vértices $v$. Iterando el proceso podemos quedarnos finalmente con una sola clase de vértices.

\underline{PASO 5}: \textit{Si la presentación tiene un par complementario $a$, $a^{-1}$, entonces tiene otro par complementario $b$,  $b^{-1}$ que aparece intercalado con el primero, en la forma $a,\dots , b, \dots ,a^{-1}, \dots , b^{-1} $.} Si no es así, entonces la presentación es de la forma $aXa^{-1}Y$, donde $X$ contiene sólo aristas de pares complementarios o pares torcidos adyacentes, igual que $Y$. Por lo tanto cada arista en $X$ se identifica con otra arista en $X$, y lo mismo pasa en $Y$. Esto implica que los vértices finales de $a$ y $a^{-1}$, que tocan únicamente $X$, se identifican sólo con vértices de $X$, y que los vértices iniciales de $a$ y $a^{-1}$ se identifican sólo con vértices de $Y$. Esto es una contradicción, puesto que todos los vértices se identifican unos con otros por el Paso 4.

\underline{PASO 6}: \textit{M admite una presentación en la que todos los pares complementarios intercalados aparecen juntos con ninguna otra arista entre ellos: $aba^{-1}b^{-1}$.} Si no es así, podemos describir la representación poligonal por la palabra $WaXbYa^{-1}Zb^{-1}$. Hacemos, como en la figura \autoref{fig:PASO6}, las siguientes transformaciones: cortamos a lo largo de $c$, pegamos por $a$, cortamos a lo largo de $d$, y pegamos finalmente por $b$. Obtenemos así la palabra $cdc^{-1}d^{-1}WZYX$, sustituyendo por tanto los dos pares adyacentes por un nuevo par intercalado $cdc^{-1}d^{-1}$, sin separar otras aristas que previamente eran adyacentes. Por lo tanto, repitiendo el proceso por cada gurpo de pares complementarios se obtiene el resultado.

\underline{PASO 7}: \textit{M es homeomorfa a la suma conexa de uno o más toros o a la suma conexa de uno o más planos proyectivos.} Este es el paso que finaliza la demostración.  Recapitulando lo hecho en los pasos anteriores, tenemos una representación de $M$ (que llamaremos \textit{representación estándar}) en la cual todos los pares torcidos aparecen uno junto al otro y todos los pares complementarios aparecen intercalados en grupos de dos, en la forma $aba^{-1}b^{-1}$. Por tanto esta es la representación de una suma conexa de copias del $\Toro$ (representadas por los bloques $aba^{-1}b^{-1}$) y de copias de $\Proyectivo$ (representadas por los $cc$). Si sólo aparecen copias de un tipo, hemos terminado.
Nos queda el caso en el que aparecen tanto pares torcidos como pares complementarios. Si es así, algún par complementario aparece junto a un par torcido, es decir, la representación está descrita por la palabra $aba^{-1}b^{-1}ccX$ o por $ccaba^{-1}b^{-1}X$. En los dos casos, tenemos la suma conexa de un toro, un plano proyectivo, y lo que represente la palabra $X$. Pero por el \autoref{lema:toro_proyectivo}, $\Toro \# \Proyectivo = \Proyectivo \# \Proyectivo \# \Proyectivo$, por lo que podemos eliminar un toro de la suma conexa. Iterando un número finito de veces, eliminamos todas las apariciones de los bloques de pares complementarios, es decir, de las copias de $\Toro$, completando así la demostración.
\end{proof}





\begin{figure}
\centering
\begin{tikzpicture}[use optics, scale=1.5, >=angle 60]
\draw [-<-] (0,0) -- (1,0);
\draw [dashed] (1,0) [out=60, in=-60] to (1,2.5);
\draw [dashed] (0,0) [out=120, in=-120] to (0,2.5);
\draw [-<-] (0,2.5) -- (0.5,3.3);
\draw [dash pattern=on 2pt off 2pt, -<-] (1,2.5) -- (0,2.5);
\draw [->-] (1,2.5) -- (.5,3.3);
\filldraw (0,2.5) circle (1pt) node[anchor=south east]{$w$};
\filldraw (1,2.5) circle (1pt) node [anchor=south west]{$x$};
\filldraw (.5,3.3) circle (1pt);
\filldraw (0,0) circle (1pt) node [anchor=north east]{$v$};
\filldraw (1,0) circle (1pt) node [anchor=north west]{$x$};
\draw (.5,3.32) node [anchor=south] {$v$};

\draw [->-] (3,.8) -- (3.5,0);
\draw [->-] (3.5,0) -- (4,.8);
\draw [->-] (4,.8) -- (3,.8);
\draw [dashed] (3,.8) [out=120, in=-120] to (3,3.3);
\draw [dashed] (4,.8) [out=60, in=-60] to (4,3.3);
\draw [->-] (3,3.3)--(4,3.3);
\filldraw (3,3.3) circle (1pt) node[anchor=south east]{$w$};
\filldraw (4,3.3) circle (1pt)  node[anchor=south west]{$x$};
\filldraw (3,.8) circle (1pt)node [anchor=north east]{$v$};
\filldraw (4,.8) circle (1pt) node [anchor=north west]{$w$};
\filldraw (3.5,0) circle (1pt);
\draw (3.5,-.02) node [anchor=north] {$w$};

\draw [->] (1.5,2.3) -- (2.3,2.3);

\draw (-.4,1.75) node [anchor=east] {$X$};
\draw (1.4,1.75) node [anchor=west] {$Y$};
\draw (.5,0) node [anchor=north] {$b$};
\draw (.5,2.5) node [anchor=north] {$c$};
\draw (0.25,2.9) node[anchor=south east]{$a$};
\draw (0.75,2.9) node [anchor=south west]{$b$};


\draw (3.5,3.32) node [anchor=south] {$c$};
\draw (2.6,1.25) node [anchor=east] {$X$};
\draw (4.6,1.25) node [anchor=west]{$Y$};
\draw (3.25,.4) node [anchor=north east]{$a$};
\draw (3.75,.4) node [anchor=north west]{$b$};
\draw (3.5,.8) node [anchor=south]{$b$};



\end{tikzpicture}
\caption{Reduciendo el número de vértices equivalentes a $v$.\label{fig:PASO4a}}
\end{figure}

\begin{figure}
\centering
\begin{tikzpicture}[use optics, scale=0.8]


%1
\draw [dashed] (-.71,1.71)--(-.71,.71);
\draw [->-] (-.71,.71) -- (0,0);
\draw [dashed] (0,0)--(1,0);
\draw [-<-](1,0)--(1.71,.71);
\draw [dashed] (1.71,.71)--(1.71,1.71);
\draw [-<-] (1.71,1.71) -- (1,2.41);
\draw [dashed] (1,2.41) -- (0,2.41);
\draw [->-] (0,2.41) -- (-.71,1.71);
\draw [dash pattern=on 2pt off 2pt] (-.71,.71)--(1,2.41);

\draw (-.35,.35) node [anchor=north east]{$b$};
\draw (.5,0) node [anchor=north] {$Y$};
\draw (1.35,.35) node [anchor=north west]{$a$};
\draw (1.71,1.21) node [anchor= west]{$Z$};
\draw (1.35,2.06) node [anchor=south west]{$b$};
\draw (.5,2.41) node [anchor=south]{$W$};
\draw (-.35,2.06) node [anchor=south east]{$a$};
\draw (-.71,1.21) node [anchor=east] {$X$};

%2
\draw [dashed, shift={(4,-.3)}] (-.71,.71) -- (0,0);
\draw [->-, shift={(4,-.3)}] (0,0)--(1,0);
\draw [dashed, shift={(4,-.3)}](1,0)--(1.71,.71);
\draw [-<-, shift={(4,-.3)}] (-.71,.71) -- (1.71,.71);


\draw [dashed, shift={(4,.3)}] (1.71,.71)--(1.71,1.71);
\draw [-<-, shift={(4,.3)}] (1.71,1.71) -- (1,2.41);
\draw [-<-, shift={(4,.3)}] (1,2.41) -- (0,2.41);
\draw [->-, shift={(4,.3)}] (0,2.41) -- (-.71,1.71);
\draw [dashed, shift={(4,.3)}] (-.71,1.71)--(-.71,.71);
\draw [-<-, shift={(4,.3)}] (-.71,.71) -- (1.71,.71);


\draw [ shift={(4,-.3)}](-.35,.35) node [anchor=north east]{$X$};
\draw [ shift={(4,-.3)}](.5,0) node [anchor=north] {$c$};
\draw [ shift={(4,-.3)}](1.35,.35) node [anchor=north west]{$W$};
\draw [ shift={(4,-.3)}](.5, .71) node [anchor= south east] {$a$};

\draw [ shift={(4,.3)}](1.71,1.21) node [anchor= west]{$Z$};
\draw [ shift={(4,.3)}](1.35,2.06) node [anchor=south west]{$b$};
\draw [ shift={(4,.3)}](.5,2.41) node [anchor=south]{$c$};
\draw [ shift={(4,.3)}](-.35,2.06) node [anchor=south east]{$b$};
\draw [ shift={(4,.3)}](-.71,1.21) node [anchor=east] {$Y$};
\draw [ shift={(4,.3)}](.5,.71) node [anchor= north west]{$a$};



%3
\draw [->-, shift={(8,0)}] (-.71,1.71)--(-.71,.71);
\draw [dashed, shift={(8,0)}] (-.71,.71) -- (0,0);
\draw [dashed, shift={(8,0)}] (0,0)--(1,0);
\draw [->-, shift={(8,0)}](1,0)--(1.71,.71);
\draw [dashed, shift={(8,0)}] (1.71,.71)--(1.71,1.71);
\draw [dashed, shift={(8,0)}] (1.71,1.71) -- (1,2.41);
\draw [-<-, shift={(8,0)}] (1,2.41) -- (0,2.41);
\draw [-<-, shift={(8,0)}] (0,2.41) -- (-.71,1.71);
\draw [dash pattern = on 2pt off 2pt, shift={(8,0)}] (1.71,0.71)--(0,2.41);


\draw[shift={(8,0)}] (-.35,.35) node [anchor=north east]{$Y$};
\draw [shift={(8,0)}](.5,0) node [anchor=north] {$X$};
\draw [shift={(8,0)}](1.35,.35) node [anchor=north west]{$c$};
\draw [shift={(8,0)}](1.71,1.21) node [anchor= west]{$W$};
\draw [shift={(8,0)}](1.35,2.06) node [anchor=south west]{$Z$};
\draw [shift={(8,0)}](.5,2.41) node [anchor=south]{$b$};
\draw [shift={(8,0)}](-.35,2.06) node [anchor=south east]{$c$};
\draw [shift={(8,0)}] (-.71,1.21) node [anchor=east] {$b$};

%4
\draw [-<-, shift={(12,-.3)}] (-.71,.71) -- (0,0);
\draw [dashed, shift={(12,-.3)}] (0,0)--(1,0);
\draw [dashed, shift={(12,-.3)}](1,0)--(1.71,.71);
\draw [->-, shift={(12,-.3)}] (-.71,.71) -- (1.71,.71);


\draw [dashed, shift={(12,.3)}] (1.71,.71)--(1.71,1.71);
\draw [dashed, shift={(12,.3)}] (1.71,1.71) -- (1,2.41);
\draw [->-, shift={(12,.3)}] (1,2.41) -- (0,2.41);
\draw [->-, shift={(12,.3)}] (0,2.41) -- (-.71,1.71);
\draw [-<-, shift={(12,.3)}] (-.71,1.71)--(-.71,.71);
\draw [->-, shift={(12,.3)}] (-.71,.71) -- (1.71,.71);


\draw [ shift={(12,-.3)}](-.35,.35) node [anchor=north east]{$d$};
\draw [ shift={(12,-.3)}](.5,0) node [anchor=north] {$W$};
\draw [ shift={(12,-.3)}](1.35,.35) node [anchor=north west]{$Z$};
\draw [ shift={(12,-.3)}](.5, .71) node [anchor= south east] {$b$};

\draw [ shift={(12,.3)}](1.71,1.21) node [anchor= west]{$Y$};
\draw [ shift={(12,.3)}](1.35,2.06) node [anchor=south west]{$X$};
\draw [ shift={(12,.3)}](.5,2.41) node [anchor=south]{$c$};
\draw [ shift={(12,.3)}](-.35,2.06) node [anchor=south east]{$d$};
\draw [ shift={(12,.3)}](-.71,1.21) node [anchor=east] {$c$};
\draw [ shift={(12,.3)}](.5,.71) node [anchor= north west]{$b$};

\draw [->] (2,3) [out=45, in=135] to (3,3);
\draw [->,shift={(4,0)}] (2,3) [out=45, in=135] to (3,3);
\draw [->, shift={(8,0)}] (2,3) [out=45, in=135] to (3,3);


\end{tikzpicture}
\caption{Poniendo los pares complementarios juntos. \label{fig:PASO6}}
\end{figure}
\clearpage







\section{Teorema de clasificación de superficies con borde}
Podemos dar ahora una clasificación de las superficies con borde, que nos servirá para comparar resultados con la prueba ZIP, donde se da directamente una clasificación de las superficies con borde.
Empecemos viendo una definición:
\begin{defin}%M%%%DEF: perforación
\label{def:perforacion}
Sea $S$ una superficie (con o sin borde). Sea $p\in \interior (S)$ y $U$ un entorno abierto de $p$ en $S$. Sea $\phi :U\to \R^n$ un homeomorfismo tal que $\phi (p)=0$. Sea $B=\phi ^{-1}(B_1(0))$. Decimos que la nueva superficie con borde $S^o=S\setminus B$ es $S$ \enfatiza{1-perforada}, y a $\partial B\subset S^o$ la llamamos \enfatiza{perforación}. Podemos repetir el proceso sobre $S^o$ sucesivamente, obteniendo $S$ $n$-perforada con un número finito $n\in \N$ de perforaciones.
\end{defin}

Vimos, en el ejemplo \autoref{eje:rep_borde} que un toro con dos perforaciones es una superficie con borde. Veamos una propiedad muy interesante:

\begin{prop}\label{prop:borde_perforaciones}
Toda superficie con borde compacta es homeomorfa a una superficie compacta con perforaciones.
\end{prop}
\begin{proof}
Sea $M$ una superficie compacta. Vamos a demostrar que el borde $\partial M$ es homeomorfo a una unión disjunta de circunferencias $\mathbb{S}^1$. Por el \autoref{corol:borde_n-1_variedad} se tiene que $\partial M$ es una 1-variedad. Salvo homeomorfismo, las 1-variedades conexas son $\R$ y $\mathbb{S}^1$. Por ser $\partial M$ un subconjunto cerrado de un compacto, es también compacto, y por tanto se tiene que cada componente conexa de $\partial M$ es homeomorfa a $\mathbb{S}^1$.

Por otro lado, vamos a ver que \textit{pegando} discos a las componentes de $\partial M$ se obtiene una superficie sin borde. Sean $S=\amalg_i \overline{\mathbb{B}}_i^2$ y $f:\partial S=\amalg_i \mathbb{S}^1\to \partial M$ un homeomorfismo. Considerando $f$ como una función de $\partial S$ en $M$, definimos la relación de equivalencia $\sim$ en la unión disjunta $M\amalg S$ generada por $a\sim f(a)$, y denotamos el espacio cociente resultante por $M^*=(M\amalg S)/\sim$. Tenemos que ver que $M^*$ es IIAN, Haussdorf y localmente homeomorfo a un espacio euclídeo de dimensión $2$. Empecemos con esto último. 

Sea $q:M\amalg S\to M^*$ la aplicación cociente, y sea $B=q(\partial M\cup \partial S)$. Notamos que $\interior M\amalg \interior S$ es un abierto saturado de $M\amalg S$, por lo que podemos restringir $q$ a una aplicación cociente que va de $\interior M\amalg \interior S$ en $M^*\setminus B$. Dado que esta restricción es inyectiva, es un homeomorfismo, y por tanto $M^*\setminus B$ es localmente homeomorfo a un espacio euclídeo de dimensión $2$. Por tanto nos queda considerar los puntos en $B$, pero esto lo tenemos por un argumento análogo al que usamos en la demostración de \autoref{prop:poligonos}. 

Por otro lado, el espacio cociente $M^*$ es IIAN por la proposición \autoref{prop:sub_IIAN}. Para probar que es Haussdorf, hay que ver que las fibras de $q$ se pueden separar por abiertos saturados. Pero esto siempre es posible escogiendo bolas coordenadas suficientemente pequeñas. Así pues, $M^*$ es una superficie. 
\end{proof}

Dada una superficie con borde compacta $M$, denotaremos por $M^*$ la superficie compacta sin borde obtenida mediante el proceso anterior, que llamaremos \enfatiza{pegar} un disco cerrado a cada componente conexa de $\partial M$. Si en cambio empezamos por una superficie compacta sin borde $M^*$ y construimos una superficie con borde haciendo perforaciones, ¿Van a ser homeomorfas las superficies resultantes de hacer las perforaciones en lugares distintos? La respuesta es afirmativa, pero antes necesitamos ver la siguiente proposición:

\begin{figure}
\centering
\begin{tikzpicture}[use optics, scale=2,>=angle 60]

\draw (.5,1.5) -- +(1,0) -- (1.5,.5)--cycle ;
\draw [dashed] (1.5,.5)[out=-90, in=-45] to (0,0) [out=135, in=180] to (.5,1.5) ;
\draw (0,0) node [anchor= north east] {$\partial M$};

\draw [shift={(3,0)}] (.5,1.5) -- +(1,0) -- (1.5,.5)--cycle ;
\draw [shift={(3,0)}, dashed] (1.5,.5)[out=-90, in=-45] to (0,0) [out=135, in=180] to (.5,1.5) ;
\draw [shift={(3,0)}] (0,0) node [anchor= north east] {$\partial M$};
\draw [shift={(3,0)}] (1.5,1.5) -- (1,1);
\draw [shift={(3,0)}] (.5,1.5)--(1.5,1);
\draw [shift={(3,0)}]  (1.5,.5) -- (1,1.5);

\draw [->](2,1.7) [out=30, in=150] to  (3,1.7);

\end{tikzpicture}
\caption{Subdivisión baricéntrica de un 2-símplice que tiene dos aristas en $\partial M$.\label{fig:baricentrica}}
\end{figure}


\begin{prop}\label{prop:sup_borde_rep}
Toda superficie con borde compacta admite una representación poligonal.
\end{prop}
\begin{proof}
Sea $M$ una superficie con borde compacta. Sabemos que $M$ es triangulable por el poliedro de un complejo simplicial en el cual hay dos clases de 1-símplices: los 1-símplices cuyo interior se corresponde a puntos interiores y los 1-símplices que se corresponden a componentes del borde de $M$. 

Podemos suponer que la triangulación satisface que ningún 1-símplice tiene los dos vértices contenidos en el borde a menos que esté contenido completamente en el borde, y un 2-símplice no tiene más de una arista en el borde. Si no fuese así, pordemos conseguir esta condición subdividiendo cada arista en dos y cada 2-símplice en seis 2-símplices mediante lo que llamamos subdivisión baricéntrica, representada en la \autoref{fig:baricentrica}. Subdividiendo más veces si es necesario, llegamos a una triangulación que satisface la siguiente condición más fuerte: Si $P_i$ y $P_j$ son  dos 2-símplices tal que cada uno tiene una arista contenida en el borde de $M$, entonces, o bien $P_i$ y $P_j$ son disjuntos, o bien tienen un vértice en común $v$, que es un vértice del borde de $M$.

Denotemos por $B_1,\dots , B_k$ las componentes de $\partial M$. Si $P$ es un 2-símplice que encuentra a $B_i$ en algún punto suyo, entonces $P$ tiene exáctamente dos aristas que tienen un vértice en $B_i$ pero que no están contenidos en $B_i$. Similarmente, si $\sigma$ es un 1-símplice que tiene un vértice en $B_i$ pero que no yace sobre $B_i$, entonces $\sigma$ es una arista de dos 2-símplices que encuentran a $B_i$. Podemos por tanto ordenar los 2-símplices que encuentran $B_i$ y los 1-símplices que son aristas suyas pero que no yacen sobre $B_i$ de la siguiente manera: 
\begin{equation*}
P_1,e_1,P_2,e_2,\dots ,P_n,e_n,P_{n+1}=P_1
\end{equation*}
donde cada $e_j$ es una arista de $P_j$ y $P_{j+1}$, y donde cada 2-símplice $P_k$ tiene por tanto como aristas a $e_{k-1}$ y $e_k$. ((Habría que demostrar que es única¿?)) Para cada componente $B_i$ podemos escribir de esa forma un ciclo de 2-símplices y aristas, que es único. Para cada componente del borde $B_i$, la unión de los 2-símplices de su ciclo asociado $\mathcal{P}_i=\bigcup_{j}P_j$ es homeomorfa a una región poligonal con una perforación. La \autoref{fig:triangulacion_perf} muestra un ejemplo cuando $n=?????$. Sean $P'_1,\dots P'_l$ los 2-símplices de la triangulación de $M$ que no pertenecen a ninguna de las $\mathcal{P}_i$, y llamamos $$\mathcal{P}=\left( \bigcup_{i=1}^{k} \mathcal{P}_i\right) \cup \left( \bigcup_{j=1}^{l} P'_j\right).$$
Ahora sólo falta aplicar sobre $\mathcal{P}$ el resto de la demostración de la \autoref{prop:superficie_tiene_representacion}.
\end{proof}

\begin{figure}
\centering
\begin{tikzpicture}[use optics,>=angle 60,scale=.8]
\draw (.38,1.23) node {$B_i$};
\draw [->](.59,1.4) [out=10,in=170] to (1.55,1.59);
\draw [->](.1,1) [out=-120, in=0] to (-.66,.65);
\draw [->] (0.71,1.02) [out=-20,in=100] to (1.18,.54);
\draw [->] (0.07, 1.45) [out=135, in=-45] to (-.48,1.98);
\draw [->] (0.45,1.56) [out=90, in=-135] to (.8,2.08);
\draw [->] (0.36,.95) -- (.24,.23);
\draw [->] (.31,1.59) [out=110, in=-120] to (0.22, 2.21);
\draw [->] (0.01,1.31) -- (-.86,1.45);
\draw [->] (.8,1.21) -- (1.76,1.05);



\fill[ fill=black,fill opacity= 0.1] (-1.0734974348383717,1.9534822942323429) -- (-1.2251018153471387,1.062806558743338) -- (-2.267381931344911,3.2800206236840523) -- cycle;
\fill[ fill=black,fill opacity= 0.1] (-2.267381931344911,3.2800206236840523) -- (-2.7221950728712114,1.4797186051424467) -- (-1.2251018153471387,1.062806558743338) -- cycle;
\fill[ fill=black,fill opacity= 0.1] (-2.7221950728712114,1.4797186051424467) -- (-2.13472809839974,-0.30163286583556315) -- (-1.2251018153471387,1.062806558743338) -- cycle;
\fill[ fill=black,fill opacity= 0.1] (-1.2251018153471387,1.062806558743338) -- (-0.5618326506212836,0) -- (-2.13472809839974,-0.30163286583556315) -- cycle;
\fill[ fill=black,fill opacity= 0.1] (-0.5618326506212836,0) -- (1,0) -- (-0.5428821030576877,-1.4197151720877181) -- cycle;
\fill[ fill=black,fill opacity= 0.1] (-2.13472809839974,-0.30163286583556315) -- (-0.5428821030576877,-1.4197151720877181) -- (-0.5618326506212836,0) -- cycle;
\fill[ fill=black,fill opacity= 0.1] (1,0) -- (1.5985297716286444,-1.4197151720877181) -- (-0.5428821030576877,-1.4197151720877181) -- cycle;
\fill[ fill=black,fill opacity= 0.1] (1,0) -- (1.8448868899553905,0.4184879415810791) -- (1.5985297716286444,-1.4197151720877181) -- cycle;
\fill[ fill=black,fill opacity= 0.1] (1.8448868899553905,0.4184879415810791) -- (3.796793289005764,0.19108137081792892) -- (1.5985297716286444,-1.4197151720877181) -- cycle;
\fill[ fill=black,fill opacity= 0.1] (1.8448868899553905,0.4184879415810791) -- (3.417782337733847,1.9155811991051512) -- (3.796793289005764,0.19108137081792892) -- cycle;
\fill[ fill=black,fill opacity= 0.1] (3.417782337733847,1.9155811991051512) -- (2.015441818027753,1.422866962451659) -- (1.8448868899553905,0.4184879415810791) -- cycle;
\fill[ fill=black,fill opacity= 0.1] (2.015441818027753,1.422866962451659) -- (1.39007374842909,2.1429877698683013) -- (3.417782337733847,1.9155811991051512) -- cycle;
\fill[ fill=black,fill opacity= 0.1] (3.417782337733847,1.9155811991051512) -- (2.849265910825971,3.3558228139384356) -- (1.39007374842909,2.1429877698683013) -- cycle;
\fill[ fill=black,fill opacity= 0.1] (1.39007374842909,2.1429877698683013) -- (1.636430866755836,4.113844716482269) -- (2.849265910825971,3.3558228139384356) -- cycle;
\fill[ fill=black,fill opacity= 0.1] (1.636430866755836,4.113844716482269) -- (0.6320518458852555,2.465147078449431) -- (1.39007374842909,2.1429877698683013) -- cycle;
\fill[ fill=black,fill opacity= 0.1] (-2.267381931344911,3.2800206236840523) -- (-0.2965249847309415,4.473905120190591) -- (-1.0734974348383717,1.9534822942323429) -- cycle;
\fill[ fill=black,fill opacity= 0.1] (-1.0734974348383717,1.9534822942323429) -- (-0.1828216993493664,2.446196530885835) -- (-0.2965249847309415,4.473905120190591) -- cycle;
\fill[ fill=black,fill opacity= 0.1] (-0.2965249847309415,4.473905120190591) -- (0.6320518458852555,2.465147078449431) -- (-0.1828216993493664,2.446196530885835) -- cycle;
\fill[ fill=black,fill opacity= 0.1] (-0.2965249847309415,4.473905120190591) -- (1.636430866755836,4.113844716482269) -- (0.6320518458852555,2.465147078449431) -- cycle;
\draw [ ] (-1.0734974348383717,1.9534822942323429)-- (-1.2251018153471387,1.062806558743338);
\draw [ ] (-1.2251018153471387,1.062806558743338)-- (-2.267381931344911,3.2800206236840523);
\draw [ ] (-2.267381931344911,3.2800206236840523)-- (-1.0734974348383717,1.9534822942323429);
\draw [ ] (-2.267381931344911,3.2800206236840523)-- (-2.7221950728712114,1.4797186051424467);
\draw [ ] (-2.7221950728712114,1.4797186051424467)-- (-1.2251018153471387,1.062806558743338);
\draw [ ] (-1.2251018153471387,1.062806558743338)-- (-2.267381931344911,3.2800206236840523);
\draw [ ] (-2.7221950728712114,1.4797186051424467)-- (-2.13472809839974,-0.30163286583556315);
\draw [ ] (-2.13472809839974,-0.30163286583556315)-- (-1.2251018153471387,1.062806558743338);
\draw [ ] (-1.2251018153471387,1.062806558743338)-- (-2.7221950728712114,1.4797186051424467);
\draw [ ] (-1.2251018153471387,1.062806558743338)-- (-0.5618326506212836,0);
\draw [ ] (-0.5618326506212836,0)-- (-2.13472809839974,-0.30163286583556315);
\draw [ ] (-2.13472809839974,-0.30163286583556315)-- (-1.2251018153471387,1.062806558743338);
\draw [ ] (-0.5618326506212836,0)-- (1,0);
\draw [ ] (1,0)-- (-0.5428821030576877,-1.4197151720877181);
\draw [ ] (-0.5428821030576877,-1.4197151720877181)-- (-0.5618326506212836,0);
\draw [ ] (-2.13472809839974,-0.30163286583556315)-- (-0.5428821030576877,-1.4197151720877181);
\draw [ ] (-0.5428821030576877,-1.4197151720877181)-- (-0.5618326506212836,0);
\draw [ ] (-0.5618326506212836,0)-- (-2.13472809839974,-0.30163286583556315);
\draw [ ] (1,0)-- (1.5985297716286444,-1.4197151720877181);
\draw [ ] (1.5985297716286444,-1.4197151720877181)-- (-0.5428821030576877,-1.4197151720877181);
\draw [ ] (-0.5428821030576877,-1.4197151720877181)-- (1,0);
\draw [ ] (1,0)-- (1.8448868899553905,0.4184879415810791);
\draw [ ] (1.8448868899553905,0.4184879415810791)-- (1.5985297716286444,-1.4197151720877181);
\draw [ ] (1.5985297716286444,-1.4197151720877181)-- (1,0);
\draw [ ] (1.8448868899553905,0.4184879415810791)-- (3.796793289005764,0.19108137081792892);
\draw [ ] (3.796793289005764,0.19108137081792892)-- (1.5985297716286444,-1.4197151720877181);
\draw [ ] (1.5985297716286444,-1.4197151720877181)-- (1.8448868899553905,0.4184879415810791);
\draw [ ] (1.8448868899553905,0.4184879415810791)-- (3.417782337733847,1.9155811991051512);
\draw [ ] (3.417782337733847,1.9155811991051512)-- (3.796793289005764,0.19108137081792892);
\draw [ ] (3.796793289005764,0.19108137081792892)-- (1.8448868899553905,0.4184879415810791);
\draw [ ] (3.417782337733847,1.9155811991051512)-- (2.015441818027753,1.422866962451659);
\draw [ ] (2.015441818027753,1.422866962451659)-- (1.8448868899553905,0.4184879415810791);
\draw [ ] (1.8448868899553905,0.4184879415810791)-- (3.417782337733847,1.9155811991051512);
\draw [ ] (2.015441818027753,1.422866962451659)-- (1.39007374842909,2.1429877698683013);
\draw [ ] (1.39007374842909,2.1429877698683013)-- (3.417782337733847,1.9155811991051512);
\draw [ ] (3.417782337733847,1.9155811991051512)-- (2.015441818027753,1.422866962451659);
\draw [ ] (3.417782337733847,1.9155811991051512)-- (2.849265910825971,3.3558228139384356);
\draw [ ] (2.849265910825971,3.3558228139384356)-- (1.39007374842909,2.1429877698683013);
\draw [ ] (1.39007374842909,2.1429877698683013)-- (3.417782337733847,1.9155811991051512);
\draw [ ] (1.39007374842909,2.1429877698683013)-- (1.636430866755836,4.113844716482269);
\draw [ ] (1.636430866755836,4.113844716482269)-- (2.849265910825971,3.3558228139384356);
\draw [ ] (2.849265910825971,3.3558228139384356)-- (1.39007374842909,2.1429877698683013);
\draw [ ] (1.636430866755836,4.113844716482269)-- (0.6320518458852555,2.465147078449431);
\draw [ ] (0.6320518458852555,2.465147078449431)-- (1.39007374842909,2.1429877698683013);
\draw [ ] (1.39007374842909,2.1429877698683013)-- (1.636430866755836,4.113844716482269);
\draw [ ] (-2.267381931344911,3.2800206236840523)-- (-0.2965249847309415,4.473905120190591);
\draw [ ] (-0.2965249847309415,4.473905120190591)-- (-1.0734974348383717,1.9534822942323429);
\draw [ ] (-1.0734974348383717,1.9534822942323429)-- (-2.267381931344911,3.2800206236840523);
\draw [ ] (-1.0734974348383717,1.9534822942323429)-- (-0.1828216993493664,2.446196530885835);
\draw [ ] (-0.1828216993493664,2.446196530885835)-- (-0.2965249847309415,4.473905120190591);
\draw [ ] (-0.2965249847309415,4.473905120190591)-- (-1.0734974348383717,1.9534822942323429);
\draw [ ] (-0.2965249847309415,4.473905120190591)-- (0.6320518458852555,2.465147078449431);
\draw [ ] (0.6320518458852555,2.465147078449431)-- (-0.1828216993493664,2.446196530885835);
\draw [ ] (-0.1828216993493664,2.446196530885835)-- (-0.2965249847309415,4.473905120190591);
\draw [ ] (-0.2965249847309415,4.473905120190591)-- (1.636430866755836,4.113844716482269);
\draw [ ] (1.636430866755836,4.113844716482269)-- (0.6320518458852555,2.465147078449431);
\draw [ ] (0.6320518458852555,2.465147078449431)-- (-0.2965249847309415,4.473905120190591);
\end{tikzpicture}
\caption{Triangulación cerca de una componente del borde de $M$.\label{fig:triangulacion_perf}}
\end{figure}
\begin{eje}
Veamos las representaciones estándar de las superficies fundamentales perforadas. Para ello, hacemos un corte $c_i$ desde un mismo vértice a cada perforación asociada a la componente $B_i$, de forma que estos no se solapen con las perforaciones. En la \autoref{fig:toro_perforado} podemos ver un ejemplo de la representación estándar de la suma conexa de dos toros con dos perforaciones.
\begin{itemize}
\item [(a)] \textit{Esfera con k perforaciones.} 
$$\langle a, c_1,\dots ,c_k, B_1, \dots ,B_k \mid aa^{-1}c_1B_1c_1^{-1}\dots c_kB_kc_k^{-1}\rangle$$ 
\item[(b)] \textit{Suma conexa de n toros con k perforaciones.}
$$\langle a_1,\dots ,a_n, b_1,\dots ,b_n, c_1,\dots ,c_k, B_1,\dots ,B_k\mid a_1b_1a_1^{-1}b_1^{-1}\dots a_nb_na_n^{-1}b_n^{-1}c_1B_1c_1^{-1}\dots c_kB_kc_k^{-1}\rangle$$ 
\item[(c)] \textit{Suma conexa de n planos proyectivos con k perforaciones.} 
$$\langle a_1,\dots ,a_n, c_1,\dots ,c_k, B_1,\dots ,B_k\mid a_1a_1\dots a_na_nc_1B_1c_1^{-1}\dots c_kB_kc_k^{-1} \rangle$$
\end{itemize}
\end{eje}

\begin{figure}[b]
\centering
\begin{subfigure}{1\linewidth}
\centering
\begin{tikzpicture}[use optics, >=angle 60, scale=2]

\fill[  ,fill=black,fill opacity=0.1] (0,0) -- (0.9238207149966959,0.3828253995531037) -- (1.3063621711370152,1.3067637277623212) -- (0.9235367715839117,2.230584442759017) -- (-0.00040155662530578324,2.6131258988993364) -- (-0.9242222716220017,2.230300499346233) -- (-1.3067637277623212,1.3063621711370155) -- (-0.9239383282092178,0.38254145614031965) -- cycle;

\draw  [->-]  (0,0) -- (0.9238207149966959,0.3828253995531037)node[midway,below] {$b_2$};
\draw  [->-](0.9238207149966959,0.3828253995531037)-- (1.3063621711370152,1.3067637277623212)node[midway, anchor=west] {$a_2$};
\draw  [-<-](1.3063621711370152,1.3067637277623212)-- (0.9235367715839117,2.230584442759017)node[midway,anchor=west] {$b_2$};
\draw  [-<-](0.9235367715839117,2.230584442759017)-- (-0.00040155662530578324,2.6131258988993364)node[midway,anchor=south west] {$a_2$};
\draw  [->-](-0.00040155662530578324,2.6131258988993364)-- (-0.9242222716220017,2.230300499346233)node[midway,anchor=south east] {$b_1$};
\draw  [->-](-0.9242222716220017,2.230300499346233)-- (-1.3067637277623212,1.3063621711370155)node[midway,anchor=east] {$a_1$};
\draw  [-<-](-1.3067637277623212,1.3063621711370155)-- (-0.9239383282092178,0.38254145614031965)node[midway,anchor=east] {$b_1$};
\draw  [-<-](-0.9239383282092178,0.38254145614031965)-- (0,0)node[midway,below] {$a_1$};
\draw  [-<-](0,0)-- (-0.5128819104663424,1.2073243438172108) node [midway, anchor=east] {$c_1$};
\draw  [-<-](0,0)-- (0.4909162594319499,1.202150126446601)node [midway, anchor=west] {$c_2$};

\filldraw[white]  (-0.5,1.5)circle (0.2929590130705814cm);
\filldraw [white] (0.5,1.5) circle (0.2979883580250803cm);
\draw  [->-={at=0.25}](-0.5,1.5) circle (0.2929590130705814cm);
\draw  [->-={at=0.25}](0.5,1.5) circle (0.2979883580250803cm);
\draw (-0.27,1.5) node [anchor=east] {$B_1$};
\draw (0.27,1.5) node [anchor=west] {$B_2$};

\filldraw (0,0) circle (1pt);
\end{tikzpicture}
\caption{ }
\end{subfigure}
\begin{subfigure}{1\linewidth}
\centering
\begin{tikzpicture}[use optics,scale=2]



\fill[  fill=black,fill opacity=0.1] (0,0) -- (0.6,0) -- (1.1405813207414512,0.2603302434705348) -- (1.5146752018566914,0.7294291329513525) -- (1.6481877622304801,1.3143858802604464) -- (1.5146752018566916,1.8993426275695406) -- (1.1405813207414515,2.3684415170503583) -- (0.6,2.6287717605208933) -- (0,2.628771760520894) -- (-0.5405813207414512,2.3684415170503588) -- (-0.9146752018566913,1.899342627569541) -- (-1.0481877622304798,1.314385880260447) -- (-0.9146752018566915,0.7294291329513529) -- (-0.5405813207414516,0.260330243470535) -- cycle;


\draw [ -<- ] (0,0)-- (0.6,0) node [anchor=north, midway] {$c_1$};
\draw [ -<- ] (0.6,0)-- (1.1405813207414512,0.2603302434705348)node [anchor=north, midway] {$B_1$};
\draw [ ->- ] (1.1405813207414512,0.2603302434705348)-- (1.5146752018566914,0.7294291329513525) node [anchor=north west, midway] {$c_1$};
\draw [ -<- ] (1.5146752018566914,0.7294291329513525)-- (1.6481877622304801,1.3143858802604464) node [midway, anchor=west] {$c_2$};
\draw [ -<- ] (1.6481877622304801,1.3143858802604464)-- (1.5146752018566916,1.8993426275695406) node [midway, anchor=west] {$B_2$};
\draw [ ->- ] (1.5146752018566916,1.8993426275695406)-- (1.1405813207414515,2.3684415170503583) node [anchor=south west, midway]  {$c_2$};
\draw [ ->- ] (1.1405813207414515,2.3684415170503583)-- (0.6,2.6287717605208933)node [midway, anchor=south] {$b_2$};
\draw [  ->-] (0.6,2.6287717605208933)-- (0,2.628771760520894)node [midway, anchor=south] {$a_2$};
\draw [ -<- ] (0,2.628771760520894)-- (-0.5405813207414512,2.3684415170503588)node [midway, anchor=south] {$b_2$};
\draw [ -<- ] (-0.5405813207414512,2.3684415170503588)-- (-0.9146752018566913,1.899342627569541)node [anchor= south east, midway] {$a_2$};
\draw [ ->- ] (-0.9146752018566913,1.899342627569541)-- (-1.0481877622304798,1.314385880260447)node [midway, anchor=east]{$b_1$};
\draw [ ->- ] (-1.0481877622304798,1.314385880260447)-- (-0.9146752018566915,0.7294291329513529) node [anchor=east, midway] {$a_1$};
\draw [  -<-] (-0.9146752018566915,0.7294291329513529)-- (-0.5405813207414516,0.260330243470535) node [anchor=north east, midway]{$b_1$};
\draw [ -<- ] (-0.5405813207414516,0.260330243470535)-- (0,0)node [anchor=north, midway] {$a_1$};

\filldraw (0,0) circle (1pt);
\filldraw (1.5146752018566916,1.8993426275695406)  circle (1pt);
\end{tikzpicture}
\caption{ }
\end{subfigure}
\caption{La superficie $\Toro \# \Toro$  con dos perforaciones.\label{fig:toro_perforado}}
\end{figure}

\begin{tma}[Clasificación de superficies con borde compactas]
Sean $M_1$ y $M_2$ superficies con borde compactas tales que $\partial M_1$ y $\partial M_2$ tienen el mismo número de componentes. Entonces $M_1$ y $M_2$ son homeomorfas si y solo si las superficies $M_1^*$ y $M_2^*$  son homeomorfas.
\end{tma}
\begin{proof}
Probaremos el teorema demostrando que $M_1$ y $M_2$ son homeomorfas a una misma representación estándar de las del ejemplo anterior.
Sea $M$ una superficie con borde compacta, y sea una representación suya como la de la \autoref{prop:sup_borde_rep}. El teorema se prueba aplicando los siete pasos de la demostración del Teorema de Clasificación de Superficies Compactas a esta representación poligonal, teniendo en cuenta que las operaciones de cortar y pegar que aparecen sucesivamente se hagan evitando las perforaciones. Así podemos asegurar que durante el proceso el número de perforaciones no cambia. Como resultado, obtenemos una de las tres representaciones estándar de las superficies sin borde, con $k$ perforaciones en su interior. Para completar la prueba, hacemos los cortes $c_1,\dots ,c_k$ desde el vértice inicial de la representación poligonal a cada una de las perforaciones $B_1,\dots ,B_k$, de forma inductiva y sin que se solapen, obteniendo uno de los polígonos estándar del ejemplo anterior.
\end{proof}




%%%%%%%%%%%%%%%%%%%%%%%%%%%%%%%%%%%%%%%%%%%%%%%%%%%%%%%%%%%%%%%%%%%%%%%%%%%%%%%%%%%%%%%%%%%%%%%%%%%%%%%%%%%%%%%%%%%%%%%%%%%%%%%%%%%%%%%%%%%%%%%%%%%%%%%%%%%%%%%%%%%%%%%%%%%%%%%%%%%%%%%%%%%%%%%%%%%%%%%%%%%%%%%%%%%%%%%%%%%%%%%%%%%%%%%%%%%%%%%%%%%%%%%%%%%%%%%%%%%%%%%%%%%%%%%%%%%%%%%%%%%%%%%%%%%%%%%%%%%%%%%%%%%%%%%%%%%%%%%%%%%%%%%%%%%%%%%%%%%%%%%%%%%%%%%%%%%%%%%%%%%%%%%%%%%%%%%%%%%%%%%%%%%%%%%%%%%%%%%%%%%%%%%%%%%%%%%%%%%%%%%%%%%%%%%%%%%%%%%%%%%%%%%%%%%%%%%%%%%%%%%%%%%%%%%%%%%%%%%%%%%%%%%%%%%%%%%%%%%%%%%%%%%%%%%%%%%%%%%%%%%%%%%%%%%%%%%%%%%%%%%%%%%%%%%%%%%%%%%%%%%%%%%%%%%%%%%%%%%%%%%%%%%%%%%%%%%%%%%%%%%%%%%%%%%%%%%%%%%%%%%%%%%%%%%%%%%%%%%%%%%%%%%%%%%%%%%%%%%%%%%%%%%%%%%%%%%%%%%%%%%%%%%%%%%%%%%%%%%%%%%%%%%%%%%%%%%%%%%%%%%%%%%%%%%%%%%%%%%%%%%%%%%%%%%%%%%%%%%%%%%%%%%%%%%%%%%%%%%%%%%%%%%%%%%%%%%%%%%%%%%%%%%%%%%%%%%%%%%%%%%%%%%%%%%%%%%%%%%%%%%%%%%%%%%%%%%%%%%%%%%%%%%%%%%%%%%%%%%%%%%%%%%%%%%%%%%%%%%%%%%%%%%%%%%%%%%%%%%%%%%%%%%%%%%%%%%%%%%%%%%%%%%%%%%%%%%%%%%%%%%%%%%%%%%%%%%%%%%%%%%%%%%%%%%%%%%%%%%%%%%%%%%%%%%%%%%%%%%%%%%%%%%%%%%%%%%%%%%%%%%%%%%%%%%%%%%%%%%%%%%%%%%%%%%%%%%%%%%%%%%%%%%%%%%%%%%%%%%%%%%%%%%%%%%%%%%%%%%%%%%%%%%%%%%%%%%%%%%%%%%%%%%%%%%%%%%%%%%%%%%%%%%%%%%%%%%%%%%%%%%%%%%%%%%%%%%%%%%%%%%%%%%%%%%%%%%%%%%%%%%%%%%%%%%%%%%%%%%%%%%%%%%%%%%%%%%%%%%%%%%%%%%%%%%%%%%%%%%%%%%%%%%%%%%%%%%%%%%%%%%%%%%%%%%%%%%%%%%%%%%%%%%%%%%%%%%%%%%%%%%%%%%%%%%%%%%%%%%%%%%%%%%%%%%%%%%%%%%%%%%%%%%%%%%%%%%%%%%%%%%%%%%%%%%%%%%%%%%%%%%%%%%%%%%%%%%%%%%%%%%%%%%%%%%%%%%%%%%%%%%%%%%%%%%%%%%%%%%%%%%%%%%%%%%%%%%%%%%%%%%%%%%%%%%%%%%%%%%%%%%%%%%%%%%%%%%%%%%%%%%%%%%%%%%%%%%%%%%%%%%%%%%%%%%%%%%%%%%%%%%%%%%%%%%%%%%%%%%%%%%%%%%%%%%%%%%%%%%%%%%%%%%%%%%%%%%%%%%%%%%%%%%%%%%%%%%%%%%%%%%%%%%%%%%%%%%%%%%%%%%%%%%%%%%%%%%%%%%%%%%%%%%%%%%%%%%%%%%%%%%%%%%%%%%%%%%%%%%%%%%%%%%%%%%%%%%%%%%%%%%%%%%%%%%%%%%%%%%%%%%%%%%%%%%%%%%%%%%%%%%%%%%%%%%%%%%%%%%%%%%%%%%%%%%%%%%%%%%%

\chapter{La prueba ZIP de Conway}

\section{Cremalleras}


Conway utiliza las cremalleras (\textit{zips} en inglés) para describir cómo actúan las identificaciones topológicas. Cada cremallera actúa sobre una o dos perforaciones de una superficie. Están formadas por dos \textit{zips} (dos partes dentadas) fijadas la/s perforación/es y un \textit{zipper} (el deslizador). Al cerrar el \textit{zipper}, las \textit{zips} se juntan identificándose. Trato de dar una definición rigurosa:


\begin{defin}%%%DEF: ZIP
Sea $S$ una superficie con borde compacta. Una \enfatiza{cremallera} es una identificación entre dos subconjuntos cerrados del borde de $S$. A este par lo llamamos \enfatiza{par-zip}.
\end{defin}


En la \textit{prueba ZIP}, Conway nos explica gráficamente las posibles formas de unir cremalleras. 

\begin{defin}
Sea $S$ una superficie. Definimos cuatro formas elementales de identificar pares-\textit{zip} en perforaciones de 
$S$ perforada:

\begin{itemize}
\item[1.] \enfatiza{Cap}: Los pares zip yacen cada uno sobre la mitad de una misma perforación con orientaciones opuestas (\autoref{fig:cap}).
\item[2.] \enfatiza{Crosscap}: Los pares zip yacen cada uno sobre la mitad de una misma perforación con la misma orientación (\autoref{fig:crosscap}).
\item[3.] \enfatiza{Handle}: Los pares zip yacen cada uno sobre una perforación distinta de S con orientaciones opuestas (\autoref{fig:handle}).
\item[4.] \enfatiza{Crosshandle}: Los pares zip yacen cada uno sobre una perforación distinta de S con la misma orientación (\autoref{fig:crosshandle}).
\end{itemize}
\end{defin}


\begin{figure}
\begin{subfigure}{1\textwidth} %%%%FIG: Cap
\centering
\begin{tikzpicture} [use optics]
%\draw [help lines, step=1mm](0,-2) grid (6,2);
\fill [color=gray!10] (-3,-1.5) -- (-5,0) -- (-3,1) -- (-1,0) -- cycle;
\fill [gray!10](3,-1.5) -- (5,0) -- (3,1) -- (1,0) -- cycle;
\fill [gray!20] (2,-.1) [out=80, in=180] to (3,0.7) [out=0, in=100] to (4,-.1) arc[x radius=1, y radius=0.5, start angle=0, end angle=-180];
\fill [white] (-4,-.1) arc[x radius=1, y radius=0.5, start angle=180, end angle=-180];
\draw (-3,-1.5) -- (-5,0) -- (-3,1) -- (-1,0) -- cycle;
\draw (3,-1.5) -- (5,0) -- (3,1) -- (1,0) -- cycle;
\draw [-<-={at=0.1}, ->-={at=0.6}] (-4,-.1) arc[x radius=1, y radius=0.5, start angle=180, end angle=-180];
\draw [dashed](2,-.1) arc[x radius=1, y radius=0.5, start angle=180, end angle=0];
\draw (2,-.1) arc[x radius=1, y radius=0.5, start angle=180, end angle=360];
\draw[-<-={at=0.48}] [dash pattern= on 50pt off 2pt on 2pt off 2pt on 2pt off 2pt on 2pt off 2pt on 2pt off 2pt] (2.4,-.5) [out=90, in=180] to (3.1,.685) [out=0, in=90] to (3.6,.3);
\draw (2,-.1) [out=80, in=180] to (3,0.7) [out=0, in=100] to (4,-.1);
%\draw (2.4,-.5)-- (3.6,.3);

\fill [color=black] (-3.6,-.5) circle (1.pt);
\fill [color=black] (-2.4,.3) circle (1.pt);
\fill [color=black] (3.6,.3) circle (1.pt);
\fill [color=black] (2.4,-.5) circle (1.pt);

\end{tikzpicture}
\caption{Construcción del \textit{cap}.\label{fig:cap}}
\end{subfigure}



\begin{subfigure}{1\textwidth} %%%%FIG: Crosscap
\centering
\begin{tikzpicture} [use optics]
%\draw [help lines, step=1mm](0,-2) grid (6,2);
\fill [color=gray!10] (-3,-1.5) -- (-5,0) -- (-3,1) -- (-1,0) -- cycle;
\fill [gray!10](3,-1.5) -- (5,0) -- (3,1) -- (1,0) -- cycle;
\fill [white] (-4,-.1) arc[x radius=1, y radius=0.5, start angle=180, end angle=-180];
\fill [gray!20](2,-.1) [out=90,in=180] to (3.1,2)[out=0, in=90] to (4,-.1)  arc[x radius=1, y radius=0.5, start angle=0, end angle=-180];
\draw (-3,-1.5) -- (-5,0) -- (-3,1) -- (-1,0) -- cycle;
\draw[dash pattern= on 103pt off 2pt on 2pt off 2pt on 2pt off 2pt on 2pt off 2pt on 2pt off 2pt on 2pt off 2pt on 2pt off 2pt on 2pt off 2pt on 2pt off 2pt on 2pt off 2pt on 2pt off 2pt on 2pt off 2pt on 2pt off 2pt on 2pt off 2pt  on 2pt off 2pt  on 2pt off 3pt  on 150pt] (3,-1.5) -- (5,0) -- (3,1) -- (1,0);
\draw (1,0) -- (3,-1.5);
\draw [-<-={at=0.4}, -<-={at=0.9}] (-4,-.1) arc[x radius=1, y radius=0.5, start angle=180, end angle=-180];
\draw [dashed](2,-.1) arc[x radius=1, y radius=0.5, start angle=180, end angle=0];
\draw (2,-.1) arc[x radius=1, y radius=0.5, start angle=180, end angle=360];
\draw (2,-.1) [out=90,in=180] to (3.1,2)[out=0, in=90] to (4,-.1);
\draw (3.1,2) -- (3.1,0.7);

\fill [color=black] (-3.6,.3) circle (1.pt);
\fill [color=black] (-2.4,-.5) circle (1.pt);

%\fill [color=black] (-4,-.1) circle (1.pt);
%\fill [color=black] (-2,-.1) circle (1.pt);

%\fill [color=black] (-3.6,-.5) circle (1.pt);
%\fill [color=black] (-2.4,.3) circle (1.pt);


\draw (2.17,1.15) [out=-80, in=-135, looseness=0.6] to (3.1,1.2) [out=-45, in=-90, looseness=0.6] to (3.9,1.2);
\draw [dashed] (2.17,1.15) [out=90, in=135, looseness=0.6] to (3.1,1.2) [out=45, in=90, looseness=0.6] to (3.9,1.2);
\end{tikzpicture}
\caption{Construcción del \textit{crosscap}.\label{fig:crosscap}}
\end{subfigure}


\begin{subfigure}{1\textwidth} %%%%FIG: Handle
\centering

\begin{tikzpicture} [use optics]
%\draw [help lines, step=1mm](-6,-2) grid (6,2);

\fill [gray!10] (-3,-1.5) -- (-5,0) -- (-3,1) -- (-1,0) -- cycle;
\fill [white](-4.4,-.05) arc[x radius=0.5, y radius=0.25, start angle=180, end angle=540];
\fill [white](-1.6,-.05) arc [x radius=0.5, y radius=0.25, start angle=0, end angle=360];
\fill [gray!10](3,-1.5) -- (5,0) -- (3,1) -- (1,0) -- cycle;
\fill [gray!20](1.6,-.05) [out=90, in=180] to (3,2) [out=0, in=90] to (4.4,-.05) arc [x radius=0.5, y radius=0.25, start angle=0, end angle=-180]  [out=90, in=0] to (3,1.2) [out=180, in=90] to (2.6,-.05) arc [x radius=0.5, y radius=0.25, start angle=0, end angle=-180] ;

\draw (-3,-1.5) -- (-5,0) -- (-3,1) -- (-1,0) -- cycle;
\draw [dash pattern= on 19pt off 2 pt on 2pt off 2pt on 2pt off 2pt on 2pt off 2pt on 2pt off 2pt on 2pt off 2pt  on 2pt off 2pt on 2pt off 2pt  on 1pt off 1pt on 24pt](5,0) -- (3,1);
\draw [dash pattern= on 12pt off 2pt on 2pt off 2pt on 2pt off 2pt on 2pt off 2pt on 2pt off 2pt on 2pt off 2pt on 2pt off 2pt on 2pt off 2pt on 50pt ](3,1) -- (1,0);
\draw (1,0) -- (3,-1.5) -- (5,0);
\draw [-<-={at=0}](-4.4,-.05) arc[x radius=0.5, y radius=0.25, start angle=180, end angle=540];
\draw [->-={at=0}](-1.6,-.05) arc [x radius=0.5, y radius=0.25, start angle=0, end angle=360];

\draw (1.6,-.05) arc[x radius=0.5, y radius=0.25, start angle=180, end angle=360];
\draw  (4.4,-.05)arc [x radius=0.5, y radius=0.25, start angle=0, end angle=-180];
\draw [dashed](1.6,-.05) arc[x radius=0.5, y radius=0.25, start angle=180, end angle=0];
\draw  [dashed](4.4,-.05)arc [x radius=0.5, y radius=0.25, start angle=0, end angle=180];

\draw (1.6,-.05) [out=90, in=180] to (3,2) [out=0, in=90] to (4.4,-.05);
\draw (2.6,-.05) [out=90, in=180] to (3,1.2) [out=0, in=90]to (3.4,-.05);
\draw [->-](3,1.2) [out=180, in=180, looseness=0.7] to (3,2);
\draw [dashed] (3,1.2)[out=0, in=0, looseness=0.7] to (3,2);
\fill (-3.4,-.05) circle (1pt);
\fill (-2.6,-.05) circle (1pt);
\end{tikzpicture}

\caption{Construcción del \textit{handle}. \label{fig:handle}}
\end{subfigure}


\begin{subfigure}{1\textwidth} 
\centering
\begin{tikzpicture}[use optics]
%\draw [help lines, step=1mm](-6,-2) grid (6,2);

\fill [gray!10] (-3,-1.5) -- (-5,0) -- (-3,1) -- (-1,0) -- cycle;
\fill [white](-4.4,-.05) arc[x radius=0.5, y radius=0.25, start angle=180, end angle=540];
\fill [white](-1.6,-.05) arc [x radius=0.5, y radius=0.25, start angle=0, end angle=360];
\fill [gray!10](3,-1.5) -- (5,0) -- (3,1) -- (1,0) -- cycle;
\fill [gray!20](1.6,-.05) [out=90, in=180] to (3,2) [out=0, in=90] to (4.4,-.05) arc [x radius=0.5, y radius=0.25, start angle=0, end angle=-180]  [out=90, in=0] to (3,1.2) [out=180, in=90] to (2.6,-.05) arc [x radius=0.5, y radius=0.25, start angle=0, end angle=-180] ;

\draw (-3,-1.5) -- (-5,0) -- (-3,1) -- (-1,0) -- cycle;
\draw [dash pattern= on 19pt off 2 pt on 2pt off 2pt on 2pt off 2pt on 2pt off 2pt on 2pt off 2pt on 2pt off 2pt  on 2pt off 2pt on 2pt off 2pt  on 1pt off 1pt on 24pt](5,0) -- (3,1);
\draw [dash pattern= on 12pt off 2pt on 2pt off 2pt on 2pt off 2pt on 2pt off 2pt on 2pt off 2pt on 2pt off 2pt on 2pt off 2pt on 2pt off 2pt on 50pt ](3,1) -- (1,0);
\draw (1,0) -- (3,-1.5) -- (5,0);
\draw [-<-={at=0}](-4.4,-.05) arc[x radius=0.5, y radius=0.25, start angle=180, end angle=540];
\draw [-<-={at=0}](-1.6,-.05) arc [x radius=0.5, y radius=0.25, start angle=0, end angle=360];

\draw (1.6,-.05) arc[x radius=0.5, y radius=0.25, start angle=180, end angle=360];
\draw  (4.4,-.05)arc [x radius=0.5, y radius=0.25, start angle=0, end angle=-180];
\draw [dashed](1.6,-.05) arc[x radius=0.5, y radius=0.25, start angle=180, end angle=0];
\draw  [dashed](4.4,-.05)arc [x radius=0.5, y radius=0.25, start angle=0, end angle=180];

\draw (2.1,1.56) [out=-90, in=-135, looseness=0.5] to (3, 1.56) [out=-45, in=-90, looseness=0.5] to (3.9, 1.56);
\draw [dashed] (2.1,1.56) [out=90, in=135, looseness=0.5] to (3, 1.56) [out=45, in=90, looseness=0.5] to (3.9, 1.56);
\draw (1.6,-.05) [out=90, in=180] to (3,2) [out=0, in=90] to (4.4,-.05);
\draw (2.6,-.05) [out=90, in=180] to (3,1.2) [out=0, in=90]to (3.4,-.05);
%\draw (3,1.2) [out=180, in=180, looseness=0.2] to (3,2);
\draw (3,1.2)-- (3,2);
%\draw [dashed] (3,1.2)[out=0, in=0, looseness=0.7] to (3,2);
\fill (-3.4,-.05) circle (1pt);
\fill (-2.6,-.05) circle (1pt);

\end{tikzpicture}
\caption{Construcción del \textit{crosshandle}.\label{fig:crosshandle}}
\end{subfigure}
\caption{Construcciones elementales con cremalleras.}
\end{figure}
\clearpage
%
%
%
%
%
\begin{comment}
\begin{figure}[h]%%%%FIG: Disco esfera perforada
\centering
\begin{tikzpicture}[line cap=round,line join=round,>=triangle 45,x=1.0cm,y=1.0cm]
\fill[gray!20](3,0) circle (2cm);
\fill[color=gray!20][rotate around={-14.04:(-1.34,0.02)}] (-1.34,0.02) ellipse (0.32cm and 0.24cm);
\fill[color=white][rotate around={-14.04:(3.33,1.31)}] (3.33,1.31) ellipse (0.32cm and 0.24cm);
\draw[rotate around={-14.04:(-1.34,0.02)}] (-1.34,0.02) ellipse (0.32cm and 0.24cm);
\draw[rotate around={-14.04:(3.33,1.31)}] (3.33,1.31) ellipse (0.32cm and 0.24cm);
\draw(3,0) circle (2cm);
\draw(1,0) arc[x radius=2, y radius=0.5, start angle=180, end angle=360];
\draw[dashed](5,0) arc[x radius=2, y radius=0.5, start angle=0, end angle=180];
\draw(0,0) node[anchor=north] {$ \approx $};
%\draw [out=-14.04, in=0, looseness=2] (-1.28,0.26) to (-1.5,-2);
%\draw [out=180, in=165.96, looseness=4] (-1.5,-2) to (-1.28,0.26);
\end{tikzpicture}
\caption{El disco cerrado como una esfera perforada.\label{fig:disco_esfera_perforada}}
\end{figure}
\end{comment}


\begin{prop}\label{prop:suma_esfera}
Sean $M,M'$ dos superficies conexas (con o sin borde) y sean $P=\langle S\mid W\rangle$ y $P'=\langle S'\mid W'\rangle$ sus respectivas representaciones poligonales de una sola cara. Se tiene entonces:
\begin{itemize}
\item[(i)] La suma conexa de una esfera y $S$ es homeomorfa a $S$.
\item[(ii)] Sea $|\mathcal{P}|$ la realización geométrica de $P$. Sea $\partial B$ una perforación sobre $\Esfera$, y sea $\phi$ un homeomorfismo entre las aristas de $|\mathcal{P}|$ y $\partial B$. Si identificamos ahora los pares de segmentos sobre $\partial B$ de la imagen de $\phi$, obtenemos la misma superficie $S$. 
\item[(iii)] Con una construcción análoga a la del apartado anterior, si hacemos una perforación sobre una superficie $S'$ con el borde con etiquetas asociadas a una representación poligonal de $S$, al identifacarlas obtenemos la suma conexa $S\# S'$. 
\end{itemize}
\end{prop}
\begin{proof}
Sea la suma conexa de $\Esfera$ y $M$, cuya representación poligonal viene dada por $\langle a,S\mid aa^{-1}W\rangle$. Plegando por $a$ se obtiene (i).

Para demostrar (ii), sea $\langle a,c,S\mid aa^{-1}cWc^{-1}\rangle$ la representación poligonal de la esfera con una perforación asociada a $M$. Si plegamos por $a$, rotamos, y volvemos a plegar por $c$ obtenemos la palabra $\langle S\mid W\rangle$ que es la representación poligonal de $M$.

Finalmente para (iii), sea $\langle S,S',B,c\mid W'cWc^{-1}\rangle$, que es la representación de $M'$ con una perforación asociada a $M$. Desplegando por $a$ se tiene $$\langle S,S',a,B,c\mid W'aa^{-1}cWc^{-1}\rangle$$ que es la suma conexa de $M'$ y $\Esfera$ con una perforación asociada a $M$. Por (ii), esto es equivalente a la suma conexa de $M'$ y $M$. 
\end{proof}


\begin{comment}
\begin{obs}\label{obs:suma_esfera}
Sea $S$ una superficie conexa y $P=\langle A\mid W\rangle$ una representación poligonal suya de una sola cara (recordemos que esto siempre se puede dar), y sea $|\mathcal{P}|$ su realización geométrica. Sea $\partial B$ una perforación sobre $\Esfera$, y sea $\phi$ un homeomorfismo entre las aristas de $|\mathcal{P}|$ y $\partial B$. Si identificamos ahora los pares de segmentos sobre $\partial B$ de la imagen de $\phi$, obtenemos la misma superficie $S$. Esto se da puesto que una esfera perforada es homeomorfa al disco cerrado (\autoref{fig:disco_esfera_perforada}), y por tanto tenemos que hacer la construcción anterior es equivalente a partir de su representación poligonal y generar su realización geométrica. Pero esto también nos indica que $S\# \Esfera=S$. (Ver \autoref{fig:suma_toro_esfera}).
Por el mismo argumento vemos también que hacer una perforación sobre una superficie $S'$, con el borde con etiquetas asociadas a una representación poligonal de $S$ e identifacarlas da lugar a la suma conexa $S\# S'$. 
\end{obs}


\begin{obs}\label{obs:asa}%%%%EJE: Asas y toros
Dada una superficie $S$, la suma conexa $S\# \Toro$ puede visualizarse como el espacio que se obtiene al pegarle un \textit{handle} a $S$. De forma más precisa, sea $S_0$ que denota a $S$ con dos perforaciones. Entonces $S_0$ y $\mathbb{S}^1\times I$ son ambas superficies con borde, el cual está formado por dos componentes conexas, que como sabemos  son ambas homeomorfas a la union disjunta de dos copias de $\mathbb{S}^1$. Sea $\tilde{S}$ el espacio obtenido pegando $S_0$ y $\mathbb{S}^1\times I$ por sus bordes con un proceso análogo al de la \autoref{prop:borde_perforaciones}. Este espacio cociente es homeomorfo a $S\# \Toro$. La razón se puede ver en la \autoref{fig:toro_asa}, y es que podemos obtener un espacio homeomorfo a $S\# \Toro$ primero quitando un disco abierto de $S$, luego pegando un disco cerrado con dos discos abiertos quitados (es decir, la parte gris de la figura), y finalmente pegando al borde de la construcción el cilindro $\Esfera \times I$. Dado que la primera operación da resultado a un espacio homeomorfo a $S$ con dos discos abiertos quitados, el resultado es el mismo que si quitamos directamente a $S$ dos discos abiertos y entonces pegamos el cilindro a su frontera. 
%%165%%
\end{obs}
\end{comment}

\begin{figure}%%%% FIG: suma conexa toro esfera
\centering
\begin{tikzpicture}[line cap=round,line join=round,>=triangle 45,x=1cm,y=1cm, scale=0.75]
\fill [black!10] (0.5,0) arc[x radius=0.5, y radius=0.5, start angle=0, end angle=-180];
\fill[color=black!15] (0.5,0) arc[x radius=0.5, y radius=0.23, start angle=0, end angle=360];
\draw [rotate around={0:(-5.5,0)}] (-5.5,0) ellipse (2.5cm and 2cm);
\draw(0,0) circle (0.5cm);
\draw (0.5,0) arc[x radius=0.5, y radius=0.23, start angle=0, end angle=-180];
\draw [dashed] (0.5,0) arc[x radius=0.5, y radius=0.23, start angle=0, end angle=180];
\draw [rotate around={0:(5.7,0)}] (5.7,0) ellipse (2.5cm and 2cm);
\draw [dashed] (6.67,1.32) arc[x radius=0.45, y radius=0.22, start angle=0, end angle=180];
\draw (6.67,1.32) arc[x radius=0.45, y radius=0.22, start angle=0, end angle=-180];
\filldraw [color=black, fill=black!10] [rotate around={0:(-5,1.32)}] (-5.0,1.32) ellipse (0.45cm and 0.22cm);
\draw (5.77,1.32) arc[x radius=0.45, y radius=0.45, start angle=180, end angle=0];
\draw (-6.1,0) arc[x radius=0.4, y radius=0.2, start angle=180, end angle=0];
\draw (-6.4,0.31) arc[x radius=0.7, y radius=0.4, start angle=180, end angle=360];
\draw (-1.8,0.56) node[anchor=north west] {$\#$};
\draw (1.22,0.56) node[anchor=north west] {$=$};
\draw (5.1,0) arc[x radius=0.4, y radius=0.2, start angle=180, end angle=0];
\draw (4.8,0.31) arc[x radius=0.7, y radius=0.4, start angle=180, end angle=360];
\end{tikzpicture}
\caption{Suma conexa de un toro y una esfera.\label{fig:suma_toro_esfera}}
\end{figure}






\begin{prop}\label{prop:handletoro_crosshandleklein}
\begin{itemize}

\item[(a)] La esfera con un handle es homeomorfa al toro $\Toro$.
\item [(b)] La esfera con un crosshandle es homeomorfa a la botella de Klein $K$.

\end{itemize}
\end{prop}
\begin{proof}
Sea la representación de la esfera con un par-zip asociado al handle (ver \autoref{fig:handle_toro}) $$\langle a,c_1,c_2,z\mid aa^{-1} c_1^{-1}zc_1c_2^{-1}z^{-1}c_1\rangle.$$ Doblamos por $a$ y nos queda
\[
\langle c_1,c_2,z\mid c_1^{-1}zc_1c_2^{-1}z^{-1}c_2\rangle
\]
y podemos finalmente reetiquetar $e=c_2c_1^{-1}$ para obtener la representación de $\Toro$ 

\[
\langle e,z\mid eze^{-1}z^{-1}\rangle
\]

Similarmente, para demostrar (b), sea la representación de la esfera con un par-zip asociado al crosshandle (ver \autoref{fig:crosshandle_klein}) $$\langle a,c_1,c_2,z\mid aa^{-1} c_1^{-1}zc_1c_2^{-1}zc_1\rangle.$$ Doblamos por $a$ y nos queda
\[
\langle c_1,c_2,z\mid c_1^{-1}zc_1c_2^{-1}zc_2\rangle
\]
y reetiquetando $e=c_2c_1^{-1}$, obtenemos finalmente la representación de la botella de Klein

\[
\langle e,z\mid eze^{-1}z\rangle
\]


\end{proof}

\begin{figure}%%%%FIG: Toro-Asa
\centering
\begin{tikzpicture}
%1


\fill[gray!20] (-3.3, 0.2) -- (-3.3,0) arc[x radius=0.7, y radius=0.2, start angle=0, end angle=-180] (-4.7,0) -- (-4.7, 0.2)  [out=90, in=270] to (-5,1) arc[x radius=0.25, y radius=0.1, start angle=180, end angle=360] arc[x radius=0.5, y radius=0.3, start angle=180, end angle=360]  arc[x radius=0.25, y radius=0.1, start angle=180, end angle=360] [out=270, in=90] to (-3.3,0.2) ;


\draw (-3.3, 0.2) -- (-3.3,0) arc[x radius=0.7, y radius=0.2, start angle=0, end angle=-180] (-4.7,0) -- (-4.7, 0.2)  [out=90, in=270] to (-5,1) arc[x radius=0.25, y radius=0.1, start angle=180, end angle=360] arc[x radius=0.5, y radius=0.3, start angle=180, end angle=360]  arc[x radius=0.25, y radius=0.1, start angle=180, end angle=360] [out=270, in=90] to (-3.3,0.2) ;






\draw (-3,1) arc[x radius=1, y radius=0.8, start angle=0, end angle=180]; \draw [dashed] (-5,1)  arc [x radius=0.25, y radius=0.1, start angle=180, end angle=0];
\draw (-4.5,1) arc [x radius=0.5, y radius=0.3, start angle=180, end angle=0];
\draw (-3.5,1) [dashed] arc [x radius=0.25, y radius=0.1, start angle=180, end angle=0];

\draw (-5.8,-0.5) [out=0, in=270] to (-4.7,0);
\draw [dashed](-3.3,0) arc[x radius=0.7, y radius=0.2, start angle=0, end angle=180];
\draw (-3.3,0) [out=270, in=180] to (-2.2,-0.5);

%2



\draw (1,1) arc[x radius=1, y radius=0.8, start angle=0, end angle=180];
\draw [dashed] (-1,1)  arc [x radius=0.25, y radius=0.1, start angle=180, end angle=0];
\draw (-0.5,1) arc [x radius=0.5, y radius=0.3, start angle=180, end angle=0];
\draw (0.5,1) [dashed] arc [x radius=0.25, y radius=0.1, start angle=180, end angle=0];


\fill[color=gray!20] (-0.5,1) arc[x radius=0.5, y radius=0.3, start angle=180, end angle=360] arc[x radius=0.25, y radius=0.1, start angle=180, end angle=360] [out=270, in=90] to (1.2,0) arc[x radius=1.2, y radius=0.3, start angle=0, end angle=-180] [out=90, in=270] to (-1,1) arc[x radius=0.25, y radius=0.1, start angle=180, end angle=360];
\draw (-0.5,1) arc[x radius=0.5, y radius=0.3, start angle=180, end angle=360] arc[x radius=0.25, y radius=0.1, start angle=180, end angle=360] [out=270, in=90] to (1.2,0) arc[x radius=1.2, y radius=0.3, start angle=0, end angle=-180] [out=90, in=270] to (-1,1) arc[x radius=0.25, y radius=0.1, start angle=180, end angle=360];

\draw (-1.8,-0.5) [out=0, in=270] to (-1.2,0);

\draw [dashed] (1.2,0) arc[x radius=1.2, y radius=0.3, start angle=0, end angle=180];
\draw (1.2,0) [out=270, in=180] to (1.8,-0.5);

%3


\draw (5.2,0) [out=270, in=180] to (5.8, -0.5);
\draw (2.2,-0.5) [out=0, in=270] to (2.8, 0);
\fill[gray!20](2.8,0) arc[x radius=1.2, y radius=0.3, start angle=180, end angle=540];
\draw (2.8,0) arc[x radius=1.2, y radius=0.3, start angle=180, end angle=360];

\fill[white] (3,0) -- (3, 0.5) arc[x radius=1, y radius=.8, start angle=180, end angle=0] -- +(0,-0.5) arc[x radius=0.25, y radius=.1, start angle=0, end angle=-180] -- +(0,.5) arc[x radius=0.5, y radius=0.3, start angle=0, end angle=180] -- +(0,-.5) arc[x radius=0.25, y radius=0.1, start angle=0, end angle=-180];

\draw (3,0) -- (3, 0.5) arc[x radius=1, y radius=.8, start angle=180, end angle=0] -- +(0,-0.5) arc[x radius=0.25, y radius=.1, start angle=0, end angle=-180] -- +(0,.5) arc[x radius=0.5, y radius=0.35, start angle=0, end angle=180] -- +(0,-.5) arc[x radius=0.25, y radius=0.1, start angle=0, end angle=-180];

\draw[dashed] (3,0) arc[x radius=0.25, y radius=0.1, start angle=180, end angle=0] (5,0) arc[x radius=0.25, y radius=0.1, start angle=0, end angle=180];
\draw[dash pattern=on 7.5pt off 2.2pt on 2.2pt off 2.2pt on 2.2pt off 2.2pt on 2.2pt off 1.8 pt on 28.4pt off 2pt on 2pt off 2pt on 2pt off 2pt on 2pt off 2.4pt] (2.8,0) arc[x radius=1.2, y radius=0.3, start angle=180, end angle=0];

\draw (-2,1) node {$ \approx $};
\draw (2,1) node {$ \approx $};


\end{tikzpicture}
\caption{Suma conexa de una superficie y un toro visto como pegar un \textit{asa}. \label{fig:toro_asa}}
\end{figure}

\begin{corol}
Sea $S$ una superficie. Los siguientes espacios son homeomorfos:
\begin{itemize}
\item[a)] $S$ con un cap y $S$.
\item[b)] $S$ con un crosscap y $S\# \Proyectivo$.
\item[c)] $S$ con un handle y $S\# \Toro$.
\item[d)] $S$ con un crosshandle y $S\# K$ (siendo $K$ la botella de Klein). 
\end{itemize}
\end{corol}


\begin{comment}

\section{Representación de superficies con cremalleras}
Veamos un método para representar las perforaciones y cremalleras sobre colecciones de esferas, y definamos unas operaciones sobre estas representaciones que nos den espacios topológicos equivalentes, tal como hicimos en el Capítulo 1. La idea es de Wildberger \cite{wildberger}.

Sean $\Esfera_1,\dots ,\Esfera_k$ esferas con perforaciones, algunas con pares-zip. 
\begin{itemize}

\item Para cada esfera, etiquetamos las cremalleras asociadas a un mismo par-zip con el mismo símbolo. Para las partes de perforaciones que no tengan cremallera, las etiquetamos cada una con un símbolo distinto. 
\item A cada perforación le asociaremos una palabra, formada a partir de las etiquetas que acabamos de crear. Empezamos arbitrariamente por una, digamos $a$, y si esta está asociada a una cremallera, escribiremos $a$ o $a^{-1}$ dependiendo de si su orientación se corresponde con la de las agujas del reloj; si no está asociada a una cremallera escribiremos simplemente $a$. Recorreremos la perforación en orden en sentido contrario de las agujas del reloj reptiendo el proceso anterior hasta llegar a la anterior etiqueta de la que hemos partido.
\item Cada esfera la representaremos por la yuxtaposición de las palabras entre paréntesis asociadas a las perforaciones, y una usperficie formada por varias esferas la escribiremos como la suma de sus representaciones.
\end{itemize}
\begin{eje}\label{eje:rep_perf}
Sea una esfera perforada como la de la izquierda de la \autoref{fig:rep_perf}. Hay dos pares-zip, uno de ellos con las cremalleras en la misma perforación y el otro en perforaciones distintas. Etiquetamos las primeras por $a$ y las segundas por $b$ y etiquetamos el resto de segmentos de las perforaciones sin cremallera por $x,y,z$. Por tanto, una representación de nuestra esfera sería $$(a^{-1}xab)(by)(z)$$
\end{eje}

\begin{figure}%%%%FIG: Representación esferas
\centering
\begin{tikzpicture}[use optics, line cap=round,line join=round,>=triangle 45,x=1.0cm,y=1.0cm, scale=0.8]

\fill[gray!10](-3,0) circle (3cm);
\fill[gray!10] (5,0) circle (3cm);

\fill[white][rotate around={-41.01:(-1.54,1.6)}](-1.54,1.6) ellipse (0.8cm and 0.51cm);
\fill[white](-4.14,-1.36) circle (0.4cm);
\fill[white][rotate around={33.69:(-1.79,-1.86)}] (-1.79,-1.86) ellipse (0.4cm and 0.31cm);
\fill[white][rotate around={-41.01:(6.46,1.6)}] (6.46,1.6) ellipse (0.8cm and 0.51cm);
\fill[white](3.86,-1.36) circle (0.4cm);
\fill[white][rotate around={33.69:(6.21,-1.86)}] (6.21,-1.86) ellipse (0.4cm and 0.31cm);

\draw (0,0) arc[x radius=3, y radius=.9, start angle=0, end angle=-180] (8,0) arc[x radius=3, y radius=.9, start angle=0, end angle=-180] ;
\draw [dashed] (0,0) arc[x radius=3, y radius=.9, start angle=0, end angle=180] (8,0) arc[x radius=3, y radius=0.9, start angle=0, end angle=180] ;

\draw(-3,0) circle (3cm);
\draw [rotate around={-41.01:(-1.54,1.6)}, -<<-={at=0.39}, ->-={at=0.2}, ->>-={at=0.9}] (-1.54,1.6) ellipse (0.8cm and 0.51cm);
\draw  [->-={at=0.45}](-4.14,-1.36) circle (0.4cm);
\draw [rotate around={33.69:(-1.79,-1.86)}] (-1.79,-1.86) ellipse (0.4cm and 0.31cm);

\draw (5,0) circle (3cm);
\draw [rotate around={-41.01:(6.46,1.6)}, -<-={at=0.39}, ->-={at=0.2}, ->-={at=0.9}, ->-={at=0.6}] (6.46,1.6) ellipse (0.8cm and 0.51cm);
\draw[->-={at=0.45}, ->-={at=0.91}](3.86,-1.36) circle (0.4cm);
\draw [rotate around={33.69:(6.21,-1.86)}, ->-={at=0.8}] (6.21,-1.86) ellipse (0.4cm and 0.31cm);

\fill [color=black] (-2.1,2.17) circle (1.5pt);
\fill [color=black] (-0.86,1.45) circle (1.5pt);
\fill [color=black] (-1.31,2.07) circle (1.5pt);
\fill [color=black] (-1.98,1.31) circle (1.5pt);
\fill [color=black] (-4.35,-1.7) circle (1.5pt);
\fill [color=black] (-3.99,-0.98) circle (1.5pt);
\fill [color=black] (5.9,2.17) circle (1.5pt);
\fill [color=black] (7.14,1.45) circle (1.5pt);
\fill [color=black] (6.69,2.07) circle (1.5pt);
\fill [color=black] (6.02,1.31) circle (1.5pt);
\fill [color=black] (3.65,-1.7) circle (1.5pt);
\fill [color=black] (4.01,-0.98) circle (1.5pt);
\fill [color=black] (5.97,-1.67) circle (1.5pt);

\draw[anchor=east] (3.47,-1.29) node {$ b $};
\draw[anchor=south west] (6.97,1.71) node {$ b $};
\draw[anchor=north] (6.68,0.89) node {$ a $};
\draw[anchor=south] (6.29,2.24) node {$ a $};
\draw[anchor=east] (5.79,1.72) node {$ x $};
\draw[anchor=north west] (4.23,-1.51) node {$ y $};
\draw[anchor=north west] (6.46,-2.05) node {$ z $};


\end{tikzpicture}
\caption{Representación de una esfera con perforaciones y cremalleras.	 \label{fig:rep_perf}}
\end{figure}
\end{comment}

\section{Teorema de Clasificación}

\begin{defin}%%%%DEF: Ordinaria
Una superficie con borde se dice ordinaria si es homeomorfa a una colección finita de esferas cada una con un número finito de \textit{handles}, \textit{crosshandles}, \textit{crosscaps} y perforaciones.
\end{defin}


\begin{lema}%%%%LEMA: Superficie ordinaria zips
Sea $S$ una superficie con borde con un par-zip tal que cada cremallera está en una componente conexa de su borde. Entonces, si $S$ es ordinaria antes de identificar las cremalleras, es ordinaria también después.\label{lema:superficie_ordinaria}
\end{lema}
\begin{proof}
Consideramos el caso en que las dos cremalleras ocupan cada una una perforación en su totailidad. Entonces al identificarlas se tiene un \textit{handle} (\autoref{fig:handle}) o un \textit{crosshandle} (\autoref{fig:crosshandle}), dependiendo de sus respectivas orientaciones. Si las dos perforaciones pertenecen a componentes conexas de $S$ distintas, entonces identificando obtenemos la suma conexa de las dos componentes. 

Consideramos ahora el caso en el que las dos cremalleras yacen sobre la misma perforación y la cubren totalmente. Identificándolas nos da o bien un \textit{cap} (\autoref{fig:cap}) o bien un \textit{crosscap} (\autoref{fig:crosscap}), dependiendo de sus respectivas orientaciones.

Finalmente, consideramos los varios casos en que las cremalleras no ocupan perforaciones en su totalidad. Para empezar, supongamos que la superficie ordinaria de la que partimos es una esfera. Tenemos varios casos:
\begin{itemize}
\item[(1)] \textit{Una perforación con un par-zip y dos componentes conexas distintas del borde.} Tenemos dos posibilidades dependiendo de la orientación de las cremalleras. En la primera 
\begin{align*}
\langle a,z,B_1,B_2,c\mid aa^{-1} c^{-1}zB_2z^{-1}B_1c\rangle & & \\
\approx & \, \langle z,B_1,B_2,c\mid c^{-1}zB_2z^{-1}B_1c\rangle &\text{(doblar por } a\text{)}\\
\approx & \, \langle z,B_1,B_2\mid zB_2z^{-1}B_1\rangle &\text{(doblar por \textit{c})}
\end{align*}
que es una representación del cilindro, es decir, una esfera con dos perforaciones, y por tanto una superficie ordinaria. Cambiando la aparición de $z^{-1}$ por la de $z$, obtenemos mediante un proceso análogo al anterior la representación
\[ \langle z,B_1,B_2\mid zB_2zB_1 \rangle
\]
que es la representación de un crosscap con dos perforaciones, es decir, una superficie ordinaria.
\item[(2)] \textit{Una perforación con un par-zip y una componente conexa del borde.} Como antes, dependiendo de la orientación podemos partir de 
\begin{align*}
\langle a,z,B,c\mid aa^{-1} c^{-1}zBz^{-1}c\rangle & & \\
\approx & \, \langle z,B,c\mid c^{-1}zBz^{-1}c\rangle &\text{(doblar por } a\text{)}\\
\approx & \, \langle z,B\mid zBz^{-1}\rangle &\text{(doblar por \textit{c})}
\end{align*}
que es una representación de una esfera perforada. Cambiando la aparición de $z^{-1}$ por $z$, llegamos a la representación
\[
\langle z,B,\mid zBz\rangle
\]
que es una representación de $\Proyectivo$ perforado, es decir, un crosscap con una perforación, y por lo tanto una superficie ordinaria.
\item[(3)] \textit{Dos perforaciones con un par-zip y dos componentes conexas distisntas del borde.} Por un lado
\begin{align*}
\langle a,z,B_1,B_2,c_1,c_2\mid aa^{-1} c_2^{-1}zB_2c_2c_1z^{-1}B_1c_1\rangle & & \\
\approx  \, \langle z,B_1,B_2,c_1,c_2\mid c_2^{-1}B_2zc_2c_1^{-1}z^{-1}B_1c_1\rangle &&\text{(doblar por } a\text{)}\\
\approx  \, \langle e,z,B_1,B_2\mid eB_2zze^{-1}z^{-1}B_1\rangle &&\text{(rotar y consolidar } e=c_1c_2^{-1} \text{)}
\end{align*}
que es la representación de un toro con dos perforaciones. De la misma forma, cambiando la orientación de la aparición de $z^{-1}$ llegamos a 
\[
\langle e,z,B_1,B_2\mid eB_2ze^{-1}zB_1\rangle
\]
que es la botella de Klein con dos perforaciones, lo que es lo mismo, una esfera con crosshandle y dos perforaciones, es decir, una superficie ordinaria.

\item[(4)] \textit{Dos perforaciones con un par-zip y una componente conexa distinta del borde.} Empezando con la representación
\begin{align*}
\langle a,z,B,c_1,c_2\mid aa^{-1} c_2^{-1}Bzc_2c_1^{-1}z^{-1}c_1\rangle & & \\
\approx  \, \langle z,B,c_1,c_2\mid c_2^{-1}Bzc_2c_1^{-1}z^{-1}c_1\rangle &&\text{(doblar por } a\text{)}\\
\approx  \, \langle e,z,B\mid eBze^{-1}z^{-1}\rangle &&\text{(rotar y consolidar } e=c_1c_2^{-1} \text{)}
\end{align*}
que es una representación de $\Toro$ con una perforación, o lo que es lo mismo, una esfera con un handle y una perforación, es decir, una superficie ordinaria. De la misma forma, cambiando $z^{-1}$ por $z$ obtenemos
\[
\langle e,z,B\mid eBze^{-1}z\rangle
\]
que es la representación de la botella de Klein con una perforación, es decir, una esfera con un crosshandle y una perforación, y por tanto una superficie ordinaria.
\end{itemize}
Por tanto, aplicando la demostración de la \autoref{prop:suma_esfera} (iii), demostramos los cuatro puntos anteriores para cremalleras sobre superficies arbitrarias, por lo que queda entonces demostrada la proposición.

\end{proof}


\begin{tma}[Teorema de clasificación, versión preeliminar] %M%%%TEO: Clasificación preeliminar
Toda superficie compacta es ordinaria.
\end{tma}
\begin{proof}
Sea $S$ una superficie compacta. Sabemos, por el \nameref{teo:rado}, que $S$ está triangulada por un poliedro $|K|$ asociado a un complejo simplicial $K$ tal que cada 1-símplice que contiene puntos interiores de $S$ es una cara de exáctamente dos 2-símplices, y cada 1-símplice que contiene puntos del borde de $S$ es cara de exáctamente un 2-símplice. Si sobre los primeros 1-símplices ponemos una cremallera distinta, en los 2-símplices habrá algunos 1-símplices que se identifiquen. Llamemos $K_2=\left\{\sigma_1,\dots ,\sigma_j\right\}$, donde cada $\sigma_i \text{ es un 2-símplice para todo } i=1\dots ,j$. $K_2$ es una superficie ordinaria, pues cada $\sigma_i$ es homeomorfo a una esfera perforada. Si identificamos ahora las cremalleras una a una, por el \autoref{lema:superficie_ordinaria} y por inducción, la superficie resultante es ordinaria.
\end{proof}


\begin{lema}\label{lema:1}
Un crosshandle es homeomorfo a dos crosscaps.
\end{lema}
\begin{proof}
Por la \autoref{prop:handletoro_crosshandleklein}, la esfera con un crosshandle es homeomorfa a una botella de klein. Por la \autoref{lema:klein_proyectivoproyectivo}, la botella de Klein es homeomorfa a la suma conexa $\Proyectivo \# \Proyectivo$, o lo que es lo mismo, dos crosscaps.
\end{proof}

\begin{lema}\label{lema:2}
Handles y crosshandles son equivalentes en la presencia de crosscaps.
\end{lema}
\begin{proof}
Sea una esfera con un par-zip asociado a un handle y otro asociado a un crosscap. Sabemos que, identificando las cremalleras obtenemos un espacio homeomorfo a la suma conexa $\Toro \# \Proyectivo$. Pero por la \autoref{lema:toro_proyectivo}, este espacio es homeomorfo a $\Proyectivo \# \Proyectivo \# \Proyectivo$, que a su vez, es homeomorfo al plano proyectivo con un crosshandle, que a su vez es homeomorfo a una esfera con un crosshandle y un crosscap.
\end{proof}
%((Ahora no entiendo por qué dado todo lo anterior Conway sólo da la representación de superficies sin borde))
\begin{tma}[Clasificación de superficies]
Toda superficie compacta es homeomorfa o bien a una esfera con handles o bien a una esfera con crosscaps.
\end{tma}
\begin{proof}
Por la versión preeliminar del teorema de clasificación, una superficie compacta es homeomorfa a una esfera con handles, crosshandles y crosscaps.
\begin{itemize}
\item \textit{Caso 1:} Al menos hay un crosshandle en nuestra superficie. Por el \autoref{lema:1}, cada crosshandle es homeomorfo a dos crosscaps, por lo que la superficie es homeomorfa a una esfera con solamente crosscaps y handles. Como al menos hay un crosscap, cada handle es homeomorfo a un crosshandle (\autoref{lema:2}), que es a us vez homeomorfo a dos crosscaps (\autoref{lema:1} de nuevo), quedando una esfera sólo con crosscaps.
\item \textit{Caso 2:} No hay ni crosshandles ni crosscap en la superficie. La superficie es entonces homeomorfa a una esfera sólo con handles.
\end{itemize}
\end{proof}
%\begin{prop}
%Si $S$ es una superficie compacta y conexa, $S$ admite una representación de una sola cara.
%\end{prop}


%A la perforación $\partial B_i$ la llamamos \enfatiza{perforación} asociada a $W_i$, y a la imagen por $\phi _i$ de cada par de segmentos $a$ la llamamos \enfatiza{cremallera} asociada a $a\in W_i$.


%\begin{defin}%M%%%DEF: zips asociadas a una representación poligonal de una superficie
%Sea $S$ una superficie. Sea $|\mathcal{P}|$ la realización geométrica de una representación poligonal $P=\langle A\mid W_1,\dots W_n\rangle$ de la superficie. Para cada $W_i$, con $i=1,\dots n$, sea $\partial B_i$ una perforación sobre una esfera $\Esfera_i$, y sea $\phi _i$ un homeomorfismo entre los lados de $|\mathcal{P}|$ y $\partial B_i$. A la perforación $\partial B_i$ la llamamos \enfatiza{perforación} asociada a $W_i$, y a la imagen de $\phi _i$ la llamamos \enfatiza{cremallera} asociada a $W_i$.
%\end{defin}



\chapter{Teorema de Clasificación, segunda parte}
Vamos a empezar recordando brevemente las presentaciones de grupos. 

\begin{defin}
Sea $S$ un conjunto, y $R\subset F(S)$, donde $F(S)$ es el grupo libre en $S$. Llamamos \enfatiza{presentación de un grupo} al par $\langle S \mid R\rangle$. A los elementos de $S$ los llamamos \enfatiza{generadores}, y a los de $R$ \enfatiza{relaciones}. La presentación de un grupo determina un grupo, que también denotamos por $\langle S \mid R \rangle$, y que viene dado por el siguiente grupo cociente: 
\[
\langle S\mid R\rangle = F(S)/\overline{R}
\]
donde $\overline{R}$ representa la intersección de todos los subgrupos normales de $F(S)$ que contienen a $R$.
\end{defin}
Cada relación $r\in R$ determina un producto de potencias de generadores que vale 1 en el cociente.
\begin{defin}
Sea ahora un grupo arbitrario $G$. Decimos que una presentación de un grupo $\langle S \mid R \rangle$ es una \enfatiza{presentación de G} si existe un isomorfismo $\langle S \mid R\rangle \cong G$. 
\end{defin}
Los elementos de $G$ claramente generan $G$, y por la propiedad característica de los grupos libres, la aplicación identidad de $G$ en sí mismo se extiende a un homomorfismo único $\Phi: F(G) \to G$. Si ponemos $R= \Ker{\Phi}$, entonces, por el primer teorema de isomorfía tenemos que $G\cong F(G)/R$. Como $R$ es normal, $R=\overline{R}$, y por tanto $G$ tiene como presentación $\langle G\mid R\rangle$. Esta presentación trivial es muy ineficiente, pues normalmente $F(G)$ y $R$ son más grandes que $G$. 
Si $G$ admite una presentación $\langle S\mid R	\rangle$ en las cuales $S$ y $R$ son conjuntos finitos, decimos que $G$ tiene una \enfatiza{presentación finita}.




\subsection{Grupos fundamentales de las Superficies Compactas}

\begin{tma}
Sea $M$ una superficie con una presentación suya de una cara $\langle a_1, \dots , a_n \mid W\rangle$. Entonces una presentación del grupo fundamental $\pi_1 (M)$ viene dada por $\langle a_1, \dots , a_n \mid W\rangle$.
\end{tma}

\begin{corol}
Los grupos fundamentales de las superficies compactas estándar tienen como presentación:
\begin{itemize}
\item[(a)] $\pi_1 (\Esfera ) \cong \langle \emptyset \mid \emptyset \rangle $ (el grupo trivial).
\item[(b)] $\pi_1 ( \Toro \# \dots \Toro) \cong \langle \beta_1 , \gamma_1 ,\dots ,\beta_n , \gamma_n \mid \beta_1 \gamma_1 \beta^{-1}_1 \gamma^{-1}_1\dots \beta_n \gamma_n \beta^{-1}_n \gamma^{-1}_n \rangle$.
\item[(c)] $\pi_1(\Proyectivo \# \dots \# \Proyectivo \rangle \cong \langle \beta_1 , \dots , \beta_n \mid \beta_1^{2} \dots \beta_n^{2} \rangle $. 
\end{itemize}
\end{corol}


\begin{defin}
Dado un grupo $G$, el \enfatiza{subgrupo conmutador de G}, denotado por $\left[ G, G \right]$, es el subgrupo de $G$ generado por todos los elementos de la forma $\alpha \beta \alpha^{-1} \beta^{-1}$, para $\alpha \beta \in G$. 

Al grupo cociente $\Ab{(G)} = G / \left[ G,G \right]$ lo llamamos el \enfatiza{abelianizado de G}. 
\end{defin}

Como el isomorfismo $F:G_1 \to G_2$ lleva el subgrupo conmutador de $G_1$ al de $G_2$, entonces grupos isomorfos tienen abelianizados isomorfos. 

\begin{prop}
Los grupos fundamentales de las superficies estándar tienen los siguientes abelianizados:
\begin{itemize}
\item $\Ab{(\pi_1 (\Esfera ))} = \{ 1 \}$.
\item $\Ab{(\pi_1 (\Toro \# \dots \# \Toro ))} \cong \mathbb{Z}^{2n}$.
\item $\Ab{(\pi_1 (\Proyectivo \# \dots \# \Proyectivo ))} \cong \mathbb{Z}^{n-1} \times \mathbb{Z}/2$.
\end{itemize}
\end{prop}


\begin{tma}[Teorema de Clasificación, segunda parte]
Toda superficie compacta es homeomorfa a exáctamente una de las superficies estándar.
\end{tma}

\begin{tma}
Para $n\geq 2$, $\mathbb{S}^n$ es simplemente conexa.
\end{tma}

\begin{lema}
Si un grupo abeliano $G$ tiene una base finita, entonces toda base finita tiene el mismo número de elementos.
\end{lema}


\begin{defin}
Si $G$ es un grupo abeliano libre con una base finita, decimos que $G$ tiene \enfatiza{rango finito}, y decimos que $G$ es de \enfatiza{rango n} si $n$ es el número de elementos de cualquier base.
\end{defin}

\begin{defin}
Sea $G$ un grupo abeliano. Decimos que un elemento suyo $g\in G$ es un elemento de torsión si $ng=0$ para algún $n\in \mathbb{Z}\setminus \{ 0 \}$. Dado que si $ng=n'g'=0$, entonces $nn'(g+g')=0$, y por lo tanto el conjunto de todos los elementos de torsión de $G$ es un subgrupo suyo, denotado por $G_{tor}$. Decimos que $G$ es un grupo sin torsión si el único elemento de torsión es el 0. 
\end{defin}


\appendix
\chapter{Algunos teoremas de topología importantes}


\section{Cocientes topológicos}
Empecemos recordando las nociones básicas de cocientes topológicos.
\begin{defin}
Sea $X$ un espacio topológico e $Y$ un conjunto. Sea $f:X\to Y$ una aplicación sobreyectiva. Definimos la \enfatiza{topología cociente} sobre $Y$ inducida por $f$ de la siguiente manera: $V\subseteq Y$ es abierto si y solo si $f^{-1}(V)$ es abierto e n$X$.
\end{defin}
\begin{defin}
Si $X$ e $Y$ son espacios topológicos, decimos que  $q:X\to Y$ es una \enfatiza{aplicación cociente} o \enfatiza{identificación} si es sobreyectiva e $Y$ tiene la topología cociente inducida por $q$.
\end{defin}
Se deduce de la definición que una aplicación cociente es siempre continua.\\
Sea ahora $X$ un espacio topológico y $\sim$ una relación de equivalencia en $X$. Si $X/\sim$ es el conjunto de las clases de equivalencia, sea $q:X\to X/\sim$ la aplicación que lleva cada $x\in X$ a su clase de equivalencia. Entonces decimos que $X/\sim$ junto con la topología cociente inducida por $q$ es un \enfatiza{espacio cociente} de $X$ por la relación de equivalencia $\sim$.\\
Estudiemos ahora condiciones bajo las cuales una aplicación entre dos espacios topológicos es una aplicación cociente.\\
Sean $X$ e $Y$ espacios topológicos y sea $q:X\to Y$ una aplicación. Si $y\in Y$ decimos que un subconjunto del tipo $q^{-1}(y)\subseteq X$ es una \enfatiza{fibra} de $q$. Un conjunto $U\subseteq X$ tal que existe un $V\subseteq Y$ con $U=q^{-1}(V)$ decimos que es \enfatiza{saturado con respecto de q}.
\begin{prop}
Una aplicación $q:X\to Y$ continua y sobreyectiva es una aplicación cociente si y solo si lleva abiertos saturados a abiertos saturados.
\end{prop}

\begin{prop}[Propiedades de las aplicaciones cociente]
Se tienen las siguientes propiedades:
\begin{itemize}
\item La composición de aplicaciones cociente es una aplicación cociente.
\item Una aplicación cociente sobreyectiva es un homeomorfismo.
\item Si $q:X\to Y$ es una aplicación cociente y $U\subseteq X$ es un abierto (o cerrado) saturado, entonces la restricción $q|_U:U\to q(U)$ es una aplicación cociente.
\end{itemize}
\end{prop}
\begin{prop}
Si $q:X\to Y$ es una aplicación continua, sobreyectiva y abierta o cerrada, entonces es una aplicación cociente.
\end{prop}

\begin{tma}[Lema de la aplicación cerrada]\label{teo:aplicac_cerrada}
Sea $F$ una aplicación continua de un espacio topológico compacto en un espacio topológico Hausdorff. Entonces:
\begin{itemize}
\item[(a)]$F$ es una aplicación cerrada.
\item[(b)] Si $F$ es sobreyectiva, entonces es una aplicación cociente.
\item[(c)] Si $F$ es inyectiva, entonces es una inmersión topológica. %?????
\item[(d)] Si $F$ es biyectiva, entonces es un homeomorfismo.
\end{itemize}
\end{tma}

\begin{tma}[Unicidad de espacios cociente]
\label{teo:unicidad_espacio_cociente}
Supongamos $q_1:X\to Y_1$ y $q_2:X\to Y_2$ son aplicaciones cociente que hacen las mismas identificaciones, es decir, tales que $q_1(x)=q_1(x')$ si y solo si $q_2(x)=q_2(x')$. Entonces existe un único homeomorfismo $\phi:Y_1\to Y_2$ tal que $\phi \circ q_1=q_2$.
\end{tma}

\begin{prop}[Pasando al cociente]
Supongamos que $q:X\to Y$ es una aplicación cociente, $Z$ es un espacio topológico y $f:X\to Z$ es una aplicación continua tal que si $q(x)=q(x')$, entonces $f(x)=f(x')$. Entonces existe una única aplicación continua $\tilde{f}:Y\to Z$ tal que $f=\tilde{f} \circ q$.
\end{prop}

\begin{tma}[Lema de pegado]
Sean $A,B$ dos subconjuntos abiertos (o cerrados) de un espacio topológico $X$, tal que $X=A\cap B$, y sea $Y$ otro espacio topológico. Si la restricción de $f:X\to Y$ a $A$ y $B$ es continua, entonces $f$ es continua.
\end{tma}
\chapter{CW-complejos}\label{sec:CW}
Utilizo la definición inductiva de CW-complejo dada por Hatcher \cite{hatcher} y algunas propiedades expuestas por Lee \cite{lee1}.


\begin{defin}
Una \enfatiza{n-celda abierta} es un espacio topológico homeomorfo a la bola abierta unidad $\mathbb{B}^n$, y una \enfatiza{n-celda} cerrada es un espacio homemorfo a $\overline{\mathbb{B}}^n$. 
\end{defin}

Toda bola abierta o cerrada en $\R^n$ es claramente una $n$-celda. El siguiente teorema nos proporciona más ejemplos:

\begin{prop}% TEOREMA 5.1 LEE%%%% CONVEXO HOMEOM ESFERA
\label{teo:convexo_homeom_esfera} 
Si $D\subseteq \R^n$ es un conjunto compacto y convexo con interior no vacío, entonces $D$ es una $n$-celda cerrada y su interior es una $n$-celda abierta. De hecho, dado $p\in \mathring{D}$, entonces existe un homeomorfismo $F:\overline{\mathbb{B}}^n\to D$ que envía $0$ a $p$, $\overline{\mathbb{B}}^n$ a $\mathring{D}$, y $\mathbb{S}^{n-1}$ a $\partial D$.
\end{prop}
\begin{proof}
Sea $p\in D$ un punto de su interior. Si reemplazamos $D$ por su imagen mediante la traslación $x\mapsto x-p$, que es un homeomorfismo de $\R^n$ en sí mismo, podemos asumir que $p=0\in \mathring{D}$. Entonces existe un $\varepsilon >0 $ tal que la bola $B_{\varepsilon}(0)$ está contenida en $D$. Usando la dilatación $x\mapsto x/\varepsilon$, podemos asumir que $\mathbb{B}^n = B_ 1(0) \subseteq D$.
La clave de la demostración es la siguiente: \emph{cada semirecta cerrada empezando en el origen interseca $\partial
D$ en exactamente un punto}. Sea $R$ una semirecta así. Dado que $D$ es compacto, su intersección con $R$ es compacta. Por tanto existe un punto $x_0$ en su intersección tal que en él su distancia al origen asume el máximo. Es claro %lo es?
que pertenece a la frontera de $D$. Para ver que el punto es único, veamos que el segmento que une $0$ y $x_0$ está formado enteramente por puntos interiores de $D$ excepto por el $x_0$ mismo. Cualquier punto en este segmento distinto de $x_o$ se puede escribir de la forma $\lambda x_0 $ para $0\leq \lambda <1. $ Supongamos $z\in B_{1-\lambda}(\lambda x_0)$, y sea $y=(z-\lambda x_0)/(1-\lambda )$. Como $|z-\lambda x_0|<|1-\lambda|$ se tiene que $|y|<1$, y por tanto $y\in B_ 1(0)\subseteq D$ (ver \autoref{fig:convexo_esfera}). Como $y$ y $x_0$ están en $D$ y $z=\lambda x_0 + (1-\lambda )y$, se sigue de la convexidad que $\in D$. Por tanto la bola abierta $B_{1-\lambda}(\lambda x_0)$ está contenida en $D$, lo que implica que $\lambda x_0$ es un punto interior.

Definimos ahora la aplicación $f:\partial D \to \mathbb{S}^{n-1}$ por 
$$f(x)=\frac{x}{|x|}$$

$f(x)$ es el punto donde el segmento desde el origen hasta $x$ interseca la esfera unidad. Como $f$ es la restricción de una función continua, es continua, y por el parágrafo anterior es biyectiva. Dado que $\partial D$ es compacta, $f$ es un homeomorfismo por el teorema de la aplicación cerrada (\autoref{teo:aplicac_cerrada}).

Finalmente definimos $F:\overline{\mathbb{S}}^n \to D$ por 
$$F(x)= \left\{\begin{array}{lc}
				|x|f^{-1}\left(\dfrac{x}{|x|}\right), & x\neq 0;
				\\0, & x=0.

\end{array}
\right. $$
$F$ es continua fuera del origen por serlo $f^{-1}$, y en el origen porque por ser $f^{-1}$ acotada $F(x) \to 0$ cuando $x\to 0$. Geometricamente, $F$ manda cada segmento radial que conecta 0 con un punto de $\mathbb{S}^{n-1}$ al segmento radial desde $0$ hasta el punto $f^{-1}(w)\in \partial D$. Por convexidad, $F$ toma valores en $D$. La aplicación $F$ es inyectiva, pues puntos de distintas semirectas van a parar a distintas semirectas, y cada segmento radial va linealmente a su imagen. Es sobreyectiva pues cada punto $y \in D$ está en una semirecta empezando en 0. Por el teorema de la aplicación cerrada, $F$ es un homeomorfismo.
\end{proof}


\begin{figure}%%%%FIG: CONVEXO ESFERA%

\begin{center}

\begin{tikzpicture}[line cap=round,line join=round,>=triangle 45,x=0.65cm,y=0.65cm, scale=0.8]
\draw [shift={(7.74,-8.14)}] plot[domain=1.68:2.51,variable=\t]({1*14.71*cos(\t r)+0*14.71*sin(\t r)},{0*14.71*cos(\t r)+1*14.71*sin(\t r)});
\draw [shift={(-0.34,-1.89)}] plot[domain=2.56:4.7,variable=\t]({1*4.5*cos(\t r)+0*4.5*sin(\t r)},{0*4.5*cos(\t r)+1*4.5*sin(\t r)});
\draw [shift={(6.31,-14.62)}] plot[domain=1.46:2.25,variable=\t]({1*10.61*cos(\t r)+0*10.61*sin(\t r)},{0*10.61*cos(\t r)+1*10.61*sin(\t r)});
\draw [shift={(5.58,1.03)}] plot[domain=-1.21:1.47,variable=\t]({1*5.47*cos(\t r)+0*5.47*sin(\t r)},{0*5.47*cos(\t r)+1*5.47*sin(\t r)});
\draw (0,0)-- (10.97,0.11);
\draw(0,0) circle (1.65cm);
\draw (0,0)-- (3.74,0.04);
\draw(3.74,0.04) circle (0.66cm);
\draw (-0.1,-0.01) node[anchor=north ] {$0 $};
\draw (10.97,0.11)-- (0.08,0.79);
\draw (-0.37,1.65) node[anchor=north west] {$y $};
\draw (3.6,1.1) node[anchor=north west] {$z $};
\draw (3.3,-0.15) node[anchor=north west] {$ \lambda x_0 $};
\draw (11.17,0.53) node[anchor=north west] {$ x_0 $};
\draw (-3.78,2.92) node[anchor=north west] {$D$};
\draw (-1.10,3.5) node[anchor=north west] {$B_1(0)$};
\draw (2.85,2.2) node[anchor=north west] {$B_{1-\lambda}(\lambda x_0)$};
\begin{scriptsize}
\fill [color=black] (10.97,0.11) circle (1.5pt);
\fill [color=black] (0,0) circle (1.5pt);
\fill [color=black] (3.74,0.04) circle (1.5pt);
\fill [color=black] (0.08,0.79) circle (1.5pt);
\fill [color=black] (3.8,0.56) circle (1.5pt);
\end{scriptsize}
\end{tikzpicture}

\end{center}

\caption{Demostración de que sólo hay un punto de la frontera en la semirecta.\label{fig:convexo_esfera}}

\end{figure}


Esto nos muestra que un intervalo cerrado es una 1-celda cerrada, toda región poligonal es una 2-celda cerrada, y un tetraedro sólido es una 3-celda cerrada. Por convención, los conjuntos unitarios son 0-celdas abiertas y cerradas a la vez.

\begin{obs}
Sea $D$ una $n$-celda. Entonces $D$ es una variedad con borde por serlo $\overline{\mathbb{B}}^n$. Denotamos por $\partial D$ e $\interior{D}$ respectivamente a las imágenes de $\mathbb{S}^{n-1}$ y $\mathbb{B}^n$ por un homeomorfismo $F:\overline{\mathbb{B}}^n\to D$, tal que $\partial D$ es homeomorfo a $\mathbb{S}^{n-1}$ e $\interior{D}$ es una $n$-celda abierta.
\end{obs}

\begin{defin}\label{def:cw}%%%%DEF: CW
Un \enfatiza{CW-complejo} es un espacio topológico $X$ construido de la siguiente manera:
\begin{itemize}
\item[(1)] Empezamos con un espacio discreto $X^0$, cuyos puntos consideramos 0-celdas.
\item[(2)] Inductivamente, formamos el \enfatiza{$n$-esqueleto} $X^n$ a partir de $X^{n-1}$ pegando una colección (que puede ser vacía) de $n$-celdas.
\item[(3)] Definimos $X=\bigcup_n X^n$, y definimos la siguiente topología, que es coherente con la familia $\left\{X^n\right\}$: un conjunto $A\subset X$ es abierto (o cerrado) si y solo si $A\cap X^n$ es abierto (o cerrado) en $X^n$, para todo $n$. 
\end{itemize}
\end{defin}

\begin{defin}
Si $X=X^n$ para algún $n$, entonces se dice que $X$ es de dimensión finita, y decimos que la dimensión de $X$ es $n$.
\end{defin}

\begin{prop}\label{prop:cw_variedad}%%%%PROP: CW complejo variedad
Sea $X$ un CW-complejo con un conjunto numerable de celdas. Si $X$ es localmente homeomorfo a un espacio euclídeo, entonces es una variedad.
\end{prop}

\begin{prop}
Si $M$ es una $n$-variedad no vacía y un CW-complejo, entonces $n$ es también la dimensión de $M$ como CW-complejo.
\end{prop}



\begin{comment}
\chapter{Espacios adjunción}

\chapter{Homología}

\chapter{Orientabilidad}

\chapter{Espacios cociente}

\chapter{Definiciones}

\begin{defin}%%%%DEF: Finitamente local
Una colección de subconjuntos de un espacio topológico $X$ se dice \enfatiza{localmente finita} si cada punto del espacio tiene un entorno que interseca sólo un número finito de subconjuntos de la colección.
\end{defin}

\begin{defin}
es coherente
\end{defin}
\end{comment}

\begin{thebibliography}{99}

\bibitem{lee1}
J. M. Lee.
\textit{Introduction to Topological Manifolds}. Graduate text in mathematics, Springer - Verlag New York, 2011.

\bibitem{juanjo}
V. Muñoz, J. J. Madrigal. 
\textit{Topología Algebráica}. Sanz y Torres, 2015.


\bibitem{hatcher_torus}
A. Hatcher.
\textit{The Kirby Torus Trick for Surfaces}, 2013.
\\\url{http://front.math.ucdavis.edu/1312.3518}

\bibitem{rado}

T. Radó.
\textit{Über den Begriff der Riemannschen Fläche, Acta Sci}. Math. Szeged. 2 1925,
101–121.

\bibitem{munkres}
J. R. Munkres.
\textit{Elements of Agebraic Topology}. Addison-Wesley, 1984.

\bibitem{hatcher}
A. Hatcher.
\textit{Algebraic Topology}. 2001. 
\\\url{http://pi.math.cornell.edu/~hatcher/AT/ATpage.html}

\bibitem{dehn}
M. Dehn, P. Heegard.
\textit{Analysis situs}, Enzyklopädie der Math. Wiss. 1907, 153-220.


\bibitem{seifert}
H. Seifert, W. Threlfall.
\textit{A Textbook of Topology}, 1era edición. Academic, New York. 1980.

\bibitem{massey}
W. S. Massey. 
\textit{Algebraic Topology, an introduction}, Springer - Verlag. 1970.

\bibitem{wildberger}
N. J. Wildberger.
\textit{Algebraic Topolgy: A Begginer's Course}. Lecture 19.
\\\url{http://www.wildegg.com/youtube-algebraic-topology.html}

\end{thebibliography}

\end{document}


