\documentclass[10pt]{report}
\usepackage[utf8]{inputenc}

\usepackage[spanish]{babel} % Idioma
\selectlanguage{spanish}
\usepackage{graphicx} %imagenes
\usepackage{verbatim} % Para comment


\usepackage{appendix} %Para el apéndice


\usepackage{ragged2e} %para alinear texto izqda y derecha



\usepackage{geometry}
%\geometry{ % Margenes
%  a4paper,
 % left=20mm,
 % right=20mm,
  %top=20mm,
 % bottom=20mm
%}
%\setlength{\parindent}{0pt} % Quitar indentado parrafos automatico

\usepackage{amsmath} % Movidas útiles
\usepackage{amssymb} % Simbolos mates
\usepackage{amsthm} % Personalizar teoremas (mas abajo continuacion)
\usepackage{thmtools} % Mas movidas teoremas

\usepackage{tikz} % Para pictures
\usetikzlibrary{decorations.markings,arrows} %decoracion en tikz
\usetikzlibrary{shadows,arrows,positioning,shapes.geometric}
\usetikzlibrary{decorations,decorations.markings}
\usepackage{etoolbox}
\usetikzlibrary{optics}
\usepackage{pgf,tikz}
\usetikzlibrary{arrows}
\usetikzlibrary{babel}



\usepackage{sectsty} % Personalizar titulos secciones
\sectionfont{\underline} % Titulo seccion subrayado

% Comandos simbolos utiles 
\newcommand{\C}{\mathbb{C}}
\newcommand{\R}{\mathbb{R}}
\newcommand{\Q}{\mathbb{Q}}
\newcommand{\Z}{\mathbb{Z}}
\newcommand{\N}{\mathbb{N}}
\newcommand{\Epsilon}{\mathcal{E}}

\newcommand{\norm}[1]{\left\lVert#1\right\rVert} % Comando para normas

% Estilos teoremas (Mejorable)
\theoremstyle{definition}
\newtheorem{defin}{Definición}[section]
\newtheorem{tma}[defin]{Teorema}
\newtheorem*{tma*}{Teorema}
\newtheorem{corol}[defin]{Corolario}
\newtheorem{prop}[defin]{Proposición}
\newtheorem{lema}[defin]{Lema}
%\newcommand{\demo}{Demostración.\\}
%\newcommand{\ok}{\hfill$\square$}
%\theoremstyle{remark}
\newtheorem{obs}[defin]{Observación}
\newtheorem{eje}[defin]{Ejemplo}
\graphicspath{ {images/} }
\usepackage{afterpage}

\newcommand\blankpage{%
    \null
    \thispagestyle{empty}%
    \newpage}


% Cabeceras
\renewcommand{\title}{ZIP}
\newcommand{\subtitle}{Trabajo fin de grado}
\renewcommand{\maketitle}{{\Large{\textbf{\title}}}\\\\{\Large \subtitle}\\\rule{17cm}{0.4pt}\\}

\begin{document}
%%%%%%%%PORTADA%%%%%%%%%%%

%%% PORTADA%%%%%%
\begin{titlepage} %Creo que esto es para la numeración de páginas
\begin{center} %Que todo quede centradito

% Todo esto de abajo habría que retocarlo pero así sirve de ejemplo
\huge\textsc{Universidad Complutense de Madrid}\\[0.2in]
\includegraphics[scale=0.8]{comlu}\\[0.1in] %Introduce la imagen y la reescala, inserta un pequeño hueco con lo de debajo

\Large{Facultad de Matemáticas}\\[0.5in] %inserta un hueco mayor con lo de abajo
 %la linea horizontal
\Large{Trabajo de Fin de Grado}\\[.1in]
\Huge {Un tratamiento riguroso de la prueba ZIP}\\[0.2in]



\vfill %Llenar verticalmente
\Large {Juan Valero Oliet}\\[0.5in]
\vfill 
Dirigido por:\\
Manuel Alonso Morón\\[.1in]
\Large{Junio de 2020}
\end{center}

\end{titlepage}


%\afterpage{\blankpage}

%%%%%%%%%%%%%PORTADA%%%%%%%%%%%%%%%%%%%%%5

\tableofcontents
\chapter{Definiciones preeliminares}


\section{Variedades}

\subsection{Variedades y superficies}


Los espacios topológicos de los que nos vamos a ocupar en el siguiente trabajo son las variedades.%, que son los más relevantes desde el punto de vista de la geometría.


\begin{defin}
Una \textbf{\emph{variedad topológica}} (de ahora en adelante \emph{variedad}) es un espacio topológico Hausdorff, II AN y localmente homeomorfo a $R^n$, para algún $n\geq 0$.
\end{defin}

Como la propiedad ``ser localmente homeomorfo a $\R^n$'' es local, toda propiedad local de $\R^n$ se traslada a una variedad. Así, las variedades son localmente compactas, I AN, localmente conexas, localmente conexas por caminos y localmente simplemente conexas.\\



%COPIADO lo de abajo, cambiar las palabras!!!!!!!!!!!!

El \textit{teorema de invarianza del dominio} dice que si $W\subset \R^n$ y $W'\subset \R^m$ son abiertos y existe $\phi: W \rightarrow W'$ homeomorfismo, entonces $n=m$. Esto implica que, dado un punto $p\in X$ de una variedad, hay un único $n=n(p)$ tal que un entorno $U^p$ es homeomorfo a un abierto $U'\subset \R^n$. Llamamos $n(p)$ la dimensión en p. Claramente, para todo punto $q\in U$ podemos tomar $U$ como entorno de $q$, y por tanto $n(q)=n(p)$. Luego en toda la componente conexa de $p$, el $n$ que aparece es el mismo, y lo llamaremos dimensión de dicha componente conexa. Nótese que si escribimos $X=\sqcup X_i$, con $X_i$ componentes conexas de $X$, todas las $X_i$ son variedades. Si todas las $X_i$ tienen la misma dimensión $n$, entonces escribimos $n=dim X$ , y decimos que $X$ es una $n$-variedad.\\



\begin{eje}
\begin{itemize}
\item Las 0-variedades son espacios discretos numerables. La única 0-variedad conexa es un punto.
\item Existen dos 1-variedades conexas salvo homeomorfismo: la recta $\R$ y el círculo $\mathbb{S}^1$

\end{itemize}


\end{eje}

\begin{defin}
Una \textbf{\emph{superficie}} es una $2$-variedad.
\end{defin}

\begin{eje}
	\begin{itemize}
		\item La esfera $\mathbb{S}^2=\{(x,y,z) \in \R^3 \mid x^2+y^2+z^2=1\}$


		\item El toro $\mathbb{T}^2=\{(x,y,z)\in \R^3 \mid (\sqrt{x^2+y^2}-2)^2+z^2=1\}$
	\end{itemize}

\end{eje}

\


\begin{figure}
	\begin{center}
		\begin{tikzpicture}
			\draw (0,0) circle (2cm);
			\draw (2,0) arc[x radius=2, y radius=0.7, start angle=0, end angle=-180];
			\draw [dashed] (2,0) arc[x radius=2, y radius=0.7, start angle=0, end angle=180];
			\draw (6,-0.1) ellipse (2.9cm and 1.4 cm);
			\draw (7.5,0.1) arc[x radius=1.5, y radius=0.4, start angle=0, end angle=-180];
			\draw (7.3,-0.1) arc[x radius=1.3, y radius=0.3, start angle=0, end angle=180]; 
		\end{tikzpicture}
	\caption{$\mathbb{S}^2$ y $\mathbb{T}^2$}
	\end{center}
\end{figure}




%%%%%%%%%%%%%%%%%%%%%%%%%%%%%%%%%%%%%%%%%%%%%%%%%%%%%%%%%%%%%%%%%%%%%%%%%%%%%%%%%%%%%%%%%%%%%%%%%%%%%%%%%%%%%%%%%%%%%%%%%%%%%%%%%%%%%%%%%%%%%%%%%%%%%%%%%%%%%%%%%%%%%%%%%%%%%%%%%%%%%%%%%%%%%%%%%%%%%%%%%%%%%%%%%%%%%%%%%%%%%%%%%%%%%%%%%%%%%%%%%%%%%%%%%%%%%%%%%%%%%%%%%%%%%%%%%%%%%%%%%%%%%%%%%%%%%%%%%%%%%%%%%%%%%%%%%%%%%%%%%%%%%%%%%%%%%%%%%%%%%%

\subsection{Suma conexa de variedades}


Sean $V_1$ y $V_2$ dos $n$-variedades conexas. Dados $p_1\in V_1$ y $p_2\in V_2$ sean $U_1^{p_1}\subset V_1$, $U_2^{p_2}\subset V_2$  entornos de $p_1$ y $p_2$ respectivamente, y sean $\phi_1:U_1\to\R^n$ y $\phi_2:U_2\to\R^n$ dos homeomorfismos tales que $\phi_1(p_1)=0$ y $\phi_2(p_2)=0$. Si llamamos $B_1=\phi_1^{-1}(B_1(0))\subset V_1$ y $B_2=\phi_2^{-1}(B_1(0))\subset V_2$, consideremos $V_1^o=V_1-B_1$, $V_2^o=V_2-B_2$ y $V_1^o \sqcup V_2^o$ con la topología unión disjunta.
Se define la relación de equivalencia $\sim$ en la que si $x_1\in S_1=\phi_1^{-1}(\partial B_1(0))$, $x_2\in S_2=\phi_2^{-1}(\partial B_1(0))$, entonces $x_1\sim x_2$ si y sólo si $\phi_1(x_1)=\phi_2(x_2)$, y se considera el cociente 

$$X=\frac{V_1^o\sqcup V_2^o}{\sim}.$$

\begin{defin}
A $X$ se le llama \textbf{\textit{suma conexa}} de $V_1$ y $V_2$, y se denota por $X=V_1\#V_2$.
\end{defin}
\begin{figure}
\begin{tikzpicture}[line cap=round,line join=round,>=triangle 45,x=1.0cm,y=1.0cm]
%\clip(-8.56,-6.52) rectangle (6.77,3.86);
\draw [rotate around={0:(-5.5,0)}] (-5.5,0) ellipse (2.5cm and 2cm);
\draw(0,0) circle (0.5cm);
\fill [black!10] (0.5,0) arc[x radius=0.5, y radius=0.5, start angle=0, end angle=-180];
%\draw [rotate around={0:(0,0)}] (0,0) ellipse (0.5cm and 0.23cm);
\filldraw [color=black, fill=black!15] (0.5,0) arc[x radius=0.5, y radius=0.23, start angle=0, end angle=-180];
\filldraw [color=black, fill=black!15][dashed] (0.5,0) arc[x radius=0.5, y radius=0.23, start angle=0, end angle=180];

\draw [rotate around={0:(5.7,0)}] (5.7,0) ellipse (2.5cm and 2cm);
%\draw [dashed] [rotate around={0:(6.22,1.32)}] (6.22,1.32) ellipse (0.45cm and 0.22cm);
\draw [dashed] (6.67,1.32) arc[x radius=0.45, y radius=0.22, start angle=0, end angle=180];
\draw (6.67,1.32) arc[x radius=0.45, y radius=0.22, start angle=0, end angle=-180];

\filldraw [color=black, fill=black!10] [rotate around={0:(-5,1.32)}] (-5.0,1.32) ellipse (0.45cm and 0.22cm);
%\fill [black!10] [rotate around={0:(6.22,1.32)}] (6.22,1.32) ellipse (0.45cm and 0.28cm);
%\draw(6.22,1.32) circle (0.45cm);
\draw (5.77,1.32) arc[x radius=0.45, y radius=0.45, start angle=180, end angle=0];
%\draw [ultra thin] (5.77,1.32) arc[x radius=0.45, y radius=0.45, start angle=-180, end angle=0];

\draw (-6.1,0) arc[x radius=0.4, y radius=0.2, start angle=180, end angle=0];
\draw (-6.4,0.31) arc[x radius=0.7, y radius=0.4, start angle=180, end angle=360];
\draw (-1.8,0.56) node[anchor=north west] {$\#$};
\draw (1.22,0.56) node[anchor=north west] {$=$};
\draw (5.1,0) arc[x radius=0.4, y radius=0.2, start angle=180, end angle=0];
\draw (4.8,0.31) arc[x radius=0.7, y radius=0.4, start angle=180, end angle=360];
\end{tikzpicture}
\end{figure}
%%%%%%%%%%%%DIBUJO DE SUMA CONEXA%%%%%%%%%%%%
\begin{prop}
Sean $V_1$ y $V_2$ variedades. Entonces $X=V_1\#V_2$ es una variedad.
\end{prop}
\begin{proof}
Denotemos la proyección $\pi:M_1^o\sqcup M_2^o\to X$. Sea $S=\pi (S_1)=\pi (S_2)$. Tenemosdos abiertos $U_j=M_j^o-S_j$, $j=1,2$ saturados. Por tanto, $\pi :U_ j\to \pi (U_j)=U_j'$ es homeomorfismo. Esto implica que $X$ es localmente $\R^n$ en los puntos de $U_1'\cup U_2'$ . Además ahí la topología es Hausdorff y IIAN.
Veamos ahora qué ocurre para un punto $p\in S$. Se tiene que $p=\pi (p_1)=\pi (p_2)$, $p_j\in S_j$, $j=1,2$, y $\varphi_j(p_j)=x_0 \in \partial B_1(0) \in \R^n$. Tomamos un entorno $V\subset \partial B_1(0)$ de $x_0$ en $\partial B_1(0)$, con lo que $\hat{V}=\{rx|r\in (1-\varepsilon , 1+\varepsilon ), x\in V\}$ es entorno de $x_0$ en $\R^n$, y $\hat{V}-B_1(0)=\{rx|r\in [1, 1+\varepsilon ), x \in V\}$. Sea $V_j=\varphi_j^{-1}(\hat{V}-B_1(0))\subset M_j^o$, que es entorno de $p_j$. Claramente $V_1\sqcup V_2$ es abierto saturado de $M_1^o\sqcup M_2^o$, luego $$\tilde{V}=\pi (V_1\sqcup V_2)$$ es entorno de $p$ en $X$. Veamos ahora que es homeomorfo a un abierto de $\R^n$. Sea
\begin{align*}
\Phi : & V_1\sqcup V_2  \to  V\times (1-\varepsilon , 1+ \varepsilon ), \\
& q_1\in V_1  \mapsto  (x,r), r=\norm{\varphi_1(q_1)}, x=\varphi_1(q_1)/r,\\
& q_2\in V_2  \mapsto  (x,2-r), r= \norm{\varphi_2(q_2)}, x= \varphi_2(q_2)/r.
\end{align*}
Por tanto, $\Phi : V_1 \to V \times [1, 1+\varepsilon)$ y $\Phi : V_2 \to V \times (1- \varepsilon, 1]$ son homeomorfismos. Además, $q_1 \sim q_2$ si y sólo si $\Phi(q_1)=\Phi(q_2)$. De este modo, $\Phi$ induce una aplicación continua y biyectiva $$\overline{\Phi} : \tilde{V} \to V \times (1- \varepsilon, 1+ \varepsilon)$$.

$\overline{\Phi}$ es abierta: si tomamos un abierto básico saturado de $V_1\sqcup V_2$, o bien está totalmente incluido en $V_1-S_1$ o en $V_2-S_2$, en cuyo caso su imagen es un abierto de $V\times (1-\varepsilon ,1)$ o $V\times (1,1+\varepsilon )$, o bien interseca a $S_1$ y $S_2$. En ese caso se puede asumir que es un abierto de la forma $W_1\sqcup W_2$, construido como antes y donde hemos partido de un $W\subset V \subset \partial B_1(0)$. Entonces $\overline{\Phi} (\tilde{W})= W \times (1-\delta , 1+\delta )$ con $0<\delta \leq \varepsilon$, $\tilde{W} = \pi (W_1 \sqcup W_2)$. Luego $\overline{\Phi}$ es un homeomorfismo, y $\tilde{V}$ es homeomorfo a un abierto de $\R^n$.

Los abiertos construidos, $\tilde{V} \subset X$, se pueden tomar en cantidad numerable para formar una base de la topología, con lo cual $X$ es IIAN. También, dado un $q\in U_j'$, $j=1,2$, y un $p\in S$, se puede tomar un abierto $\tilde{V}$ entorno de $p$ disjunto de un entorno pequeño de $q$. Y si tomamos $p, p' \in S$ distintos, los abiertos $\tilde{V}$,$\tilde{V}'$ construidos partiendo de $V,V'\subset \partial B_1(0)$ disjuntos, serán disjuntos. Luego $X$ es Hausdorff.
\end{proof}



\section{Representación de superficies}



Para el teorema de clasificación necesitamos un método uniforme de representación de las superficies compactas. Representaremos todas las superficies como cocientes de polígonos con $2n$ lados. 


%%%%%%%%% QUIZAS INCLUIR UNA DEFINICION INFORMAL%%%%%%%%%%%%%%
\begin{defin}
Sea $S$ un conjunto. Una \textbf{\textit{palabra en $S$}} es una $k$-tupla ordenada de símbolos, cada uno de la forma $a$ o $a^{-1}$, para cierto $a\in S$.
\end{defin}

\begin{defin}
\label{def:rep_pol}
Una \textbf{\textit{representación poligonal}}, que denotaremos por $$\mathcal{P}=\langle S\mid W_1,...,W_k\rangle$$ es un conjunto finito S junto con un número finito de palabras $W_1,..,W_k$ de longitud $3$ o más, tal que para todo $a\in S$ existe un $W_i$ tal que $a\in W_i$. Por cuestiones de notación, cuando el conjunto $S$ esté descrito listando sus elementos, quitaremos los corchetes que rodean los elementos de $S$ y denotaremos las palabras $W_i$ por youxtaposición. Por ejemplo, la presentación con $S=\{a,b\}$ y la palabra $W=(a,b,a^{-1},b^{-1})$ se escribe $\langle a,b \mid aba^{-1}b^{-1}\rangle$. Permitimos el caso especial de $S=\{a\}$ y palabras de longitud $2$, es decir $\langle a\mid aa\rangle$, $\langle a\mid a^{-1}a^{-1}\rangle$, $\langle a\mid aa^{-1}\rangle$ y $\langle a\mid a^{-1}a\rangle$.
\end{defin}
\begin{defin}
Toda representación poligonal $\mathcal{P}$ da lugar a un espacio topológico $|\mathcal{P}|$, llamado \textbf{\textit{realización geométrica de $\mathcal{P}$}} . $|\mathcal{P}|$ se obtiene de la siguiente manera:
\begin{itemize}
\item[1.] Para cada $W_i\in \mathcal{P}$ de longitud $k$, sea $P_i$ el $k$-polígono centrado en el origen con lados de longitud 1 y tal que un lado yace sobre el eje $OY$.
\item[2.] Se define una correspondencia uno a uno entre los símbolos de $W_ i$ y los lados de $P_i$ en orden inverso a las agujas del reloj, empezando por el que yace en el eje $OY$.
\item[3.] Sea $|\mathcal{P}|$ el espacio cociente de $\coprod_i P_i$ determinado identificando lados que tengan el mismo símbolo, conforme al homeomorfismo afín que hace coincidir los primeros vértices de lo lados con una dada etiqueta $a$ y los últimos vertices de los que tienen la correspondiente etiqueta $a^{-1}$ (en el sentido a las agujas del reloj).
\end{itemize}


Si $|\mathcal{P}|$ es una de las representaciones poligonales de un solo elemento, decimos que $|\mathcal{P}|$ es la esfera $\mathbb{S}^2$ si la palabra es $aa^{-1}$ o $a^{-1}a$, y el plano proyectivo $\mathbb{P}^2$ si es $aa$ o $a^{-1}a^{-1}$.

\end{defin}

\begin{defin}

 Las regiones interiores, los lados y los vértices de cada polígono $P_i$ se llaman \textbf{\emph{caras, lados y vértices de la presentación}}. El número de caras es el mismo que el número de palabras, y el número de lados coincide con la suma de la longitud de las palabras.
Para un lado etiquetado $a$, el \textbf{\emph{vértice inicial}} es el primero en el sentido de las agujas del reloj, y el otro es el \textbf{\emph{vértice final}}. Para un lado etiquetado $a^{-1}$, estas definiciones se invierten. 
\end{defin}

\begin{defin}
Una representación poligonal es una \textbf{\emph{representación de una superficie}} si para todo $a\in S$, $a$ ocurre exáctamente dos veces en $W_1,...,W_k$ como $a$ o como $a^{-1}$.
\end{defin}

\begin{defin}
Si $X$ es un espacio topológico y $\mathcal{P}$ una representación poligonal cuya realización geométrica es homeomorfa a $\mathcal{P}$, decimos que $\mathcal{P}$ es una \textbf{\emph{representación de $X$}}.
\end{defin}

\begin{obs}
Un espacion topológico que admite una representación con una sola cara es conexo, pues es homeomorfo al cociente de una región poligonal conexa. Con más de una cara, puede ser o no conexo.
\end{obs}

Veamos la representación de algunas superficies importantes. Para ello vamos a necesitar la siguiente proposición:



\begin{figure}[h]
\begin{center}


\label{fig:convexo_esfera}
\definecolor{uququq}{rgb}{0.25,0.25,0.25}
\begin{tikzpicture}[line cap=round,line join=round,>=triangle 45,x=0.65cm,y=0.65cm]
%\clip(-6.02,-3.95) rectangle (8.11,5.63);
\draw [shift={(7.74,-8.14)}] plot[domain=1.68:2.51,variable=\t]({1*14.71*cos(\t r)+0*14.71*sin(\t r)},{0*14.71*cos(\t r)+1*14.71*sin(\t r)});
\draw [shift={(-0.34,-1.89)}] plot[domain=2.56:4.7,variable=\t]({1*4.5*cos(\t r)+0*4.5*sin(\t r)},{0*4.5*cos(\t r)+1*4.5*sin(\t r)});
\draw [shift={(6.31,-14.62)}] plot[domain=1.46:2.25,variable=\t]({1*10.61*cos(\t r)+0*10.61*sin(\t r)},{0*10.61*cos(\t r)+1*10.61*sin(\t r)});
\draw [shift={(5.58,1.03)}] plot[domain=-1.21:1.47,variable=\t]({1*5.47*cos(\t r)+0*5.47*sin(\t r)},{0*5.47*cos(\t r)+1*5.47*sin(\t r)});
\draw (0,0)-- (10.97,0.11);
\draw(0,0) circle (1.65cm);
\draw (0,0)-- (3.74,0.04);
\draw(3.74,0.04) circle (0.66cm);
\draw (-0.1,-0.01) node[anchor=north ] {$0 $};
\draw (10.97,0.11)-- (0.08,0.79);
\draw (-0.37,1.65) node[anchor=north west] {$y $};
\draw (3.6,1.1) node[anchor=north west] {$z $};
\draw (3.3,-0.15) node[anchor=north west] {$ \lambda x_0 $};
\draw (11.17,0.53) node[anchor=north west] {$ x_0 $};
\draw (-3.78,2.92) node[anchor=north west] {$D$};
\draw (-1.10,3.5) node[anchor=north west] {$B_1(0)$};
\draw (2.85,2.2) node[anchor=north west] {$B_{1-\lambda}(\lambda x_0)$};
\begin{scriptsize}
\fill [color=black] (10.97,0.11) circle (1.5pt);
\fill [color=uququq] (0,0) circle (1.5pt);
\fill [color=black] (3.74,0.04) circle (1.5pt);
\fill [color=black] (0.08,0.79) circle (1.5pt);
\fill [color=black] (3.8,0.56) circle (1.5pt);
\end{scriptsize}
\end{tikzpicture}
\end{center}
\caption{Demostración de que sólo hay un punto de la frontera en la semirecta.}
\end{figure}
\begin{prop}% TEOREMA 5.1 LEE
\label{teo:convexo_homeom_esfera} 
Si $D\subseteq \R^n$ es un conjunto compacto y convexo con interior no vacío, entonces $D$ es homeomorfo a $\overline{\mathbb{B}}^n.$ De hecho, dado $p\in \mathring{D}$, entonces existe un homeomorfismo $F:\overline{\mathbb{B}}^n\to D$ que envía $0$ a $p$, $\overline{\mathbb{B}}^n$ a $\mathring{D}$, y $\mathbb{S}^{n-1}$ a $\partial D$.
\end{prop}
\begin{proof}
Sea $p\in D$ un punto de su interior. Si reemplazamos $D$ por su imagen mediante la traslación $x\mapsto x-p$, que es un homeomorfismo de $\R^n$ en sí mismo, podemos asumir que $p=0\in \mathring{D}$. Entonces existe un $\varepsilon >0 $ tal que la bola $B_{\varepsilon}(0)$ está contenida en $D$. Usando la dilatación $x\mapsto x/\varepsilon$, podemos asumir que $\mathbb{B}^n = B_ 1(0) \subseteq D$.
La clave de la demostración es la siguiente: \emph{cada semirecta cerrada empezando en el origen interseca $\partial
D$ en exactamente un punto}. Sea $R$ una semirecta tal. Dado que $D$ es compacto, su interrsección con R es compacta. Por tanto existe un punto $x_0$ en su intersección tal que en él su distancia al origen asume el máximo. Es claro %lo es?
que pertenece a la frontera de $D$. Para ver que el punto es único, veamos que el segmento que une $0$ y $x_0$ está formado enteramente por puntos interiores de $D$ excepto por el $x_0$ mismo. Cualquier punto en este segmento distinto de $x_o$ se puede escribir de la forma $\lambda x_o $ para $0\leq \lambda <1. $ Supongamos $z\in B_{1-\lambda}(\lambda x_0)$, y sea $y=(z-\lambda x_0)/(1-\lambda )$. Como $|z-\lambda x_0|<|1-\lambda|$ se tiene que $|y|<1$, y por tanto $y\in B_ 1(0)\subseteq D$ (\ref{fig:convexo_esfera}). Como $y$ y $x_0$ están en $D$ y $z=\lambda x_0 + (1-\lambda )y$, se sigue de la convexidad que $\in D$. Por tanto la bola abierta $B_{1-\lambda}(\lambda x_0)$ está contenida en $D$, lo que implica que $\lambda x_0$ es un punto interior.

Definimos ahora la aplicación $f:\partial D \to \mathbb{S}^{n-1}$ por 
$$f(x)=\frac{x}{|x|}$$.

$f(x)$ es el punto donde el segmento desde el origen hasta $x$ interseca la esfera unidad. Como $f$ es la restricción de una función continua, es continua, y por el parágrafo anterior es biyectiva. Dado que $\partial D$ es compacta, $f$ es un homeomorfismo por el teorema de la aplicación cerrada (\ref{teo:aplicac_cerrada}).

Finalmente definimos $F:\overline{\mathbb{S}}^n \to D$ por 
$$F(x)= \left\{\begin{array}{lc}
				|x|f^{-1}\left(\dfrac{x}{|x|}\right), & x\neq 0;
				\\0, & x=0.

\end{array}
\right. $$
$F$ es continua fuera del origen por serlo $f^{-1}$, y en el origen porque por ser $f^{-1}$ acotada $F(x) \to 0$ cuando $x\to 0$. Geometricamente, $F$ manda cada segmento radial que conecta 0 con un punto de $\mathbb{S}^{n-1}$ al segmento radial desde $0$ hasta el punto $f^{-1}(w)\in \partial D$. Por convexidad, $F$ toma valores en $D$. La aplicación $F$ es inyectiva, pues puntos de distintas semirectas van a parar a distintas semirectas, y cada segmento radial va linealmente a su imagen. Es sobreyectiva pues cada punto $y \in D$ está en una semirecta empezando en 0. Por el teorema de la aplicación cerrada \ref{teo:aplicac_cerrada}, $F$ es un homeomorfismo.
\end{proof}















\begin{figure}[h]
\begin{center}
\label{fig:esfera_cociente_circunferencia}
\begin{tikzpicture}[line cap=round,line join=round,>=triangle 45,x=1.5cm,y=1.5cm]
%\clip(-5.98,-3.38) rectangle (6.76,3.11);
\draw [<-][shift={(0,1.25)}] plot[domain=0.64:2.5,variable=\t]({1*1.25*cos(\t r)+0*1.25*sin(\t r)},{0*1.25*cos(\t r)+1*1.25*sin(\t r)});
\draw (0,3) node[anchor=north] {$\pi$};
\draw (2,1.5) arc[x radius=0.7, y radius=1.5, start angle=90, end angle=270];
\draw [->] (2,0) -- (0.87,-1.68);
\draw(2,0) circle (2.25cm);
\draw(-2,0) circle (2.25cm);
\draw (-3.12,1)-- (-0.88,1);
\draw (-3.41,0.5)-- (-0.59,0.5);
\draw (-3.41,-0.5)-- (-0.59,-0.5);
\draw (-3.12,-1)-- (-0.88,-1);
\draw [->] (2,0) -- (2,2);
\draw [dashed] (3.1,1) arc[x radius=1.1, y radius=0.2, start angle=0, end angle=180];
\draw (3.1,1) arc[x radius=1.1, y radius=0.2, start angle=0, end angle=-180];
\draw [dashed] (3.4,0.5) arc[x radius=1.4, y radius=0.2, start angle=0, end angle=180];
\draw (3.4,0.5) arc[x radius=1.4, y radius=0.2, start angle=0, end angle=-180];
\draw [dashed] (3.5,0) arc[x radius=1.5, y radius=0.2, start angle=0, end angle=180];
\draw (3.5,0) arc[x radius=1.5, y radius=0.2, start angle=0, end angle=-180];
\draw [dashed] (3.4,-0.5) arc[x radius=1.4, y radius=0.2, start angle=0, end angle=180];
\draw (3.4,-0.5) arc[x radius=1.4, y radius=0.2, start angle=0, end angle=-180];
\draw [dashed] (3.10,-1) arc[x radius=1.10, y radius=0.2, start angle=0, end angle=180];
\draw (3.10,-1) arc[x radius=1.10, y radius=0.2, start angle=0, end angle=-180];
\draw (-3.5,0)-- (-0.5,0);
\draw (-0.6,0.1)--(-0.5,0)--(-0.4,0.1);
\draw (-3.6,0.13)--(-3.5,0)--(-3.4,0.13);
\draw [->] (2,0) -- (4,0);
\begin{scriptsize}
\fill [color=black] (-2,1.5) circle (2.0pt);
\fill [color=black] (-2,-1.5) circle (2.0pt);
\fill [color=black] (2,1.5) circle (2.0pt);
\fill [color=black] (2,-1.5) circle (2.0pt);
\fill [color=black] (-2,1.5) circle (2.0pt);
%\fill [color=black] (1.3,-0.17) circle (2.0pt);
\draw (1.2,-0.07)--(1.3,-0.17)--(1.4,-0.07);
\end{scriptsize}
\end{tikzpicture}
\end{center}

\caption{La esfera como cociente de una circunferencia.}
\end{figure}
\begin{prop}
La esfera $\mathbb{S}^2$ es homeomorfa a los siguientes espacios cociente: 
%%%%%%%%%%o cocientes??????????%%%%%
\begin{itemize}
\item[(a)] El disco cerrado $\overline{\mathbb{B}}^2\subseteq \mathbb{R}^2$ módulo la relación de equivalencia generada por $(x,y)\sim (-x,y)$, para $(x,y)\in \partial \overline{\mathbb{B}}^2$
\item[(b)] El cuadrado $S=\{(x,y):|x|+|y|\leq 1\}$ módulo la relación de equivalencia generada por $(x,y)\sim(-x,y)$ para $(x,y)\in \partial S$.
\end{itemize}
\end{prop}
\begin{proof}
Para ver que cada espacio es homeomorfo a la esfera, daremos una aplicación cociente desde el espacio dado a la esfera que haga las mismas identificaciones que la relación de equivalencia, y entonces apelaremos a la unicidad del espacio cociente. (Teorema \ref{teo:unicidad_espacio_cociente})\\
Para (a), vamos a definir una aplicación del disco en la esfera que envuelve cada paralelo con un segmento horizontal del disco (ver Figura \ref{fig:esfera_cociente_circunferencia})
Formalmente, esta aplicación $\pi:\overline{\mathbb{B}}^2\to \mathbb{S}^2$ vienen dada por 
$$\pi(x,y)=\left\{\begin{array}{lc}
			(-\sqrt{1-y^2} \cos\dfrac{\pi x}{\sqrt{1-y^2}}, -\sqrt{1-y^2}, y), & y\neq \pm 1 \\
			\\(0,0,y), & y=\pm1 

\end{array}
\right.$$
Es claro que $\pi$ es continua y hace las mismas identificaciones que la relación de equivalencia. Por ser sobreyectiva, es una aplicación cociente (\emph{Teorema de la aplicación cerrada \ref{teo:aplicac_cerrada}}).

Para probar (b), sea $\alpha:S\to \overline{\mathbb{B}}^2$ el homeomorfismo construido en la demostración de \ref{teo:convexo_homeom_esfera} que manda linealmente cada segmento radial entre el origen y la frontera de $S$ al segmento paralelo entre centro del disco y su frontera. Hagamos ahora $\beta=\pi \circ \alpha : S \to \mathbb{S}^2$, donde $\pi$ es la aplicación cociente del parágrafo anterior. Tenemos entonces que $\beta$ identifica $(x,y)$ y $(-x,y)$ cuando $(x,y)\in \partial S$, y por otro lado es inyectia, así que hace las mismas identificaciones que la aplicación cociente definida en (b), completando así la demostración (ver figura \ref{fig:esfera_cuadrado}). 
\end{proof}



\definecolor{uququq}{rgb}{0.25,0.25,0.25}
\definecolor{qqqqff}{rgb}{0,0,1}

\begin{figure}[h]

\begin{center}
\label{fig:esfera_cuadrado}
\begin{tikzpicture}[use optics][line cap=round,line join=round,>=triangle 45,x=1.0cm,y=1.0cm]
%\clip(-15.84,-6.32) rectangle (13.31,6.01);
\draw[-<-={at=0.125},->-={at=0.375}, -<<-={at=0.625}, ->>-={at=0.875} ](0,0) circle (2cm);
\draw [-<-={at=0.5}] (-8,0)-- (-6,2);
\draw [-<-={at=0.5}](-4,0)-- (-6,2);
\draw [-<<-={at=0.5}](-8,0)-- (-6,-2);
\draw [->>-={at=0.5}](-6,-2)-- (-4,0);
\draw(6,0) circle (2cm);
%\draw [rotate around={0:(6,0)},dash pattern=on 5pt off 5pt] (6,0) ellipse (2cm and 0.7cm);
\draw (8,0) arc[x radius=2, y radius=0.7, start angle=0, end angle=-180];
\draw [dashed] (8,0) arc[x radius=2, y radius=0.7, start angle=0, end angle=180];
\draw [->] (-8.5,0) -- (-3.5,0);
\draw [->] (-6,-2.5) -- (-6,2.5);
\draw [->] (-2.5,0) -- (2.5,0);
\draw [->] (0,-2.5) -- (0,2.5);
\draw [->] (3.5,0) -- (8.5,0);
\draw [->] (6,0) -- (6,2.5);
\draw [->] (6,0) -- (4.51,-2.5);
\draw [<-][shift={(-3,1.25)}] plot[domain=0.64:2.5,variable=\t]({1*1.25*cos(\t r)+0*1.25*sin(\t r)},{0*1.25*cos(\t r)+1*1.25*sin(\t r)});
\draw [<-] [shift={(3,1.25)}] plot[domain=0.64:2.5,variable=\t]({1*1.25*cos(\t r)+0*1.25*sin(\t r)},{0*1.25*cos(\t r)+1*1.25*sin(\t r)});
\draw (-3.1,3.21) node[anchor=north west] {$ \alpha $};
\draw (2.89,3.23) node[anchor=north west] {$ \pi $};
\begin{scriptsize}
\fill [color=qqqqff] (0,2) circle (1.5pt);
\fill [color=uququq] (-8,0) circle (1.5pt);
\fill [color=uququq] (-4,0) circle (1.5pt);
\fill [color=uququq] (-6,-2) circle (1.5pt);
\fill [color=uququq] (-6,2) circle (1.5pt);
\fill [color=uququq] (-2,0) circle (1.5pt);
\fill [color=uququq] (2,0) circle (1.5pt);
\fill [color=uququq] (0,-2) circle (1.5pt);
\fill [color=uququq] (0,2) circle (1.5pt);
\end{scriptsize}
\end{tikzpicture}
\caption{La esfera como cociente de un cuadrado.}

\end{center}
\end{figure}

\begin{corol}
$\mathbb{S}^2=\langle a|aa^{-1}\rangle=\langle a,b| aba^{-1}b^{-1}\rangle$
\end{corol}
\begin{proof}
$\mathbb{S}^2=\langle a|aa^{-1}\rangle$ por la definición \ref{def:rep_pol}. La segunda igualdad es consecuencia de la proposición anterior.


\end{proof}

\begin{prop}
El plano proyectivo $\mathbb{P}^2$ es homeomorfo a los siguientes espacios cociente:
\begin{itemize}
\item[(a)] El disco cerrado $\overline{\mathbb{B}}^2$ módulo la relación de equivalencia generada por $(x,y) \sim (-x,-y)$ para cada $(x,y)\in \partial \overline{\mathbb{B}}^2$.
\item[(b)] La región cuadrada $S=\{(x,y):|x|+|y|\leq 1\} $ módulo la relación de equivalencia generada por $(x,y)\sim (-x,-y) $ para todo $(x,y)\in \partial S$.
\end{itemize}

\end{prop}
\begin{proof}
Sea $p:\mathbb{S}^2 \to \mathbb{P}^2$ la aplicación cociente dada por la relación de equivalencia $\sim$ generada por $(x,y) \sim (-x,-y)$ para cada $(x,y)\in \mathbb{S}^2$, que representa $\mathbb{P}^2$ como el cociente de una esfera. %AQUI FALTA UN POCO DE EXPLICACION, EJEMPLO 4.54 DEL LEE
Si $F:\overline{\mathbb{B}}^2 \to \mathbb{S}^2$ es la aplicación que manda el disco en el emisferio superior de la esfera por $F(x,y)=(x,y,\sqrt{1-x^2-y^2})$, entonces $p\o F:\overline{\mathbb{B}}^2 \to \mathbb{S}^2/\sim$ es sobreyectiva (---lo demuestro?) y es así una aplicación cociente por el teorema de la aplicación cerrada (\ref{teo:aplicac_cerrada}). La aplicación identifica únicamente $(x,y)\in \partial \overline{\mathbb{B}}^2$ con $(-x,-y)\in \partial \overline{\mathbb{B}}^2$, por lo que $\mathbb{P}^2$ es homeomorfo al espacio cociente resultante.
PARTE B

\end{proof}
%%%%%%%%%%%%%%%%%%%%%%%%%%%%%%%%%%%%%%%%%%%%%%%%%%%%%%%%%%%%%%%%%%%%%%%%%%%%%%%%%%%%%%%%%%%%%%%%%%%%%%%%%%%%%%%%%%%%%%%%%%%%%%%%%%%%%%%%%%%%%%%%%%%%%%%%%%%%%%%%%%%%%%%%%%%%%%%%%%%%%%%%%%%%%%%%%%%%%%%%%%%%%%%%%%%%%%%%%%%%%%%%%%%%%%%%%%%%%%%%%%%%%%%%%%%%%%%%%%%%%%%%%%%%%%%%%%%%%%%%%%%%%%%%%%%%%%%%%%%%%%%%%%%%%%%%%%%%%%%%%%%%%%%%%%%%%%%%%%%%%%%%%%%%%%%%%%%%%%%%%%%%%%%%%%%%%%%%%%%%%%%%%%%%%%%%%%%%%%%%%%%%%%%%%%%%%%%%%%%%%%%%%%%%%%%%%%%%%%%%%%%%%%%%%%%%%%%%%%%%%%%%%%%%%%%%%%%%%%%%%%%%%%%%%%%%%%%%%%%%%%%%%%%%%%%%%%%%

\begin{comment}
\begin{figure}

\definecolor{uququq}{rgb}{0.25,0.25,0.25}



\begin{tikzpicture}[line cap=round,line join=round,>=triangle 45,x=1.0cm,y=1.0cm]
%\clip(-5.39,-4) rectangle (8.94,3.29);
\draw(0,0) circle (2cm);
\draw (-2.2,-0.2)-- (-1.99,0.2);
\draw (-1.99,0.2)-- (-1.85,-0.2);
\draw (1.85,-0.2)-- (1.99,0.2);
\draw (1.99,0.2)-- (2.2,-0.2);
\draw (2.5,0.15) node[anchor=north west] {$ a $};
\draw (-2.5,0.15) node[anchor=north east] {$ a $};
\draw (0.00,-2.3) node[anchor=north] {$ \mathbb{S}^2 $};
\begin{scriptsize}
\fill [color=uququq] (0,2) circle (1.5pt);
\fill [color=uququq] (0,-2) circle (1.5pt);
\end{scriptsize}
\end{tikzpicture}


\begin{tikzpicture}[line cap=round,line join=round,>=triangle 45,x=1.0cm,y=1.0cm]
%\clip(-4.3,-5.46) rectangle (18.82,6.3);
\draw(0,0) circle (2cm);
\draw (-2.2,-0.2)-- (-1.99,0.2);
\draw (-1.99,0.2)-- (-1.85,-0.2);
\draw (2.5,0.24) node[anchor=north west] {$ a $};
\draw (-2.5,0.24) node[anchor=north east] {$ a $};
\draw (0,-2.3) node[anchor=north] {$ \mathbb{P}^2 $};
\draw (1.85,0.2)-- (1.99,-0.2);
\draw (1.99,-0.2)-- (2.2,0.2);
\begin{scriptsize}
\fill [color=uququq] (0,2) circle (1.5pt);
\fill [color=uququq] (0,-2) circle (1.5pt);
\end{scriptsize}
\end{tikzpicture}
\caption{Representación poligonal de $\mathbb{S}^2 y \mathbb{P}^2$}
\end{figure}

\end{comment}

%
%
%
%
%
%
%
%
%
%
%
%
%
%
%
%
%



\appendix
\chapter{Teoremas Usados}

\begin{tma}[Lema de la aplicación cerrada]
\label{teo:aplicac_cerrada}
Sea $F$ una aplicación continua de un espacio topológico compacto en un espacio topológico Hausdorff. Entonces:
\begin{itemize}
\item[(a)]$F$ es una aplicación cerrada.
\item[(b)] Si $F$ es sobreyectiva, entonces es una aplicación cociente.
\item[(c)] Si $F$ es inyectiva, entonces es una inmersión topológica. %?????
\item[(d)] Si $F$ es biyectiva, entonces es un homeomorfismo.
\end{itemize}
\end{tma}

\begin{tma}[Unicidad de espacios cociente]
\label{teo:unicidad_espacio_cociente}
Supongamos $q_1:X\to Y_1$ y $q_2:X\to Y_2$ son aplicaciones cociente que hacen las mismas identificaciones, es decir, tales que $q_1(x)=q_1(x')$ si y solo si $q_2(x)=q_2(x')$. Entonces existe un único homeomorfismo $\phi:Y_1\to Y_2$ tal que $\phi \circ q_1=q_2$.
\end{tma}
\end{document}
