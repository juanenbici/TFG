\documentclass[10pt]{report}
\usepackage[utf8]{inputenc}

\usepackage[spanish]{babel} % Idioma
\selectlanguage{spanish}
\usepackage{graphicx} %imagenes
\usepackage{verbatim} % Para comment


\usepackage{appendix} %Para el apéndice


\usepackage{ragged2e} %para alinear texto izqda y derecha



\usepackage{geometry}
%\geometry{ % Margenes
%  a4paper,
 % left=20mm,
 % right=20mm,
  %top=20mm,
 % bottom=20mm
%}
%\setlength{\parindent}{0pt} % Quitar indentado parrafos automatico

\usepackage{amsmath} % Movidas útiles
\usepackage{amssymb} % Simbolos mates
\usepackage{amsthm} % Personalizar teoremas (mas abajo continuacion)
\usepackage{thmtools} % Mas movidas teoremas

\usepackage{tikz} % Para pictures
\usetikzlibrary{decorations.markings,arrows} %decoracion en tikz
\usetikzlibrary{shadows,arrows,positioning,shapes.geometric}
\usetikzlibrary{decorations,decorations.markings}
\usepackage{etoolbox}
\usetikzlibrary{optics}
\usepackage{pgf,tikz}
\usetikzlibrary{arrows}
\usetikzlibrary{babel}

\usepackage{sectsty} % Personalizar titulos secciones
\sectionfont{\underline} % Titulo seccion subrayado

% Comandos simbolos utiles 
\newcommand{\C}{\mathbb{C}}
\newcommand{\R}{\mathbb{R}}
\newcommand{\Q}{\mathbb{Q}}
\newcommand{\Z}{\mathbb{Z}}
\newcommand{\N}{\mathbb{N}}
\newcommand{\Epsilon}{\mathcal{E}}

\newcommand{\norm}[1]{\left\lVert#1\right\rVert} % Comando para normas

% Estilos teoremas (Mejorable)
\theoremstyle{definition}
\newtheorem{defin}{Definición}[section]
\newtheorem{tma}[defin]{Teorema}
\newtheorem*{tma*}{Teorema}
\newtheorem{corol}[defin]{Corolario}
\newtheorem{prop}[defin]{Proposición}
\newtheorem{lema}[defin]{Lema}
%\newcommand{\demo}{Demostración.\\}
%\newcommand{\ok}{\hfill$\square$}
%\theoremstyle{remark}
\newtheorem{obs}[defin]{Observación}
\newtheorem{eje}[defin]{Ejemplo}
\graphicspath{ {images/} }
\usepackage{afterpage}

\newcommand\blankpage{%
    \null
    \thispagestyle{empty}%
    \newpage}


% Cabeceras
\renewcommand{\title}{ZIP}
\newcommand{\subtitle}{Trabajo fin de grado}
\renewcommand{\maketitle}{{\Large{\textbf{\title}}}\\\\{\Large \subtitle}\\\rule{17cm}{0.4pt}\\}

\begin{document}
%%%%%%%%PORTADA%%%%%%%%%%%

%%% PORTADA%%%%%%
\begin{titlepage} %Creo que esto es para la numeración de páginas
\begin{center} %Que todo quede centradito

% Todo esto de abajo habría que retocarlo pero así sirve de ejemplo
\huge\textsc{Universidad Complutense de Madrid}\\[0.2in]
\includegraphics[scale=0.8]{comlu}\\[0.1in] %Introduce la imagen y la reescala, inserta un pequeño hueco con lo de debajo

\Large{Facultad de Matemáticas}\\[0.5in] %inserta un hueco mayor con lo de abajo
 %la linea horizontal
\Large{Trabajo de Fin de Grado}\\[.1in]
\Huge {Un tratamiento riguroso de la prueba ZIP}\\[0.2in]



\vfill %Llenar verticalmente
\Large {Juan Valero Oliet}\\[0.5in]
\vfill 
Dirigido por:\\
Manuel Alonso Morón\\[.1in]
\Large{Junio de 2020}
\end{center}

\end{titlepage}


%\afterpage{\blankpage}

%%%%%%%%%%%%%PORTADA%%%%%%%%%%%%%%%%%%%%%5

\tableofcontents
\chapter{Definiciones preeliminares}


\section{Variedades}

\subsection{Variedades y superficies}


Los espacios topológicos de los que nos vamos a ocupar en el siguiente trabajo son las variedades.%, que son los más relevantes desde el punto de vista de la geometría.


\begin{defin}
Una \textbf{\emph{variedad topológica}} (de ahora en adelante \emph{variedad}) es un espacio topológico Hausdorff, II AN y localmente homeomorfo a $R^n$, para algún $n\geq 0$.
\end{defin}

Como la propiedad ``ser localmente homeomorfo a $\R^n$'' es local, toda propiedad local de $\R^n$ se traslada a una variedad. Así, las variedades son localmente compactas, I AN, localmente conexas, localmente conexas por caminos y localmente simplemente conexas.\\



%COPIADO lo de abajo, cambiar las palabras!!!!!!!!!!!!

El \textit{teorema de invarianza del dominio} dice que si $W\subset \R^n$ y $W'\subset \R^m$ son abiertos y existe $\phi: W \rightarrow W'$ homeomorfismo, entonces $n=m$. Esto implica que, dado un punto $p\in X$ de una variedad, hay un único $n=n(p)$ tal que un entorno $U^p$ es homeomorfo a un abierto $U'\subset \R^n$. Llamamos $n(p)$ la dimensión en p. Claramente, para todo punto $q\in U$ podemos tomar $U$ como entorno de $q$, y por tanto $n(q)=n(p)$. Luego en toda la componente conexa de $p$, el $n$ que aparece es el mismo, y lo llamaremos dimensión de dicha componente conexa. Nótese que si escribimos $X=\sqcup X_i$, con $X_i$ componentes conexas de $X$, todas las $X_i$ son variedades. Si todas las $X_i$ tienen la misma dimensión $n$, entonces escribimos $n=dim X$ , y decimos que $X$ es una $n$-variedad.\\



\begin{eje}
\begin{itemize}
\item Las 0-variedades son espacios discretos numerables. La única 0-variedad conexa es un punto.
\item Existen dos 1-variedades conexas salvo homeomorfismo: la recta $\R$ y el círculo $\mathbb{S}^1$

\end{itemize}


\end{eje}

\begin{defin}
Una \textbf{\emph{superficie}} es una $2$-variedad.
\end{defin}

\begin{eje}
	\begin{itemize}
		\item La esfera $\mathbb{S}^2=\{(x,y,z) \in \R^3 \mid x^2+y^2+z^2=1\}$


		\item El toro $\mathbb{T}^2=\{(x,y,z)\in \R^3 \mid (\sqrt{x^2+y^2}-2)^2+z^2=1\}$
	\end{itemize}

\end{eje}

\


\begin{figure}
	\begin{center}
		\begin{tikzpicture}
			\draw (0,0) circle (2cm);
			\draw (2,0) arc[x radius=2, y radius=0.7, start angle=0, end angle=-180];
			\draw [dashed] (2,0) arc[x radius=2, y radius=0.7, start angle=0, end angle=180];
			\draw (6,-0.1) ellipse (2.9cm and 1.4 cm);
			\draw (7.5,0.1) arc[x radius=1.5, y radius=0.4, start angle=0, end angle=-180];
			\draw (7.3,-0.1) arc[x radius=1.3, y radius=0.3, start angle=0, end angle=180]; 
		\end{tikzpicture}
	\caption{$\mathbb{S}^2$ y $\mathbb{T}^2$}
	\end{center}
\end{figure}




%%%%%%%%%%%%%%%%%%%%%%%%%%%%%%%%%%%%%%%%%%%%%%%%%%%%%%%%%%%%%%%%%%%%%%%%%%%%%%%%%%%%%%%%%%%%%%%%%%%%%%%%%%%%%%%%%%%%%%%%%%%%%%%%%%%%%%%%%%%%%%%%%%%%%%%%%%%%%%%%%%%%%%%%%%%%%%%%%%%%%%%%%%%%%%%%%%%%%%%%%%%%%%%%%%%%%%%%%%%%%%%%%%%%%%%%%%%%%%%%%%%%%%%%%%%%%%%%%%%%%%%%%%%%%%%%%%%%%%%%%%%%%%%%%%%%%%%%%%%%%%%%%%%%%%%%%%%%%%%%%%%%%%%%%%%%%%%%%%%%%%

\subsection{Suma conexa de variedades}


Sean $V_1$ y $V_2$ dos $n$-variedades conexas. Dados $p_1\in V_1$ y $p_2\in V_2$ sean $U_1^{p_1}\subset V_1$, $U_2^{p_2}\subset V_2$  entornos de $p_1$ y $p_2$ respectivamente, y sean $\phi_1:U_1\to\R^n$ y $\phi_2:U_2\to\R^n$ dos homeomorfismos tales que $\phi_1(p_1)=0$ y $\phi_2(p_2)=0$. Si llamamos $B_1=\phi_1^{-1}(B_1(0))\subset V_1$ y $B_2=\phi_2^{-1}(B_1(0))\subset V_2$, consideremos $V_1^o=V_1-B_1$, $V_2^o=V_2-B_2$ y $V_1^o \sqcup V_2^o$ con la topología unión disjunta.
Se define la relación de equivalencia $\sim$ en la que si $x_1\in S_1=\phi_1^{-1}(\partial B_1(0))$, $x_2\in S_2=\phi_2^{-1}(\partial B_1(0))$, entonces $x_1\sim x_2$ si y sólo si $\phi_1(x_1)=\phi_2(x_2)$, y se considera el cociente 

$$X=\frac{V_1^o\sqcup V_2^o}{\sim}.$$

\begin{defin}
A $X$ se le llama \textbf{\textit{suma conexa}} de $V_1$ y $V_2$, y se denota por $X=V_1\#V_2$.
\end{defin}
%%%%%%%%%%%%DIBUJO DE SUMA CONEXA%%%%%%%%%%%%
\begin{prop}
$X$ es una variedad.
\end{prop}
\begin{proof}
Denotemos la proyección $\pi:V_1\#V_2\to X$ %%%%%%%%DEMOSTRACION%%%%%%%%%%%
\end{proof}



\section{Representación de superficies}

Para el teorema de clasificación necesitamos un método uniforme de representación de las superficies compactas. Representaremos todas las superficies como cocientes de polígonos con $2n$ lados. 
%%%%%%%%% QUIZAS INCLUIR UNA DEFINICION INFORMAL%%%%%%%%%%%%%%
\begin{defin}
Sea $S$ un conjunto. Una \textbf{\textit{palabra en $S$}} es una $k$-tupla ordenada de símbolos, cada uno de la forma $a$ o $a^{-1}$, para cierto $a\in S$.
\end{defin}

\begin{defin}
\label{def:rep_pol}
Una \textbf{\textit{representación poligonal}}, que denotaremos por $$\mathcal{P}=\langle S\mid W_1,...,W_k\rangle$$ es un conjunto finito S junto con un número finito de palabras $W_1,..,W_k$ de longitud $3$ o más, tal que para todo $a\in S$ existe un $W_i$ tal que $a\in W_i$. Por cuestiones de notación, cuando el conjunto $S$ esté descrito listando sus elementos, quitaremos los corchetes que rodean los elementos de $S$ y denotamos las palabras $W_i$ por youxtaposición. Por ejemplo, la presentación con $S=\{a,b\}$ y la palabras $W=(a,b,a^{-1},b^{-1})$ se escribe $\langle a,b \mid aba^{-1}b^{-1}\rangle$. Permitimos el caso especial de $S=\{a\}$ y palabras de longitud $2$, es decir $\langle a\mid aa\rangle$, $\langle a\mid a^{-1}a^{-1}\rangle$, $\langle a\mid aa^{-1}\rangle$ y $\langle a\mid a^{-1}a\rangle$.
\end{defin}
\begin{defin}
Toda representación poligonal $\mathcal{P}$ da lugar a un espacio topológico $|\mathcal{P}|$, llamado \textbf{\textit{realización geométrica de $\mathcal{P}$}} . $|\mathcal{P}|$ se obtiene de la siguiente manera:
\begin{itemize}
\item[1.] Para cada $W_i\in \mathcal{P}$ de longitud $k$, sea $P_i$ el $k$-polígono centrado en el origen con lados de longitud 1 y tal que un lado yace sobre el eje $OY$.
\item[2.] Se define una correspondencia uno a uno entre los símbolos de $W_ i$ y los lados de $P_i$ en orden inverso a las agujas del reloj, empezando por el que yace en el eje $OY$.
\item[3.] Sea $|\mathcal{P}|$ el espacio cociente de $\coprod_i P_i$ determinado identificando lados que tengan el mismo símbolo, conforme al homeomorfismo afín que hace coincidir los primeros vértices de lo lados con una dada etiqueta $a$ y los últimos vertices de los que tienen la correspondiente etiqueta $a^{-1}$ (en el sentido a las agujas del reloj).
\end{itemize}


Si $|\mathcal{P}|$ es una de las representaciones poligonales de un solo elemento, decimos que $|\mathcal{P}|$ es la esfera $\mathbb{S}^2$ si la palabra es $aa^{-1}$ o $a^{-1}a$, y el plano proyectivo $\mathbb{P}^2$ si es $aa$ o $a^{-1}a^{-1}$.

\end{defin}
\begin{figure}
\begin{center}
\begin{tikzpicture}[use optics]


\draw[put arrow={at=0.12}, put arrow={at=0.39}] (-1,0) -- (0,1) -- (1,0) -- (0,-1) -- cycle;
\draw (-1.5,0) -- (1.5,0);
\draw (0,-1.5) -- (0,1.5);
\end{tikzpicture}
\caption{Representación poligonal de $\mathbb{S}^2$}
\end{center}
\end{figure}

\begin{defin}

 Las regiones interiores, los lados y los vértices de cada polígono $P_i$ se llaman \textbf{\emph{caras, lados y vértices de la presentación}}. El número de caras es el mismo que el número de palabras, y el número de lados coincide con la suma de la longitud de las palabras.
Para un lado etiquetado $a$, el \textbf{\emph{vértice inicial}} es el primero en el sentido de las agujas del reloj, y el otro es el \textbf{\emph{vértice final}}. Para un lado etiquetado $a^{-1}$, estas definiciones se invierten. 
\end{defin}

\begin{defin}
Una representación poligonal es una \textbf{\emph{representación de una superficie}} si para todo $a\in S$, $a$ ocurre exáctamente dos veces en $W_1,...,W_k$ como $a$ o como $a^{-1}$.
\end{defin}

\begin{defin}
Si $X$ es un espacio topológico y $\mathcal{P}$ una representación poligonal cuya realización geométrica es homeomorfa a $\mathcal{P}$, decimos que $\mathcal{P}$ es una \textbf{\emph{representación de $X$}}.
\end{defin}

\begin{obs}
Un espacion topológico que admite una representación con una sola cara es conexo, pues es homeomorfo al cociente de una región poligonal conexa. Con más de una cara, puede ser o no conexo.
\end{obs}

Veamos la representación de algunas superficies importantes.

\begin{prop}
La esfera $\mathbb{S}^2$ es homeomorfa a los siguientes espacios cociente: 
%%%%%%%%%%o cocientes??????????%%%%%
\begin{itemize}
\item[(a)] El disco cerrado $\overline{\mathbb{B}}^2\subseteq \mathbb{R}^2$ módulo la relación de equivalencia generada por $(x,y)\sim (-x,y)$, para $(x,y)\in \partial \overline{\mathbb{B}}^2$
\item[(b)] El cuadrado $C=\{(x,y):|x|+|y|\leq 1\}$ módulo la relación de equivalencia generada por $(x,y)\sim(-x,y)$ para $(x,y)\in \partial S$.
\end{itemize}
\end{prop}
\begin{proof}
Para ver que cada espacio es homeomorfo a la esfera, daremos una aplicación cociente desde el espacio dado a la esfera que haga las mismas identificaciones que la relación de equivalencia, y entonces apelaremos a la unicidad del espacio cociente.\\
Para (a), vamos a definir una aplicación del disco en la esfera que envuelve cada paralelo con un segmento horizontal del disco (ver %figura)
Formalmente, esta aplicación $\pi:\overline{\mathbb{B}}^2\to \mathbb{S}^2$ vienen dada por 
$$\pi(x,y)=\left\{\begin{array}{lc}
			(-\sqrt{1-y^2} \cos\dfrac{\pi x}{\sqrt{1-y^2}}, -\sqrt{1-y^2}, y), & y\neq \pm 1 \\
			\\(0,0,y), & y=\pm1 

\end{array}
\right.$$
Es claro que $\pi$ es continua y hace las mismas identificaciones que la relación de equivalencia. Por ser sobreyectiva, es una aplicación cociente (\emph{Teorema de la aplicación cerrada \ref{teo:aplicac_cerrada}})

Para probar (b), sea $\alpha:S\to \overline{\mathbb{B}}^2$ el homeomorfismo 
\end{proof}



\begin{corol}
$\mathbb{S}^2=\langle a|aa^{-1}\rangle=\langle a,b| aba^{-1}b^{-1}\rangle$
\end{corol}
\begin{proof}
$\mathbb{S}^2=\langle a|aa^{-1}\rangle$ por la definición \ref{def:rep_pol}. La segunda igualdad es consecuencia de la proposición anterior.
%%%%%%%%%%%%Falta demo%%%%%%%%%%%%%%%%%%%%%%%%%%%%%


\end{proof}

\begin{figure}

\definecolor{uququq}{rgb}{0.25,0.25,0.25}



\begin{tikzpicture}[line cap=round,line join=round,>=triangle 45,x=1.0cm,y=1.0cm]
%\clip(-5.39,-4) rectangle (8.94,3.29);
\draw(0,0) circle (2cm);
\draw (-2.2,-0.2)-- (-1.99,0.2);
\draw (-1.99,0.2)-- (-1.85,-0.2);
\draw (1.85,-0.2)-- (1.99,0.2);
\draw (1.99,0.2)-- (2.2,-0.2);
\draw (2.5,0.15) node[anchor=north west] {$ a $};
\draw (-2.5,0.15) node[anchor=north east] {$ a $};
\draw (0.00,-2.3) node[anchor=north] {$ \mathbb{S}^2 $};
\begin{scriptsize}
\fill [color=uququq] (0,2) circle (1.5pt);
\fill [color=uququq] (0,-2) circle (1.5pt);
\end{scriptsize}
\end{tikzpicture}


\begin{tikzpicture}[line cap=round,line join=round,>=triangle 45,x=1.0cm,y=1.0cm]
%\clip(-4.3,-5.46) rectangle (18.82,6.3);
\draw(0,0) circle (2cm);
\draw (-2.2,-0.2)-- (-1.99,0.2);
\draw (-1.99,0.2)-- (-1.85,-0.2);
\draw (2.5,0.24) node[anchor=north west] {$ a $};
\draw (-2.5,0.24) node[anchor=north east] {$ a $};
\draw (0,-2.3) node[anchor=north] {$ \mathbb{P}^2 $};
\draw (1.85,0.2)-- (1.99,-0.2);
\draw (1.99,-0.2)-- (2.2,0.2);
\begin{scriptsize}
\fill [color=uququq] (0,2) circle (1.5pt);
\fill [color=uququq] (0,-2) circle (1.5pt);
\end{scriptsize}
\end{tikzpicture}
\caption{Representación poligonal de $\mathbb{2}^2 y \mathbb{P}^2$}
\end{figure}

\appendix
\chapter{Teoremas Usados}

\begin{tma}[Teorema de la aplicación cerrada]
Sea
\end{tma}
\begin{tma}
Si $D\subseteq \R^n$ es un conjunto compacto y convexo con interior no vacío, entonces $D$ es homeomorfo a $\overline{\mathbb{B}}^n.$ De hecho, dado $p\in \mathring{D}$, entonces existe un homeomorfismo $F:\overline{\mathbb{B}}^n\to D$ que envía $0$ a $p$, $\overline{\mathbb{B}}^n$ a $\mathring{D}$, y $\mathbb{S}^{n-1}$ a $\partial D$.
\end{tma}
\end{document}
