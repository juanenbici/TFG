\documentclass[10pt]{report}
\usepackage[utf8]{inputenc}

\usepackage[spanish]{babel} % Idioma
\selectlanguage{spanish}
\usepackage{graphicx} %imagenes
\usepackage{verbatim} % Para comment

\usepackage{geometry}
\geometry{ % Margenes
  a4paper,
  left=20mm,
  right=20mm,
  top=20mm,
  bottom=20mm
}
\setlength{\parindent}{0pt} % Quitar indentado parrafos automatico

\usepackage{amsmath} % Movidas útiles
\usepackage{amssymb} % Simbolos mates
\usepackage{amsthm} % Personalizar teoremas (mas abajo continuacion)
\usepackage{thmtools} % Mas movidas teoremas
\usepackage{tikz} % Para pictures
\usepackage{sectsty} % Personalizar titulos secciones
\sectionfont{\underline} % Titulo seccion subrayado

% Comandos simbolos utiles 
\newcommand{\C}{\mathbb{C}}
\newcommand{\R}{\mathbb{R}}
\newcommand{\Q}{\mathbb{Q}}
\newcommand{\Z}{\mathbb{Z}}
\newcommand{\N}{\mathbb{N}}
\newcommand{\Epsilon}{\mathcal{E}}

\newcommand{\norm}[1]{\left\lVert#1\right\rVert} % Comando para normas

% Estilos teoremas (Mejorable)
\newtheorem{defin}{Definición}
\newtheorem{tma}{Teorema}
\newtheorem*{tma*}{Teorema}
\newtheorem{corol}{Corolario}
\newtheorem{prop}{Proposición}
\newtheorem{lema}[tma]{Lema}
\newcommand{\demo}{Demostración.\\}
\newcommand{\ok}{\hfill$\square$}
\theoremstyle{remark}
\newtheorem{obs}{Observación}
\newtheorem{eje}{Ejemplo}
\graphicspath{ {images/} }
\usepackage{afterpage}

\newcommand\blankpage{%
    \null
    \thispagestyle{empty}%
    \newpage}


% Cabeceras
\renewcommand{\title}{ZIP}
\newcommand{\subtitle}{Trabajo fin de grado}
\renewcommand{\maketitle}{{\Large{\textbf{\title}}}\\\\{\Large \subtitle}\\\rule{17cm}{0.4pt}\\}

\begin{document}
%%%%%%%%PORTADA%%%%%%%%%%%

%%% PORTADA%%%%%%
\begin{titlepage} %Creo que esto es para la numeración de páginas
\begin{center} %Que todo quede centradito

% Todo esto de abajo habría que retocarlo pero así sirve de ejemplo
\huge\textsc{Universidad Complutense de Madrid}\\[0.2in]
\includegraphics[scale=0.8]{comlu}\\[0.1in] %Introduce la imagen y la reescala, inserta un pequeño hueco con lo de debajo

\Large{Facultad de Matemáticas}\\[0.5in] %inserta un hueco mayor con lo de abajo
 %la linea horizontal
\Large{Trabajo de Fin de Grado}\\[.1in]
\Huge {Un tratamiento riguroso de la prueba ZIP}\\[0.2in]



\vfill %Llenar verticalmente
\Large {Juan Valero Oliet}\\[0.5in]
\vfill 
Dirigido por:\\
Manuel Alonso Morón\\[.1in]
\Large{Junio de 2020}
\end{center}

\end{titlepage}


%\afterpage{\blankpage}

%%%%%%%%%%%%%PORTADA%%%%%%%%%%%%%%%%%%%%%5


\chapter{Definiciones preeliminares}


\section{Variedades}

\subsection{Variedades y superficies}


Los espacios topológicos de los que nos vamos a ocupar en el siguiente trabajo son las variedades.%, que son los más relevantes desde el punto de vista de la geometría.


\begin{defin}
Una variedad topológica (de ahora en adelante variedad) es un espacio topológico Hausdorff, II AN y localmente homeomorfo a $R^n$, para algún $n\geq 0$.
\end{defin}

Como la propiedad ``ser localmente homeomorfo a $\R^n$'' es local, toda propiedad local de $\R^n$ se traslada a una variedad. Así, las variedades son localmente compactas, I AN, localmente conexas, localmente conexas por caminos y localmente simplemente conexas.\\



%COPIADO lo de abajo, cambiar las palabras!!!!!!!!!!!!

El \textit{teorema de invarianza del dominio} dice que si $W\subset \R^n$ y $W'\subset \R^m$ son abiertos y existe $\phi: W \rightarrow W'$ homeomorfismo, entonces $n=m$. Esto implica que, dado un punto $p\in X$ de una variedad, hay un único $n=n(p)$ tal que un entorno $U^p$ es homeomorfo a un abierto $U'\subset \R^n$. Llamamos $n(p)$ la dimensión en p. Claramente, para todo punto $q\in U$ podemos tomar $U$ como entorno de $q$, y por tanto $n(q)=n(p)$. Luego en toda la componente conexa de $p$, el $n$ que aparece es el mismo, y lo llamaremos dimensión de dicha componente conexa. Nótese que si escribimos $X=\sqcup X_i$, con $X_i$ componentes conexas de $X$, todas las $X_i$ son variedades. Si todas las $X_i$ tienen la misma dimensión $n$, entonces escribimos $n=dim X$ , y decimos que $X$ es una $n$-variedad.\\



\begin{eje}
\begin{itemize}
\item Las 0-variedades son espacios discretos numerables. La única 0-variedad conexa es un punto.
\item Existen dos 1-variedades conexas salvo homeomorfismo: la recta $\R$ y el círculo $\mathbb{S}^1$

\end{itemize}


\end{eje}

\begin{defin}
A las 2-variedades las llamamos superficies.
\end{defin}

\begin{eje}
	\begin{itemize}
		\item La esfera $\mathbb{S}^2=\{(x,y,z) \in \R^3 \mid x^2+y^2+z^2=1\}$


		\item El toro $\mathbb{T}^2=\{(x,y,z)\in \R^3 \mid (\sqrt{x^2+y^2}-2)^2+z^2=1\}$
	\end{itemize}

\end{eje}

\


\begin{figure}
	\begin{center}
		\begin{tikzpicture}
			\draw (0,0) circle (2cm);
			\draw (2,0) arc[x radius=2, y radius=0.7, start angle=0, end angle=-180];
			\draw [dashed] (2,0) arc[x radius=2, y radius=0.7, start angle=0, end angle=180];
			\draw (6,-0.1) ellipse (2.9cm and 1.4 cm);
			\draw (7.5,0.1) arc[x radius=1.5, y radius=0.4, start angle=0, end angle=-180];
			\draw (7.3,-0.1) arc[x radius=1.3, y radius=0.3, start angle=0, end angle=180]; 
		\end{tikzpicture}
	\caption{$\mathbb{S}^2$ y $\mathbb{T}^2$}
	\end{center}
\end{figure}








% Aquí deberían ir dibujos de la esfera pero ni idea






\subsection{Suma conexa de variedades}


Sean $V_1$ y $V_2$ dos $n$-variedades conexas. Dados $p_1\in V_1$ y $p_2\in V_2$ sean $U_1^{p_1}\subset V_1$, $U_2^{p_2}\subset V_2$  entornos de $p_1$ y $p_2$ respectivamente, y sean $\phi_1:U_1\to\R^n$ y $\phi_2:U_2\to\R^n$ dos homeomorfismos tales que $\phi_1(p_1)=0$ y $\phi_2(p_2)=0$. Si llamamos $B_1=\phi_1^{-1}(B_1(0))\subset V_1$ y $B_2=\phi_2^{-1}(B_1(0))\subset V_2$, consideremos $V_1^o=V_1-B_1$, $V_2^o=V_2-B_2$ y $V_1^o \sqcup V_2^o$ con la topología unión disjunta.
Se define la relación de equivalencia $\sim$ en la que si $x_1\in S_1=\phi_1^{-1}(\partial B_1(0))$, $x_2\in S_2=\phi_2^{-1}(\partial B_1(0))$, entonces $x_1\sim x_2$ si y sólo si $\phi_1(x_1)=\phi_2(x_2)$, y se considera el cociente 

$$X=\frac{V_1^o\sqcup V_2^o}{\sim}.$$

\begin{defin}
A $X$ se le llama suma conexa de $V_1$ y $V_2$, y se denota por $X=V_1\#V_2$.
\end{defin}
%%%%%%%%%%%%DIBUJO DE SUMA CONEXA%%%%%%%%%%%%
\begin{prop}
$X$ es una variedad.
\end{prop}
\begin{demo}
Denotemos la proyección $\pi:V_1\#V_2\to X$ af
\end{demo}









\end{document}
